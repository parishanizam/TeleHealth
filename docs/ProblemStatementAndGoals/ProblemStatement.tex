\documentclass{article}

\usepackage{tabularx}
\usepackage{booktabs}

\title{Problem Statement and Goals\\\progname}

\author{\authname}

\date{}

%% Comments

\usepackage{color}

\newif\ifcomments\commentstrue %displays comments
%\newif\ifcomments\commentsfalse %so that comments do not display

\ifcomments
\newcommand{\authornote}[3]{\textcolor{#1}{[#3 ---#2]}}
\newcommand{\todo}[1]{\textcolor{red}{[TODO: #1]}}
\else
\newcommand{\authornote}[3]{}
\newcommand{\todo}[1]{}
\fi

\newcommand{\wss}[1]{\authornote{blue}{SS}{#1}} 
\newcommand{\plt}[1]{\authornote{magenta}{TPLT}{#1}} %For explanation of the template
\newcommand{\an}[1]{\authornote{cyan}{Author}{#1}}

%% Common Parts

\newcommand{\progname}{Software Engineering} % PUT YOUR PROGRAM NAME HERE
\newcommand{\authname}{Team \#22, TeleHealth Insights
\\ Mitchell Weingust
\\ Parisha Nizam
\\ Promish Kandel
\\ Jasmine Sun-Hu} % AUTHOR NAMES                  

\usepackage{hyperref}
    \hypersetup{colorlinks=true, linkcolor=blue, citecolor=blue, filecolor=blue,
                urlcolor=blue, unicode=false}
    \urlstyle{same}
                                


\begin{document}

\maketitle

\begin{table}[hp]
\caption{Revision History} \label{TblRevisionHistory}
\begin{tabularx}{\textwidth}{llX}
\toprule
\textbf{Date} & \textbf{Developer(s)} & \textbf{Change}\\
\midrule
Sept 20 & Promish Kandel& Added Initial problem statement\\
Sept 22 & Promish Kandel & Modified Problem, Added Goals, Inputs/Outputs, Environment\\
Sept 23 & Promish Kandel & Modified Goals, Added Challenge Level and Extras\\
Sept 23 & Mitchell Weingust & Proofread and Modified: Problem Statement, Goals, Stretch Goals, Challenge Level and Extras\\
\bottomrule
\end{tabularx}
\end{table}

\section{Problem Statement}
\subsection{Problem}
\hspace{2em}Children with speech difficulties often require regular assessments to track their progress
during speech therapy. These assessments are vital in monitoring development, identifying issues, and
adjusting treatment plans accordingly. For bilingual children, these assessments can be challenging
because of the shortage of bilingual  speech language pathologists (SLPs).  Parents, who are the key
informants with understanding of multiple languages spoken in the home, have the potential to become
helpers in bilingual assessments, especially in telehealth settings. However, existing bilingual language
assessment tools are often designed and built based on SLPs’ in-person practices, and lack designs
specifically addressing parents’ needs and behaviors when they act as at-home test administrators. The lack of such design
considerations could also affect the outcome from these assessments, when SLPs need to assess and understand
children’s results from remote assessments facilitated by parents. Hence, there are opportunities to build a
system that better supports bilingual language assessments at home settings for children and parents, through
providing better guidance and instructions for parents, capturing more contextual data to complement the
results for SLPs, and engaging children in these assessments.


\subsection{Inputs and Outputs}
\subsubsection{Input}
\begin{itemize}
\item Video Stream of the individual taking the assessment
\item Selecting answers via mouse clicks
\item Microphone Audio
\end{itemize}
\subsubsection{Output}
\begin{itemize}
\item Audio analysis for background noise
\item Video analysis for keyboard and face movement
\item Analysis of the selected answer for each question
\item Summary of results and analysis details for clinicians
\end{itemize}

\subsection{Stakeholders}
\begin{itemize}
    \item Parents with children that have speech difficulties
    \item Children with speech difficulties (who need to take language assessments)
    \item Clinicians (SLPs) who work with children that have speech difficulties
    \item Project Researcher, Dr. Yao Du, Clinician Assistant, Professor at the University of Southern California
    \item Project Supervisor, Dr. Irene Ye Yuan, Assistant Professor in the Department of Computing and Software at McMaster University
\end{itemize}

\subsection{Environment}
\subsubsection{Software}
Software should be cross-compatible amongst Linux, MacOS and Windows operating system
\subsubsection{Hardware}
Hardware required includes any personal computer with:
\begin{itemize}
    \item Sounds Output (Speaker, Headphones)
    \item Microphone
    \item Webcam (Internal/External)
\end{itemize}

\section{Goals}
\begin{itemize}
    \item Intuitive and helpful interface that can guide parents to effectively administer language tests.
    \begin{itemize}
        \item The application should be easy to navigate with clear and meaningful symbols.
        It should also provide feedback so that end users are aware of their interactions being processed throughout the assessment.
    \end{itemize}
    
    \item Engaging interface and interaction for children when taking the assessment.
    \begin{itemize}
        \item The webpage should have a simple but visually appealing design to keep children engaged throughout the assessment.
        \item The webpage should have colours and images to attract attention to the assessment's questions and selections.
    \end{itemize}
    
    \item Provide reliable assessment results for SLPs by capturing additional contextual data and preliminary analysis
    \begin{itemize}
        \item The application should provide additional information to SLPs to identify background interference and signs of bias or test complications.
        \item The application should filter out noise and be able to identify multiple users to detect additional guidance from others.
    \end{itemize}
    \item Data security to ensure health/sensitive records are stored and accessed securely
    \begin{itemize}
        \item The software should use a security protocol to store and retrieve sensitive data from a secure database
    \end{itemize}
    \item Provide cross-platform integration for different screen sizes 
    \begin{itemize}
        \item The webpage should be accessible for parents and children, regardless of the chosen device, by rendering correctly on all screen sizes and formats. 
    \end{itemize}
\end{itemize}


\section{Stretch Goals}
\begin{itemize}
    \item Provide a detailed interface for clinicians to analyze stored data
    \begin{itemize}
        \item The software should provide a web interface for clinicians to upload assessment results, for storing and analyzing data more efficiently.
    \end{itemize}
\end{itemize}


\section{Challenge Level and Extras}

Our challenge level is general as the project scope is limited in terms of how much research is required. The required domain knowledge
is basic web-design in a stack of our choice. We are also planning on using open-source large language models for audio and video processing.\\
\indent The extras for this project includes:
\begin{itemize}
    \item User Documentation: Providing users with a guide on how to use and better understand the system.
    \item Usability Testing: Provide the team information on how usable the design is, along with improvements on how the system looks and feels
\end{itemize}

\newpage{}

\section*{Appendix --- Reflection}

The purpose of reflection questions is to give you a chance to assess your own
learning and that of your group as a whole, and to find ways to improve in the
future. Reflection is an important part of the learning process.  Reflection is
also an essential component of a successful software development process.  

Reflections are most interesting and useful when they're honest, even if the
stories they tell are imperfect. You will be marked based on your depth of
thought and analysis, and not based on the content of the reflections
themselves. Thus, for full marks we encourage you to answer openly and honestly
and to avoid simply writing ``what you think the evaluator wants to hear.''

Please answer the following questions.  Some questions can be answered on the
team level, but where appropriate, each team member should write their own
response:


\begin{enumerate}
    \item What went well while writing this deliverable? 
    \item What pain points did you experience during this deliverable, and how
    did you resolve them?
    \item How did you and your team adjust the scope of your goals to ensure
    they are suitable for a Capstone project (not overly ambitious but also of
    appropriate complexity for a senior design project)?
\end{enumerate}  

\end{document}