\documentclass{article}

\usepackage{tabularx}
\usepackage{booktabs}

\title{Problem Statement and Goals\\\progname}

\author{\authname}

\date{}

%% Comments

\usepackage{color}

\newif\ifcomments\commentstrue %displays comments
%\newif\ifcomments\commentsfalse %so that comments do not display

\ifcomments
\newcommand{\authornote}[3]{\textcolor{#1}{[#3 ---#2]}}
\newcommand{\todo}[1]{\textcolor{red}{[TODO: #1]}}
\else
\newcommand{\authornote}[3]{}
\newcommand{\todo}[1]{}
\fi

\newcommand{\wss}[1]{\authornote{blue}{SS}{#1}} 
\newcommand{\plt}[1]{\authornote{magenta}{TPLT}{#1}} %For explanation of the template
\newcommand{\an}[1]{\authornote{cyan}{Author}{#1}}

%% Common Parts

\newcommand{\progname}{Software Engineering} % PUT YOUR PROGRAM NAME HERE
\newcommand{\authname}{Team \#22, TeleHealth Insights
\\ Mitchell Weingust
\\ Parisha Nizam
\\ Promish Kandel
\\ Jasmine Sun-Hu} % AUTHOR NAMES                  

\usepackage{hyperref}
    \hypersetup{colorlinks=true, linkcolor=blue, citecolor=blue, filecolor=blue,
                urlcolor=blue, unicode=false}
    \urlstyle{same}
                                


\begin{document}

\maketitle

\begin{table}[hp]
\caption{Revision History} \label{TblRevisionHistory}
\begin{tabularx}{\textwidth}{llX}
\toprule
\textbf{Date} & \textbf{Developer(s)} & \textbf{Change}\\
\midrule
Sept 20 & Promish Kandel& - Added Initial problem statement\\
Sept 22 & Promish Kandel & - Modified Problem, Added Goals, Inputs/Outputs, Environment\\
Sept 23 & Promish Kandel & - Modified Goals, Added Challenge Level and Extras\\
Sept 23 & Mitchell Weingust & - Proofread and Modified: Problem Statement, Goals, Stretch Goals, Challenge Level and Extras\\
Spet 23 & Promish, Mitchell, Parisha and Jasmine & - Added Reflection\\
Spet 23 & Promish, Mitchell, Parisha and Jasmine & - Final review\\
\bottomrule
\end{tabularx}
\end{table}

\section{Problem Statement}
\subsection{Problem}
\hspace{2em}Children with speech difficulties often require regular assessments to track their progress
during speech therapy. These assessments are vital in monitoring development, identifying issues, and
adjusting treatment plans accordingly. For bilingual children, these assessments can be challenging
because of the shortage of bilingual  speech language pathologists (SLPs).  Parents, who are the key
informants with understanding of multiple languages spoken in the home, have the potential to become
helpers in bilingual assessments, especially in telehealth settings. However, existing bilingual language
assessment tools are often designed and built based on SLPs’ in-person practices, and lack designs
specifically addressing parents’ needs and behaviors when they act as at-home test administrators. The lack of such design
considerations could also affect the outcome from these assessments, when SLPs need to assess and understand
children’s results from remote assessments facilitated by parents. Hence, there are opportunities to build a
system that better supports bilingual language assessments at home settings for children and parents, through
providing better guidance and instructions for parents, capturing more contextual data to complement the
results for SLPs, and engaging children in these assessments.


\subsection{Inputs and Outputs}
\subsubsection{Input}
\begin{itemize}
\item Video Stream of the individual taking the assessment
\item Selecting answers via mouse clicks
\item Microphone Audio
\end{itemize}
\subsubsection{Output}
\begin{itemize}
\item Audio analysis for background noise
\item Video analysis for keyboard and face movement
\item Analysis of the selected answer for each question
\item Summary of results and analysis details for clinicians
\end{itemize}

\subsection{Stakeholders}
\begin{itemize}
    \item Parents with children that have speech difficulties
    \item Children with speech difficulties (who need to take language assessments)
    \item Clinicians (SLPs) who work with children that have speech difficulties
    \item Project Researcher, Dr. Yao Du, Clinician Assistant, Professor at the University of Southern California
    \item Project Supervisor, Dr. Irene Ye Yuan, Assistant Professor in the Department of Computing and Software at McMaster University
\end{itemize}

\subsection{Environment}
\subsubsection{Software}
Software should be cross-compatible amongst Linux, MacOS, Windows, IOS and Android operating systems.
\subsubsection{Hardware}
Hardware required includes any personal computer or mobile device with:
\begin{itemize}
    \item Sound Output (Speaker, Headphones)
    \item Microphone
    \item Webcam (Internal/External)
\end{itemize}

\section{Goals}
\begin{itemize}
    \item Intuitive and helpful interface that can guide parents to effectively administer language tests.
    \begin{itemize}
        \item The application should be easy to navigate with clear and meaningful symbols.
        It should also provide feedback so that end users are aware of their interactions being processed throughout the assessment.
    \end{itemize}
    
    \item Engaging interface and interaction for children when taking the assessment.
    \begin{itemize}
        \item The webpage should have a simple but visually appealing design to keep children engaged throughout the assessment.
        \item The webpage should have colours and images to attract attention to the assessment's questions and selections.
    \end{itemize}
    
    \item Provide reliable assessment results for SLPs by capturing additional contextual data and preliminary analysis.
    \begin{itemize}
        \item The application should provide additional information to SLPs to identify background interference and signs of bias or test complications.
        \item The application should filter out noise and be able to identify multiple users to detect additional guidance from others.
    \end{itemize}
    \item Data security to ensure health/sensitive records are stored and accessed securely.
    \begin{itemize}
        \item The software should use a security protocol to store and retrieve sensitive data from a secure database.
    \end{itemize}
    \item Provide cross-platform integration for different devices and screen sizes.
    \begin{itemize}
        \item The webpage should be accessible for parents and children, regardless of the chosen device, by rendering correctly on all screen sizes and formats. 
    \end{itemize}
\end{itemize}


\section{Stretch Goals}
\begin{itemize}
    \item Provide an interface for clinicians to analyze stored data.
    \begin{itemize}
        \item The software should provide a web interface for clinicians to upload assessment results, for storing and analyzing data more efficiently.
    \end{itemize}
\end{itemize}


\section{Challenge Level and Extras}

Our challenge level is general as the project scope is limited in terms of how much research is required. The required domain knowledge
is basic web-design in a stack of our choice. We are also planning on using open-source large language models for audio and video processing.\\
\indent The extras for this project includes:
\begin{itemize}
    \item User Documentation: Providing users with a guide on how to use and better understand the system.
    \item Usability Testing: Receive user feedback on usability of design of the application, including improvements on how the system looks and functions.
\end{itemize}

\newpage{}

\section*{Appendix --- Reflection}

The purpose of reflection questions is to give you a chance to assess your own
learning and that of your group as a whole, and to find ways to improve in the
future. Reflection is an important part of the learning process.  Reflection is
also an essential component of a successful software development process.  

Reflections are most interesting and useful when they're honest, even if the
stories they tell are imperfect. You will be marked based on your depth of
thought and analysis, and not based on the content of the reflections
themselves. Thus, for full marks we encourage you to answer openly and honestly
and to avoid simply writing ``what you think the evaluator wants to hear.''

Please answer the following questions.  Some questions can be answered on the
team level, but where appropriate, each team member should write their own
response:


\begin{enumerate}
    \item What went well while writing this deliverable? 
    \begin{itemize}
        \item Promish Kandel: Setting up meetings was easy during this deliverable, as everyone was online and ready to start. We had questions about our goals, which Chris answered. Our supervisor helped guide the scope of our problem statement and goals, providing us with more insight into our project.
        \item Mitchell Weingust: The team was able to effectively split up parts, without relying on other team members to complete their parts. The team also communicated with each other to help one another out by reviewing documents for spelling and grammar. The team effectively collaborated with each other to ensure that the documents got completed to a standard all members were happy with, prior to the deadline, giving enough time for a team review and feedback.
        \item Parisha Nizam: For this deliverable, most aspects went well. We defined a great system to seperate tasks, review work and work together. This deliverable helped us define our goals, and process of how we will be tackling this project which made a very clear guideline. All sections were completed as detailed as possible to ensure we are on the same page and have a good understanding of our mission for the project and generally how we will be creating it
        \item Jasmine Sun-Hu: The team worked well together throughout this deliverable which led to a fairly smooth completion. Tasks were divided up easily among team members, and each member completed their assigned parts in a timely manner. The team met up several times over the timeline of the deliverable to make sure everyone was up-to-date with each other's work and any changes made, and helped each other through small problems such as setting up LaTeX to pdf in vs code or latex syntax errors. Everyone made the effort to communicate with each other and shared issues and ideas freely.
    \end{itemize}
    \item What pain points did you experience during this deliverable, and how
    did you resolve them?
    \begin{itemize}
        \item Promish Kandel: I had a fever as the due date approached, which made getting work done more challenging. Some aspects of the problem statement, such as how it should be written, weren't well defined, making it difficult to know how to format or write it. I thought of the problem statement was one to two sentences from previous classes but looking 
        at past examples, it had a different format, which I felt it wasn't covered in depth in class.
        \item Mitchell Weingust: A pain point I experienced during this deliverable was accurately describing the goals (and their significance). The team understood the problem we wanted to accomplish, but effectively describing the unique selling points, and phrasing them as goals were more difficult. I resolved this pain point by discussing with the team what excited us most about the project, and we found that the most interesting aspects could be grouped and summarized into 5 main goals. This also helped us figure out the goals' significance, which further motivated the project.
        \item Parisha Nizam: A pain point i experienced during this deliverable was pinpointing the most important risk for the proof of concept. When building a new application with unfamiliar tools, technology, and handling  sensitive info, there is significant potential for many issues to arise.  In order to determine the main risk, my team had a very thorough discussion along with our supervisor to effectively decide what to consider and how we plan on managing the risks.
        \item Jasmine Sun-Hu: One of the pain points we faced was defining our goals in a way that clearly separated five distinct objectives. During brainstorming sessions a lot of our ideas seemed to overlap, such as an intuitive interface vs  ease of use vs accessibility. To address this issue, we reached out to our supervisor for feedback as we knew they were knowledgeable with the project scope. With their feedback, we were able to refine our goals with more clarity and distinction from each other.
    \end{itemize}
    \item How did you and your team adjust the scope of your goals to ensure
    they are suitable for a Capstone project (not overly ambitious but also of
    appropriate complexity for a senior design project)?
    \begin{itemize}
        \item  Team answer: The team adjusted the scope of our goals for the project by first discussing the main ideas and requirements we had with our project supervisor. From there, our supervisor helped us narrow down the goals to those that are the most important to the system's success. The final goals we decided upon are of an appropriate complexity for a final Capstone, senior design project, as they are requirements for the system to have the desired, full functionality. As well, the goals are precise and measurable, and focus on different, integral aspects of the system.
    \end{itemize}
\end{enumerate}  

\end{document}