\documentclass[12pt, titlepage]{article}

\usepackage{booktabs}
\usepackage{tabularx}
\usepackage{hyperref}
\usepackage{enumitem}
\usepackage{pifont}
\usepackage{mdframed}

\newcommand{\checkbox}{\ding{113}} % 113 corresponds to an open square box

\hypersetup{
    colorlinks,
    citecolor=blue,
    filecolor=black,
    linkcolor=red,
    urlcolor=blue
}
\usepackage[round]{natbib}

%% Comments

\usepackage{color}

\newif\ifcomments\commentstrue %displays comments
%\newif\ifcomments\commentsfalse %so that comments do not display

\ifcomments
\newcommand{\authornote}[3]{\textcolor{#1}{[#3 ---#2]}}
\newcommand{\todo}[1]{\textcolor{red}{[TODO: #1]}}
\else
\newcommand{\authornote}[3]{}
\newcommand{\todo}[1]{}
\fi

\newcommand{\wss}[1]{\authornote{blue}{SS}{#1}} 
\newcommand{\plt}[1]{\authornote{magenta}{TPLT}{#1}} %For explanation of the template
\newcommand{\an}[1]{\authornote{cyan}{Author}{#1}}

%% Common Parts

\newcommand{\progname}{Software Engineering} % PUT YOUR PROGRAM NAME HERE
\newcommand{\authname}{Team \#22, TeleHealth Insights
\\ Mitchell Weingust
\\ Parisha Nizam
\\ Promish Kandel
\\ Jasmine Sun-Hu} % AUTHOR NAMES                  

\usepackage{hyperref}
    \hypersetup{colorlinks=true, linkcolor=blue, citecolor=blue, filecolor=blue,
                urlcolor=blue, unicode=false}
    \urlstyle{same}
                                


\begin{document}

\title{System Verification and Validation Plan for \progname{}} 
\author{\authname}
\date{\today}
	
\maketitle

\pagenumbering{roman}

\section*{Revision History}

\begin{tabularx}{\textwidth}{p{3cm}p{2cm}X}
\toprule {\bf Date} & {\bf Version} & {\bf Notes}\\
\midrule
Date 1 & 1.0 & Notes\\
Date 2 & 1.1 & Notes\\
\bottomrule
\end{tabularx}

~\\
\wss{The intention of the VnV plan is to increase confidence in the software.
However, this does not mean listing every verification and validation technique
that has ever been devised.  The VnV plan should also be a \textbf{feasible}
plan. Execution of the plan should be possible with the time and team available.
If the full plan cannot be completed during the time available, it can either be
modified to ``fake it'', or a better solution is to add a section describing
what work has been completed and what work is still planned for the future.}

\wss{The VnV plan is typically started after the requirements stage, but before
the design stage.  This means that the sections related to unit testing cannot
initially be completed.  The sections will be filled in after the design stage
is complete.  the final version of the VnV plan should have all sections filled
in.}

\newpage

\tableofcontents

\listoftables
\wss{Remove this section if it isn't needed}

\listoffigures
\wss{Remove this section if it isn't needed}

\newpage

\section{Symbols, Abbreviations, and Acronyms}

\renewcommand{\arraystretch}{1.2}
\begin{tabular}{l l} 
  \toprule		
  \textbf{symbol} & \textbf{description}\\
  \midrule 
  T & Test\\
  \bottomrule
\end{tabular}\\

\wss{symbols, abbreviations, or acronyms --- you can simply reference the SRS
  \citep{SRS} tables, if appropriate}

\wss{Remove this section if it isn't needed}

\newpage

\pagenumbering{arabic}

This document ... \wss{provide an introductory blurb and roadmap of the
  Verification and Validation plan}

\section{General Information}

\subsection{Summary}

  The software being tested TeleHealth Insights, is a web-based at-home bilingual
  speech assessment system with video and audio analysis features. The system is designed to provide clear 
  guidance to parents when administering the assessment to their children, in an environment where speech-language
  pathologists (SLPs) are unavailable. By streamlining the assessment pro-
  cess, the project aims to provide a convenient and comprehensive solution
  for SLPs to assess and support their patient's speech and language development remotely.

  TeleHealth System comprises of several software components to efficiently provide 
  a platform for children assessments to occur while actively recording for clinicians to review and analyze. 

  \subsubsection{User Interface}

  We must ensure that the interface is easy to navigate for all stakeholders involved, the parents, 
  clinicians as well as the parents. The system should be simple and testing will include ensuring 
  there is an intuitive layouts, clear instructions and labels, as well as a logical workflow.

  \subsubsection{Video and Audio Recording}

  Ensure set up of hardware devices used on application is intuitive and done properly. 

  \subsubsection{Video and Audio Analysis}

  Integrate web-based application to store recorded audio and video footage. The system will be tested to ensure it is able to 
  appropriately analyze and identify flags and bias when conducting assessments. The analyzed data should be presented 
  in a easy to understand representation

\subsection{Objectives}


The Vnv Plan will focus on the following objectives
\subsubsection{Ensure Accuracy and Correctness of Software Reading and Identifying Bias}
The primary objective of this VnV Plan is to verify the system's ability to accurately record 
and store video and audio of clients during speech therapy assessments, while also ensuring the AI model can 
effectively detect disturbances and potential biases in the assessment process. This includes generating 
a report or dashboard to present findings, ensuring results are accurate and unbiased.

\begin{itemize}
  \item Check correctness of the AI model's ability to flag valid disturbances in assessment recordings.
  \item Verify the AI's ability to accurately interpret the recordings and present valuable information to clinicians
\end{itemize}

\subsubsection{Security and Authentication}

Second main objective of this VnV plan is to ensure security and strong authentication capabilities of the software. 
\begin{itemize}
  \item Check security and anonymity of all patient data (recording), ensuring its secure storage of information and access.
  \item Check software correctness, ensuring authorized users are allowed in the system, and their appropriate role. 
  \subitem{Parents should be able to view the assessment and complete it with their child.}
  \subitem{Clinicians should be able to view assessment results and recordings}
  \subitem{Admins should be able to view and create accounts.}
  \item Verify that unauthorized users are effectively blocked from accessing 
  patient data, diagnosis reports, or any system interfaces related to sensitive information.
\end{itemize}

\subsubsection{Demonstrate Usability}
The final objective is to ensure the system is intuitive and user-friendly for parents, children, and medical professionals. 
The goal is to create an application that not only stores accurate results but also enables 
users to complete tasks quickly and efficiently.

\begin{itemize}
  \item Confirm that the user interface is user-friendly, allowing clinicians and parents to login, authenticate 
  and access relevant data or complete assessments respectively with ease. This will be done via Usability testing as outlined
  in section 4.2 under Usability as well as the Usability Survey in Appendix.   
\end{itemize}


Some objectives that are considered out of scope for this VnV plan due to resource limitations include:

\begin{itemize}
  \item \textbf{Adherence to Industry Standards: } Full compliance with every industry standard for user interface elements is not feasible due to time 
  constraints and the higher priority of AI model accuracy.
  \item \textbf{External Library Verification: } The System will be using external libraries and 
  open source  for its interface and AI model. The correctness of these will not be independently verified but verified as a whole with the system. 
\end{itemize}

\subsection{Challenge Level and Extras}

\subsubsection{Challenge Level} 

Our challenge level is general as the project scope is limited in terms of how much
research is required. The required domain knowledge is basic web-design in a
stack of our choice. We are also planning on using open-source large language
models for audio and video processing.

\subsubsection{Extras}
Our project has the following extras;
\begin{itemize}
  \item \textbf{User Documentation: } Providing users with a guide on how to use and
  better understand the system. 
  \item \textbf{Usability Testing: } Receive user feedback on usability of design of the application, 
including improvements on how the system looks and functions. This will help to ensure an 
intuitive and easy system for both clinicians and parents/children to navigate through.
\end{itemize}


\subsection{Relevant Documentation}

\wss{Reference relevant documentation.  This will definitely include your SRS
  and your other project documents (design documents, like MG, MIS, etc).  You
  can include these even before they are written, since by the time the project
  is done, they will be written.  You can create BibTeX entries for your
  documents and within those entries include a hyperlink to the documents.}

  The three main relevant documentation that helped guide us in System Verification and Validation Plan (VnV) was Software Requirements Specification
  (SRS), Development plan and MIS.  
  
  \begin{itemize}
    \item The SRS document was crucial to define descriptions, rationals and fit criteria for all of the systems main functional 
    and non functional requirements. This outlines the basis of what needs to be verified and tested in the VnV Plan. It also outlines the 
    traceability matrix that needs to be followed.
    \item Development Plan: Defined the main Goals, Objectives and extras for the system, aiding the team to understand what
     needs to be done and focused on. 
    \item MIS : The Module Interface Specification (MIS) will shape the scope of unit testing in the VnV plan and guides 
    the development of both functional and nonfunctional tests to verify specific software functions and attributes.
  \end{itemize}
  

\citet{SRS}


\section{Plan}

\wss{Introduce this section.  You can provide a roadmap of the sections to
  come.}

\subsection{Verification and Validation Team}

\wss{Your teammates.  Maybe your supervisor.
  You should do more than list names.  You should say what each person's role is
  for the project's verification.  A table is a good way to summarize this information.}

\subsection{SRS Verification Plan}

\wss{List any approaches you intend to use for SRS verification.  This may
  include ad hoc feedback from reviewers, like your classmates (like your
  primary reviewer), or you may plan for something more rigorous/systematic.}

\wss{If you have a supervisor for the project, you shouldn't just say they will
read over the SRS.  You should explain your structured approach to the review.
Will you have a meeting?  What will you present?  What questions will you ask?
Will you give them instructions for a task-based inspection?  Will you use your
issue tracker?}

\wss{Maybe create an SRS checklist?}

\subsection{Design Verification Plan}

\wss{Plans for design verification}

\wss{The review will include reviews by your classmates}

\wss{Create a checklists?}


The design verification plan outlines the strategies that the team will use to verify the usability and correctness of our
 system. The following outlines the plan for verifying the User Interface of the software as captured by system design
\begin{itemize}
  \item Shall conduct a quick review with supervisor after design documents (MIS) have been completed 
  \item Conduct peer review sessions from classmates to provide critical suggestions on improvements of the system. 
  \item Conduct final formal review with clinician Dr. Du and the team, following defined SRS and MIS documents. 
  \item Conduct ad-hoc review(s) with other teams.
\end{itemize}

The following checklist will be used to verify the system's design documents over the course of completion \\\\
\checkbox Are all requirements traceable to at least one feature/module in the MIS? \\
\checkbox Do all modules and components follow SOLID design principles? \\
\checkbox Have the creation of all modules been tracked via issues and closed once reviewed? \\
\checkbox Are all inputs and outputs of system components defined well and correctly? \\



\subsection{Verification and Validation Plan Verification Plan}

\wss{The verification and validation plan is an artifact that should also be
verified.  Techniques for this include review and mutation testing.}

\wss{The review will include reviews by your classmates}

\wss{Create a checklists?}

\subsection{Implementation Verification Plan}

The following outlines the plan for verifying implementation of the system, using both static and dynamic techniques 

\begin{itemize}
  \item Walkthrough of key components of the system with the supervisor (UI design, user authentication, video and audio analysis)
  \item Walkthrough of each components with other teammates 
  \item Running unit tests (to be implemented in VnV Plan) to verify that the implementation matches the specified design 
  outlined in MIS. This will be done automatically (dynamic) using Github Actions for each pull request made, following
  CI/CD principles. 
  \item Running system tests (both functional and non functional) described in the VnV plan to verify that the implementation 
  meets requirements
  \item Running Static analyzers including linters to help discover any potential errors or bugs in the code. Linters 
  and tools will also be used to help make the code more organized and readable. 
  \item Major code commits will be reviewed by at least one other team member before merging to main branch to ensure
   consistency and minimize chance of conflicts. 
\end{itemize}


\subsection{Automated Testing and Verification Tools}

\wss{What tools are you using for automated testing.  Likely a unit testing
  framework and maybe a profiling tool, like ValGrind.  Other possible tools
  include a static analyzer, make, continuous integration tools, test coverage
  tools, etc.  Explain your plans for summarizing code coverage metrics.
  Linters are another important class of tools.  For the programming language
  you select, you should look at the available linters.  There may also be tools
  that verify that coding standards have been respected, like flake9 for
  Python.}

\wss{If you have already done this in the development plan, you can point to
that document.}

\wss{The details of this section will likely evolve as you get closer to the
  implementation.}

\subsection{Software Validation Plan}

\wss{If there is any external data that can be used for validation, you should
  point to it here.  If there are no plans for validation, you should state that
  here.}

\wss{You might want to use review sessions with the stakeholder to check that
the requirements document captures the right requirements.  Maybe task based
inspection?}

\wss{For those capstone teams with an external supervisor, the Rev 0 demo should 
be used as an opportunity to validate the requirements.  You should plan on 
demonstrating your project to your supervisor shortly after the scheduled Rev 0 demo.  
The feedback from your supervisor will be very useful for improving your project.}

\wss{For teams without an external supervisor, user testing can serve the same purpose 
as a Rev 0 demo for the supervisor.}

\wss{This section might reference back to the SRS verification section.}

\section{System Tests}

\wss{There should be text between all headings, even if it is just a roadmap of
the contents of the subsections.}

\subsection{Tests for Functional Requirements}

\subsubsection{System Set Up}

\begin{itemize}
  \item FR-ST-SS1
    \begin{mdframed}[linewidth=0.5mm]
      \textbf{Control:} Manual \par
      \textbf{Initial State:} User is logged into the system and has access to the assessment information page. \par
      \textbf{Input:} User navigates to the page where assessment information is displayed before starting hardware checks. \par
      \textbf{Output:} User is able to view relevant information about the assessment before beginning any hardware checks. \par
      \textbf{Test Case Derivation:} Users need to be aware of the assessment requirements and goals prior to starting 
      the hardware setup to ensure they are prepared. This follows expected output found in FR-SS1 in (SRS) Document \par
      \textbf{How the test will be performed:}
      \begin{enumerate}[noitemsep]
        \item Log in as a user and navigate to the assessment information section.
        \item Verify that the page displays necessary details about the assessment, such as the purpose, requirements, and overview.
        \item Confirm that the information is accessible and readable to the user.
      \end{enumerate}
    \end{mdframed}

  \item FR-ST-SS2
    \begin{mdframed}[linewidth=0.5mm]
      \textbf{Control:} Manual \par
      \textbf{Initial State:} User is on the hardware check section of the system with audio devices connected. \par
      \textbf{Input:} User initiates the audio hardware check through the system. \par
      \textbf{Output:} User receives confirmation that the audio input and output devices are functioning correctly. \par
      \textbf{Test Case Derivation:} Verifying that audio devices are functional before the assessment prevents issues of recording audio during
      the assessment, which could effect proper identification of bias. This follows expected output found in FR-SS2 in (SRS) Document\par
      \textbf{How the test will be performed:}
      \begin{enumerate}[noitemsep]
        \item Start the audio hardware check.
        \item Confirm that the system prompts the user to test both audio input (microphone) and output (speakers/headphones).
        \item Verify that the system indicates successful audio detection when the test is performed.
        \item Test with a non-functional audio device and confirm that the system displays an appropriate error.
      \end{enumerate}
    \end{mdframed}

  \item FR-ST-SS3
    \begin{mdframed}[linewidth=0.5mm]
      \textbf{Control:} Manual \par
      \textbf{Initial State:} User is on the hardware check section of the system with a video device connected. \par
      \textbf{Input:} User initiates the video hardware check through the system. \par
      \textbf{Output:} User receives confirmation that the video capturing device is functioning correctly. \par
      \textbf{Test Case Derivation:} Ensuring video functionality prevents disruptions in assessments that require 
      visual input or interaction and ensures 
      proper recording is taken for accurate data analysis. This follows expected output found in FR-SS3 in (SRS) Document \par
      \textbf{How the test will be performed:}
      \begin{enumerate}[noitemsep]
        \item Start the video hardware check.
        \item Confirm that the system activates the video device and displays a live feed.
        \item Verify that the system confirms successful video detection if the feed is displayed correctly.
        \item Test with a non-functional video device and confirm that the system displays an error.
      \end{enumerate}
    \end{mdframed}

  \item FR-ST-SS4
    \begin{mdframed}[linewidth=0.5mm]
      \textbf{Control:} Manual \par
      \textbf{Initial State:} User has successfully completed audio and video hardware checks. \par
      \textbf{Input:} User proceeds to the tutorial section after completing hardware checks. \par
      \textbf{Output:} User is directed to a tutorial that explains the assessment process in a step-by-step manner. \par
      \textbf{Test Case Derivation:} Providing a tutorial ensures that users understand the assessment process,
       improving accuracy and compliance. This follows expected output found in FR-SS4 in (SRS) Document \par
      \textbf{How the test will be performed:}
      \begin{enumerate}[noitemsep]
        \item Complete the audio and video hardware checks.
        \item Verify that the system automatically navigates the user to a tutorial section upon completion of the hardware checks.
        \item Confirm that the tutorial provides clear, step-by-step instructions on how to proceed with the assessment.
      \end{enumerate}
    \end{mdframed}

  \item FR-ST-SS5
    \begin{mdframed}[linewidth=0.5mm]
      \textbf{Control:} Manual \par
      \textbf{Initial State:} User has completed the tutorial and is ready to begin the assessment. \par
      \textbf{Input:} User initiates the start of the assessment through the system. \par
      \textbf{Output:} User is taken to the first assessment question, and the assessment begins. \par
      \textbf{Test Case Derivation:} Allowing users to start the assessment on their own terms helps them feel prepared and reduces errors. 
      This follows expected output found in FR-SS5 in (SRS) Document\par
      \textbf{How the test will be performed:}
      \begin{enumerate}[noitemsep]
        \item After completing the tutorial, select the option to start the assessment.
        \item Confirm that the system directs the user to the first assessment question.
        \item Verify that the assessment interface is properly displayed and ready for the user to begin answering.
      \end{enumerate}
    \end{mdframed}

\end{itemize}

\subsubsection{Assessment Interface}

\begin{itemize}
  \item FR-ST-AI1
    \begin{mdframed}[linewidth=0.5mm]
      \textbf{Control:} Automatic \par
      \textbf{Initial State:} User has started the assessment and is ready to begin answering questions. \par
      \textbf{Input:} User initiates the assessment. \par
      \textbf{Output:} System begins recording both audio and video, with an indicator showing recording is active. \par
      \textbf{Test Case Derivation:} Audio and video recording are essential for capturing user responses accurately,
       and an indicator reassures users that recording is in progress. This follows expected output found in FR-AI1 in (SRS) Document. \par
      \textbf{How the test will be performed:}
      \begin{enumerate}[noitemsep]
        \item Start the assessment as a user.
        \item Confirm that the system begins recording audio and video.
        \item Verify that a visible indicator shows that recording is ongoing.
      \end{enumerate}
    \end{mdframed}

  \item FR-ST-AI2
    \begin{mdframed}[linewidth=0.5mm]
      \textbf{Control:} Automatic \par
      \textbf{Initial State:} User is on a new question within the assessment. \par
      \textbf{Input:} System progresses to a new question in the assessment. \par
      \textbf{Output:} The system plays the corresponding audio prompt for the new question. \par
      \textbf{Test Case Derivation:} The audio prompt ensures users understand each question and assessment is running as intended.
      This follows expected output found in FR-AI2 in (SRS) Document. \par
      \textbf{How the test will be performed:}
      \begin{enumerate}[noitemsep]
        \item Progress to a new question in the assessment.
        \item Verify that the system automatically plays the corresponding audio prompt.
      \end{enumerate}
    \end{mdframed}

  \item FR-ST-AI3
    \begin{mdframed}[linewidth=0.5mm]
      \textbf{Control:} Automatic \par
      \textbf{Initial State:} User has progressed to a new question in the assessment. \par
      \textbf{Input:} System loads a new question. \par
      \textbf{Output:} System displays all possible answer options for the user to select from. \par
      \textbf{Test Case Derivation:} Presenting options ensures the user can make a response selection based
       on the question, system must mark answer select as right or wrong. 
       This follows expected output found in FR-AI3 in (SRS) Document. \par
      \textbf{How the test will be performed:}
      \begin{enumerate}[noitemsep]
        \item Go to a new question in the assessment.
        \item Confirm that all answer options associated with the question are displayed.
      \end{enumerate}
    \end{mdframed}

  \item FR-ST-AI4
    \begin{mdframed}[linewidth=0.5mm]
      \textbf{Control:} Manual \par
      \textbf{Initial State:} User is viewing the answer options for a question in the assessment. \par
      \textbf{Input:} User selects one of the displayed answer options. \par
      \textbf{Output:} System highlights or otherwise indicates the user’s selected option. \par
      \textbf{Test Case Derivation:} Visual confirmation of selection minimizes user error and allows
       the user to review their choice before confirmation. 
       This follows expected output found in FR-AI4 in (SRS) Document  \par
      \textbf{How the test will be performed:}
      \begin{enumerate}[noitemsep]
        \item Select an answer option for a question.
        \item Verify that the system indicates the selected option visually (e.g., highlighting).
      \end{enumerate}
    \end{mdframed}

  \item FR-ST-AI5
    \begin{mdframed}[linewidth=0.5mm]
      \textbf{Control:} Manual \par
      \textbf{Initial State:} User has selected an answer option and is ready to proceed. \par
      \textbf{Input:} User confirms their selection. \par
      \textbf{Output:} System moves the user to the next question or stage. \par
      \textbf{Test Case Derivation:} Confirmation helps prevent unintended answers, ensuring 
      accuracy in user responses. This follows expected output found in FR-AI5 in (SRS) Document. \par
      \textbf{How the test will be performed:}
      \begin{enumerate}[noitemsep]
        \item Select and confirm an answer for a question.
        \item Verify that the system progresses to the next question or stage upon confirmation.
      \end{enumerate}
    \end{mdframed}

  \item FR-ST-AI6
    \begin{mdframed}[linewidth=0.5mm]
      \textbf{Control:} Automatic \par
      \textbf{Initial State:} User is in the process of answering questions within the assessment. \par
      \textbf{Input:} User enters and exits each question. \par
      \textbf{Output:} System records timestamps for entry and exit for each question. \par
      \textbf{Test Case Derivation:} Timestamping provides valuable tracking information for 
      synchronizing recordings and analyzing response timing. 
      This follows expected output found in FR-AI6 in (SRS) Document \par
      \textbf{How the test will be performed:}
      \begin{enumerate}[noitemsep]
        \item Begin the assessment and progress through several questions.
        \item Verify that the system logs entry and exit timestamps for each question.
      \end{enumerate}
    \end{mdframed}

  \item FR-ST-AI7
    \begin{mdframed}[linewidth=0.5mm]
      \textbf{Control:} Automatic \par
      \textbf{Initial State:} User has reached the final question in the assessment. \par
      \textbf{Input:} User completes the final question and confirms the selection. \par
      \textbf{Output:} System displays a message informing the user that the assessment is complete. \par
      \textbf{Test Case Derivation:} Notifying the user of completion provides a clear end to 
      the assessment and allows users to exit confidently. 
      This follows expected output found in FR-AI7 in (SRS) Document. \par
      \textbf{How the test will be performed:}
      \begin{enumerate}[noitemsep]
        \item Complete the final question in the assessment and confirm the selection.
        \item Verify that the system displays a completion message.
      \end{enumerate}
    \end{mdframed}
\end{itemize}

\pagebreak

\subsection{Tests for Nonfunctional Requirements}

\subsubsection{Security}

The test cases below ensures that the system meets essential 
security requirements including authentication of users and encryption of 
confidential data in the system
		
\begin{itemize}
  \begin{item}
      SR-ST-AC1
      \begin{mdframed}[linewidth=0.5mm]
          \textbf{Type:} Dynamic \par
          \textbf{Initial State:} System has multiple user roles: Admin, Parent, and Clinician. \par
          \textbf{Input/Condition:}   User with Admin role attempts to create and assign accounts to clinicians \par
          \textbf{Output/Results:}  Only Admin users can access and execute functions related 
          to clinician account creation \par
          \textbf{How the test will be performed:}
          \begin{enumerate}[noitemsep]
            \item Log in as an Admin and attempt to create and view clinician accounts. Verify that the action succeeds
            \item Log in as a non-Admin (e.g., Parent or Clinician) and try to access the same 
            function. Confirm that access is denied with an appropriate error message.
          \end{enumerate}
      \end{mdframed}
  \end{item}

  \begin{item}
    SR-ST-AC2
    \begin{mdframed}[linewidth=0.5mm]
      \textbf{Type:} Dynamic \par
      \textbf{Initial State:} Parent role created and available for testing. \par
      \textbf{Input/Condition:} User with Parent role logs in and attempts to complete assessments. \par
      \textbf{Output/Results:} Parent users can create their account, complete assessments, and log out successfully. \par
      \textbf{How the test will be performed:}
      \begin{enumerate}[noitemsep]
        \item Log in as a Parent user and attempt to start and complete an assessment. Verify successful completion and logout.
        \item Attempt to access administrative functions (e.g., creating clinician accounts). 
        Verify that access is denied with an appropriate message.
      \end{enumerate}
    \end{mdframed}
  
    \begin{item}
    SR-ST-AC3
      \begin{mdframed}[linewidth=0.5mm]
        \textbf{Type:} Dynamic \par
        \textbf{Initial State:} Clinician role with restricted access is created and available for testing. \par
        \textbf{Input/Condition:} User with Clinician role logs in and attempts to view assessment results. \par
        \textbf{Output/Results:} Clinician users can view assessment results but cannot start or 
        complete assessments as a Parent user would. \par
        \textbf{How the test will be performed:}
        \begin{enumerate}[noitemsep]
          \item Log in as a Clinician and attempt to view completed assessment results. Confirm access is granted.
          \item Attempt to start or complete an assessment and confirm access is denied with an 
          error message indicating unauthorized action.
        \end{enumerate}
      \end{mdframed}
    \end{item}

    \begin{item}
    \end{item} SR-ST-AC4
    \begin{mdframed}[linewidth=0.5mm]
      \textbf{Type:} Dynamic \par
      \textbf{Initial State:} System with user login functionality. \par
      \textbf{Input/Condition:} Users attempt to log in with correct and incorrect credentials. \par
      \textbf{Output/Results:} Users can only log in with correct credentials; unauthorized access attempts are denied. \par
      \textbf{How the test will be performed:}
      \begin{enumerate}[noitemsep]
        \item Attempt to log in with valid credentials for multiple roles (Admin, Parent, Clinician). Confirm successful login.
        \item Attempt to log in with incorrect credentials (e.g., incorrect username or password). 
        Confirm that login is denied and an error message is shown.
      \end{enumerate}
    \end{mdframed}
  \end{item}

  \begin{item}
    SR-ST-P1
    \begin{mdframed}[linewidth=0.5mm]
      \textbf{Type:} Static \par
      \textbf{Initial State:} System ready for pre-release review. \par
      \textbf{Input/Condition:} Review documentation to ensure adherence to data protection and privacy laws in the region. \par
      \textbf{Output/Results:} Confirm all applicable data protection requirements are met. \par
      \textbf{How the test will be performed:}
      \begin{enumerate}[noitemsep]
        \item Conduct a documentation review with legal and compliance teams, focusing on privacy policies, data handling, and retention practices.
        \item Verify that all data collection, storage, and usage adhere to regional privacy laws (e.g., GDPR if applicable).
      \end{enumerate}
    \end{mdframed}
  \end{item}

  \begin{item}
    SR-ST-P2
    \begin{mdframed}[linewidth=0.5mm]
      \textbf{Type:} Static, Automated \par
      \textbf{Initial State:} System with sensitive data handling enabled. \par
      \textbf{Input/Condition:} Examine data in transit and at rest. \par
      \textbf{Output/Results:} Data remains encrypted according to standard encryption protocols during transit and at rest. \par
      \textbf{How the test will be performed:}
      \begin{enumerate}[noitemsep]
        \item Use network monitoring tools to capture data packets during transmission to ensure data is encrypted.
        \item Review database configuration to verify that data at rest is encrypted. Decrypting should only be 
        possible by authorized clinician accounts.
      \end{enumerate}
    \end{mdframed}
  \end{item}

  \begin{item}
    SR-ST-P3
    \begin{mdframed}[linewidth=0.5mm]
      \textbf{Type:} Static \par
      \textbf{Initial State:} System configured for data collection. \par
      \textbf{Input/Condition:} Examine data storage for PII. \par
      \textbf{Output/Results:} System does not store any personal identifiable information beyond username and assessment recordings. \par
      \textbf{How the test will be performed:}
      \begin{enumerate}[noitemsep]
        \item Conduct a database inspection and audit to ensure no PII (e.g., address, date of birth, names) is stored.
        \item Verify data schema to confirm that only usernames and recordings are stored.
      \end{enumerate}
    \end{mdframed}
  \end{item}

  \begin{item}
    SR-ST-IM1
    \begin{mdframed}[linewidth=0.5mm]
      \textbf{Type:} Dynamic \par
      \textbf{Initial State:} System is in the account creation phase, requiring users to set up passwords. \par
      \textbf{Input/Condition:} User attempts to create an account with both weak and strong passwords. \par
      \textbf{Output/Results:} Account creation is completed only when a strong password, meeting specified 
      security criteria, is entered. Weak passwords are rejected with an error message detailing the password requirements. \par
      \textbf{How the test will be performed:}
      \begin{enumerate}[noitemsep]
        \item Attempt to create an account with a weak password (e.g., fewer than 8 characters, no special 
        characters or numbers). Verify that the system rejects the password and displays an error message with the password requirements.
        \item Attempt to create an account with a password that meets all the specified criteria (e.g., at 
        least 8 characters, containing upper and lowercase letters, numbers, and special characters). 
        Verify that account creation is successful.
        \item Repeat with different combinations of weak and strong passwords to confirm consistent 
        enforcement of the password policy.
      \end{enumerate}
    \end{mdframed}
  \end{item}

\end{itemize}

\subsubsection{Compliance}

The test cases below ensures that the system meets essential 
compliance requirements including security of the system, following the rules of the clinician and law.
		
\begin{itemize}
  \begin{item}
    SR-ST-STD1
    \begin{mdframed}[linewidth=0.5mm]
      \textbf{Type:} Static, Automated \par
      \textbf{Initial State:} User assessment data, including video and audio recordings, is stored in the system. \par
      \textbf{Input/Condition:} Examine the storage configuration and security measures applied to user assessment data. \par
      \textbf{Output/Results:} User assessment data is securely stored and associated only with usernames, 
      ensuring no additional personally identifiable information is included. Data at rest is protected by strong encryption. \par
      \textbf{How the test will be performed:}
      \begin{enumerate}[noitemsep]
        \item Review database schema to confirm that assessment data is linked only to usernames and does not
         include any additional identifiable information.
        \item Conduct an inspection of the encryption protocols used for stored data to verify compliance with
         security standards.
        \item Attempt unauthorized access to stored assessment data to ensure encryption and security measures
         prevent access by unauthorized users.
      \end{enumerate}
    \end{mdframed}
  \end{item}

\end{itemize}


\subsection{Traceability Between Test Cases and Requirements}

\wss{Provide a table that shows which test cases are supporting which
  requirements.}

\section{Unit Test Description}

\wss{This section should not be filled in until after the MIS (detailed design
  document) has been completed.}

\wss{Reference your MIS (detailed design document) and explain your overall
philosophy for test case selection.}  

\wss{To save space and time, it may be an option to provide less detail in this section.  
For the unit tests you can potentially layout your testing strategy here.  That is, you 
can explain how tests will be selected for each module.  For instance, your test building 
approach could be test cases for each access program, including one test for normal behaviour 
and as many tests as needed for edge cases.  Rather than create the details of the input 
and output here, you could point to the unit testing code.  For this to work, you code 
needs to be well-documented, with meaningful names for all of the tests.}

\subsection{Unit Testing Scope}

\wss{What modules are outside of the scope.  If there are modules that are
  developed by someone else, then you would say here if you aren't planning on
  verifying them.  There may also be modules that are part of your software, but
  have a lower priority for verification than others.  If this is the case,
  explain your rationale for the ranking of module importance.}

\subsection{Tests for Functional Requirements}

\wss{Most of the verification will be through automated unit testing.  If
  appropriate specific modules can be verified by a non-testing based
  technique.  That can also be documented in this section.}

\subsubsection{Module 1}

\wss{Include a blurb here to explain why the subsections below cover the module.
  References to the MIS would be good.  You will want tests from a black box
  perspective and from a white box perspective.  Explain to the reader how the
  tests were selected.}

\begin{enumerate}

\item{test-id1\\}

Type: \wss{Functional, Dynamic, Manual, Automatic, Static etc. Most will
  be automatic}
					
Initial State: 
					
Input: 
					
Output: \wss{The expected result for the given inputs}

Test Case Derivation: \wss{Justify the expected value given in the Output field}

How test will be performed: 
					
\item{test-id2\\}

Type: \wss{Functional, Dynamic, Manual, Automatic, Static etc. Most will
  be automatic}
					
Initial State: 
					
Input: 
					
Output: \wss{The expected result for the given inputs}

Test Case Derivation: \wss{Justify the expected value given in the Output field}

How test will be performed: 

\item{...\\}
    
\end{enumerate}

\subsubsection{Module 2}

...

\subsection{Tests for Nonfunctional Requirements}

\wss{If there is a module that needs to be independently assessed for
  performance, those test cases can go here.  In some projects, planning for
  nonfunctional tests of units will not be that relevant.}

\wss{These tests may involve collecting performance data from previously
  mentioned functional tests.}

\subsubsection{Module ?}
		
\begin{enumerate}

\item{test-id1\\}

Type: \wss{Functional, Dynamic, Manual, Automatic, Static etc. Most will
  be automatic}
					
Initial State: 
					
Input/Condition: 
					
Output/Result: 
					
How test will be performed: 
					
\item{test-id2\\}

Type: Functional, Dynamic, Manual, Static etc.
					
Initial State: 
					
Input: 
					
Output: 
					
How test will be performed: 

\end{enumerate}

\subsubsection{Module ?}

...

\subsection{Traceability Between Test Cases and Modules}

\wss{Provide evidence that all of the modules have been considered.}
				
\bibliographystyle{plainnat}

\bibliography{../../refs/References}

\newpage

\section{Appendix}

This is where you can place additional information.

\subsection{Symbolic Parameters}

The definition of the test cases will call for SYMBOLIC\_CONSTANTS.
Their values are defined in this section for easy maintenance.

\subsection{Usability Survey Questions?}

\wss{This is a section that would be appropriate for some projects.}

\newpage{}
\section*{Appendix --- Reflection}

\wss{This section is not required for CAS 741}

The information in this section will be used to evaluate the team members on the
graduate attribute of Lifelong Learning.

The purpose of reflection questions is to give you a chance to assess your own
learning and that of your group as a whole, and to find ways to improve in the
future. Reflection is an important part of the learning process.  Reflection is
also an essential component of a successful software development process.  

Reflections are most interesting and useful when they're honest, even if the
stories they tell are imperfect. You will be marked based on your depth of
thought and analysis, and not based on the content of the reflections
themselves. Thus, for full marks we encourage you to answer openly and honestly
and to avoid simply writing ``what you think the evaluator wants to hear.''

Please answer the following questions.  Some questions can be answered on the
team level, but where appropriate, each team member should write their own
response:


\begin{enumerate}
  \item What went well while writing this deliverable? 
  \item What pain points did you experience during this deliverable, and how
    did you resolve them?
  \item What knowledge and skills will the team collectively need to acquire to
  successfully complete the verification and validation of your project?
  Examples of possible knowledge and skills include dynamic testing knowledge,
  static testing knowledge, specific tool usage, Valgrind etc.  You should look to
  identify at least one item for each team member.
  \item For each of the knowledge areas and skills identified in the previous
  question, what are at least two approaches to acquiring the knowledge or
  mastering the skill?  Of the identified approaches, which will each team
  member pursue, and why did they make this choice?
\end{enumerate}

\end{document}