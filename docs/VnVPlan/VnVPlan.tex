\documentclass[12pt, titlepage]{article}

\usepackage{booktabs}
\usepackage{tabularx}
\usepackage{hyperref}
\hypersetup{
    colorlinks,
    citecolor=blue,
    filecolor=black,
    linkcolor=red,
    urlcolor=blue
}
\usepackage[round]{natbib}
\usepackage{enumitem, amssymb}
\newlist{todolist}{itemize}{2}
\setlist[todolist]{label=$\square$}
\usepackage{tcolorbox}

\usepackage[none]{hyphenat}
\usepackage{amssymb}
\usepackage{enumitem}
\usepackage{mdframed}
\usepackage{parskip}
\usepackage{longtable}
\usepackage{float}
\usepackage{diagbox}

%% Comments

\usepackage{color}

\newif\ifcomments\commentstrue %displays comments
%\newif\ifcomments\commentsfalse %so that comments do not display

\ifcomments
\newcommand{\authornote}[3]{\textcolor{#1}{[#3 ---#2]}}
\newcommand{\todo}[1]{\textcolor{red}{[TODO: #1]}}
\else
\newcommand{\authornote}[3]{}
\newcommand{\todo}[1]{}
\fi

\newcommand{\wss}[1]{\authornote{blue}{SS}{#1}} 
\newcommand{\plt}[1]{\authornote{magenta}{TPLT}{#1}} %For explanation of the template
\newcommand{\an}[1]{\authornote{cyan}{Author}{#1}}

%% Common Parts

\newcommand{\progname}{Software Engineering} % PUT YOUR PROGRAM NAME HERE
\newcommand{\authname}{Team \#22, TeleHealth Insights
\\ Mitchell Weingust
\\ Parisha Nizam
\\ Promish Kandel
\\ Jasmine Sun-Hu} % AUTHOR NAMES                  

\usepackage{hyperref}
    \hypersetup{colorlinks=true, linkcolor=blue, citecolor=blue, filecolor=blue,
                urlcolor=blue, unicode=false}
    \urlstyle{same}
                                


\begin{document}

\title{System Verification and Validation Plan for \progname{}} 
\author{\authname}
\date{\today}
	
\maketitle

\pagenumbering{roman}

\section*{Revision History}

\begin{tabularx}{\textwidth}{p{3cm}p{2cm}X}
\toprule {\bf Date} & {\bf Version} & {\bf Notes}\\
\midrule
Date 1 & 1.0 & Notes\\
Date 2 & 1.1 & Notes\\
\bottomrule
\end{tabularx}

~\\
\wss{The intention of the VnV plan is to increase confidence in the software.
However, this does not mean listing every verification and validation technique
that has ever been devised.  The VnV plan should also be a \textbf{feasible}
plan. Execution of the plan should be possible with the time and team available.
If the full plan cannot be completed during the time available, it can either be
modified to ``fake it'', or a better solution is to add a section describing
what work has been completed and what work is still planned for the future.}

\wss{The VnV plan is typically started after the requirements stage, but before
the design stage.  This means that the sections related to unit testing cannot
initially be completed.  The sections will be filled in after the design stage
is complete.  the final version of the VnV plan should have all sections filled
in.}

\newpage

\tableofcontents

\listoftables
\listoffigures
\wss{Remove this section if it isn't needed}

\newpage

\section{Symbols, Abbreviations, and Acronyms}

\renewcommand{\arraystretch}{1.2}
\begin{tabular}{l l} 
  \toprule		
  \textbf{symbol} & \textbf{description}\\
  \midrule 
  SRS & Software Requirements Specification\\
  VnV & Verification and Validation\\
  \bottomrule
\end{tabular}\\

\wss{symbols, abbreviations, or acronyms --- you can simply reference the SRS
  \citep{SRS} tables, if appropriate}

\wss{Remove this section if it isn't needed}

\newpage

\pagenumbering{arabic}

This document ... \wss{provide an introductory blurb and roadmap of the
  Verification and Validation plan}

\section{General Information}

\subsection{Summary}

\wss{Say what software is being tested.  Give its name and a brief overview of
  its general functions.}

\subsection{Objectives}

\wss{State what is intended to be accomplished.  The objective will be around
  the qualities that are most important for your project.  You might have
  something like: ``build confidence in the software correctness,''
  ``demonstrate adequate usability.'' etc.  You won't list all of the qualities,
  just those that are most important.}

\wss{You should also list the objectives that are out of scope.  You don't have 
the resources to do everything, so what will you be leaving out.  For instance, 
if you are not going to verify the quality of usability, state this.  It is also 
worthwhile to justify why the objectives are left out.}

\wss{The objectives are important because they highlight that you are aware of 
limitations in your resources for verification and validation.  You can't do everything, 
so what are you going to prioritize?  As an example, if your system depends on an 
external library, you can explicitly state that you will assume that external library 
has already been verified by its implementation team.}

\subsection{Challenge Level and Extras}

\wss{State the challenge level (advanced, general, basic) for your project.
Your challenge level should exactly match what is included in your problem
statement.  This should be the challenge level agreed on between you and the
course instructor.  You can use a pull request to update your challenge level
(in TeamComposition.csv or Repos.csv) if your plan changes as a result of the
VnV planning exercise.}

\wss{Summarize the extras (if any) that were tackled by this project.  Extras
can include usability testing, code walkthroughs, user documentation, formal
proof, GenderMag personas, Design Thinking, etc.  Extras should have already
been approved by the course instructor as included in your problem statement.
You can use a pull request to update your extras (in TeamComposition.csv or
Repos.csv) if your plan changes as a result of the VnV planning exercise.}

\subsection{Relevant Documentation}

\wss{Reference relevant documentation.  This will definitely include your SRS
  and your other project documents (design documents, like MG, MIS, etc).  You
  can include these even before they are written, since by the time the project
  is done, they will be written.  You can create BibTeX entries for your
  documents and within those entries include a hyperlink to the documents.}

\citet{SRS}

\wss{Don't just list the other documents.  You should explain why they are relevant and 
how they relate to your VnV efforts.}

\section{Plan}

\wss{Introduce this section.  You can provide a roadmap of the sections to
  come.}

\subsection{Verification and Validation Team}

\wss{Your teammates.  Maybe your supervisor.
  You should do more than list names.  You should say what each person's role is
  for the project's verification.  A table is a good way to summarize this information.}
\begin{table}[h!]
  \centering
  \begin{tabular}{ | m{5cm} | m{8cm} | }
    \toprule
    Name & Roles and Responsibilities \\
    \hline
    Mitchell Weingust & \begin{itemize}
      \item Audio Analysis Model verification
      \item System Architecture and Design Validation
      \item SRS Verification
    \end{itemize}\\
    \hline
    Parisha Nizam & \begin{itemize}
      \item Frontend Interface Verification
      \item Backend Database Verification
      \item VnV Verification
    \end{itemize}\\
    \hline
    Promish Kandel & \begin{itemize}
      \item Frontend Interface Verification
      \item Video Analysis Model verification
      \item VnV Verification
    \end{itemize}\\
    \hline
    Jasmine Sun-Hu & \begin{itemize}
      \item Backend Database Verification
      \item System Architecture and Design Validation
      \item SRS Verification
    \end{itemize}\\
    \hline
    Dr. Irene Yuan & \begin{itemize}
      \item Providing feedback (including Hands-On) during project development
    \end{itemize}\\
    \hline
    Dr. Yao Du & \begin{itemize}
      \item Providing written feedback on user experiences and testing
    \end{itemize}\\
    \hline
    Chris Schankula & \begin{itemize}
      \item Providing feedback during project development
      \item Revision recommendations
    \end{itemize}\\
    \bottomrule
  \end{tabular}
  \caption{Verification and Validation Team Table}
\end{table}

\subsection{SRS Verification Plan}

\wss{List any approaches you intend to use for SRS verification.  This may
  include ad hoc feedback from reviewers, like your classmates (like your
  primary reviewer), or you may plan for something more rigorous/systematic.}

\wss{If you have a supervisor for the project, you shouldn't just say they will
read over the SRS.  You should explain your structured approach to the review.
Will you have a meeting?  What will you present?  What questions will you ask?
Will you give them instructions for a task-based inspection?  Will you use your
issue tracker?}

\wss{Maybe create an SRS checklist?}

\subsection{Design Verification Plan}

\wss{Plans for design verification}

\wss{The review will include reviews by your classmates}

\wss{Create a checklists?}

\subsection{Verification and Validation Plan Verification Plan}

As the verification and validation plan is an artifact, it must be verified too. The team's verification of the VnV plan follows:
\begin{itemize}
  \item Peer reviews by classmates, including other teams' peer reviews, to identify areas of improvement and general feedback
  \item Documentation review by the project's supervisor, Dr. Irene Ye Yuan, to ensure that the team's planned verification and validation plan is realistic and feasible
  \item Teammate documentation reviews, to provide critical feedback and ensure that all intended goals and outcomes are met
  \item Mutation testing to ensure that changes to aspects of the source plan can be detected by test cases.
\end{itemize}

The below checklist will be used, in addition, to ensure the team's VnV plan is correct and complete.

\begin{todolist}
\item Does the VnV Plan verify all functional requirements are met?
\item Does the VnV Plan verify all non-functional requirements are met?
\item Have all peer-review issues been addressed and closed?
\item Have all members of the Verification and Validation Team contributed to the review and approved the document?
\item Are all aspects of the system boundary being verified, validated, and tested?
\item Do the system tests cover all requirements mentioned in the SRS?
\item Did the test cases detect mutations and give desired outputs?
\item Did the test cases' expected output match the actual output?
\item Is there a process for documenting and resolving defects?
\end{todolist}

\wss{The verification and validation plan is an artifact that should also be
verified.  Techniques for this include review and mutation testing.}

\wss{The review will include reviews by your classmates}

\wss{Create a checklists?}

\subsection{Implementation Verification Plan}

\wss{You should at least point to the tests listed in this document and the unit
  testing plan.}

\wss{In this section you would also give any details of any plans for static
  verification of the implementation.  Potential techniques include code
  walkthroughs, code inspection, static analyzers, etc.}

\wss{The final class presentation in CAS 741 could be used as a code
walkthrough.  There is also a possibility of using the final presentation (in
CAS741) for a partial usability survey.}

\subsection{Automated Testing and Verification Tools}

\wss{What tools are you using for automated testing.  Likely a unit testing
  framework and maybe a profiling tool, like ValGrind.  Other possible tools
  include a static analyzer, make, continuous integration tools, test coverage
  tools, etc.  Explain your plans for summarizing code coverage metrics.
  Linters are another important class of tools.  For the programming language
  you select, you should look at the available linters.  There may also be tools
  that verify that coding standards have been respected, like flake9 for
  Python.}

\wss{If you have already done this in the development plan, you can point to
that document.}

\wss{The details of this section will likely evolve as you get closer to the
  implementation.}

\subsection{Software Validation Plan}

\wss{If there is any external data that can be used for validation, you should
  point to it here.  If there are no plans for validation, you should state that
  here.}

\wss{You might want to use review sessions with the stakeholder to check that
the requirements document captures the right requirements.  Maybe task based
inspection?}

\wss{For those capstone teams with an external supervisor, the Rev 0 demo should 
be used as an opportunity to validate the requirements.  You should plan on 
demonstrating your project to your supervisor shortly after the scheduled Rev 0 demo.  
The feedback from your supervisor will be very useful for improving your project.}

\wss{For teams without an external supervisor, user testing can serve the same purpose 
as a Rev 0 demo for the supervisor.}

\wss{This section might reference back to the SRS verification section.}

\section{System Tests}

\wss{There should be text between all headings, even if it is just a roadmap of
the contents of the subsections.}

\subsection{Tests for Functional Requirements}

\wss{Subsets of the tests may be in related, so this section is divided into
  different areas.  If there are no identifiable subsets for the tests, this
  level of document structure can be removed.}

\wss{Include a blurb here to explain why the subsections below
  cover the requirements.  References to the SRS would be good here.}

\subsubsection{Authentication}

\wss{It would be nice to have a blurb here to explain why the subsections below
  cover the requirements.  References to the SRS would be good here.  If a section
  covers tests for input constraints, you should reference the data constraints
  table in the SRS.}

\hspace{2em}The test cases below focus on ensuring users can safely and securely login, create and access
their accounts without worrying about others accessing their information.

\begin{itemize}
  \begin{item}
      FR-ST-A1
      \begin{mdframed}[linewidth=0.5mm]
          \textbf{Control:} Manual \par
          \textbf{Initial State:} User has a Parent account already created and stored in the database \par
          \textbf{Input:} Selection of Parent account role for login \par
          \textbf{Output:} The expected result is the Parent account role is selected and User is brought to the Parent login screen \par
          \textbf{Test Case Derivation:} The expected output is justified based on FR-A1 in the SRS document \par
          \textbf{How the test will be performed:}
          \begin{enumerate}[noitemsep]
            \item Select 'Login' to go to login screen
            \item Select 'Parent' when prompted to select between Parent and Clinician roles
            \item User is brought to the Parent login screen
          \end{enumerate}
      \end{mdframed}
  \end{item}
  \begin{item}
    FR-ST-A2
    \begin{mdframed}[linewidth=0.5mm]
      \textbf{Control:} Manual \par
      \textbf{Initial State:} User has a Clinician account already created and stored in the database \par
      \textbf{Input:} Selection of Clinician account role for login \par
      \textbf{Output:} The expected result is the Clinician account role is selected and User is brought to the Clinician login screen \par
      \textbf{Test Case Derivation:} The expected output is justified based on FR-A1 in the SRS document \par
      \textbf{How the test will be performed:}
      \begin{enumerate}[noitemsep]
        \item Select 'Login' to go to login screen
        \item Select 'Clinician' when prompted to select between Parent and Clinician roles
        \item User is brought to the Clinician login screen
      \end{enumerate}
  \end{mdframed}
  \end{item}
  \begin{item}
    FR-ST-A3
    \begin{mdframed}[linewidth=0.5mm]
      \textbf{Control:} Manual \par
      \textbf{Initial State:} User does not have a Parent account stored in the database \par
      \textbf{Input:} Selection of 'Create Account', with a username that does not exist in the database, upon attempting to access the system \par
      \textbf{Output:} The expected result is a new Parent account is created \par
      \textbf{Test Case Derivation:} The expected output is justified based on FR-A2 in the SRS document \par
      \textbf{How the test will be performed:}
      \begin{enumerate}[noitemsep]
        \item Select 'Create Account' to go to create account screen
        \item Enter unique username that is not in the database
        \item Enter account credentials (to complete account create process)
        \item Parent account is created
      \end{enumerate}
  \end{mdframed}
  \end{item}
  \begin{item}
    FR-ST-A4
    \begin{mdframed}[linewidth=0.5mm]
      \textbf{Control:} Manual \par
      \textbf{Initial State:} User does not have a Parent account stored in the database \par
      \textbf{Input:} Selection of 'Create Account', with a username that exists in the database, upon attempting to access the system \par
      \textbf{Output:} The expected result is a new Parent account fails to be created \par
      \textbf{Test Case Derivation:} The expected output is justified based on FR-A2 in the SRS document \par
      \textbf{How the test will be performed:}
      \begin{enumerate}[noitemsep]
        \item Select 'Create Account' to go to create account screen
        \item Enter username that already exists in the database
        \item System communicates the account could not be created
        \item System prompts user to select a new username
        \item Parent account is not created
      \end{enumerate}
  \end{mdframed}
  \end{item}
\begin{item}
  FR-ST-A5
  \begin{mdframed}[linewidth=0.5mm]
    \textbf{Control:} Manual \par
    \textbf{Initial State:} User has Admin privileges, attempting to create a new Clinician account \par
    \textbf{Input:} Admin user selects option to 'Create Account', with a username that does not exist in the database, upon attempting to access the system \par
    \textbf{Output:} The expected result is a new Clinician account is created \par
    \textbf{Test Case Derivation:} The expected output is justified based on FR-A3 in the SRS document \par
    \textbf{How the test will be performed:}
    \begin{enumerate}[noitemsep]
      \item Admin user is logged into their account
      \item Select 'Create Account' to go to create account screen
      \item Enter unique username that is not in the database
      \item Enter account credentials (to complete account create process)
      \item Clinician account is created
    \end{enumerate}
\end{mdframed}
\end{item}
\begin{item}
  FR-ST-A6
  \begin{mdframed}[linewidth=0.5mm]
    \textbf{Control:} Manual \par
    \textbf{Initial State:} User has Admin privileges, attempting to create a new Clinician account \par
    \textbf{Input:} Admin user selects option to 'Create Account', with a username that exists in the database, upon attempting to access the system \par
    \textbf{Output:} The expected result is a new Clinician account fails to be created \par
    \textbf{Test Case Derivation:} The expected output is justified based on FR-A3 in the SRS document \par
    \textbf{How the test will be performed:}
    \begin{enumerate}[noitemsep]
      \item Admin user is logged into their account
      \item Select 'Create Account' to go to create account screen
      \item Enter username that already exists in the database
      \item System communicates the account could not be created
      \item System prompts admin user to select a new username
      \item Clinician account is not created
    \end{enumerate}
\end{mdframed}
\end{item}
\begin{item}
  FR-ST-A7
  \begin{mdframed}[linewidth=0.5mm]
    \textbf{Control:} Manual \par
    \textbf{Initial State:} User is on their corresponding role's login page, with an account already created and stored in the database \par
    \textbf{Input:} Unique username and corresponding password that exists in the database \par
    \textbf{Output:} The expected result is a successful login to a user's account \par
    \textbf{Test Case Derivation:} The expected output is justified based on FR-A4 in the SRS document \par
    \textbf{How the test will be performed:}
    \begin{enumerate}[noitemsep]
      \item On login screen
      \item Enter unique username
      \item Enter corresponding password
      \item Select login to enter account
      \item Logged into account
    \end{enumerate}
\end{mdframed}
\end{item}
\begin{item}
  FR-ST-A8
  \begin{mdframed}[linewidth=0.5mm]
    \textbf{Control:} Manual \par
    \textbf{Initial State:} User is logged into their account \par
    \textbf{Input:} Selection of 'logout' \par
    \textbf{Output:} The expected result is a successful logout from a user's account \par
    \textbf{Test Case Derivation:} The expected output is justified based on FR-A5 in the SRS document \par
    \textbf{How the test will be performed:}
    \begin{enumerate}[noitemsep]
      \item Logged into account
      \item Select 'logout'
      \item System logs user out of their account
      \item Logout confirmation is displayed to the user
    \end{enumerate}
\end{mdframed}
\end{item}
\end{itemize}

\subsubsection{Area of Testing2}

...

\subsection{Tests for Nonfunctional Requirements}

\wss{The nonfunctional requirements for accuracy will likely just reference the
  appropriate functional tests from above.  The test cases should mention
  reporting the relative error for these tests.  Not all projects will
  necessarily have nonfunctional requirements related to accuracy.}

\wss{For some nonfunctional tests, you won't be setting a target threshold for
passing the test, but rather describing the experiment you will do to measure
the quality for different inputs.  For instance, you could measure speed versus
the problem size.  The output of the test isn't pass/fail, but rather a summary
table or graph.}

\wss{Tests related to usability could include conducting a usability test and
  survey.  The survey will be in the Appendix.}

\wss{Static tests, review, inspections, and walkthroughs, will not follow the
format for the tests given below.}

\wss{If you introduce static tests in your plan, you need to provide details.
How will they be done?  In cases like code (or document) walkthroughs, who will
be involved? Be specific.}

\subsubsection{Usability and Humanity}
\hspace{2em}The test cases below ensures that the system meets usability and humanity
requirements for users to have an enjoyable and accessible experience.

\begin{itemize}
  \item UH-ST-EOU1 (covers UH-EOU1, UH-EOU2, UH-LI1, UH-UP1, UH-AR1)
  \begin{mdframed}[linewidth=0.5mm]
      \textbf{Type:} Usability, Manual\par
      \textbf{Initial State:} System is complete, functional, and ready for user interaction. \par
      \textbf{Input/Condition:} Users complete one full assessment using the system. \par
      \textbf{Output/Results:} User answers questions in the Usability Survey (6.2), and results are culminated \par
      \textbf{How the test will be performed:}
      \begin{enumerate}[noitemsep]
        \item User have access to the system
        \item User completes one full assessment using the system
        \item Upon completion of the assessment, user is requested to fill out a usability survey
        \item Results are stored
        \item Usability scores are averaged across users
      \end{enumerate}
  \end{mdframed}
  \item UH-ST-PI1 (covers UH-PI1)
  \begin{mdframed}[linewidth=0.5mm]
      \textbf{Type:} Static \par
      \textbf{Initial State:} System, including assessments, have been completed. \par
      \textbf{Input/Condition:} List of available languages to perform assessments in is available to be selected and listed\par
      \textbf{Output/Results:} Count the number of available languages for the assessment \par
      \textbf{How the test will be performed:}
      \begin{enumerate}[noitemsep]
        \item View list of available languages
        \item Count number of languages are available for the assessment
      \end{enumerate}
  \end{mdframed}
  \item UH-ST-LI2 (covers UH-LI2)
  \begin{mdframed}[linewidth=0.5mm]
      \textbf{Type:} Manual \par
      \textbf{Initial State:} User documentation has been completed and made available to users. \par
      \textbf{Input/Condition:} Link to documentation is available on the system's frontend interface, and can be accessed\par
      \textbf{Output/Results:} Verify link takes user to access documentation \par
      \textbf{How the test will be performed:}
      \begin{enumerate}[noitemsep]
        \item Select 'documentation'
        \item User goes to documentation screen
        \item User has access to view up-to-date, available documentation
      \end{enumerate}
  \end{mdframed}
\end{itemize}

\subsubsection{Operational and Environmental}
\hspace{2em}The test cases below ensures that the system can be used in a variety of environments,
along with the requirements for which users are expected to use the system within, and the capabilities
and qualities the system has to interact with adjacent systems in the environment.

\begin{itemize}
  \item OE-ST-EPE1 (covers OE-EPE1)
  \begin{mdframed}[linewidth=0.5mm]
      \textbf{Type:} Usability, Manual\par
      \textbf{Initial State:} System is complete, functional, and ready for user interaction. \par
      \textbf{Input/Condition:} Testing the system, including the assessment, on a variety of screen sizes \par
      \textbf{Output/Results:} The system's displayed elements will scale appropriately to different screen sizes \par
      \textbf{How the test will be performed:}
      \begin{enumerate}[noitemsep]
        \item User logs into the system
        \item User completes one full assessment using the system
        \item Upon completion of the assessment, user is requested to fill out a usability survey
        \item User answers questions about their screensize and if the test scaled accordingly
        \item Results are stored for review
      \end{enumerate}
  \end{mdframed}
  \item OE-ST-WE1 (covers OE-WE1, OE-WE2)
  \begin{mdframed}[linewidth=0.5mm]
      \textbf{Type:} Dynamic, Manual \par
      \textbf{Initial State:} System, including assessments, have been completed. \par
      \textbf{Input/Condition:} User attempts to start system setup\par
      \textbf{Output/Results:} Device verification displayed on-screen, informing the user that the environment they're in is suitable for the assessment \par
      \textbf{How the test will be performed:}
      \begin{enumerate}[noitemsep]
        \item Select 'system setup'
        \item System checks if connected to the internet
        \item System checks audio input is not noisy
        \item System checks video input is clear
        \item System displays to the user their device is ready for the assessment to be used in the current environment
      \end{enumerate}
  \end{mdframed}
  \item OE-ST-IA1 (covers OE-IA1)
  \begin{mdframed}[linewidth=0.5mm]
      \textbf{Type:} Functional, Dynamic \par
      \textbf{Initial State:} System is connected to external server for retrieving and storing data \par
      \textbf{Input/Condition:} Assessment is complete, and results need to be stored\par
      \textbf{Output/Results:} Verify results are stored in the external server \par
      \textbf{How the test will be performed:}
      \begin{enumerate}[noitemsep]
        \item Complete assessment
        \item Access external server
        \item Check if results have been uploaded to server
        \item Access results to ensure data has been uploaded successfully
      \end{enumerate}
  \end{mdframed}
\end{itemize}
		
\paragraph{Title for Test}

\begin{enumerate}

\item{test-id1\\}

Type: Functional, Dynamic, Manual, Static etc.
					
Initial State: 
					
Input/Condition: 
					
Output/Result: 
					
How test will be performed: 
					
\item{test-id2\\}

Type: Functional, Dynamic, Manual, Static etc.
					
Initial State: 
					
Input: 
					
Output: 
					
How test will be performed: 

\end{enumerate}

\subsubsection{Area of Testing2}

...

\subsection{Traceability Between Test Cases and Requirements}

\wss{Provide a table that shows which test cases are supporting which
  requirements.}

\section{Unit Test Description}

\wss{This section should not be filled in until after the MIS (detailed design
  document) has been completed.}

\wss{Reference your MIS (detailed design document) and explain your overall
philosophy for test case selection.}  

\wss{To save space and time, it may be an option to provide less detail in this section.  
For the unit tests you can potentially layout your testing strategy here.  That is, you 
can explain how tests will be selected for each module.  For instance, your test building 
approach could be test cases for each access program, including one test for normal behaviour 
and as many tests as needed for edge cases.  Rather than create the details of the input 
and output here, you could point to the unit testing code.  For this to work, you code 
needs to be well-documented, with meaningful names for all of the tests.}

\subsection{Unit Testing Scope}

\wss{What modules are outside of the scope.  If there are modules that are
  developed by someone else, then you would say here if you aren't planning on
  verifying them.  There may also be modules that are part of your software, but
  have a lower priority for verification than others.  If this is the case,
  explain your rationale for the ranking of module importance.}

\subsection{Tests for Functional Requirements}

\wss{Most of the verification will be through automated unit testing.  If
  appropriate specific modules can be verified by a non-testing based
  technique.  That can also be documented in this section.}

\subsubsection{Module 1}

\wss{Include a blurb here to explain why the subsections below cover the module.
  References to the MIS would be good.  You will want tests from a black box
  perspective and from a white box perspective.  Explain to the reader how the
  tests were selected.}

\begin{enumerate}

\item{test-id1\\}

Type: \wss{Functional, Dynamic, Manual, Automatic, Static etc. Most will
  be automatic}
					
Initial State: 
					
Input: 
					
Output: \wss{The expected result for the given inputs}

Test Case Derivation: \wss{Justify the expected value given in the Output field}

How test will be performed: 
					
\item{test-id2\\}

Type: \wss{Functional, Dynamic, Manual, Automatic, Static etc. Most will
  be automatic}
					
Initial State: 
					
Input: 
					
Output: \wss{The expected result for the given inputs}

Test Case Derivation: \wss{Justify the expected value given in the Output field}

How test will be performed: 

\item{...\\}
    
\end{enumerate}

\subsubsection{Module 2}

...

\subsection{Tests for Nonfunctional Requirements}

\wss{If there is a module that needs to be independently assessed for
  performance, those test cases can go here.  In some projects, planning for
  nonfunctional tests of units will not be that relevant.}

\wss{These tests may involve collecting performance data from previously
  mentioned functional tests.}

\subsubsection{Module ?}
		
\begin{enumerate}

\item{test-id1\\}

Type: \wss{Functional, Dynamic, Manual, Automatic, Static etc. Most will
  be automatic}
					
Initial State: 
					
Input/Condition: 
					
Output/Result: 
					
How test will be performed: 
					
\item{test-id2\\}

Type: Functional, Dynamic, Manual, Static etc.
					
Initial State: 
					
Input: 
					
Output: 
					
How test will be performed: 

\end{enumerate}

\subsubsection{Module ?}

...

\subsection{Traceability Between Test Cases and Modules}

\wss{Provide evidence that all of the modules have been considered.}
				
\bibliographystyle{plainnat}

\bibliography{../../refs/References}

\newpage

\section{Appendix}

This is where you can place additional information.

\subsection{Symbolic Parameters}

The definition of the test cases will call for SYMBOLIC\_CONSTANTS.
Their values are defined in this section for easy maintenance.

\subsection{Usability Survey Questions?}
The following questions depict the first draft of the team's usability.\\
The usability survey will be conducted after participants have engaged with testing and using the system for at least one iteration of the assessment.\\

\newcommand{\likertScale}{
    \begin{center}
        Strongly Disagree \hfill Disagree \hfill Neutral \hfill Agree \hfill Strongly Agree
    \end{center}
}

\newcommand{\insertAnswerHere}{
  \begin{tcolorbox}[width=0.5\textwidth,
    colframe=black,
    colback=white,
    boxrule=0.1mm,
    sharp corners]
  \textit{Insert answer here...}
  \end{tcolorbox}
}

\textbf{Please select the statement that best describes your experience for each of the following:}
\begin{enumerate}
  \item Learning how to use the system was easy:\likertScale
  \item Setting up the system was easy:\likertScale
  \item I found the assessment interface easy to use:\likertScale
  \item Navigating the interface was easy:\likertScale
  \item All the button interactions reacted and responded how I thought they should:\likertScale
  \item The information on screen was easy to read and understand:\likertScale
  \item I like the organization of the assessment interface:\likertScale
  \item I enjoyed my overall experience using the TeleHealth Insights platform:\likertScale
\end{enumerate}
\textbf{Answer the following:}

\begin{enumerate}
    \setcounter{enumi}{8}
    \item What was the most difficult part of using the platform? \insertAnswerHere
    \item Did you encounter any bugs/problems while using the platform? If so, what were they? \insertAnswerHere
    \item What was your favourite part of the experience? Why? \insertAnswerHere
    \item What was your least favourite part of the experience? Why? \insertAnswerHere
    \item Were there any aspects of the platform that you found unnecessary? Why? \insertAnswerHere
    \item Which part of the platform needs the most improvement? Why? \insertAnswerHere
    \item What would you like changed to make the platform easier to use? \insertAnswerHere
    \item What is the device you ran the system on? \insertAnswerHere
    \item Did the screen's visuals scale appropriately to the screen size? \insertAnswerHere
    \item Do you have any additional feedback? \insertAnswerHere
\end{enumerate}

\newpage{}
\section*{Appendix --- Reflection}

\wss{This section is not required for CAS 741}

The information in this section will be used to evaluate the team members on the
graduate attribute of Lifelong Learning.

The purpose of reflection questions is to give you a chance to assess your own
learning and that of your group as a whole, and to find ways to improve in the
future. Reflection is an important part of the learning process.  Reflection is
also an essential component of a successful software development process.  

Reflections are most interesting and useful when they're honest, even if the
stories they tell are imperfect. You will be marked based on your depth of
thought and analysis, and not based on the content of the reflections
themselves. Thus, for full marks we encourage you to answer openly and honestly
and to avoid simply writing ``what you think the evaluator wants to hear.''

Please answer the following questions.  Some questions can be answered on the
team level, but where appropriate, each team member should write their own
response:


\begin{enumerate}
  \item What went well while writing this deliverable?\\
  \newline
  \hspace{2em} Mitchell: One of the things that went well during this deliverable was splitting up
  the System Tests for both Functional Requirements and Nonfunctional Requirements. The team decided
  to assign the functional and nonfunctional tests to the team members that wrote those particular
  requirements in the SRS. This allowed the team to use their prior knowledge on the requirements
  to develop detailed test plans to accurately test the requirements. This also meant that team members
  that were familiar with the requirements knew the limitations of the tests, and could improve upon
  given feedback to strengthen the plan further.\\
  \item What pain points did you experience during this deliverable, and how
    did you resolve them?\\
  \newline
  \hspace{2em} Mitchell: One of the pain points I experienced during this deliverable was understanding the
  format needed for the System Tests section. I found it difficult to get started because of the formality
  of the tests. As well, I wanted to make sure that the tests were consistent among team members, and a difference
  in formatting would be difficult for someone reading the document to understand. To resolve this, the team decided
  to follow a consistent formatting outline, and follow along a sample so that everyone knew what test cases
  should look like for this deliverable.\\
  \item What knowledge and skills will the team collectively need to acquire to
  successfully complete the verification and validation of your project?
  Examples of possible knowledge and skills include dynamic testing knowledge,
  static testing knowledge, specific tool usage, Valgrind etc.  You should look to
  identify at least one item for each team member.\\
  \newline
  \hspace{2em} Mitchell: One of the pieces of knowledge and skills the team collectively needs to acquire to 
  successfully complete the verification and validation of the project is understanding how to properly perform
  usability testing. This will be important to the team because it is one of the project's chosen extras, so understanding
  usability testing is crucial for the success of the project. As well, it will inform us on how users actually engage,
  interact, and understand our system. Another skill the team will need to acquire is dynamic testing using different frameworks.
  It will be beneficial to perform dynamic testing to get verify each of our desired outcomes of tests against the actual outcomes.
  This will provide the team with accurate, reliable results and conclusions.\\
  \item For each of the knowledge areas and skills identified in the previous
  question, what are at least two approaches to acquiring the knowledge or
  mastering the skill?  Of the identified approaches, which will each team
  member pursue, and why did they make this choice?\\
  \newline
  \hspace{2em} Mitchell: One approach to acquire knowledge for usability testing is referring to SFWRENG 4HC3 course notes,
  as usability testing is one of the major topics the course focuses on, with lots of examples and details. To obtain knowledge for dynamic tests, I plan on learning pytest to learn about dynamic testing in Python, which will be the
  team's language of choice for developing machine learning models.\\
\end{enumerate}

\end{document}