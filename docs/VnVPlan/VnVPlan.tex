\documentclass[12pt, titlepage]{article}

\usepackage{booktabs}
\usepackage{tabularx}
\usepackage{hyperref}
\usepackage{enumitem}
\usepackage{pifont}
\usepackage{mdframed}

\newcommand{\checkbox}{\ding{113}} % 113 corresponds to an open square box

\hypersetup{
    colorlinks,
    citecolor=blue,
    filecolor=black,
    linkcolor=red,
    urlcolor=blue
}
\usepackage[round]{natbib}
\usepackage{enumitem, amssymb}
\newlist{todolist}{itemize}{2}
\setlist[todolist]{label=$\square$}
\usepackage{tcolorbox}

\usepackage[none]{hyphenat}
\usepackage{amssymb}
\usepackage{enumitem}
\usepackage{mdframed}
\usepackage{parskip}
\usepackage{longtable}
\usepackage{float}
\usepackage{diagbox}
\usepackage{array}
\usepackage{geometry}
\usepackage{pdflscape}
\geometry{margin=1in}

\input{../Comments}
\input{../Common}

\begin{document}

\title{System Verification and Validation Plan for \progname{}} 
\author{\authname}
\date{\today}
	
\maketitle

\pagenumbering{roman}

\section*{Revision History}

\begin{table}[hp]
  \caption{Revision History} \label{TblRevisionHistory}
  \begin{tabularx}{\textwidth}{p{1.5cm}p{1cm}p{3.5cm}X}
  \toprule {\textbf{Date}} & {\textbf{Vers.}} & {\textbf{Contributors}} & {\textbf{Notes}}\\
  \midrule
  10/30/24 & 1.0 & Mitchell Weingust & Added: 3.1 Verification and Validation Team, 3.4 Verification and Validation Plan, 6.2 Usability Survey Questions \\
  10/30/24 & 1.1 & Promish Kandel & Added: 3.6 Automated Testing and Verification Tools, 3.7 Software Validation Plan, 4.1 Data Collection and Storage, Video and Audio Data Analysis FR tests, 4.2 Performance NFR tests \\
  10/30/24 & 1.2 & Parisha Nizam & Added: 2.1 Summary, 2.2 Objectives, 2.3 Challenge Level and Extras, 2.4 Relevant Documentation \\
  11/01/24 & 1.3 & Jasmine Sun-Hu & Added: 3.2 SRS Validation Plan, 4.1 Data Processing and Display, 4.2 Look and Feel FR tests, 4.2 Maintenance and Support NFR tests, 4.2 Cultural NFR tests \\
  11/02/24 & 1.4 & Mitchell Weingust & Added: 4.1 Authentication FR tests, 4.2 Usability and Humanity NFR tests, 4.2 Operational and Environmental NFR tests \\
  11/02/24 & 1.5 & Parisha Nizam & Added: 4.1 System Set Up \& Assessment Interface FR tests, 4.2 Security and Compliance NFR Tests \\
  11/04/24 & 1.6 & Mitchell Weingust & Added: Reflection \\
  11/04/24 & 1.7 & Jasmine Sun-Hu & Added: 6.1 Symbolic Parameters, Reflection \\
  11/04/24 & 1.8 & Promish Kandel & Added: 1 Symbols, Abbreviations and Acronyms, Reflection \\
  11/04/24 & 1.9 & Parisha Nizam & Added: Reflection \\
  03/22/25 & 2.1 & Jasmine Sun-Hu & Implemented TA Feedback: \href{https://github.com/parishanizam/TeleHealth/issues/292}{Adjusted formatting and added subsection intro text}, \href{https://github.com/parishanizam/TeleHealth/issues/293}{Added HIPAA adherence}\\
  03/23/25 & 2.2 & Jasmine Sun-Hu & Implemented Peer Feedback: \href{https://github.com/parishanizam/TeleHealth/issues/224}{Added test case derivations to NFRs}, \href{https://github.com/parishanizam/TeleHealth/issues/223}{Clarified objectives}, \href{https://github.com/parishanizam/TeleHealth/issues/222}{Adjusted FR-ST-DSC5 test case method}, \\
  \bottomrule
  \end{tabularx}
\end{table}


\newpage

\tableofcontents

\listoftables

\newpage

\section{Symbols, Abbreviations, and Acronyms}

\renewcommand{\arraystretch}{1.2}
\begin{tabular}{l l} 
  \toprule		
  \textbf{symbol} & \textbf{description}\\
  \midrule 
  SRS & Software Requirements Specification\\
  VnV & Verification and Validation\\
  PII & Personally Identifiable Information\\
  \bottomrule
\end{tabular}\\

\newpage

\pagenumbering{arabic}

\hspace{2 em} This document contains the team's verification and validation plan for the TeleHealth
Insights project. This document features general information, overall plan, system tests,
the eventual addition of unit tests, and usability survey questions.

\section{General Information}

\subsection{Summary}

  The software being tested TeleHealth Insights, is a web-based at-home bilingual
  speech assessment system with video and audio analysis features. The system is designed to provide clear 
  guidance to parents when administering the assessment to their children, in an environment where speech-language
  pathologists (SLPs) are unavailable. By streamlining the assessment pro-
  cess, the project aims to provide a convenient and comprehensive solution
  for SLPs to assess and support their patient's speech and language development remotely.

  TeleHealth System comprises of several software components to efficiently provide 
  a platform for children assessments to occur while actively recording for clinicians to review and analyze. 

  \subsubsection{User Interface}

  We must ensure that the interface is easy to navigate for all stakeholders involved, the parents, 
  clinicians as well as the parents. The system should be simple and testing will include ensuring 
  there is an intuitive layouts, clear instructions and labels, as well as a logical workflow.

  \subsubsection{Video and Audio Recording}

  Ensure set up of hardware devices used on application is intuitive and done properly. 

  \subsubsection{Video and Audio Analysis}

  Integrate web-based application to store recorded audio and video footage. The system will be tested to ensure it is able to 
  appropriately analyze and identify flags and bias when conducting assessments. The analyzed data should be presented 
  in a easy to understand representation

  \subsection{Objectives}

The VnV Plan aims to evaluate whether the system meets key project goals in terms of recording quality, AI performance, data security, and ease of use. The following subsections clarify the scope and definitions of the main objectives within the context of this project.

\subsubsection{Ensure Accuracy and Correctness of Software Reading and Identifying Bias}
The primary objective of this VnV Plan is to ensure that the system can:
\begin{itemize}
  \item Correctly record video and audio: The system should capture audio and video that meet minimum quality thresholds, such as resolution, clarity, and synchronization between video and audio streams, as defined in performance requirements (Section 4.2.3).
  \item Accurately detect disturbances or bias: The AI analysis models should flag disturbances (e.g., background noise, occluded faces, off-task behavior) and potential assessment biases (e.g., speech delays, distracted environment) with a predefined success rate (e.g., \texttt{VERY\_HIGH\_SUCCESS\_RATE}) in comparison with human-reviewed ground truth.
  \item Provide valid outputs: Results of the analysis should be correctly summarized in clinician-facing dashboards or reports, with visualizations that align with recorded content.
\end{itemize}

\subsubsection{Security and Authentication}
The second objective is to verify that the system enforces strong authentication protocols and protects sensitive data throughout its lifecycle. Specifically:
\begin{itemize}
  \item Data privacy and encryption: All assessment recordings and user data must be securely stored and encrypted both at rest and in transit, meeting compliance requirements (e.g., HIPAA, as referenced in Section 4.2.8).
  \item Role-based access control: Only authorized users should be able to perform role-specific tasks:
    \begin{itemize}
      \item Parents can complete assessments with their children.
      \item Clinicians can view assessment results and recordings.
      \item Admins can manage account creation and system configuration.
    \end{itemize}
  \item Access restrictions: Attempts by unauthorized users to access protected interfaces or data must be detected and blocked.
\end{itemize}

\subsubsection{Demonstrate Usability}
The final objective is to confirm that the system is intuitive and accessible for parents, children, and clinicians. In this context:
\begin{itemize}
  \item User-friendly interface: The UI should follow best practices in accessibility, clarity, and layout. Users should be able to perform common tasks (e.g., starting an assessment, reviewing results) with minimal instructions and no more than a defined number of clicks (\texttt{MAX\_CLICKS}).
  \item Efficient task completion: Success is measured by user testing metrics (e.g., task completion rate, time to complete), and feedback collected via the usability survey in the Appendix (Section 6.2).
  \item Child accessibility: The assessment interface must be suitable for children aged 6–12, who should be able to navigate the assessment independently (Section 4.2.2).
\end{itemize}

\subsubsection*{Out of Scope}
Some objectives that are considered out of scope for this VnV plan due to resource limitations include:
\begin{itemize}
  \item Adherence to UI Industry Standards: Full compliance with all published UI standards is not feasible due to time constraints. The focus is on practical usability and functional aesthetics rather than exhaustive WCAG or HIG compliance.
  \item Verification of Third-Party Libraries: External open-source libraries used in the frontend or AI model will not be independently audited. Instead, system-level testing will ensure that their integration functions as expected.
\end{itemize}
  
\subsection{Challenge Level and Extras}

\subsubsection{Challenge Level} 

Our challenge level is general as the project scope is limited in terms of how much
research is required. The required domain knowledge is basic web-design in a
stack of our choice. We are also planning on using open-source large language
models for audio and video processing.

\subsubsection{Extras}
Our project has the following extras;
\begin{itemize}
  \item \textbf{User Documentation: } Providing users with a guide on how to use and
  better understand the system. 
  \item \textbf{Usability Testing: } Receive user feedback on usability of design of the application, 
including improvements on how the system looks and functions. This will help to ensure an 
intuitive and easy system for both clinicians and parents/children to navigate through.
\end{itemize}


\subsection{Relevant Documentation}

  The three main relevant documentation that helped guide us in System Verification and Validation Plan (VnV) was Software Requirements Specification
  (SRS), Development plan and MIS.  
  
  \begin{itemize}
    \item The SRS document was crucial to define descriptions, rationals and fit criteria for all of the systems main functional 
    and non functional requirements. This outlines the basis of what needs to be verified and tested in the VnV Plan. It also outlines the 
    traceability matrix that needs to be followed.
    \item Development Plan: Defined the main Goals, Objectives and extras for the system, aiding the team to understand what
     needs to be done and focused on. 
    \item MIS : The Module Interface Specification (MIS) will shape the scope of unit testing in the VnV plan and guides 
    the development of both functional and nonfunctional tests to verify specific software functions and attributes.
  \end{itemize}

  % \citet{SRS}

\newpage

\section{Plan}
\hspace{2em}This section includes plans for how the team's verification and validation processes will
be implemented. This covers the team and its roles, the SRS verification plan, the design
verification plan, the verification and validation plan verification plan, the implementation
verification plan, automated testing and verification tools, and the software validation plan.

\subsection{Verification and Validation Team}

\hspace{2em}The following table displays the roles and responsibilities of each team member.

\begin{table}[h!]
  \centering
  \begin{tabular}{ | m{5cm} | m{8cm} | }
    \toprule
    Name & Roles and Responsibilities \\
    \hline
    Mitchell Weingust & \vspace{2mm}\begin{itemize}
      \item Audio Analysis Model verification
      \item System Architecture and Design Validation
      \item SRS Verification
    \end{itemize}\\
    \hline
    Parisha Nizam & \vspace{2mm}\begin{itemize}
      \item Frontend Interface Verification
      \item Backend Database Verification
      \item VnV Verification
    \end{itemize}\\
    \hline
    Promish Kandel & \vspace{2mm}\begin{itemize}
      \item Frontend Interface Verification
      \item Video Analysis Model verification
      \item VnV Verification
    \end{itemize}\\
    \hline
    Jasmine Sun-Hu & \vspace{2mm}\begin{itemize}
      \item Backend Database Verification
      \item System Architecture and Design Validation
      \item SRS Verification
    \end{itemize}\\
    \hline
    Dr. Irene Yuan & \vspace{2mm}\begin{itemize}
      \item Providing feedback (including Hands-On) during project development
    \end{itemize}\\
    \hline
    Dr. Yao Du & \vspace{2mm}\begin{itemize}
      \item Providing written feedback on user experiences and testing
    \end{itemize}\\
    \hline
    Chris Schankula & \vspace{2mm}\begin{itemize}
      \item Providing feedback during project development
      \item Revision recommendations
    \end{itemize}\\
    \bottomrule
  \end{tabular}
  \caption{Verification and Validation Team Table}
\end{table}

\newpage
\subsection{SRS Verification Plan}

The following approaches will be used to verify the SRS:

\begin{itemize}
  \item \textbf{Ad hoc Peer Review}: Informal reviews of the SRS will be conducted after every major revision 
  to the SRS with classmates who will serve as peer reviewers. This will provide new perspectives that can 
  help identify ambiguities, missing requirements and/or areas of improvement. 
  \begin{itemize}
    \item Peer reviewers will submit feedback using GitHub issue tracker for organized task assignment and tracking.
  \end{itemize}
  \item \textbf{Supervisor Review}: Before every major SRS revision submission, the team will 
  send a copy of the SRS to the project supervisor a week before meeting, along with a checklist highlighing 
  priority areas for their feedback. The team will also prepare questions about the requirements related to 
  interfaces and usability which is the supervisor's area of expertise. During the review meeting, the team will 
  first review the supervisor's initial feedback with the supervisor, and then ask the prepared questions.
  \begin{itemize}
    \item The meeting notes will be documented using GitHub issue tracker.
  \end{itemize}
  \item \textbf{Internal Team Walkthrough}: The team will hold a collaborative session 
  before every major SRS revision submission where the team will discuss each section one 
  by one to verify that all requirements are understood by all members, and that each section 
  is consistent with the project goals/objectives. A checklist will act as a guide to highlight the key concepts 
  the review should focus on, and will help ensure no critical areas are missed in review. The checklist can be found 
  below.
  \begin{itemize}
    \item Any corrections or modifications that need to be made will be noted in the team 
    meeting's GitHub issue tracker.
  \end{itemize}
\end{itemize}

The following checklist will be used for the internal team walkthrough: \\
\begin{todolist}
  \item Are all major functions required for the website (interface, backend, recording, analysis, storage) covered?
  \item Does each function have clear input and output specifications?
  \item Are all requirements written with consistent terminology?
  \item Do all requirements avoid conflict with each other?
  \item Does each requirement avoid ambiguous language?
  \item Are all requirements verifiable and testable?
  \item Is each requirement written in a way all team members and stakeholders can understand?
  \item Are any assumptions about user behavior or system behaviour explicitly stated?
\end{todolist}

\subsection{Design Verification Plan}


The design verification plan outlines the strategies that the team will use to verify the usability and correctness of our
 system. The following outlines the plan for verifying the User Interface of the software as captured by system design
\begin{itemize}
  \item Shall conduct a quick review with supervisor after design documents (MIS) have been completed 
  \item Conduct peer review sessions from classmates to provide critical suggestions on improvements of the system. 
  \item Conduct final formal review with clinician Dr. Du and the team, following defined SRS and MIS documents. 
  \item Conduct ad-hoc review(s) with other teams.
\end{itemize}

The following checklist will be used to verify the system's design documents over the course of completion.
\begin{todolist}
\item Are all requirements traceable to at least one feature/module in the MIS?
\item Do all modules and components follow SOLID design principles?
\item Have the creation of all modules been tracked via issues and closed once reviewed?
\item Are all inputs and outputs of system components defined well and correctly?
\end{todolist}

\subsection{Verification and Validation Plan Verification Plan}

As the verification and validation plan is an artifact, it must be verified too. The team's verification of the VnV plan follows:
\begin{itemize}
  \item Peer reviews by classmates, including other teams' peer reviews, to identify areas of improvement and general feedback
  \item Documentation review by the project's supervisor, Dr. Irene Ye Yuan, to ensure that the team's planned verification and validation plan is realistic and feasible
  \item Teammate documentation reviews, to provide critical feedback and ensure that all intended goals and outcomes are met
  \item Mutation testing to ensure that changes to aspects of the source plan can be detected by test cases.
\end{itemize}

The below checklist will be used, in addition, to ensure the team's VnV plan is correct and complete.

\begin{todolist}
\item Does the VnV Plan verify all functional requirements are met?
\item Does the VnV Plan verify all non-functional requirements are met?
\item Have all peer-review issues been addressed and closed?
\item Have all members of the Verification and Validation Team contributed to the review and approved the document?
\item Are all aspects of the system boundary being verified, validated, and tested?
\item Do the system tests cover all requirements mentioned in the SRS?
\item Did the test cases detect mutations and give desired outputs?
\item Did the test cases' expected output match the actual output?
\item Is there a process for documenting and resolving defects?
\end{todolist}

\subsection{Implementation Verification Plan}

The following outlines the plan for verifying implementation of the system, using both static and dynamic techniques 

\begin{itemize}
  \item Walkthrough of key components of the system with the supervisor (UI design, user authentication, video and audio analysis)
  \item Walkthrough of each components with other teammates 
  \item Running unit tests (to be implemented in VnV Plan) to verify that the implementation matches the specified design 
  outlined in MIS. This will be done automatically (dynamic) using Github Actions for each pull request made, following
  CI/CD principles. 
  \item Running system tests (both functional and non functional) described in the VnV plan to verify that the implementation 
  meets requirements
  \item Running Static analyzers including linters to help discover any potential errors or bugs in the code. Linters 
  and tools will also be used to help make the code more organized and readable. 
  \item Major code commits will be reviewed by at least one other team member before merging to main branch to ensure
   consistency and minimize chance of conflicts. 
\end{itemize}


\subsection{Automated Testing and Verification Tools}

The following are the automated testing and verification tools to be used during the validation and 
verification process for the software being tested in this VnV plan:
\begin{itemize}
  \item Unit Tests: Jest, Pytest
  \item Linters: Flake9, Prettier, ESLint
  \item Continuous Integration: GitHub Actions
\end{itemize}
\hspace{2em} For our code coverage, we will use Istanbul and Coverage.py. Istanbul is a code coverage 
tool that works with JavaScript testing frameworks like Jest. It helps developers see how much of their 
code is tested by creating reports that show untested lines and functions. Coverage.py is a code coverage 
tool for Python, which we use for our machine learning model. It measures how much of the code runs during 
tests and generates reports in different formats, helping developers find untested parts of their Python 
applications and improve test coverage.


\subsection{Software Validation Plan}

Our plan for validating the software includes review sessions with 
stakeholders and extensive user testing. Dr.Yao Du, one of our key 
stakeholders, will provide the software to clinicians and patients for 
real-world testing in clinical settings. This will allow us to gather 
valuable feedback on the software’s functionality and usability in 
actual healthcare environments. In addition to this field testing, we 
will conduct structured user testing sessions where participants will 
simulate the experience of being a patient. During these sessions, 
users will navigate through the software, interacting with its features,
and afterward, they will share their insights on what worked well and 
what didn’t. Overall, using both field testing and targeted user testing, the information
gathered will help us refine the 
software and ensure it meets the needs of its intended users. 

\newpage
\section{System Tests}
\hspace{2em}This section includes systems tests for functional requirements and nonfunctional
requirements mentioned in the SRS. For traceability, the team created a traceability
matrix to correlate the system tests to the requirement(s) they cover.

\subsection{Tests for Functional Requirements}
\hspace{2em}The following section covers all the functional requirements specified in the project's
SRS document. Each of the functional requirements are covered by a test in section 4.1. The coverage
can be traced in Table 3.

\subsubsection{Authentication}

\hspace{2em}The test cases below focus on ensuring users can safely and securely login, create and access
their accounts without worrying about others accessing their information.

\begin{itemize}
  \begin{item}
      FR-ST-A1
      \begin{mdframed}[linewidth=0.5mm]
          \textbf{Control:} Manual \par
          \textbf{Initial State:} User has a Parent account already created and stored in the database \par
          \textbf{Input:} Selection of Parent account role for login \par
          \textbf{Output:} The expected result is the Parent account role is selected and User is brought to the Parent login screen \par
          \textbf{Test Case Derivation:} The expected output is justified based on FR-A1 in the SRS document \par
          \textbf{How the test will be performed:}
          \begin{enumerate}[noitemsep]
            \item Select 'Login' to go to login screen
            \item Select 'Parent' when prompted to select between Parent and Clinician roles
            \item User is brought to the Parent login screen
          \end{enumerate}
      \end{mdframed}
  \end{item}
  \begin{item}
    FR-ST-A2
    \begin{mdframed}[linewidth=0.5mm]
      \textbf{Control:} Manual \par
      \textbf{Initial State:} User has a Clinician account already created and stored in the database \par
      \textbf{Input:} Selection of Clinician account role for login \par
      \textbf{Output:} The expected result is the Clinician account role is selected and User is brought to the Clinician login screen \par
      \textbf{Test Case Derivation:} The expected output is justified based on FR-A1 in the SRS document \par
      \textbf{How the test will be performed:}
      \begin{enumerate}[noitemsep]
        \item Select 'Login' to go to login screen
        \item Select 'Clinician' when prompted to select between Parent and Clinician roles
        \item User is brought to the Clinician login screen
      \end{enumerate}
  \end{mdframed}
  \end{item}
  \begin{item}
    FR-ST-A3
    \begin{mdframed}[linewidth=0.5mm]
      \textbf{Control:} Manual \par
      \textbf{Initial State:} User does not have a Parent account stored in the database \par
      \textbf{Input:} Selection of 'Create Account', with a username that does not exist in the database, upon attempting to access the system \par
      \textbf{Output:} The expected result is a new Parent account is created \par
      \textbf{Test Case Derivation:} The expected output is justified based on FR-A2 in the SRS document \par
      \textbf{How the test will be performed:}
      \begin{enumerate}[noitemsep]
        \item Select 'Create Account' to go to create account screen
        \item Enter unique username that is not in the database
        \item Enter account credentials (to complete account create process)
        \item Parent account is created
      \end{enumerate}
  \end{mdframed}
  \end{item}
  \begin{item}
    FR-ST-A4
    \begin{mdframed}[linewidth=0.5mm]
      \textbf{Control:} Manual \par
      \textbf{Initial State:} User does not have a Parent account stored in the database \par
      \textbf{Input:} Selection of 'Create Account', with a username that exists in the database, upon attempting to access the system \par
      \textbf{Output:} The expected result is a new Parent account fails to be created \par
      \textbf{Test Case Derivation:} The expected output is justified based on FR-A2 in the SRS document \par
      \textbf{How the test will be performed:}
      \begin{enumerate}[noitemsep]
        \item Select 'Create Account' to go to create account screen
        \item Enter username that already exists in the database
        \item System communicates the account could not be created
        \item System prompts user to select a new username
        \item Parent account is not created
      \end{enumerate}
  \end{mdframed}
  \end{item}
\begin{item}
  FR-ST-A5
  \begin{mdframed}[linewidth=0.5mm]
    \textbf{Control:} Manual \par
    \textbf{Initial State:} User has Admin privileges, attempting to create a new Clinician account \par
    \textbf{Input:} Admin user selects option to 'Create Account', with a username that does not exist in the database, upon attempting to access the system \par
    \textbf{Output:} The expected result is a new Clinician account is created \par
    \textbf{Test Case Derivation:} The expected output is justified based on FR-A3 in the SRS document \par
    \textbf{How the test will be performed:}
    \begin{enumerate}[noitemsep]
      \item Admin user is logged into their account
      \item Select 'Create Account' to go to create account screen
      \item Enter unique username that is not in the database
      \item Enter account credentials (to complete account create process)
      \item Clinician account is created
    \end{enumerate}
\end{mdframed}
\end{item}
\begin{item}
  FR-ST-A6
  \begin{mdframed}[linewidth=0.5mm]
    \textbf{Control:} Manual \par
    \textbf{Initial State:} User has Admin privileges, attempting to create a new Clinician account \par
    \textbf{Input:} Admin user selects option to 'Create Account', with a username that exists in the database, upon attempting to access the system \par
    \textbf{Output:} The expected result is a new Clinician account fails to be created \par
    \textbf{Test Case Derivation:} The expected output is justified based on FR-A3 in the SRS document \par
    \textbf{How the test will be performed:}
    \begin{enumerate}[noitemsep]
      \item Admin user is logged into their account
      \item Select 'Create Account' to go to create account screen
      \item Enter username that already exists in the database
      \item System communicates the account could not be created
      \item System prompts admin user to select a new username
      \item Clinician account is not created
    \end{enumerate}
\end{mdframed}
\end{item}
\begin{item}
  FR-ST-A7
  \begin{mdframed}[linewidth=0.5mm]
    \textbf{Control:} Manual \par
    \textbf{Initial State:} User is on their corresponding role's login page, with an account already created and stored in the database \par
    \textbf{Input:} Unique username and corresponding password that exists in the database \par
    \textbf{Output:} The expected result is a successful login to a user's account \par
    \textbf{Test Case Derivation:} The expected output is justified based on FR-A4 in the SRS document \par
    \textbf{How the test will be performed:}
    \begin{enumerate}[noitemsep]
      \item On login screen
      \item Enter unique username
      \item Enter corresponding password
      \item Select login to enter account
      \item Logged into account
    \end{enumerate}
\end{mdframed}
\end{item}
\begin{item}
  FR-ST-A8
  \begin{mdframed}[linewidth=0.5mm]
    \textbf{Control:} Manual \par
    \textbf{Initial State:} User is logged into their account \par
    \textbf{Input:} Selection of 'logout' \par
    \textbf{Output:} The expected result is a successful logout from a user's account \par
    \textbf{Test Case Derivation:} The expected output is justified based on FR-A5 in the SRS document \par
    \textbf{How the test will be performed:}
    \begin{enumerate}[noitemsep]
      \item Logged into account
      \item Select 'logout'
      \item System logs user out of their account
      \item Logout confirmation is displayed to the user
    \end{enumerate}
\end{mdframed}
\end{item}
\end{itemize}

\subsubsection{Data Collection and Storage}

\hspace{2em}The test cases below focus on ensuring reliable storage and retrieval 
of multimedia data, privacy compliance by excluding PII, accurate grouping under 
unique identifiers, and long-term report accessibility, meeting all data storage 
and organization requirements.
\begin{itemize}
  \item FR-ST-DSC1
  \begin{mdframed}[linewidth=0.5mm]
      \textbf{Control:} Automatic \par
      \textbf{Initial State:} Database is empty or initialized with test data. \par
      \textbf{Input:} Multimedia files (video, audio, and structured data files). \par
      \textbf{Output:} The expected result is a success message in console for both storing data and retrieving data \par
      \textbf{Test Case Derivation:} Ensures the database meets storage \\capacity and integrity requirements as per FR-DSC1 in SRS \\document. \par
      \textbf{How the test will be performed:}
      \begin{enumerate}[noitemsep]
        \item Insert a session containing multimedia files (video, audio, JSON) into the database.
        \item Retrieve the session files and check for data integrity by \\comparing size of stored data with retrieved data.
        \item Verify that the retrieved files are uncorrupted and correctly match the original files.
      \end{enumerate}
  \end{mdframed}

  \item FR-ST-DSC2
  \begin{mdframed}[linewidth=0.5mm]
      \textbf{Control:} Manual \par
      \textbf{Initial State:} Database is set up to store assessment session data. \par
      \textbf{Input:} Video, audio files, flagged occurrences, and timestamps for each assessment question. \par
      \textbf{Output:} The expected result is the creation of a JSON file that contains flagged occurrences and timestamps which is stored alongside the data. \par
      \textbf{Test Case Derivation:} Allows the system to store all important information for clinicians to use. \par
      \textbf{How the test will be performed:}
      \begin{enumerate}[noitemsep]
        \item Insert a test assessment session with video and audio files, flagged \\ occurrences, and timestamps.
        \item Query the database to retrieve the session’s data and verify the presence of a JSON file with accuracy of flagged occurrences and timestamps.
      \end{enumerate}
  \end{mdframed}

  \item FR-ST-DSC3
  \begin{mdframed}[linewidth=0.5mm]
      \textbf{Control:} Automatic \par
      \textbf{Initial State:} Database configured to prevent storage of personally identifiable information (PII). \par
      \textbf{Input:} Attempted insertion of a record containing personally \\identifiable information. \par
      \textbf{Output:} The expected result is that the database rejects any PII-containing records and stores only anonymized data. \par
      \textbf{Test Case Derivation:} Ensures compliance with privacy standards, verifying that no PII is stored in the database, in alignment with HIPAA data minimization and protection requirements. \par
      \textbf{How the test will be performed:}
      \begin{enumerate}[noitemsep]
        \item Attempt to insert a record with PII (e.g., name, address).
        \item Verify that the system blocks or anonymizes PII, preventing its storage.
        \item Retrieve all clinician-accessible data and confirm the absence of PII.
      \end{enumerate}
  \end{mdframed}

  \item FR-ST-DSC4
  \begin{mdframed}[linewidth=0.5mm]
      \textbf{Control:} Manual \par
      \textbf{Initial State:} Database initialized and ready for storing user session data. \par
      \textbf{Input:} Multiple sessions, each with unique user identifiers. \par
      \textbf{Output:} The expected result is that all session data is stored and grouped correctly according to the unique user identifiers. \par
      \textbf{Test Case Derivation:} Confirms database's grouping and \\retrievable capabilities, ensuring accurate data organization.\par
      \textbf{How the test will be performed:}
      \begin{enumerate}[noitemsep]
        \item Insert multiple sessions into the database, each tagged with a unique user identifier.
        \item Query the database for each user identifier and verify that all associated session data is correctly grouped.
        \item Confirm no data is incorrectly associated or left unassociated.
      \end{enumerate}
  \end{mdframed}

  \item FR-ST-DSC5
  \begin{mdframed}[linewidth=0.5mm]
      \textbf{Control:} Manual \par
      \textbf{Initial State:} Database initialized and ready to store reports with unique \\ identifiers. \par
      \textbf{Input:} Assessment report linked to a patient’s unique identifier. \par
      \textbf{Output:} The expected result is that the report is successfully stored, linked to the corresponding patient identifier, and retrievable for at least \\ MAX\_STORAGE\_TIME. \par
      \textbf{Test Case Derivation:} Verifies long-term storage and retrievability of assessment reports by simulating a shortened retention period (e.g. 7 days), and confirming report accessibility at the start, midpoint, and just before expiration. This ensures confidence in meeting the real retention requirement. \par
      \textbf{How the test will be performed:}
      \begin{enumerate}[noitemsep]
        \item Insert a test report with a unique patient identifier into the database.
        \item Simulate passage of time or use timestamps to emulate 3 key stages: day 1, midpoint, and just before test retention period ends.
        \item At each point, retrieve the report using the identifier and confirm its presence and correctness.
        \item Optionally, test behavior just after simulated expiration to ensure proper handling (retained if policy allows or removed if enforced).
      \end{enumerate}
  \end{mdframed}

\end{itemize}

\subsubsection{Video and Audio Data Analysis}

\hspace{2em}The test cases below focus on ensuring the video analysis model can reliably access 
session recordings, accurately detect and log speech disturbances, and correctly flag 
disturbances with timestamps, questions, and user responses to support efficient clinical review.
\begin{itemize}
  \item FR-ST-VDA1
  \begin{mdframed}[linewidth=0.5mm]
      \textbf{Control:} Automatic \par
      \textbf{Initial State:} Completed assessment sessions are available in the database, with video and audio recordings accessible for processing. \par
      \textbf{Input:} Request by the analysis model to access video and audio data from a completed session. \par
      \textbf{Output:} The expected result is that all videos requested are processed with a success message in the logs \par
      \textbf{Test Case Derivation:} Verifies that the model has reliable access to stored multimedia data, which is critical for processing and analysis. \par
      \textbf{How the test will be performed:}
      \begin{enumerate}[noitemsep]
        \item Retrieve the video and audio recordings from several completed sessions.
        \item Check that the model successfully accesses the multimedia files for each session without data access errors.
        \item Verify that the files are correctly loaded for analysis with no corruption or access issues.
      \end{enumerate}
  \end{mdframed}

  \item FR-ST-VDA2
  \begin{mdframed}[linewidth=0.5mm]
      \textbf{Control:} Automatic \par
      \textbf{Initial State:} The analysis model is initialized and ready to process test video and audio data. \par
      \textbf{Input:} Video and audio data containing speech disturbances, interruptions, and other irregularities for analysis. \par
      \textbf{Output:} The expected results is an accuracy of VERY\_HIGH\_SUCCESS\_RATE in a JSON file for number of disturbances found by the model \par
      \textbf{Test Case Derivation:} Confirms that the model’s disturbance detection meets the accuracy requirement, reducing bias in the analysis process. \par
      \textbf{How the test will be performed:}
      \begin{enumerate}[noitemsep]
        \item Run the model on a test dataset containing known speech disturbances.
        \item Compare the disturbances identified by the model with human observations for accuracy validation.
        \item Verify that the model achieves at least VERY\_HIGH\_SUCCESS\_RATE \\ accuracy in identifying and logging disturbances.
      \end{enumerate}
  \end{mdframed}

  \item FR-ST-VDA3
  \begin{mdframed}[linewidth=0.5mm]
      \textbf{Control:} Automatic \par
      \textbf{Initial State:} Video and audio data with disturbances has been processed by the analysis model. \par
      \textbf{Input:} Disturbances identified by the model, requiring flags with associated timestamps, assessment questions, and user answers. \par
      \textbf{Output:} The expected result is an accuracy of VERY\_HIGH\_SUCCESS\_RATE in a JSON file for timestamp accuracy\par
      \textbf{Test Case Derivation:} Ensures clinicians can quickly access relevant parts of the assessment with accuracy, aiding in efficient diagnosis. \par
      \textbf{How the test will be performed:}
      \begin{enumerate}[noitemsep]
        \item Process a test assessment session with the model, identifying and flagging disturbances.
        \item Retrieve flagged disturbances and confirm each has an accurate timestamp, associated question, and user response.
        \item Compare the flagged data with human observations and verify at least \\VERY\_HIGH\_SUCCESS\_RATE accuracy in the model’s associations.
      \end{enumerate}
  \end{mdframed}
\end{itemize}

\subsubsection{Data Processing and Display}

\hspace{2em}This set of test cases will help confirm the system's data retrieval, report generation, 
and display functionalities to ensure the clinician experience aligns with the project’s goals.

\begin{itemize}
  \item FR-ST-DPD1
  \begin{mdframed}[linewidth=0.5mm]
      \textbf{Control:} Automatic \par
      \textbf{Initial State:} Database is populated with processed assessment data. \par
      \textbf{Input:} Query request for a specific patient’s processed assessment data. \par
      \textbf{Output:} The expected result is the successful retrieval of all relevant assessment data, displayed without errors within \\MAX\_PROCESSING\_TIME \par
      \textbf{Test Case Derivation:} Ensures the system can retrieve data efficiently for report generation, meeting retrieval speed and \\completeness requirements. \par
      \textbf{How the test will be performed:}
      \begin{enumerate}[noitemsep]
        \item Query the database for a test patient’s processed assessment results. 
        \item Measure and record retrieval time, ensuring it does not exceed\\
        MAX\_PROCESSING\_TIME. 
        \item Verify that all required data is retrieved and matches the stored information.
      \end{enumerate}
  \end{mdframed}

  \item FR-ST-DPD2
  \begin{mdframed}[linewidth=0.5mm]
      \textbf{Control:} Automatic \par
      \textbf{Initial State:} Database has assessment data including flagged occurrences,\\
      timestamps, and patient performance metrics. \par
      \textbf{Input:} Trigger for report generation based on a retrieved \\assessment dataset. \par
      \textbf{Output:} The expected result is a generated report containing all required data within MAX\_PROCESSING\_TIME. \par
      \textbf{Test Case Derivation:} Confirms that report generation is \\complete, accurate, and within performance constraints. \par
      \textbf{How the test will be performed:}
      \begin{enumerate}[noitemsep]
        \item Retrieve a patient’s assessment data from the database. 
        \item Trigger report generation. 
        \item Confirm the report includes flagged occurrences, timestamps, and \\ performance metrics. 
        \item Measure and confirm report generation time does not exceed\\ MAX\_PROCESSING\_TIME.
      \end{enumerate}
  \end{mdframed}

  \item FR-ST-DPD3
  \begin{mdframed}[linewidth=0.5mm]
      \textbf{Control:} Manual \par
      \textbf{Initial State:} Generated report available in the database. \par
      \textbf{Input:} Clinician dashboard query to display the generated report. \par
      \textbf{Output:} Report displayed in the clinician’s dashboard with \\accurate formatting, charts, and tables, fully loaded within \\MAX\_PROCESSING\_TIME. \par
      \textbf{Test Case Derivation:} Validates the report display function, ensuring usability and speed requirements are met. \par
      \textbf{How the test will be performed:}
      \begin{enumerate}[noitemsep]
        \item Query the clinician’s dashboard to load the report. 
        \item Confirm that the report is displayed with correct charts, \\tables, and formatting. 
        \item Verify full loading of the report within \\MAX\_PROCESSING\_TIME.
      \end{enumerate}
  \end{mdframed}

  \item FR-ST-DPD4
  \begin{mdframed}[linewidth=0.5mm]
      \textbf{Control:} Manual \par
      \textbf{Initial State:} Database has stored reports for previous sessions, each with a unique patient identifier. \par
      \textbf{Input:} Clinician request to access a specific previously generated report. \par
      \textbf{Output:} The expected result is successful retrieval and display of the requested report without errors, within \\MAX\_PROCESSING\_TIME. \par
      \textbf{Test Case Derivation:} Ensures that clinicians can reliably access and view past reports, supporting longitudinal patient assessment.\par
      \textbf{How the test will be performed:}
      \begin{enumerate}[noitemsep]
        \item Query the database for a stored report using a unique patient identifier. 
        \item Verify that the retrieved report is complete and accurate. 
        \item Confirm that the report is displayed within \\MAX\_PROCESSING\_TIME.
      \end{enumerate}
  \end{mdframed}

\end{itemize}

\subsubsection{System Set Up}

\hspace{2em}This set of test cases verifies that the system provides users with the ability to 
review assessment information, complete necessary hardware checks, and receive guidance before beginning an assessment.

\begin{itemize}
  \item FR-ST-SS1
    \begin{mdframed}[linewidth=0.5mm]
      \textbf{Control:} Manual \par
      \textbf{Initial State:} User is logged into the system and has access to the assessment information page. \par
      \textbf{Input:} User navigates to the page where assessment information is displayed before starting hardware checks. \par
      \textbf{Output:} User is able to view relevant information about the assessment before beginning any hardware checks. \par
      \textbf{Test Case Derivation:} Users need to be aware of the assessment requirements and goals prior to starting 
      the hardware setup to ensure they are prepared. This follows expected output found in FR-SS1 in (SRS) Document \par
      \textbf{How the test will be performed:}
      \begin{enumerate}[noitemsep]
        \item Log in as a user and navigate to the assessment information section.
        \item Verify that the page displays necessary details about the assessment, such as the purpose, requirements, and overview.
        \item Confirm that the information is accessible and readable to the user.
      \end{enumerate}
    \end{mdframed}

  \item FR-ST-SS2
    \begin{mdframed}[linewidth=0.5mm]
      \textbf{Control:} Manual \par
      \textbf{Initial State:} User is on the hardware check section of the system with audio devices connected. \par
      \textbf{Input:} User initiates the audio hardware check through the system. \par
      \textbf{Output:} User receives confirmation that the audio input and output devices are functioning correctly. \par
      \textbf{Test Case Derivation:} Verifying that audio devices are functional before the assessment prevents issues of recording audio during
      the assessment, which could effect proper identification of bias. This follows expected output found in FR-SS2 in (SRS) Document\par
      \textbf{How the test will be performed:}
      \begin{enumerate}[noitemsep]
        \item Start the audio hardware check.
        \item Confirm that the system prompts the user to test both audio input \\ (microphone) and output (speakers/headphones).
        \item Verify that the system indicates successful audio detection when the test is performed.
        \item Test with a non-functional audio device and confirm that the system displays an appropriate error.
      \end{enumerate}
    \end{mdframed}

  \item FR-ST-SS3
    \begin{mdframed}[linewidth=0.5mm]
      \textbf{Control:} Manual \par
      \textbf{Initial State:} User is on the hardware check section of the system with a video device connected. \par
      \textbf{Input:} User initiates the video hardware check through the system. \par
      \textbf{Output:} User receives confirmation that the video capturing device is functioning correctly. \par
      \textbf{Test Case Derivation:} Ensuring video functionality prevents disruptions in assessments that require 
      visual input or interaction and ensures 
      proper recording is taken for accurate data analysis. This follows expected output found in FR-SS3 in (SRS) Document \par
      \textbf{How the test will be performed:}
      \begin{enumerate}[noitemsep]
        \item Start the video hardware check.
        \item Confirm that the system activates the video device and displays a live feed.
        \item Verify that the system confirms successful video detection if the feed is displayed correctly.
        \item Test with a non-functional video device and confirm that the system displays an error.
      \end{enumerate}
    \end{mdframed}

  \item FR-ST-SS4
    \begin{mdframed}[linewidth=0.5mm]
      \textbf{Control:} Manual \par
      \textbf{Initial State:} User has successfully completed audio and video hardware checks. \par
      \textbf{Input:} User proceeds to the tutorial section after completing hardware checks. \par
      \textbf{Output:} User is directed to a tutorial that explains the assessment process in a step-by-step manner. \par
      \textbf{Test Case Derivation:} Providing a tutorial ensures that users understand the assessment process,
       improving accuracy and compliance. This follows expected output found in FR-SS4 in (SRS) Document \par
      \textbf{How the test will be performed:}
      \begin{enumerate}[noitemsep]
        \item Complete the audio and video hardware checks.
        \item Verify that the system automatically navigates the user to a tutorial section upon completion of the hardware checks.
        \item Confirm that the tutorial provides clear, step-by-step instructions on how to proceed with the assessment.
      \end{enumerate}
    \end{mdframed}

  \item FR-ST-SS5
    \begin{mdframed}[linewidth=0.5mm]
      \textbf{Control:} Manual \par
      \textbf{Initial State:} User has completed the tutorial and is ready to begin the assessment. \par
      \textbf{Input:} User initiates the start of the assessment through the system. \par
      \textbf{Output:} User is taken to the first assessment question, and the assessment begins. \par
      \textbf{Test Case Derivation:} Allowing users to start the assessment on their own terms helps them feel prepared and reduces errors. 
      This follows expected output found in FR-SS5 in (SRS) Document\par
      \textbf{How the test will be performed:}
      \begin{enumerate}[noitemsep]
        \item After completing the tutorial, select the option to start the assessment.
        \item Confirm that the system directs the user to the first assessment question.
        \item Verify that the assessment interface is properly displayed and ready for the user to begin answering.
      \end{enumerate}
    \end{mdframed}

\end{itemize}

\subsubsection{Assessment Interface}

\hspace{2em}This set of test cases ensures the assessment interface supports a smooth and accurate user experience by verifying the 
functionality of question presentation, response recording, navigation, and feedback mechanisms throughout the assessment.

\begin{itemize}
  \item FR-ST-AI1
    \begin{mdframed}[linewidth=0.5mm]
      \textbf{Control:} Automatic \par
      \textbf{Initial State:} User has started the assessment and is ready to begin answering questions. \par
      \textbf{Input:} User initiates the assessment. \par
      \textbf{Output:} System begins recording both audio and video, with an indicator showing recording is active. \par
      \textbf{Test Case Derivation:} Audio and video recording are essential for capturing user responses accurately,
       and an indicator reassures users that recording is in progress. This follows expected output found in FR-AI1 in (SRS) Document. \par
      \textbf{How the test will be performed:}
      \begin{enumerate}[noitemsep]
        \item Start the assessment as a user.
        \item Confirm that the system begins recording audio and video.
        \item Verify that a visible indicator shows that recording is ongoing.
      \end{enumerate}
    \end{mdframed}

  \item FR-ST-AI2
    \begin{mdframed}[linewidth=0.5mm]
      \textbf{Control:} Automatic \par
      \textbf{Initial State:} User is on a new question within the assessment. \par
      \textbf{Input:} System progresses to a new question in the assessment. \par
      \textbf{Output:} The system plays the corresponding audio prompt for the new question. \par
      \textbf{Test Case Derivation:} The audio prompt ensures users understand each question and assessment is running as intended.
      This follows expected output found in FR-AI2 in (SRS) Document. \par
      \textbf{How the test will be performed:}
      \begin{enumerate}[noitemsep]
        \item Progress to a new question in the assessment.
        \item Verify that the system automatically plays the corresponding audio prompt.
      \end{enumerate}
    \end{mdframed}

  \item FR-ST-AI3
    \begin{mdframed}[linewidth=0.5mm]
      \textbf{Control:} Automatic \par
      \textbf{Initial State:} User has progressed to a new question in the assessment. \par
      \textbf{Input:} System loads a new question. \par
      \textbf{Output:} System displays all possible answer options for the user to select from. \par
      \textbf{Test Case Derivation:} Presenting options ensures the user can make a response selection based
       on the question, system must mark answer select as right or wrong. 
       This follows expected output found in FR-AI3 in (SRS) Document. \par
      \textbf{How the test will be performed:}
      \begin{enumerate}[noitemsep]
        \item Go to a new question in the assessment.
        \item Confirm that all answer options associated with the question are displayed.
      \end{enumerate}
    \end{mdframed}

  \item FR-ST-AI4
    \begin{mdframed}[linewidth=0.5mm]
      \textbf{Control:} Manual \par
      \textbf{Initial State:} User is viewing the answer options for a question in the assessment. \par
      \textbf{Input:} User selects one of the displayed answer options. \par
      \textbf{Output:} System highlights or otherwise indicates the user’s selected option. \par
      \textbf{Test Case Derivation:} Visual confirmation of selection minimizes user error and allows
       the user to review their choice before confirmation. 
       This follows expected output found in FR-AI4 in (SRS) Document  \par
      \textbf{How the test will be performed:}
      \begin{enumerate}[noitemsep]
        \item Select an answer option for a question.
        \item Verify that the system indicates the selected option visually (e.g., highlighting).
      \end{enumerate}
    \end{mdframed}

  \item FR-ST-AI5
    \begin{mdframed}[linewidth=0.5mm]
      \textbf{Control:} Manual \par
      \textbf{Initial State:} User has selected an answer option and is ready to proceed. \par
      \textbf{Input:} User confirms their selection. \par
      \textbf{Output:} System moves the user to the next question or stage. \par
      \textbf{Test Case Derivation:} Confirmation helps prevent unintended answers, ensuring 
      accuracy in user responses. This follows expected output found in FR-AI5 in (SRS) Document. \par
      \textbf{How the test will be performed:}
      \begin{enumerate}[noitemsep]
        \item Select and confirm an answer for a question.
        \item Verify that the system progresses to the next question or stage upon \\ confirmation.
      \end{enumerate}
    \end{mdframed}

  \item FR-ST-AI6
    \begin{mdframed}[linewidth=0.5mm]
      \textbf{Control:} Automatic \par
      \textbf{Initial State:} User is in the process of answering questions within the assessment. \par
      \textbf{Input:} User enters and exits each question. \par
      \textbf{Output:} System records timestamps for entry and exit for each question. \par
      \textbf{Test Case Derivation:} Timestamping provides valuable tracking information for 
      synchronizing recordings and analyzing response timing. 
      This follows expected output found in FR-AI6 in (SRS) Document \par
      \textbf{How the test will be performed:}
      \begin{enumerate}[noitemsep]
        \item Begin the assessment and progress through several questions.
        \item Verify that the system logs entry and exit timestamps for each question.
      \end{enumerate}
    \end{mdframed}

  \item FR-ST-AI7
    \begin{mdframed}[linewidth=0.5mm]
      \textbf{Control:} Automatic \par
      \textbf{Initial State:} User has reached the final question in the assessment. \par
      \textbf{Input:} User completes the final question and confirms the selection. \par
      \textbf{Output:} System displays a message informing the user that the assessment is complete. \par
      \textbf{Test Case Derivation:} Notifying the user of completion provides a clear end to 
      the assessment and allows users to exit confidently. 
      This follows expected output found in FR-AI7 in (SRS) Document. \par
      \textbf{How the test will be performed:}
      \begin{enumerate}[noitemsep]
        \item Complete the final question in the assessment and confirm the selection.
        \item Verify that the system displays a completion message.
      \end{enumerate}
    \end{mdframed}
\end{itemize}

\pagebreak

\subsection{Tests for Nonfunctional Requirements}

\hspace{2em}The following section covers all the nonfunctional requirements specified in the project's SRS document.
Each of the nonfunctional requirements are covered by a test in section 4.2. The coverage can be traced in Table 4.

\subsubsection{Look and Feel Requirements}

These test cases ensure that all appearance and style requirements are\\
addressed effectively, covering navigation, user-friendliness, brand \\
consistency, visual appeal, and responsiveness.

\begin{itemize} 
  \item LF-ST-LFR1
  \begin{mdframed}[linewidth=0.5mm] 
    \textbf{Type:} Dynamic \par 
    \textbf{Initial State:} Platform initialized and navigable on a test \\device, prepared for user testing with adults and children. \par 
    \textbf{Input/Condition:} Conduct user tests with participants \\performing core tasks like starting an assessment, navigating \\menus, and viewing results. \par 
    \textbf{Output/Results:} Expected results include: 
    \begin{itemize} 
      \item No more than three levels of navigation depth, and each screen presents no more than six main options. 
      \item VERY\_HIGH\_SUCCESS\_RATE of users complete core tasks within\\
      MAX\_CLICKS clicks. 
      \item HIGH\_SUCCESS\_RATE of children aged 6-12 can \\complete a sample assessment independently. 
    \end{itemize} \par
    \textbf{Test Case Derivation:} This test ensures that the user interface design meets key look-and-feel requirements by validating navigation simplicity, ease of use, and child accessibility. It covers constraints on UI depth and option count (LF-AR1), intuitive navigation and task flow (LF-AR2, LF-AR4), and confirms that the platform is usable without adult guidance by children aged 6-12. \par
    \textbf{How the test will be performed:} 
    \begin{enumerate}[noitemsep] 
      \item Observe and record the number of clicks taken by each user to complete key tasks. 
      \item Inspect the navigation structure to ensure it meets depth and option \\constraints. 
      \item Test with children to verify completion of assessments \\independently. 
    \end{enumerate} 
  \end{mdframed}

  \item LF-ST-LFR2
  \begin{mdframed}[linewidth=0.5mm] 
    \textbf{Type:} Static and Dynamic \par 
    \textbf{Initial State:} Platform with finalized colour schemes, fonts, and brand assets loaded and available for inspection and user \\interaction tests. \par 
    \textbf{Input/Condition:} Perform visual inspection and feedback \\collection, along with response-time measurements for interactive elements. \par 
    \textbf{Output/Results:} Expected results include: 
    \begin{itemize} 
      \item Compliance with brand guidelines, style guide, and use of no more than three calming, neutral/pastel tones. 
      \item Positive feedback from child participants indicating a calm, non-stressful environment. 
      \item VERY\_HIGH\_SUCCESS\_RATE consistency in design across all pages. 
      \item HIGH\_SUCCESS\_RATE of user interactions provide immediate feedback \\within SHORT\_PROCESSING\_TIME. 
    \end{itemize} \par 
    \textbf{Test Case Derivation:} This test validates the system’s adherence to aesthetic and branding requirements (LF-AR3), confirms user interaction feedback timing (LF-AR5), and ensures a calm and accessible visual experience for children (LF-SR1, LF-SR2). Both objective checks and subjective user feedback are used to confirm compliance and usability standards. \par
    \textbf{How the test will be performed:} 
    \begin{enumerate}[noitemsep] 
      \item Compare platform’s design elements against the client’s brand and style guidelines. 
      \item Collect feedback from child participants about the visual \\atmosphere. 
      \item Test and measure feedback response times for various \\interactions. 
    \end{enumerate} 
  \end{mdframed} 
\end{itemize}

\subsubsection{Usability and Humanity}
\hspace{2em}The test cases below ensures that the system meets usability and humanity
requirements for users to have an enjoyable and accessible experience.

\begin{itemize}
  \item UH-ST-EOU1
  \begin{mdframed}[linewidth=0.5mm]
      \textbf{Type:} Usability, Manual\par
      \textbf{Initial State:} System is complete, functional, and ready for user interaction. \par
      \textbf{Input/Condition:} Users complete one full assessment using the system. \par
      \textbf{Output/Results:} User answers questions in the Usability Survey (6.2), and results are culminated and averaged.\\
      Averages should be at least 'Agree' on the answer scale in section 6.2. \par
      \textbf{Test Case Derivation:} This test confirms that the system is easy to use, accessible, and intuitive (UH-EOU1, UH-EOU2, UH-LI1, UH-UP1, UH-AR1) based on user feedback collected after completing an assessment. \par
      \textbf{How the test will be performed:}
      \begin{enumerate}[noitemsep]
        \item User has access to the system
        \item User completes one full assessment using the system
        \item Upon completion of the assessment, user is requested to fill out a usability survey
        \item Results are stored
        \item Usability scores for questions 1 through 8 are averaged across users
      \end{enumerate}
  \end{mdframed}

  \item UH-ST-PI1
  \begin{mdframed}[linewidth=0.5mm]
      \textbf{Type:} Static \par
      \textbf{Initial State:} System, including assessments, have been completed. \par
      \textbf{Input/Condition:} List of available languages to perform assessments in is available to be selected and listed \par
      \textbf{Output/Results:} Count the number of available languages for the assessment \par
      \textbf{Test Case Derivation:} This test ensures that users have access to multiple supported languages as outlined in UH-PI1. \par
      \textbf{How the test will be performed:}
      \begin{enumerate}[noitemsep]
        \item View list of available languages
        \item Count how many languages are available for the assessment
      \end{enumerate}
  \end{mdframed}

  \item UH-ST-LI2
  \begin{mdframed}[linewidth=0.5mm]
      \textbf{Type:} Manual \par
      \textbf{Initial State:} User documentation has been completed and made available to users. \par
      \textbf{Input/Condition:} Link to documentation is available on the system's frontend interface, and can be accessed \par
      \textbf{Output/Results:} Verify link takes user to access documentation \par
      \textbf{Test Case Derivation:} This test validates that accessible help documentation is present and reachable from the system interface as required in UH-LI2. \par
      \textbf{How the test will be performed:}
      \begin{enumerate}[noitemsep]
        \item Select 'documentation'
        \item User goes to documentation screen
        \item User has access to view up-to-date, available documentation
      \end{enumerate}
  \end{mdframed}
\end{itemize}

\subsubsection{Performance}
\hspace{2em}The test cases below ensures that the system meets essential performance metrics, 
including quick page load times, low latency in video and audio recording, high video 
resolution, and efficient report generation. 

\begin{itemize}
  \item PR-ST-SL1
  \begin{mdframed}[linewidth=0.5mm]
      \textbf{Type:} Dynamic\par
      \textbf{Initial State:} System initialized, strong internet connection \par
      \textbf{Input/Condition:} User navigates to different web pages within the system. \par
      \textbf{Output/Results:} The expected output is that each web page loads fully with all functionalities within MAX\_LOAD\_TIME. \par
      \textbf{Test Case Derivation:} Verifies system responsiveness and performance under expected usage conditions as required by PR-SL1. \par
      \textbf{How the test will be performed:}
      \begin{enumerate}[noitemsep]
        \item Navigate to various web pages within the system.
        \item Measure the time taken for each page to load completely.
        \item Confirm that all pages load within MAX\_LOAD\_TIME.
      \end{enumerate}
  \end{mdframed}

  \item PR-ST-SL2
  \begin{mdframed}[linewidth=0.5mm]
      \textbf{Type:} Static \par
      \textbf{Initial State:} Video and audio recording session initialized. \par
      \textbf{Input/Condition:} Audio and video session of user performing gestures while talking \par
      \textbf{Output/Results:} The expected output is the latency between actions and \\ recorded playback remains under SHORT\_PROCESSING\_TIME, ensuring \\ synchronization. \par
      \textbf{Test Case Derivation:} Ensures time-sensitive performance requirements for synchronization between user behavior and system response as required by PR-SL2. \par
      \textbf{How the test will be performed:}
      \begin{enumerate}[noitemsep]
        \item Begin recording a session.
        \item Have the user perform timed actions while recording.
        \item Play back the recording and measure the latency between actions and their corresponding timestamps.
        \item Confirm latency does not exceed SHORT\_PROCESSING\_TIME.
      \end{enumerate}
  \end{mdframed}

  \item PR-ST-SL3
  \begin{mdframed}[linewidth=0.5mm]
      \textbf{Type:} Dynamic \par
      \textbf{Initial State:} System configured to store video data. \par
      \textbf{Input/Condition:} Video recorded and stored from assessment session. \par
      \textbf{Output/Results:} The expected output is the video quality is at least \\ AVERAGE\_RESOLUTION when retrieving or storing. \par
      \textbf{Test Case Derivation:} Ensures the system maintains minimum quality standards for video data as required by PR-SL3. \par
      \textbf{How the test will be performed:}
      \begin{enumerate}[noitemsep]
        \item Record a session and store the video.
        \item Retrieve the stored video and verify its resolution.
        \item Confirm the resolution is AVERAGE\_RESOLUTION or higher.
      \end{enumerate}
  \end{mdframed}
\end{itemize}

\hspace{2em}The test cases below focus on verifying the system's precision in detecting and analyzing speech and gesture disturbances, ensuring timestamp alignment, and maintaining 100\% accuracy in assessment data.

\begin{itemize}
  \item PR-ST-PA1
  \begin{mdframed}[linewidth=0.5mm]
      \textbf{Type:} Manual, Dynamic \par
      \textbf{Initial State:} Video and Audio analysis model is loaded with sample audio data. \par
      \textbf{Input/Condition:} Data containing speech patterns and video with known \\ disturbances. \par
      \textbf{Output/Results:} The expected output is that the analysis achieves \\ VERY\_HIGH\_SUCCESS\_RATE in detecting speech and gesture disturbances. \par
      \textbf{Test Case Derivation:} Verifies the precision of detection and classification in the audio/video analysis model, fulfilling PR-PA1. \par
      \textbf{How the test will be performed:}
      \begin{enumerate}[noitemsep]
        \item Load test data files with known disturbances into the analysis model.
        \item Compare detected disturbances with human-reviewed observations.
        \item Verify that the model correctly identifies at least \\ VERY\_HIGH\_SUCCESS\_RATE of disturbances.
      \end{enumerate}
  \end{mdframed}

  \item PR-ST-PA3
  \begin{mdframed}[linewidth=0.5mm]
      \textbf{Type:} Static \par
      \textbf{Initial State:} Timestamp function is synchronized with real-time actions. \par
      \textbf{Input/Condition:} User performs actions in the recorded session. \par
      \textbf{Output/Results:} The expected output is that the timestamps delay within SHORT\_PROCESSING\_TIME of the real-time action. \par
      \textbf{Test Case Derivation:} Validates synchronization accuracy between user actions and system-generated timestamps per PR-PA3. \par
      \textbf{How the test will be performed:}
      \begin{enumerate}[noitemsep]
        \item Record a session with specific user actions.
        \item Analyze timestamps and compare them to the actual timing of the actions.
        \item Confirm each timestamp falls within SHORT\_PROCESSING\_TIME margin of the real action.
      \end{enumerate}
  \end{mdframed}

  \item PR-ST-PA4
  \begin{mdframed}[linewidth=0.5mm]
      \textbf{Type:} Manual, Static \par
      \textbf{Initial State:} Assessment answer key is loaded. \par
      \textbf{Input/Condition:} Manual verification of the answer key’s accuracy. \par
      \textbf{Output/Results:} The expected output is that the answer key is \\ MAX\_SUCCESS\_RATE. \par
      \textbf{Test Case Derivation:} Confirms correctness and completeness of stored answer keys against expected outputs as defined in PR-PA4. \par
      \textbf{How the test will be performed:}
      \begin{enumerate}[noitemsep]
        \item Manually review each entry in the assessment answer key.
        \item Check for errors or inconsistencies in each answer.
        \item Confirm all answers are correct, ensuring MAX\_SUCCESS\_RATE.
      \end{enumerate}
  \end{mdframed}
\end{itemize}

\hspace{2em}The test cases below validate the system’s ability to handle errors, 
back up data reliably, and enforce strict input validation.

\begin{itemize}
  \item PR-ST-RFT1
  \begin{mdframed}[linewidth=0.5mm]
      \textbf{Type:} Dynamic \par
      \textbf{Initial State:} System is operational with error handling in place. \par
      \textbf{Input/Condition:} User initiates actions known to cause common errors. \par
      \textbf{Output/Results:} The expected output is the system displays clear error messages for at least VERY\_HIGH\_SUCCESS\_RATE of the common errors encountered. \par
      \textbf{Test Case Derivation:} Verifies the system’s ability to handle errors with \\ informative feedback, ensuring robustness and user understanding as required by PR-RFT1. \par
      \textbf{How the test will be performed:}
      \begin{enumerate}[noitemsep]
        \item Simulate common user errors, such as invalid inputs or incorrect file uploads.
        \item Observe system response and displayed error messages.
        \item Verify clarity and accuracy of error messages in at least \\VERY\_HIGH\_SUCCESS\_RATE.
      \end{enumerate}
  \end{mdframed}

  \item PR-ST-RFT2
  \begin{mdframed}[linewidth=0.5mm]
      \textbf{Type:} Dynamic \par
      \textbf{Initial State:} Database backup processes configured and operational in the system. \par
      \textbf{Input/Condition:} Monthly data backup event. \par
      \textbf{Output/Results:} The expected output is that the system performs a data backup within a MONTHLY\_BACKUP timeframe on the first of each month. \par
      \textbf{Test Case Derivation:} Ensures system reliability through automated data backups to prevent loss of critical assessment data as required by PR-RFT2. \par
      \textbf{How the test will be performed:}
      \begin{enumerate}[noitemsep]
        \item A data backup is triggered.
        \item Record the duration of the backup process.
        \item Confirm that backup completes within MONTHLY\_BACKUP.
      \end{enumerate}
  \end{mdframed}
\end{itemize}

\vspace{1em}
\hspace{2em}The test cases below ensure that the system can handle the expected user load, data storage needs, and simultaneous uploads without performance degradation.

\begin{itemize}
  \item PR-ST-CR1
  \begin{mdframed}[linewidth=0.5mm]
      \textbf{Type:} Dynamic \par
      \textbf{Initial State:} System initialized with maximum user capacity parameters. \par
      \textbf{Input/Condition:} System loaded with MIN\_USERS accounts. \par
      \textbf{Output/Results:} The expected result is that the system operates stably and manages all accounts without issues. \par
      \textbf{Test Case Derivation:} Validates the system’s ability to scale under a minimum expected user load while maintaining stable performance as described in PR-CR1. \par
      \textbf{How the test will be performed:}
      \begin{enumerate}[noitemsep]
        \item Create and load MIN\_USERS accounts into the system.
        \item Monitor system performance metrics, including stability and response time.
        \item Confirm system maintains stable performance.
      \end{enumerate}
  \end{mdframed}

  \item PR-ST-CR2
  \begin{mdframed}[linewidth=0.5mm]
      \textbf{Type:} Static \par
      \textbf{Initial State:} System is initialized with required storage capacity. \par
      \textbf{Input/Condition:} Data stored in the database approaches MIN\_STORAGE annually. \par
      \textbf{Output/Results:} The expected result is that the system accommodates \\ MIN\_STORAGE of data without loss of performance. \par
      \textbf{Test Case Derivation:} Confirms that the database can handle expected yearly data growth without system slowdowns or data loss, fulfilling PR-CR2. \par
      \textbf{How the test will be performed:}
      \begin{enumerate}[noitemsep]
        \item Load MIN\_STORAGE data into database.
        \item Monitor database performance metrics, such as access time and error rate.
        \item Confirm system’s ability to manage MIN\_STORAGE without performance impact.
      \end{enumerate}
  \end{mdframed}
\end{itemize}

\vspace{1em}
\hspace{2em}The test cases below confirm that the system can scale effectively to accommodate an increasing user base, data volume, and computational needs over time.

\begin{itemize}
  \item PR-ST-SE1
  \begin{mdframed}[linewidth=0.5mm]
      \textbf{Type:} Static \par
      \textbf{Initial State:} System operational with current allocated user amount. \par
      \textbf{Input/Condition:} Increase user base by YEARLY\_INCREASE\_PERCENTAGE. \par
      \textbf{Output/Results:} The expected result is that the system maintains performance while handling user growth. \par
      \textbf{Test Case Derivation:} Ensures the system is scalable and can maintain acceptable performance under anticipated user base growth as required by PR-SE1. \par
      \textbf{How the test will be performed:}
      \begin{enumerate}[noitemsep]
        \item Simulate a YEARLY\_INCREASE\_PERCENTAGE increase in the user base by increase user base parameters by YEARLY\_INCREASE\_PERCENTAGE.
        \item Monitor system metrics such as response time and error rate.
        \item Confirm that the system operates within acceptable performance metrics post-increase.
      \end{enumerate}
  \end{mdframed}
\end{itemize}

\vspace{1em}
\hspace{2em}The test cases below verify the system’s reliability across updates and compatibility with major operating systems.

\begin{itemize}
  \item PR-ST-LR1
  \begin{mdframed}[linewidth=0.5mm]
      \textbf{Type:} Static \par
      \textbf{Initial State:} System release build in use, with ongoing development updates. \par
      \textbf{Input/Condition:} System stability monitored over successive updates. \par
      \textbf{Output/Results:} The expected result is that the system maintains a failure rate below LOW\_FAILURE\_RATE in release builds. \par
      \textbf{Test Case Derivation:} Ensures that new updates do not compromise system reliability and meet the fault tolerance criteria of PR-LR1. \par
      \textbf{How the test will be performed:}
      \begin{enumerate}[noitemsep]
        \item Conduct routine tests on the release build during ongoing development updates.
        \item Monitor system logs for errors or malfunctions.
        \item Verify that the failure rate remains below LOW\_FAILURE\_RATE.
      \end{enumerate}
  \end{mdframed}

  \item PR-ST-LR2
  \begin{mdframed}[linewidth=0.5mm]
      \textbf{Type:} Static \par
      \textbf{Initial State:} System configured for compatibility testing across platforms. \par
      \textbf{Input/Condition:} System loaded on Windows, macOS, Linux, Android, and iOS. \par
      \textbf{Output/Results:} The expected result is that the system functions correctly across all platforms. \par
      \textbf{Test Case Derivation:} Validates cross-platform operability to ensure consistent performance across major operating systems, as required by PR-LR2. \par
      \textbf{How the test will be performed:}
      \begin{enumerate}[noitemsep]
        \item Run the system on each specified operating system.
        \item Perform standard operations and monitor user experience on each platform.
        \item Confirm that the system operates without issues on all platforms.
      \end{enumerate}
  \end{mdframed}
\end{itemize}

\subsubsection{Operational and Environmental}

\hspace{2em}The test cases below ensure that the system can be used in a variety of environments,
adhere to conditions users are expected to operate under, and maintain the necessary capabilities
to interact with external systems and devices within those environments.

\begin{itemize}
  \item OE-ST-EPE1
  \begin{mdframed}[linewidth=0.5mm]
      \textbf{Type:} Usability, Manual \par
      \textbf{Initial State:} System is complete, functional, and ready for user interaction. \par
      \textbf{Input/Condition:} Testing the system, including the assessment, on a variety of screen sizes. \par
      \textbf{Output/Results:} The system's displayed elements will scale appropriately to different screen sizes. \par
      \textbf{Test Case Derivation:} Validates OE-EPE1 by confirming the system provides a consistent and usable UI across different device screen sizes. \par
      \textbf{How the test will be performed:}
      \begin{enumerate}[noitemsep]
        \item User logs into the system.
        \item User completes one full assessment using the system.
        \item Upon completion of the assessment, user is requested to fill out a usability survey.
        \item User answers questions about their screen size and whether the test scaled accordingly.
        \item Results are stored for review.
      \end{enumerate}
  \end{mdframed}

  \item OE-ST-WE1
  \begin{mdframed}[linewidth=0.5mm]
      \textbf{Type:} Dynamic, Manual \par
      \textbf{Initial State:} System, including assessments, has been completed. \par
      \textbf{Input/Condition:} User attempts to start system setup. \par
      \textbf{Output/Results:} Device verification is displayed on-screen, informing the user that the environment is suitable for the assessment. \par
      \textbf{Test Case Derivation:} Confirms OE-WE1 and OE-WE2 by testing the system’s ability to verify internet connectivity and input quality, ensuring environmental readiness. \par
      \textbf{How the test will be performed:}
      \begin{enumerate}[noitemsep]
        \item Select 'system setup'.
        \item System checks if connected to the internet.
        \item System checks audio input is not noisy.
        \item System checks video input is clear.
        \item System displays to the user their device is ready for the assessment to be used in the current environment.
      \end{enumerate}
  \end{mdframed}

  \item OE-ST-IA1
  \begin{mdframed}[linewidth=0.5mm]
      \textbf{Type:} Functional, Dynamic \par
      \textbf{Initial State:} System is connected to an external server for retrieving and storing data. \par
      \textbf{Input/Condition:} Assessment is complete, and results need to be stored. \par
      \textbf{Output/Results:} Verify results are stored in the external server. \par
      \textbf{Test Case Derivation:} Ensures OE-IA1 by confirming system interoperability with external storage services for accurate assessment result storage. \par
      \textbf{How the test will be performed:}
      \begin{enumerate}[noitemsep]
        \item Complete the assessment.
        \item Access the external server.
        \item Check if results have been uploaded to the server.
        \item Access results to ensure data has been uploaded successfully.
      \end{enumerate}
  \end{mdframed}
\end{itemize}

\subsubsection{Maintainability and Support Requirements}

\hspace{2em}These test cases ensure the platform meets its maintenance, support, and adaptability requirements effectively, including modular design, tutorial clarity, and accessibility across devices.

\begin{itemize} 
  \item MS-ST-MSA1
  \begin{mdframed}[linewidth=0.5mm] 
    \textbf{Type:} Static and Dynamic \par 
    \textbf{Initial State:} Modular platform architecture with access to the component’s source code. A direct link to GitHub is also available on the platform. \par 
    \textbf{Input/Condition:} Perform updates to individual components and simulate user feedback submissions via the GitHub repository. \par 
    \textbf{Output/Results:} Expected results include: 
    \begin{itemize} 
      \item Each component update does not exceed \\NUM\_CODE\_LINES lines of code edited outside the updated module.
      \item Users can submit issues and feature requests directly to \\GitHub, categorized as issues, feature requests, or feedback.
    \end{itemize} \par
    \textbf{Test Case Derivation:} Validates MS-MR1 and MS-SR1 by confirming modularity of component updates and the presence of a support feedback mechanism via GitHub. \par
    \textbf{How the test will be performed:} 
    \begin{enumerate}[noitemsep] 
      \item Perform code updates on isolated components and verify \\changes are contained within NUM\_CODE\_LINES lines \\outside the component. 
      \item Test submission flow to GitHub, verifying links are accessible and that issues and requests are categorized correctly. 
    \end{enumerate} 
  \end{mdframed}

  \item MS-ST-MSA2
  \begin{mdframed}[linewidth=0.5mm] 
    \textbf{Type:} Dynamic \par 
    \textbf{Initial State:} Platform initialized with a tutorial accessible from the homepage. \par 
    \textbf{Input/Condition:} New user group follows the tutorial to \\complete primary tasks (e.g., starting an assessment). \par 
    \textbf{Output/Results:} Expected results include: 
    \begin{itemize} 
      \item HIGH\_SUCCESS\_RATE of users can complete core tasks \\correctly after following the tutorial. 
    \end{itemize} \par
    \textbf{Test Case Derivation:} Supports MS-SR2 by confirming that the platform’s tutorial is effective in helping new users complete essential tasks without prior training. \par
    \textbf{How the test will be performed:} 
    \begin{enumerate}[noitemsep] 
      \item Guide users through the tutorial and observe task completion rates. 
      \item Collect feedback on tutorial clarity and assess if 90\% of users can independently complete tasks. 
    \end{enumerate} 
  \end{mdframed}

  \item MS-ST-MSA3
  \begin{mdframed}[linewidth=0.5mm] 
    \textbf{Type:} Dynamic \par 
    \textbf{Initial State:} Platform accessible across various devices and \\screen sizes, from mobile (MIN\_SCREEN\_SIZE) to desktop \\(MAX\_SCREEN\_SIZE). \par 
    \textbf{Input/Condition:} Load and navigate the platform across \\multiple devices to evaluate responsive design and functionality. \par 
    \textbf{Output/Results:} Expected results include: 
    \begin{itemize} 
      \item MAX\_SUCCESS\_RATE of essential features are fully \\functional and readable across all screen sizes tested. 
    \end{itemize} \par
    \textbf{Test Case Derivation:} Ensures MS-AR1 by confirming that the platform is accessible and functions consistently across all screen resolutions and device types. \par
    \textbf{How the test will be performed:} 
    \begin{enumerate}[noitemsep] 
      \item Access the platform on various screen sizes and test for \\display, layout, and functionality. 
      \item Verify that all features are usable and adapt responsively without readability or functionality loss. 
    \end{enumerate} 
  \end{mdframed} 
\end{itemize}

\subsubsection{Cultural Requirements}

\hspace{2em}These tests ensure that the platform respects cultural sensitivities and provides full bilingual support, enhancing inclusivity and accessibility for diverse user groups.

\begin{itemize} 
  \item CU-ST-CUR1
  \begin{mdframed}[linewidth=0.5mm] 
    \textbf{Type:} Static and Dynamic \par 
    \textbf{Initial State:} Platform content (language and imagery) is \\finalized and presented for review. \par 
    \textbf{Input/Condition:} A cultural consultant reviews all language and imagery, and user acceptance testing gathers feedback from a diverse set of users. \par 
    \textbf{Output/Results:} Expected results include: 
    \begin{itemize} 
      \item MAX\_SUCCESS\_RATE of reviewed content is confirmed as culturally sensitive with no instances of offensive language or imagery. 
    \end{itemize} \par 
    \textbf{Test Case Derivation:} Verifies CU-CR1 by confirming the platform’s content is reviewed for cultural appropriateness and validated by user feedback from diverse backgrounds. \par 
    \textbf{How the test will be performed:} 
    \begin{enumerate}[noitemsep] 
      \item A cultural consultant examines all text and imagery for \\potential cultural insensitivity. 
      \item Conduct user acceptance testing with a diverse group and gather feedback on the platform's cultural sensitivity. 
      \item Validate that all feedback confirms no culturally insensitive content. 
    \end{enumerate} 
  \end{mdframed}

  \item CU-ST-CUR2
  \begin{mdframed}[linewidth=0.5mm] 
    \textbf{Type:} Static \par 
    \textbf{Initial State:} Platform is available in both English and Mandarin, with all interface elements and assessments translated. \par 
    \textbf{Input/Condition:} Switch between language settings to review each text,\\
    instruction, and assessment in both languages. \par 
    \textbf{Output/Results:} Expected results include: 
    \begin{itemize} 
      \item MAX\_SUCCESS\_RATE of assessment content is fully \\translated and functional in both English and Mandarin with no \\untranslated elements. 
    \end{itemize} \par 
    \textbf{Test Case Derivation:} Addresses CU-CR2 by confirming full bilingual support across the entire platform, ensuring accessibility for both English and Mandarin speakers. \par 
    \textbf{How the test will be performed:} 
    \begin{enumerate}[noitemsep] 
      \item Navigate through the platform in both English and Mandarin settings, \\ verifying translations for each section. 
      \item Confirm that all assessments, instructions, and interface \\elements are accurately translated and free from language discrepancies. 
    \end{enumerate} 
  \end{mdframed} 
\end{itemize}

\subsubsection{Security}

The test cases below ensures that the system meets essential 
security requirements including authentication of users and encryption of 
confidential data in the system
		
\begin{itemize}
  \begin{item}
      SR-ST-AC1
      \begin{mdframed}[linewidth=0.5mm]
          \textbf{Type:} Dynamic \par
          \textbf{Initial State:} System has multiple user roles: Admin, Parent, and Clinician. \par
          \textbf{Input/Condition:}   User with Admin role attempts to create and assign accounts to clinicians \par
          \textbf{Output/Results:}  Only Admin users can access and execute functions related 
          to clinician account creation \par
          \textbf{Test Case Derivation:} Ensures SR-AC1 by confirming that role-based access control restricts clinician account creation to Admin users only. \par
          \textbf{How the test will be performed:}
          \begin{enumerate}[noitemsep]
            \item Log in as an Admin and attempt to create and view clinician accounts. Verify that the action succeeds
            \item Log in as a non-Admin (e.g., Parent or Clinician) and try to access the same 
            function. Confirm that access is denied with an appropriate error message.
          \end{enumerate}
      \end{mdframed}
  \end{item}

  \begin{item}
    SR-ST-AC2
    \begin{mdframed}[linewidth=0.5mm]
      \textbf{Type:} Dynamic \par
      \textbf{Initial State:} Parent role created and available for testing. \par
      \textbf{Input/Condition:} User with Parent role logs in and attempts to complete assessments. \par
      \textbf{Output/Results:} Parent users can create their account, complete assessments, and log out successfully. \par
      \textbf{Test Case Derivation:} Validates SR-AC2 by ensuring that Parent users can perform all necessary functions without elevated privileges. \par
      \textbf{How the test will be performed:}
      \begin{enumerate}[noitemsep]
        \item Log in as a Parent user and attempt to start and complete an assessment. Verify successful completion and logout.
        \item Attempt to access administrative functions (e.g., creating clinician accounts). 
        Verify that access is denied with an appropriate message.
      \end{enumerate}
    \end{mdframed}
  \end{item}
  
  \begin{item}
    SR-ST-AC3
      \begin{mdframed}[linewidth=0.5mm]
        \textbf{Type:} Dynamic \par
        \textbf{Initial State:} Clinician role with restricted access is created and available for testing. \par
        \textbf{Input/Condition:} User with Clinician role logs in and attempts to view assessment results. \par
        \textbf{Output/Results:} Clinician users can view assessment results but cannot start or 
        complete assessments as a Parent user would. \par
        \textbf{Test Case Derivation:} Confirms SR-AC3 by testing access control boundaries for the Clinician role, ensuring view-only access. \par
        \textbf{How the test will be performed:}
        \begin{enumerate}[noitemsep]
          \item Log in as a Clinician and attempt to view completed assessment results. Confirm access is granted.
          \item Attempt to start or complete an assessment and confirm access is denied with an 
          error message indicating unauthorized action.
        \end{enumerate}
      \end{mdframed}
  \end{item}

  \begin{item}
     SR-ST-AC4
    \begin{mdframed}[linewidth=0.5mm]
      \textbf{Type:} Dynamic \par
      \textbf{Initial State:} System with user login functionality. \par
      \textbf{Input/Condition:} Users attempt to log in with correct and incorrect credentials. \par
      \textbf{Output/Results:} Users can only log in with correct credentials; unauthorized access attempts are denied. \par
      \textbf{Test Case Derivation:} Ensures SR-AC4 by verifying secure authentication mechanisms that prevent unauthorized access. \par
      \textbf{How the test will be performed:}
      \begin{enumerate}[noitemsep]
        \item Attempt to log in with valid credentials for multiple roles (Admin, Parent, Clinician). Confirm successful login.
        \item Attempt to log in with incorrect credentials (e.g., incorrect username or password). 
        Confirm that login is denied and an error message is shown.
      \end{enumerate}
    \end{mdframed}
  \end{item}

  \begin{item}
    SR-ST-P1
    \begin{mdframed}[linewidth=0.5mm]
      \textbf{Type:} Static \par
      \textbf{Initial State:} System ready for pre-release review. \par
      \textbf{Input/Condition:} Review documentation to ensure adherence to data protection and privacy laws in the region. \par
      \textbf{Output/Results:} Confirm all applicable data protection requirements are met. \par
      \textbf{Test Case Derivation:} Validates SR-P1 by confirming the system adheres to legal and regulatory privacy frameworks such as HIPAA. \par
      \textbf{How the test will be performed:}
      \begin{enumerate}[noitemsep]
        \item Conduct a documentation review with legal and compliance teams, focusing on privacy policies, data handling, and retention practices.
        \item Verify that all data collection, storage, and usage adhere to applicable privacy laws (e.g., HIPAA).
      \end{enumerate}
    \end{mdframed}
  \end{item}

  \begin{item}
    SR-ST-P2
    \begin{mdframed}[linewidth=0.5mm]
      \textbf{Type:} Static, Automated \par
      \textbf{Initial State:} System with sensitive data handling enabled. \par
      \textbf{Input/Condition:} Examine data in transit and at rest. \par
      \textbf{Output/Results:} Data remains encrypted according to standard encryption protocols during transit and at rest. \par
      \textbf{Test Case Derivation:} Verifies SR-P2 by ensuring encryption mechanisms are implemented and meet security best practices for sensitive data. \par
      \textbf{How the test will be performed:}
      \begin{enumerate}[noitemsep]
        \item Use network monitoring tools to capture data packets during transmission to ensure data is encrypted.
        \item Review database configuration to verify that data at rest is encrypted. \\ Decrypting should only be 
        possible by authorized clinician accounts.
      \end{enumerate}
    \end{mdframed}
  \end{item}

  \begin{item}
    SR-ST-P3
    \begin{mdframed}[linewidth=0.5mm]
      \textbf{Type:} Static \par
      \textbf{Initial State:} System configured for data collection. \par
      \textbf{Input/Condition:} Examine data storage for PII. \par
      \textbf{Output/Results:} System does not store any personal identifiable information beyond username and assessment recordings. \par
      \textbf{Test Case Derivation:} Ensures SR-P3 by confirming that only non-sensitive identifiers are retained, reducing privacy risk. \par
      \textbf{How the test will be performed:}
      \begin{enumerate}[noitemsep]
        \item Conduct a database inspection and audit to ensure no PII (e.g., address, date of birth, names) is stored.
        \item Verify data schema to confirm that only usernames and recordings are stored.
      \end{enumerate}
    \end{mdframed}
  \end{item}

  \begin{item}
    SR-ST-IM1
    \begin{mdframed}[linewidth=0.5mm]
      \textbf{Type:} Dynamic \par
      \textbf{Initial State:} System is in the account creation phase, requiring users to set up passwords. \par
      \textbf{Input/Condition:} User attempts to create an account with both weak and strong passwords. \par
      \textbf{Output/Results:} Account creation is completed only when a strong password, meeting specified 
      security criteria, is entered. Weak passwords are rejected with an error message detailing the password requirements. \par
      \textbf{Test Case Derivation:} Validates SR-IM1 by testing enforcement of strong password policies to protect against unauthorized access. \par
      \textbf{How the test will be performed:}
      \begin{enumerate}[noitemsep]
        \item Attempt to create an account with a weak password (e.g., fewer than 8 characters, no special 
        characters or numbers). Verify that the system rejects the password and displays an error message with the password requirements.
        \item Attempt to create an account with a password that meets all the specified criteria (e.g., at 
        least 8 characters, containing upper and lowercase letters, numbers, and special characters). 
        Verify that account creation is successful.
        \item Repeat with different combinations of weak and strong passwords to confirm consistent 
        enforcement of the password policy.
      \end{enumerate}
    \end{mdframed}
  \end{item}

\end{itemize}

\subsubsection{Compliance}

The test cases below ensures that the system meets essential 
compliance requirements including security of the system, following the rules of the clinician and law.
		
\begin{itemize}
  \begin{item}
    CR-ST-D1
    \begin{mdframed}[linewidth=0.5mm]
      \textbf{Type:} Static, Automated \par
      \textbf{Initial State:} User assessment data, including video and audio recordings, is stored in the system. \par
      \textbf{Input/Condition:} Examine the storage configuration and security measures applied to user assessment data. \par
      \textbf{Output/Results:} User assessment data is securely stored and associated only with usernames, 
      ensuring no additional personally identifiable information is \\ included. Data at rest is protected by strong encryption 
      in compliance with HIPAA Security Rule standards. \par
      \textbf{Test Case Derivation:} Verifies CR-STD1 by confirming the system enforces HIPAA-compliant storage standards, protecting sensitive health data through encryption and limiting identifiers. \par
      \textbf{How the test will be performed:}
      \begin{enumerate}[noitemsep]
        \item Review database schema to confirm that assessment data is linked only to usernames and does not
         include any additional identifiable information.
        \item Conduct an inspection of the encryption protocols used for stored data to verify compliance with
         security standards.
        \item Attempt unauthorized access to stored assessment data to ensure encryption and security measures
         prevent access by unauthorized users.
      \end{enumerate}
    \end{mdframed}
  \end{item}
\end{itemize}

\newpage 
\begin{landscape}
\subsection{Traceability Between Test Cases and Requirements}

  \begin{table}[!ht]
    \caption{Traceability Table Between System Test Cases and Functional Requirements}
    \resizebox{1.5\textwidth}{!}{
    \begin{tabular}{|l|l|l|l|l|l|l|l|l|l|l|l|l|l|l|l|l|l|l|l|l|l|l|l|l|l|l|l|l|l|l|l|l|}
    \hline
        ~ & FR-ST-A1 & FR-ST-A2 & FR-ST-A3 & FR-ST-A4 & FR-ST-A5 & FR-ST-A6 & FR-ST-A7 & FR-ST-A8 & FR-ST-DSC1 & FR-ST-DSC2 & FR-ST-DSC3 & FR-ST-DSC4 & FR-ST-DSC5 & FR-ST-VDA1 & FR-ST-VDA2 & FR-ST-VDA3 & FR-ST-DPD1 & FR-ST-DPD2 & FR-ST-DPD3 & FR-ST-DPD4 & FR-ST-SS1 & FR-ST-SS2 & FR-ST-SS3 & FR-ST-SS4 & FR-ST-SS5 & FR-ST-AI1 & FR-ST-AI2 & FR-ST-AI3 & FR-ST-AI4 & FR-ST-AI5 & FR-ST-AI6 & FR-ST-AI7 \\ \hline
        FR-A1 & X & X & ~ & ~ & ~ & ~ & ~ & ~ & ~ & ~ & ~ & ~ & ~ & ~ & ~ & ~ & ~ & ~ & ~ & ~ & ~ & ~ & ~ & ~ & ~ & ~ & ~ & ~ & ~ & ~ & ~ & ~ \\ \hline
        FR-A2 & ~ & ~ & X & X & ~ & ~ & ~ & ~ & ~ & ~ & ~ & ~ & ~ & ~ & ~ & ~ & ~ & ~ & ~ & ~ & ~ & ~ & ~ & ~ & ~ & ~ & ~ & ~ & ~ & ~ & ~ & ~ \\ \hline
        FR-A3 & ~ & ~ & ~ & ~ & X & X & ~ & ~ & ~ & ~ & ~ & ~ & ~ & ~ & ~ & ~ & ~ & ~ & ~ & ~ & ~ & ~ & ~ & ~ & ~ & ~ & ~ & ~ & ~ & ~ & ~ & ~ \\ \hline
        FR-A4 & ~ & ~ & ~ & ~ & ~ & ~ & X & ~ & ~ & ~ & ~ & ~ & ~ & ~ & ~ & ~ & ~ & ~ & ~ & ~ & ~ & ~ & ~ & ~ & ~ & ~ & ~ & ~ & ~ & ~ & ~ & ~ \\ \hline
        FR-A5 & ~ & ~ & ~ & ~ & ~ & ~ & ~ & X & ~ & ~ & ~ & ~ & ~ & ~ & ~ & ~ & ~ & ~ & ~ & ~ & ~ & ~ & ~ & ~ & ~ & ~ & ~ & ~ & ~ & ~ & ~ & ~ \\ \hline
        FR-SS1 & ~ & ~ & ~ & ~ & ~ & ~ & ~ & ~ & ~ & ~ & ~ & ~ & ~ & ~ & ~ & ~ & ~ & ~ & ~ & ~ & X & ~ & ~ & ~ & ~ & ~ & ~ & ~ & ~ & ~ & ~ & ~ \\ \hline
        FR-SS2 & ~ & ~ & ~ & ~ & ~ & ~ & ~ & ~ & ~ & ~ & ~ & ~ & ~ & ~ & ~ & ~ & ~ & ~ & ~ & ~ & ~ & X & ~ & ~ & ~ & ~ & ~ & ~ & ~ & ~ & ~ & ~ \\ \hline
        FR-SS3 & ~ & ~ & ~ & ~ & ~ & ~ & ~ & ~ & ~ & ~ & ~ & ~ & ~ & ~ & ~ & ~ & ~ & ~ & ~ & ~ & ~ & ~ & X & ~ & ~ & ~ & ~ & ~ & ~ & ~ & ~ & ~ \\ \hline
        FR-SS4 & ~ & ~ & ~ & ~ & ~ & ~ & ~ & ~ & ~ & ~ & ~ & ~ & ~ & ~ & ~ & ~ & ~ & ~ & ~ & ~ & ~ & ~ & ~ & X & ~ & ~ & ~ & ~ & ~ & ~ & ~ & ~ \\ \hline
        FR-SS5 & ~ & ~ & ~ & ~ & ~ & ~ & ~ & ~ & ~ & ~ & ~ & ~ & ~ & ~ & ~ & ~ & ~ & ~ & ~ & ~ & ~ & ~ & ~ & ~ & X & ~ & ~ & ~ & ~ & ~ & ~ & ~ \\ \hline
        FR-AI1 & ~ & ~ & ~ & ~ & ~ & ~ & ~ & ~ & ~ & ~ & ~ & ~ & ~ & ~ & ~ & ~ & ~ & ~ & ~ & ~ & ~ & ~ & ~ & ~ & ~ & X & ~ & ~ & ~ & ~ & ~ & ~ \\ \hline
        FR-AI2 & ~ & ~ & ~ & ~ & ~ & ~ & ~ & ~ & ~ & ~ & ~ & ~ & ~ & ~ & ~ & ~ & ~ & ~ & ~ & ~ & ~ & ~ & ~ & ~ & ~ & ~ & X & ~ & ~ & ~ & ~ & ~ \\ \hline
        FR-AI3 & ~ & ~ & ~ & ~ & ~ & ~ & ~ & ~ & ~ & ~ & ~ & ~ & ~ & ~ & ~ & ~ & ~ & ~ & ~ & ~ & ~ & ~ & ~ & ~ & ~ & ~ & ~ & X & ~ & ~ & ~ & ~ \\ \hline
        FR-AI4 & ~ & ~ & ~ & ~ & ~ & ~ & ~ & ~ & ~ & ~ & ~ & ~ & ~ & ~ & ~ & ~ & ~ & ~ & ~ & ~ & ~ & ~ & ~ & ~ & ~ & ~ & ~ & ~ & X & ~ & ~ & ~ \\ \hline
        FR-AI5 & ~ & ~ & ~ & ~ & ~ & ~ & ~ & ~ & ~ & ~ & ~ & ~ & ~ & ~ & ~ & ~ & ~ & ~ & ~ & ~ & ~ & ~ & ~ & ~ & ~ & ~ & ~ & ~ & ~ & X & ~ & ~ \\ \hline
        FR-AI6 & ~ & ~ & ~ & ~ & ~ & ~ & ~ & ~ & ~ & ~ & ~ & ~ & ~ & ~ & ~ & ~ & ~ & ~ & ~ & ~ & ~ & ~ & ~ & ~ & ~ & ~ & ~ & ~ & ~ & ~ & X & ~ \\ \hline
        FR-AI7 & ~ & ~ & ~ & ~ & ~ & ~ & ~ & ~ & ~ & ~ & ~ & ~ & ~ & ~ & ~ & ~ & ~ & ~ & ~ & ~ & ~ & ~ & ~ & ~ & ~ & ~ & ~ & ~ & ~ & ~ & ~ & X \\ \hline
        FR-DSC1 & ~ & ~ & ~ & ~ & ~ & ~ & ~ & ~ & X & ~ & ~ & ~ & ~ & ~ & ~ & ~ & ~ & ~ & ~ & ~ & ~ & ~ & ~ & ~ & ~ & ~ & ~ & ~ & ~ & ~ & ~ & ~ \\ \hline
        FR-DSC2 & ~ & ~ & ~ & ~ & ~ & ~ & ~ & ~ & ~ & X & ~ & ~ & ~ & ~ & ~ & ~ & ~ & ~ & ~ & ~ & ~ & ~ & ~ & ~ & ~ & ~ & ~ & ~ & ~ & ~ & ~ & ~ \\ \hline
        FR-DSC3 & ~ & ~ & ~ & ~ & ~ & ~ & ~ & ~ & ~ & ~ & X & ~ & ~ & ~ & ~ & ~ & ~ & ~ & ~ & ~ & ~ & ~ & ~ & ~ & ~ & ~ & ~ & ~ & ~ & ~ & ~ & ~ \\ \hline
        FR-DSC4 & ~ & ~ & ~ & ~ & ~ & ~ & ~ & ~ & ~ & ~ & ~ & X & ~ & ~ & ~ & ~ & ~ & ~ & ~ & ~ & ~ & ~ & ~ & ~ & ~ & ~ & ~ & ~ & ~ & ~ & ~ & ~ \\ \hline
        FR-DSC5 & ~ & ~ & ~ & ~ & ~ & ~ & ~ & ~ & ~ & ~ & ~ & ~ & X & ~ & ~ & ~ & ~ & ~ & ~ & ~ & ~ & ~ & ~ & ~ & ~ & ~ & ~ & ~ & ~ & ~ & ~ & ~ \\ \hline
        FR-VADA1 & ~ & ~ & ~ & ~ & ~ & ~ & ~ & ~ & ~ & ~ & ~ & ~ & ~ & X & ~ & ~ & ~ & ~ & ~ & ~ & ~ & ~ & ~ & ~ & ~ & ~ & ~ & ~ & ~ & ~ & ~ & ~ \\ \hline
        FR-VADA2 & ~ & ~ & ~ & ~ & ~ & ~ & ~ & ~ & ~ & ~ & ~ & ~ & ~ & ~ & X & ~ & ~ & ~ & ~ & ~ & ~ & ~ & ~ & ~ & ~ & ~ & ~ & ~ & ~ & ~ & ~ & ~ \\ \hline
        FR-VADA3 & ~ & ~ & ~ & ~ & ~ & ~ & ~ & ~ & ~ & ~ & ~ & ~ & ~ & ~ & ~ & X & ~ & ~ & ~ & ~ & ~ & ~ & ~ & ~ & ~ & ~ & ~ & ~ & ~ & ~ & ~ & ~ \\ \hline
        FR-DPD1 & ~ & ~ & ~ & ~ & ~ & ~ & ~ & ~ & ~ & ~ & ~ & ~ & ~ & ~ & ~ & ~ & X & ~ & ~ & ~ & ~ & ~ & ~ & ~ & ~ & ~ & ~ & ~ & ~ & ~ & ~ & ~ \\ \hline
        FR-DPD2 & ~ & ~ & ~ & ~ & ~ & ~ & ~ & ~ & ~ & ~ & ~ & ~ & ~ & ~ & ~ & ~ & ~ & X & ~ & ~ & ~ & ~ & ~ & ~ & ~ & ~ & ~ & ~ & ~ & ~ & ~ & ~ \\ \hline
        FR-DPD3 & ~ & ~ & ~ & ~ & ~ & ~ & ~ & ~ & ~ & ~ & ~ & ~ & ~ & ~ & ~ & ~ & ~ & ~ & X & ~ & ~ & ~ & ~ & ~ & ~ & ~ & ~ & ~ & ~ & ~ & ~ & ~ \\ \hline
        FR-DPD4 & ~ & ~ & ~ & ~ & ~ & ~ & ~ & ~ & ~ & ~ & ~ & ~ & ~ & ~ & ~ & ~ & ~ & ~ & ~ & X & ~ & ~ & ~ & ~ & ~ & ~ & ~ & ~ & ~ & ~ & ~ & ~ \\ \hline
    \end{tabular}
    }
\end{table}

\begin{table}[!ht]
  \caption{Traceability Table Between System Test Cases and Nonfunctional Requirements}
  \resizebox{1.5\textwidth}{!}{
  \begin{tabular}{|l|l|l|l|l|l|l|l|l|l|l|l|l|l|l|l|l|l|l|l|l|l|l|l|l|l|l|l|l|l|l|l|l|l|l|l|}
  \hline
      System test  & LF-ST-LFR1 & LF-ST-LFR2 & UH-ST-EOU1 & UH-ST-PI1 & UH-ST-LI2 & PR-ST-SL1 & PR-ST-SL2 & PR-ST-SL3 & PR-ST-PA1 & PR-ST-PA3 & PR-ST-PA4 & PR-ST-RFT1 & PR-ST-RFT2 & PR-ST-CR1 & PR-ST-CR2 & PR-ST-SE1 & PR-ST-LR1 & PR-ST-LR2 & OE-ST-EPE1 & EO-ST-WE1 & OE-ST-IA1 & MS-ST-MSA1 & MS-ST-MSA2 & MS-ST-MSA3 & CU-ST-CUR1 & CU-ST-CUR1 & SR-ST-AC1 & SR-ST-AC2 & SR-ST-AC3 & SR-ST-AC4 & SR-ST-IM1 & CR-ST-D1 & SR-ST-P1 & SR-ST-P2 & SR-ST-P3 \\ \hline
      LF-AR1 & X & ~ & ~ & ~ & ~ & ~ & ~ & ~ & ~ & ~ & ~ & ~ & ~ & ~ & ~ & ~ & ~ & ~ & ~ & ~ & ~ & ~ & ~ & ~ & ~ & ~ & ~ & ~ & ~ & ~ & ~ & ~ & ~ & ~ & ~ \\ \hline
      LF-AR2 & X & ~ & ~ & ~ & ~ & ~ & ~ & ~ & ~ & ~ & ~ & ~ & ~ & ~ & ~ & ~ & ~ & ~ & ~ & ~ & ~ & ~ & ~ & ~ & ~ & ~ & ~ & ~ & ~ & ~ & ~ & ~ & ~ & ~ & ~ \\ \hline
      LF-AR3 & ~ & X & ~ & ~ & ~ & ~ & ~ & ~ & ~ & ~ & ~ & ~ & ~ & ~ & ~ & ~ & ~ & ~ & ~ & ~ & ~ & ~ & ~ & ~ & ~ & ~ & ~ & ~ & ~ & ~ & ~ & ~ & ~ & ~ & ~ \\ \hline
      LF-AR4 & X & ~ & ~ & ~ & ~ & ~ & ~ & ~ & ~ & ~ & ~ & ~ & ~ & ~ & ~ & ~ & ~ & ~ & ~ & ~ & ~ & ~ & ~ & ~ & ~ & ~ & ~ & ~ & ~ & ~ & ~ & ~ & ~ & ~ & ~ \\ \hline
      LF-AR5 & ~ & X & ~ & ~ & ~ & ~ & ~ & ~ & ~ & ~ & ~ & ~ & ~ & ~ & ~ & ~ & ~ & ~ & ~ & ~ & ~ & ~ & ~ & ~ & ~ & ~ & ~ & ~ & ~ & ~ & ~ & ~ & ~ & ~ & ~ \\ \hline
      LF-SR1 & ~ & X & ~ & ~ & ~ & ~ & ~ & ~ & ~ & ~ & ~ & ~ & ~ & ~ & ~ & ~ & ~ & ~ & ~ & ~ & ~ & ~ & ~ & ~ & ~ & ~ & ~ & ~ & ~ & ~ & ~ & ~ & ~ & ~ & ~ \\ \hline
      LF-SR2 & ~ & X & ~ & ~ & ~ & ~ & ~ & ~ & ~ & ~ & ~ & ~ & ~ & ~ & ~ & ~ & ~ & ~ & ~ & ~ & ~ & ~ & ~ & ~ & ~ & ~ & ~ & ~ & ~ & ~ & ~ & ~ & ~ & ~ & ~ \\ \hline
      UH-EOU1 & ~ & ~ & X & ~ & ~ & ~ & ~ & ~ & ~ & ~ & ~ & ~ & ~ & ~ & ~ & ~ & ~ & ~ & ~ & ~ & ~ & ~ & ~ & ~ & ~ & ~ & ~ & ~ & ~ & ~ & ~ & ~ & ~ & ~ & ~ \\ \hline
      UH-EOU2 & ~ & ~ & X & ~ & ~ & ~ & ~ & ~ & ~ & ~ & ~ & ~ & ~ & ~ & ~ & ~ & ~ & ~ & ~ & ~ & ~ & ~ & ~ & ~ & ~ & ~ & ~ & ~ & ~ & ~ & ~ & ~ & ~ & ~ & ~ \\ \hline
      UH-PI1 & ~ & ~ & ~ & X & ~ & ~ & ~ & ~ & ~ & ~ & ~ & ~ & ~ & ~ & ~ & ~ & ~ & ~ & ~ & ~ & ~ & ~ & ~ & ~ & ~ & ~ & ~ & ~ & ~ & ~ & ~ & ~ & ~ & ~ & ~ \\ \hline
      UH-LI1 & ~ & ~ & X & ~ & ~ & ~ & ~ & ~ & ~ & ~ & ~ & ~ & ~ & ~ & ~ & ~ & ~ & ~ & ~ & ~ & ~ & ~ & ~ & ~ & ~ & ~ & ~ & ~ & ~ & ~ & ~ & ~ & ~ & ~ & ~ \\ \hline
      UH-LI2 & ~ & ~ & ~ & ~ & X & ~ & ~ & ~ & ~ & ~ & ~ & ~ & ~ & ~ & ~ & ~ & ~ & ~ & ~ & ~ & ~ & ~ & ~ & ~ & ~ & ~ & ~ & ~ & ~ & ~ & ~ & ~ & ~ & ~ & ~ \\ \hline
      UH-UP1 & ~ & ~ & X & ~ & ~ & ~ & ~ & ~ & ~ & ~ & ~ & ~ & ~ & ~ & ~ & ~ & ~ & ~ & ~ & ~ & ~ & ~ & ~ & ~ & ~ & ~ & ~ & ~ & ~ & ~ & ~ & ~ & ~ & ~ & ~ \\ \hline
      UH-AR1 & ~ & ~ & X & ~ & ~ & ~ & ~ & ~ & ~ & ~ & ~ & ~ & ~ & ~ & ~ & ~ & ~ & ~ & ~ & ~ & ~ & ~ & ~ & ~ & ~ & ~ & ~ & ~ & ~ & ~ & ~ & ~ & ~ & ~ & ~ \\ \hline
      PR-SL1 & ~ & ~ & ~ & ~ & ~ & X & ~ & ~ & ~ & ~ & ~ & ~ & ~ & ~ & ~ & ~ & ~ & ~ & ~ & ~ & ~ & ~ & ~ & ~ & ~ & ~ & ~ & ~ & ~ & ~ & ~ & ~ & ~ & ~ & ~ \\ \hline
      PR-SL2 & ~ & ~ & ~ & ~ & ~ & ~ & X & ~ & ~ & ~ & ~ & ~ & ~ & ~ & ~ & ~ & ~ & ~ & ~ & ~ & ~ & ~ & ~ & ~ & ~ & ~ & ~ & ~ & ~ & ~ & ~ & ~ & ~ & ~ & ~ \\ \hline
      PR-SL3 & ~ & ~ & ~ & ~ & ~ & ~ & ~ & X & ~ & ~ & ~ & ~ & ~ & ~ & ~ & ~ & ~ & ~ & ~ & ~ & ~ & ~ & ~ & ~ & ~ & ~ & ~ & ~ & ~ & ~ & ~ & ~ & ~ & ~ & ~ \\ \hline
      PR-SL4 & ~ & ~ & ~ & ~ & ~ & ~ & ~ & ~ & ~ & ~ & ~ & ~ & ~ & ~ & ~ & ~ & ~ & ~ & ~ & ~ & ~ & ~ & ~ & ~ & ~ & ~ & ~ & ~ & ~ & ~ & ~ & ~ & ~ & ~ & ~ \\ \hline
      PR-SCL1 & ~ & ~ & ~ & ~ & ~ & ~ & ~ & ~ & ~ & ~ & ~ & ~ & ~ & ~ & ~ & ~ & ~ & ~ & ~ & ~ & ~ & ~ & ~ & ~ & ~ & ~ & ~ & ~ & ~ & ~ & ~ & ~ & ~ & ~ & ~ \\ \hline
      PR-PA1 & ~ & ~ & ~ & ~ & ~ & ~ & ~ & ~ & X & ~ & ~ & ~ & ~ & ~ & ~ & ~ & ~ & ~ & ~ & ~ & ~ & ~ & ~ & ~ & ~ & ~ & ~ & ~ & ~ & ~ & ~ & ~ & ~ & ~ & ~ \\ \hline
      PR-PA2 & ~ & ~ & ~ & ~ & ~ & ~ & ~ & ~ & X & ~ & ~ & ~ & ~ & ~ & ~ & ~ & ~ & ~ & ~ & ~ & ~ & ~ & ~ & ~ & ~ & ~ & ~ & ~ & ~ & ~ & ~ & ~ & ~ & ~ & ~ \\ \hline
      PR-PA3 & ~ & ~ & ~ & ~ & ~ & ~ & ~ & ~ & ~ & X & ~ & ~ & ~ & ~ & ~ & ~ & ~ & ~ & ~ & ~ & ~ & ~ & ~ & ~ & ~ & ~ & ~ & ~ & ~ & ~ & ~ & ~ & ~ & ~ & ~ \\ \hline
      PR-PA4 & ~ & ~ & ~ & ~ & ~ & ~ & ~ & ~ & ~ & ~ & X & ~ & ~ & ~ & ~ & ~ & ~ & ~ & ~ & ~ & ~ & ~ & ~ & ~ & ~ & ~ & ~ & ~ & ~ & ~ & ~ & ~ & ~ & ~ & ~ \\ \hline
      PR-PA5 & ~ & ~ & ~ & ~ & ~ & ~ & ~ & ~ & ~ & ~ & X & ~ & ~ & ~ & ~ & ~ & ~ & ~ & ~ & ~ & ~ & ~ & ~ & ~ & ~ & ~ & ~ & ~ & ~ & ~ & ~ & ~ & ~ & ~ & ~ \\ \hline
      PR-RFT1 & ~ & ~ & ~ & ~ & ~ & ~ & ~ & ~ & ~ & ~ & ~ & X & ~ & ~ & ~ & ~ & ~ & ~ & ~ & ~ & ~ & ~ & ~ & ~ & ~ & ~ & ~ & ~ & ~ & ~ & ~ & ~ & ~ & ~ & ~ \\ \hline
      PR-RFT2 & ~ & ~ & ~ & ~ & ~ & ~ & ~ & ~ & ~ & ~ & ~ & ~ & X & ~ & ~ & ~ & ~ & ~ & ~ & ~ & ~ & ~ & ~ & ~ & ~ & ~ & ~ & ~ & ~ & ~ & ~ & ~ & ~ & ~ & ~ \\ \hline
      PR-RFT3 & ~ & ~ & ~ & ~ & ~ & ~ & ~ & ~ & ~ & ~ & ~ & X & ~ & ~ & ~ & ~ & ~ & ~ & ~ & ~ & ~ & ~ & ~ & ~ & ~ & ~ & ~ & ~ & ~ & ~ & ~ & ~ & ~ & ~ & ~ \\ \hline
      PR-CR1 & ~ & ~ & ~ & ~ & ~ & ~ & ~ & ~ & ~ & ~ & ~ & ~ & ~ & X & ~ & ~ & ~ & ~ & ~ & ~ & ~ & ~ & ~ & ~ & ~ & ~ & ~ & ~ & ~ & ~ & ~ & ~ & ~ & ~ & ~ \\ \hline
      PR-CR2 & ~ & ~ & ~ & ~ & ~ & ~ & ~ & ~ & ~ & ~ & ~ & ~ & ~ & ~ & X & ~ & ~ & ~ & ~ & ~ & ~ & ~ & ~ & ~ & ~ & ~ & ~ & ~ & ~ & ~ & ~ & ~ & ~ & ~ & ~ \\ \hline
      PR-CR3 & ~ & ~ & ~ & ~ & ~ & ~ & ~ & ~ & ~ & ~ & ~ & ~ & ~ & X & ~ & ~ & ~ & ~ & ~ & ~ & ~ & ~ & ~ & ~ & ~ & ~ & ~ & ~ & ~ & ~ & ~ & ~ & ~ & ~ & ~ \\ \hline
      PR-CR4 & ~ & ~ & ~ & ~ & ~ & ~ & ~ & ~ & ~ & ~ & ~ & ~ & ~ & X & ~ & ~ & ~ & ~ & ~ & ~ & ~ & ~ & ~ & ~ & ~ & ~ & ~ & ~ & ~ & ~ & ~ & ~ & ~ & ~ & ~ \\ \hline
      PR-SE1 & ~ & ~ & ~ & ~ & ~ & ~ & ~ & ~ & ~ & ~ & ~ & ~ & ~ & ~ & ~ & X & ~ & ~ & ~ & ~ & ~ & ~ & ~ & ~ & ~ & ~ & ~ & ~ & ~ & ~ & ~ & ~ & ~ & ~ & ~ \\ \hline
      PR-SE2 & ~ & ~ & ~ & ~ & ~ & ~ & ~ & ~ & ~ & ~ & ~ & ~ & ~ & ~ & X & ~ & ~ & ~ & ~ & ~ & ~ & ~ & ~ & ~ & ~ & ~ & ~ & ~ & ~ & ~ & ~ & ~ & ~ & ~ & ~ \\ \hline
      PR-SE3 & ~ & ~ & ~ & ~ & ~ & ~ & ~ & ~ & ~ & ~ & ~ & ~ & ~ & ~ & X & ~ & ~ & ~ & ~ & ~ & ~ & ~ & ~ & ~ & ~ & ~ & ~ & ~ & ~ & ~ & ~ & ~ & ~ & ~ & ~ \\ \hline
      PR-LR1 & ~ & ~ & ~ & ~ & ~ & ~ & ~ & ~ & ~ & ~ & ~ & ~ & ~ & ~ & ~ & ~ & X & ~ & ~ & ~ & ~ & ~ & ~ & ~ & ~ & ~ & ~ & ~ & ~ & ~ & ~ & ~ & ~ & ~ & ~ \\ \hline
      PR-LR2 & ~ & ~ & ~ & ~ & ~ & ~ & ~ & ~ & ~ & ~ & ~ & ~ & ~ & ~ & ~ & ~ & ~ & X & ~ & ~ & ~ & ~ & ~ & ~ & ~ & ~ & ~ & ~ & ~ & ~ & ~ & ~ & ~ & ~ & ~ \\ \hline
      OE-EPE1 & ~ & ~ & ~ & ~ & ~ & ~ & ~ & ~ & ~ & ~ & ~ & ~ & ~ & ~ & ~ & ~ & ~ & ~ & X & ~ & ~ & ~ & ~ & ~ & ~ & ~ & ~ & ~ & ~ & ~ & ~ & ~ & ~ & ~ & ~ \\ \hline
      OE-WE1 & ~ & ~ & ~ & ~ & ~ & ~ & ~ & ~ & ~ & ~ & ~ & ~ & ~ & ~ & ~ & ~ & ~ & ~ & ~ & X & ~ & ~ & ~ & ~ & ~ & ~ & ~ & ~ & ~ & ~ & ~ & ~ & ~ & ~ & ~ \\ \hline
      OE-WE2 & ~ & ~ & ~ & ~ & ~ & ~ & ~ & ~ & ~ & ~ & ~ & ~ & ~ & ~ & ~ & ~ & ~ & ~ & ~ & X & ~ & ~ & ~ & ~ & ~ & ~ & ~ & ~ & ~ & ~ & ~ & ~ & ~ & ~ & ~ \\ \hline
      OE-IA1 & ~ & ~ & ~ & ~ & ~ & ~ & ~ & ~ & ~ & ~ & ~ & ~ & ~ & ~ & ~ & ~ & ~ & ~ & ~ & ~ & X & ~ & ~ & ~ & ~ & ~ & ~ & ~ & ~ & ~ & ~ & ~ & ~ & ~ & ~ \\ \hline
      MS-MR1 & ~ & ~ & ~ & ~ & ~ & ~ & ~ & ~ & ~ & ~ & ~ & ~ & ~ & ~ & ~ & ~ & ~ & ~ & ~ & ~ & ~ & X & ~ & ~ & ~ & ~ & ~ & ~ & ~ & ~ & ~ & ~ & ~ & ~ & ~ \\ \hline
      MS-SR1 & ~ & ~ & ~ & ~ & ~ & ~ & ~ & ~ & ~ & ~ & ~ & ~ & ~ & ~ & ~ & ~ & ~ & ~ & ~ & ~ & ~ & X & ~ & ~ & ~ & ~ & ~ & ~ & ~ & ~ & ~ & ~ & ~ & ~ & ~ \\ \hline
      MS-SR2 & ~ & ~ & ~ & ~ & ~ & ~ & ~ & ~ & ~ & ~ & ~ & ~ & ~ & ~ & ~ & ~ & ~ & ~ & ~ & ~ & ~ & ~ & X & ~ & ~ & ~ & ~ & ~ & ~ & ~ & ~ & ~ & ~ & ~ & ~ \\ \hline
      MS-AR1 & ~ & ~ & ~ & ~ & ~ & ~ & ~ & ~ & ~ & ~ & ~ & ~ & ~ & ~ & ~ & ~ & ~ & ~ & ~ & ~ & ~ & ~ & ~ & X & ~ & ~ & ~ & ~ & ~ & ~ & ~ & ~ & ~ & ~ & ~ \\ \hline
      SR-AC1 & ~ & ~ & ~ & ~ & ~ & ~ & ~ & ~ & ~ & ~ & ~ & ~ & ~ & ~ & ~ & ~ & ~ & ~ & ~ & ~ & ~ & ~ & ~ & ~ & ~ & ~ & X & ~ & ~ & ~ & ~ & ~ & ~ & ~ & ~ \\ \hline
      SR-AC2 & ~ & ~ & ~ & ~ & ~ & ~ & ~ & ~ & ~ & ~ & ~ & ~ & ~ & ~ & ~ & ~ & ~ & ~ & ~ & ~ & ~ & ~ & ~ & ~ & ~ & ~ & ~ & X & ~ & ~ & ~ & ~ & ~ & ~ & ~ \\ \hline
      SR-AC3 & ~ & ~ & ~ & ~ & ~ & ~ & ~ & ~ & ~ & ~ & ~ & ~ & ~ & ~ & ~ & ~ & ~ & ~ & ~ & ~ & ~ & ~ & ~ & ~ & ~ & ~ & ~ & ~ & X & ~ & ~ & ~ & ~ & ~ & ~ \\ \hline
      SR-AC4 & ~ & ~ & ~ & ~ & ~ & ~ & ~ & ~ & ~ & ~ & ~ & ~ & ~ & ~ & ~ & ~ & ~ & ~ & ~ & ~ & ~ & ~ & ~ & ~ & ~ & ~ & ~ & ~ & ~ & X & ~ & ~ & ~ & ~ & ~ \\ \hline
      SR-P1 & ~ & ~ & ~ & ~ & ~ & ~ & ~ & ~ & ~ & ~ & ~ & ~ & ~ & ~ & ~ & ~ & ~ & ~ & ~ & ~ & ~ & ~ & ~ & ~ & ~ & ~ & ~ & ~ & ~ & ~ & ~ & ~ & X & ~ & ~ \\ \hline
      SR-P2 & ~ & ~ & ~ & ~ & ~ & ~ & ~ & ~ & ~ & ~ & ~ & ~ & ~ & ~ & ~ & ~ & ~ & ~ & ~ & ~ & ~ & ~ & ~ & ~ & ~ & ~ & ~ & ~ & ~ & ~ & ~ & ~ & ~ & X & ~ \\ \hline
      SR-P3 & ~ & ~ & ~ & ~ & ~ & ~ & ~ & ~ & ~ & ~ & ~ & ~ & ~ & ~ & ~ & ~ & ~ & ~ & ~ & ~ & ~ & ~ & ~ & ~ & ~ & ~ & ~ & ~ & ~ & ~ & ~ & ~ & ~ & ~ & X \\ \hline
      SR-IM1 & ~ & ~ & ~ & ~ & ~ & ~ & ~ & ~ & ~ & ~ & ~ & ~ & ~ & ~ & ~ & ~ & ~ & ~ & ~ & ~ & ~ & ~ & ~ & ~ & ~ & ~ & ~ & ~ & ~ & ~ & X & ~ & ~ & ~ & ~ \\ \hline
      CU-CR1 & ~ & ~ & ~ & ~ & ~ & ~ & ~ & ~ & ~ & ~ & ~ & ~ & ~ & ~ & ~ & ~ & ~ & ~ & ~ & ~ & ~ & ~ & ~ & ~ & X & ~ & ~ & ~ & ~ & ~ & ~ & ~ & ~ & ~ & ~ \\ \hline
      CU-CR2 & ~ & ~ & ~ & ~ & ~ & ~ & ~ & ~ & ~ & ~ & ~ & ~ & ~ & ~ & ~ & ~ & ~ & ~ & ~ & ~ & ~ & ~ & ~ & ~ & ~ & X & ~ & ~ & ~ & ~ & ~ & ~ & ~ & ~ & ~ \\ \hline
      CR-STD1 & ~ & ~ & ~ & ~ & ~ & ~ & ~ & ~ & ~ & ~ & ~ & ~ & ~ & ~ & ~ & ~ & ~ & ~ & ~ & ~ & ~ & ~ & ~ & ~ & ~ & ~ & ~ & ~ & ~ & ~ & ~ & X & ~ & ~ & ~ \\ \hline
  \end{tabular}
  }
\end{table}
\end{landscape}

\newpage
\section{Unit Test Description}
\hspace{2em}\textbf{Section 5, Unit Test Description, will be filled in prior to the revision 0 deliverable. The team decided to leave in the template instructions as a guide to indicate our desired return to the document.}\\

All the modules described in the MIS will be within the scope of unit testing for the following 4 services our application comprises of; Authentication Service, Question Bank Service, Result Storage Service, and Media Processing Service. 

\subsection{Unit Testing Scope}

All modules described in the MIS fall under the scope of unit tests and will be tested in some way, either with individual tests or tests that touch multiple components.

\subsection{Tests for Functional Requirements}

\subsection*{Authentication Service Tests}

\begin{mdframed}[linewidth=0.5mm]
\textbf{Unit Test Name:} Should Register a New Clinician \par
\textbf{Function(s) Tested:} \texttt{clinicianSignup} (in \texttt{clinician.controller.js}) \par
\textbf{Input:} Valid clinician registration data (firstname, lastname, username, email, password) \par
\textbf{Expected Output:} JSON response {"message": "Clinician created", "user": {"username": "drjohndoe"}} \par
\textbf{Relevant Test Case(s):} FR-ST-A6
\end{mdframed}

\begin{mdframed}[linewidth=0.5mm]
\textbf{Unit Test Name:} Should Return Error for Existing Username \par
\textbf{Function(s) Tested:} \texttt{clinicianSignup} (in \texttt{clinician.controller.js}) \par
\textbf{Input:} Clinician registration data with an existing username \par
\textbf{Expected Output:} HTTP error response with status 400 \par
\textbf{Relevant Test Case(s):} FR-ST-A4
\end{mdframed}

\begin{mdframed}[linewidth=0.5mm]
\textbf{Unit Test Name:} Should Login a Clinician with Valid Credentials \par
\textbf{Function(s) Tested:} \texttt{clinicianLogin} (in \texttt{clinician.controller.js}) \par
\textbf{Input:} Valid login credentials (username and password) \par
\textbf{Expected Output:} JSON response indicating successful login (i.e., \texttt{res.json} is called) \par
\textbf{Relevant Test Case(s):} FR-ST-A7
\end{mdframed}

\subsection*{Media Processing Service Tests}

\begin{mdframed}[linewidth=0.5mm]
\textbf{Unit Test Name:} Should Upload and Process Media \par
\textbf{Function(s) Tested:} POST \texttt{/media/upload} endpoint (in media processing controller) \par
\textbf{Input:} Valid video file upload (if file exists) \par
\textbf{Expected Output:} HTTP 200 response with JSON {"message": "Processing started"} \par
\textbf{Relevant Test Case(s):} FR-ST-DSC1
\end{mdframed}

\begin{mdframed}[linewidth=0.5mm]
\textbf{Unit Test Name:} Should Get Processed Media Successfully \par
\textbf{Function(s) Tested:} GET \texttt{/media/processed} endpoint (in media processing controller) \par
\textbf{Input:} No input (simple GET request) \par
\textbf{Expected Output:} HTTP 200 response with JSON containing a valid media URL \par
\textbf{Relevant Test Case(s):} FR-ST-VDA1
\end{mdframed}

\subsection*{Question Service Tests}

\begin{mdframed}[linewidth=0.5mm]
\textbf{Unit Test Name:} Should Return Correct JSON When Retrieving a Test Question Information \par
\textbf{Function(s) Tested:} GET \texttt{/questions/english/matching/1} endpoint (in question controller/route) \par
\textbf{Input:} HTTP GET request to \texttt{/questions/english/matching/1} \par
\textbf{Expected Output:} HTTP 200 response with JSON containing keys: \texttt{id} (Number), \texttt{title} (String), \texttt{sound} (String), \texttt{options} (Array of objects with \texttt{id} and \texttt{image}), and \texttt{correctAnswer} (String, e.g., "b") \par
\textbf{Relevant Test Case(s):} FR-ST-DPD1
\end{mdframed}

\subsection*{Result Service Tests}

\begin{mdframed}[linewidth=0.5mm]
\textbf{Unit Test Name:} Should Return Calculated Result of a Specific Test Result. \par
\textbf{Function(s) Tested:} \texttt{calculateResult} (in \texttt{results.controller.js}) \par
\textbf{Input:} JSON body with {"score": 10, "multiplier": 2} \par
\textbf{Expected Output:} JSON response {"result": 20} \par
\textbf{Relevant Test Case(s):} FR-ST-DPD2
\end{mdframed}

\subsection{Tests for Nonfunctional Requirements}

\subsubsection{Performance and Security Tests}

\begin{mdframed}[linewidth=0.5mm]
\textbf{Unit Test Name:} Should Encrypt Data Before Storing \par
\textbf{Function(s) Tested:} Database insertion function (e.g., \texttt{storeEncryptedData}) \par
\textbf{Input:} Plain text user data \par
\textbf{Expected Output:} Data stored in an encrypted format, verified by retrieval and decryption \par
\textbf{Relevant Test Case(s):} SR-ST-P2
\end{mdframed}

\begin{mdframed}[linewidth=0.5mm]
\textbf{Unit Test Name:} Should Prevent Unauthorized Access to User Data \par
\textbf{Function(s) Tested:} Authentication middleware (e.g., \texttt{verifyToken}) \par
\textbf{Input:} Unauthorized request to access protected resources \par
\textbf{Expected Output:} HTTP 401 response with an error message \par
\textbf{Relevant Test Case(s):} SR-ST-AC3
\end{mdframed}

\newpage

\subsection{Traceability Between Test Cases and Modules}

\begin{table}[!ht]
  \caption{Traceability Table Between System Test Cases and Modules}
  \resizebox{\textwidth}{!}{
  \begin{tabular}{|l|l|l|l|l|l|l|l|l|l|l|l|l|l|l|l|l|l|}
  \hline
      ~ & M1 & M2 & M3 & M4 & M5 & M6 & M7 & M8 & M9 & M10 & M11 & M12 & M13 & M14 & M15 & M16 & M17 \\ \hline
      FR-ST-A1 & X & X & X & ~ & X & ~ & ~ & X & ~ & ~ & ~ & ~ & ~ & ~ & ~ & ~ & ~ \\ \hline
      FR-ST-A2 & X & X & X & ~ & X & ~ & ~ & X & ~ & ~ & ~ & ~ & ~ & ~ & ~ & ~ & ~ \\ \hline
      FR-ST-A3 & ~ & X & X & ~ & X & ~ & ~ & X & ~ & ~ & ~ & ~ & ~ & ~ & ~ & ~ & ~ \\ \hline
      FR-ST-A4 & ~ & X & X & ~ & X & ~ & ~ & X & ~ & ~ & ~ & ~ & ~ & ~ & ~ & ~ & ~ \\ \hline
      FR-ST-A5 & X & ~ & X & ~ & X & ~ & ~ & X & ~ & ~ & ~ & ~ & ~ & ~ & ~ & ~ & ~ \\ \hline
      FR-ST-A6 & X & ~ & X & ~ & X & ~ & ~ & X & ~ & ~ & ~ & ~ & ~ & ~ & ~ & ~ & ~ \\ \hline
      FR-ST-A7 & ~ & ~ & ~ & ~ & X & ~ & ~ & X & ~ & ~ & ~ & ~ & ~ & ~ & ~ & ~ & ~ \\ \hline
      FR-ST-A8 & ~ & ~ & ~ & ~ & X & ~ & ~ & X & ~ & ~ & ~ & ~ & ~ & ~ & ~ & ~ & ~ \\ \hline
      FR-ST-DSC1 & ~ & ~ & ~ & X & ~ & X & ~ & X & ~ & ~ & ~ & ~ & ~ & ~ & ~ & ~ & ~ \\ \hline
      FR-ST-DSC2 & ~ & ~ & ~ & X & ~ & X & X & ~ & ~ & X & ~ & X & X & ~ & ~ & ~ & ~ \\ \hline
      FR-ST-DSC3 & ~ & ~ & ~ & X & X & ~ & ~ & ~ & ~ & ~ & ~ & ~ & ~ & ~ & ~ & ~ & ~ \\ \hline
      FR-ST-DSC4 & ~ & ~ & ~ & X & X & ~ & ~ & ~ & ~ & ~ & ~ & ~ & ~ & ~ & ~ & ~ & ~ \\ \hline
      FR-ST-DSC5 & ~ & ~ & ~ & X & ~ & ~ & ~ & X & ~ & X & X & ~ & ~ & ~ & ~ & ~ & ~ \\ \hline
      FR-ST-VDA1 & ~ & ~ & ~ & X & ~ & ~ & X & ~ & ~ & ~ & ~ & X & X & ~ & ~ & ~ & ~ \\ \hline
      FR-ST-VDA2 & ~ & ~ & ~ & X & ~ & ~ & X & X & ~ & ~ & ~ & X & X & ~ & ~ & ~ & ~ \\ \hline
      FR-ST-VDA3 & ~ & ~ & ~ & X & ~ & ~ & X & X & ~ & ~ & ~ & X & X & ~ & ~ & ~ & ~ \\ \hline
      FR-ST-DPD1 & ~ & ~ & ~ & X & ~ & X & ~ & X & ~ & X & X & ~ & ~ & ~ & ~ & ~ & ~ \\ \hline
      FR-ST-DPD2 & ~ & ~ & ~ & X & ~ & X & ~ & X & ~ & X & X & ~ & ~ & ~ & ~ & ~ & ~ \\ \hline
      FR-ST-DPD3 & X & ~ & ~ & X & ~ & X & ~ & X & ~ & X & X & ~ & ~ & ~ & ~ & ~ & ~ \\ \hline
      FR-ST-DPD4 & X & ~ & ~ & X & ~ & X & ~ & X & ~ & X & X & ~ & ~ & ~ & ~ & ~ & ~ \\ \hline
      FR-ST-SS1 & ~ & X & X & ~ & ~ & ~ & ~ & ~ & ~ & ~ & ~ & ~ & ~ & ~ & ~ & ~ & ~ \\ \hline
      FR-ST-SS2 & ~ & X & X & ~ & ~ & ~ & ~ & ~ & ~ & ~ & ~ & ~ & ~ & ~ & ~ & ~ & ~ \\ \hline
      FR-ST-SS3 & ~ & X & X & ~ & ~ & ~ & ~ & ~ & ~ & ~ & ~ & ~ & ~ & ~ & ~ & ~ & ~ \\ \hline
      FR-ST-SS4 & ~ & X & X & ~ & ~ & ~ & ~ & ~ & ~ & ~ & ~ & ~ & ~ & ~ & ~ & ~ & ~ \\ \hline
      FR-ST-SS5 & ~ & X & X & ~ & ~ & ~ & ~ & ~ & X & ~ & ~ & ~ & ~ & ~ & ~ & ~ & ~ \\ \hline
      FR-ST-AI1 & ~ & X & X & X & ~ & ~ & X & ~ & X & ~ & ~ & X & X & X & X & X & ~ \\ \hline
      FR-ST-AI2 & ~ & X & X & X & ~ & ~ & X & ~ & X & ~ & ~ & X & X & X & X & X & ~ \\ \hline
      FR-ST-AI3 & ~ & X & X & X & ~ & ~ & X & ~ & X & ~ & ~ & X & X & X & X & X & ~ \\ \hline
      FR-ST-AI4 & ~ & X & X & X & ~ & ~ & X & ~ & X & ~ & ~ & X & X & X & X & X & ~ \\ \hline
      FR-ST-AI5 & ~ & X & X & X & ~ & X & X & ~ & X & ~ & ~ & X & X & X & X & X & ~ \\ \hline
      FR-ST-AI6 & ~ & X & X & X & ~ & ~ & X & ~ & X & ~ & ~ & X & X & X & X & X & ~ \\ \hline
      FR-ST-AI7 & ~ & X & X & X & ~ & ~ & X & ~ & X & ~ & ~ & X & X & X & X & X & X \\ \hline
  \end{tabular}
  }
\end{table}

% \bibliographystyle{plainnat}

\newpage
% \bibliography{../../refs/References}

\newpage

\section{Appendix}

Below is additional information relevant to the document.

\subsection{Symbolic Parameters}

The definition of the test cases will call for SYMBOLIC\_CONSTANTS.
Their values are defined in this section for easy maintenance.

\begin{table}[h!]
  \centering
  \begin{tabular}{|c|c|}
      \hline
      \textbf{Variable Name} & \textbf{Value} \\
      \hline
      MAX\_PROCESSING\_TIME & 10 seconds \\
      \hline
      SHORT\_PROCESSING\_TIME & 0.5 seconds \\
      \hline
      MAX\_SUCCESS\_RATE & 100\% \\
      \hline
      VERY\_HIGH\_SUCCESS\_RATE & 95\% \\
      \hline
      HIGH\_SUCCESS\_RATE & 90\% \\
      \hline
      NUM\_CODE\_LINES & 10 \\
      \hline
      MIN\_SCREEN\_SIZE & 4 inches \\
      \hline
      MAX\_SCREEN\_SIZE & 27 inches \\
      \hline
      MAX\_STORAGE\_TIME & 7 years \\
      \hline
      MAX\_LOAD\_TIME & 3 seconds \\
      \hline
      AVERAGE\_RESOLUTION & 720p \\
      \hline
      MONTHLY\_BACKUP & 4 hour \\
      \hline
      MIN\_USERS & 2000 users \\
      \hline
      MIN\_STORAGE & 10TB \\
      \hline
      YEARLY\_INCREASE\_PERCENTAGE & 10\% \\
      \hline
      LOW\_FAILURE\_RATE & 1\% \\
      \hline
      MAX\_CLICKS & 5 \\
      \hline
  \end{tabular}
  \caption{Variable Names and Values}
  \label{tab:variables}
\end{table}

\subsection{Usability Survey Questions}
The following questions depict the first draft of the team's usability.\\
The usability survey will be conducted after participants have engaged with testing and using the system for at least one iteration of the assessment.\\

\newcommand{\likertScale}{
    \begin{center}
        Strongly Disagree \hfill Disagree \hfill Neutral \hfill Agree \hfill Strongly Agree
    \end{center}
}

\newcommand{\insertAnswerHere}{
  \begin{tcolorbox}[width=0.5\textwidth,
    colframe=black,
    colback=white,
    boxrule=0.1mm,
    sharp corners]
  \textit{Insert answer here...}
  \end{tcolorbox}
}

\textbf{Please select the statement that best describes your experience for each of the following:}
\begin{enumerate}
  \item Learning how to use the system was easy:\likertScale
  \item Setting up the system was easy:\likertScale
  \item I found the assessment interface easy to use:\likertScale
  \item Navigating the interface was easy:\likertScale
  \item All the button interactions reacted and responded how I thought they should:\likertScale
  \item The information on screen was easy to read and understand:\likertScale
  \item I like the organization of the assessment interface:\likertScale
  \item I enjoyed my overall experience using the TeleHealth Insights platform:\likertScale
\end{enumerate}
\textbf{Answer the following:}

\begin{enumerate}
    \setcounter{enumi}{8}
    \item What was the most difficult part of using the platform? \insertAnswerHere
    \item Did you encounter any bugs/problems while using the platform? If so, what were they? \insertAnswerHere
    \item What was your favourite part of the experience? Why? \insertAnswerHere
    \item What was your least favourite part of the experience? Why? \insertAnswerHere
    \item Were there any aspects of the platform that you found unnecessary? Why? \insertAnswerHere
    \item Which part of the platform needs the most improvement? Why? \insertAnswerHere
    \item What would you like changed to make the platform easier to use? \insertAnswerHere
    \item What is the device you ran the system on? \insertAnswerHere
    \item Did the screen's visuals scale appropriately to the screen size? \insertAnswerHere
    \item Do you have any additional feedback? \insertAnswerHere
\end{enumerate}

\newpage{}
\section*{Appendix --- Reflection}


The information in this section will be used to evaluate the team members on the
graduate attribute of Lifelong Learning.

\input{../Reflection.tex}

\begin{enumerate}
  \item What went well while writing this deliverable?\\
  \newline
  \hspace{2em} Parisha: The work of the VnV plan was divided very well, 
  with team members working on testing of non functional and functional requirements that match
  the ones they worked on for the SRS document. This made it a lot more efficient and easier to understand
  what to write for the testing requirements. \\\\
  \hspace{2em} Mitchell: One of the things that went well during this deliverable was splitting up
  the System Tests for both Functional Requirements and Nonfunctional Requirements. The team decided
  to assign the functional and nonfunctional tests to the team members that wrote those particular
  requirements in the SRS. This allowed the team to use their prior knowledge on the requirements
  to develop detailed test plans to accurately test the requirements. This also meant that team members
  that were familiar with the requirements knew the limitations of the tests, and could improve upon
  given feedback to strengthen the plan further.\\\\
  \hspace{2em} Jasmine: I think the team worked well together as usual, especially when it came to 
  distributing tasks. Our team was very efficient and decided to split test cases for the functional and 
  nonfunctional requirements in the same way we split up writing the requirements for the SRS, and the 
  rest of the document was split up fairly quickly within a team meeting. Another thing that went well 
  while writing this deliverable is organization. Now that we have written several pieces of documentation, 
  the team is comfortable navigating LaTex formatting, as well as GitHub project and issue tracking.\\\\
  \hspace{2em} Promish: One thing that went well during this deliverable was how 
  comfortable everyone became using GitHub. We created issues for each task, and 
  handling merge conflicts was faster. As a team, we quickly divided the work, 
  understanding that writing system tests for the requirements each of us specialized in 
  was the most efficient approach. We also recognized the importance of consistent naming 
  for our tests to maintain traceability with our SRS, so we decided to include "ST" in 
  the middle of our SRS requirements tags.\\\\
  \item What pain points did you experience during this deliverable, and how
    did you resolve them?\\
  \newline
  \hspace{2em} Parisha: A challenge I found was due to time constraints and responsibilities
   in other classes it become very hard to coordinate meetings with all members of the team. 
   In order to optimize our time, team members communicated progress consistently and we worked together
    in the end to merge branches ( a lot of merge conflicts due to structure) and then reviewed the documentation
    against rubric and added reflection \\\\
  \hspace{2em} Mitchell: One of the pain points I experienced during this deliverable was understanding the
  format needed for the System Tests section. I found it difficult to get started because of the formality
  of the tests. As well, I wanted to make sure that the tests were consistent among team members, and a difference
  in formatting would be difficult for someone reading the document to understand. To resolve this, the team decided
  to follow a consistent formatting outline, and follow along a sample so that everyone knew what test cases
  should look like for this deliverable.\\\\
  \hspace{2em} Jasmine: One pain point during this deliverable was clarification of document instructions, 
  such as what area testing is or what exactly the symbolic constraints were. We resolved this by preparing 
  questions to ask the TA during our informal TA meeting, and looking at examples from other capstone projects 
  from the same course completed in previous years.\\\\
  \hspace{2em} Promish: A challenge we faced during this deliverable was the 
  large number of tests we had to write within a limited timeframe. With only a week 
  to work on the document, we didn’t have time for extra meetings with our advisor and 
  TA to clarify certain sections as much as I would have liked. Overall, the team did a 
  great job with the time and resources available, but some parts of the document were a 
  bit unclear, especially regarding formatting and focus. The rubric was also quite 
  general, making it harder to understand the exact requirements.\\\\
  \item What knowledge and skills will the team collectively need to acquire to
  successfully complete the verification and validation of your project?
  Examples of possible knowledge and skills include dynamic testing knowledge,
  static testing knowledge, specific tool usage, Valgrind etc.  You should look to
  identify at least one item for each team member.\\
  \newline
  \hspace{2em} Parisha: One piece of knowledge and skills the team should acquire for the project is a good understanding 
  of all the different types of testing like static, dynamic, manual, automatic and functional testing. Understanding the difference
  between these types and what cases are the best to use them are important to design correct and efficient tests for the system.\\\\
  \hspace{2em} Mitchell: One of the pieces of knowledge and skills the team collectively needs to acquire to 
  successfully complete the verification and validation of the project is understanding how to properly perform
  usability testing. This will be important to the team because it is one of the project's chosen extras, so understanding
  usability testing is crucial for the success of the project. As well, it will inform us on how users actually engage,
  interact, and understand our system. Another skill the team will need to acquire is dynamic testing using different frameworks.
  It will be beneficial to perform dynamic testing to get verify each of our desired outcomes of tests against the actual outcomes.
  This will provide the team with accurate, reliable results and conclusions.\\\\
  \hspace{2em} Jasmine: The team will need skills in both dynamic and static testing to successfully complete the 
  verification and validation of the project, including knowledge of effective test case creation and debugging. 
  Being familiar with tools like automated test frameworks and static analysis tools that our team can use for our 
  project will be important to improving efficiency and accuracy in our testing processes. Additionally, understanding 
  quality assurance practices and common code review techniques will help make sure our validation is thorough and reliable. \\\\
  \hspace{2em} Promish: As a team, we’ll need to focus on learning to use the frameworks outlined 
  in section 3.6 to automate our unit tests. We’ll also need to understand how to evaluate our 
  SRS and the processes for making changes. Lastly, we’ll need to know where testing fits 
  into continuous development and how to follow Agile practices effectively.\\\\
  \item For each of the knowledge areas and skills identified in the previous
  question, what are at least two approaches to acquiring the knowledge or
  mastering the skill?  Of the identified approaches, which will each team
  member pursue, and why did they make this choice?\\
  \newline
  \hspace{2em} Parisha: One approach to acquire knowledge specially in manual vs automatic testing is using course material from our Software Testing course SFWRENG 3S03
  to get a good summary of the different between the two and find some small examples/ practice problems to go through from tutorial to refresh our memory/ 
  Another good approach would be to use testing modules online (such as Geeks4Geeks) to understand the different types of testing with good examples.
  I will be using approach 2 to acquire knowledge as I learn best with a combination of reading, watching and doing. \\\\
  \hspace{2em} Mitchell: One approach to acquire knowledge for usability testing is referring to SFWRENG 4HC3 course notes,
  as usability testing is one of the major topics the course focuses on, with lots of examples and details. To obtain knowledge for dynamic tests, I plan on learning pytest to learn about dynamic testing in Python, which will be the
  team's language of choice for developing machine learning models.\\\\
  \hspace{2em} Jasmine: I will choose to focus on static testing knowledge since it is something I usually overlook 
  and would like to improve on. To acquire knowledge in static testing, one approach would be to use online static 
  testing tutorials and resources, especially platforms with specialized online courses such as Coursera or Udacity. 
  Another approach would be to use an open-source static analysis tool and apply it to previous projects or example 
  codebases to get hands on experience identifying common code issues, warnings, and even security vulnerabilities.\\\\
  \hspace{2em} Promish: One of the best ways to learn a framework is to read its 
  documentation or watch YouTube tutorials. After that, testing it out in a separate IDE 
  with a small example would be helpful for something like Pytest. For a better 
  understanding of Agile development, we could look into resources that explain Agile 
  principles and methods. Additionally, consulting our TA, supervisor, and professor 
  would be valuable for understanding how our team can function effectively.\\\\

\end{enumerate}

\end{document}