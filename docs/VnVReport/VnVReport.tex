\documentclass[12pt, titlepage]{article}

\usepackage{booktabs}
\usepackage{tabularx}
\usepackage{hyperref}
\hypersetup{
    colorlinks,
    citecolor=black,
    filecolor=black,
    linkcolor=red,
    urlcolor=blue
}
\usepackage[round]{natbib}
\usepackage{mdframed}
\usepackage{enumitem}
\usepackage{parskip}

%% Comments

\usepackage{color}

\newif\ifcomments\commentstrue %displays comments
%\newif\ifcomments\commentsfalse %so that comments do not display

\ifcomments
\newcommand{\authornote}[3]{\textcolor{#1}{[#3 ---#2]}}
\newcommand{\todo}[1]{\textcolor{red}{[TODO: #1]}}
\else
\newcommand{\authornote}[3]{}
\newcommand{\todo}[1]{}
\fi

\newcommand{\wss}[1]{\authornote{blue}{SS}{#1}} 
\newcommand{\plt}[1]{\authornote{magenta}{TPLT}{#1}} %For explanation of the template
\newcommand{\an}[1]{\authornote{cyan}{Author}{#1}}

%% Common Parts

\newcommand{\progname}{Software Engineering} % PUT YOUR PROGRAM NAME HERE
\newcommand{\authname}{Team \#22, TeleHealth Insights
\\ Mitchell Weingust
\\ Parisha Nizam
\\ Promish Kandel
\\ Jasmine Sun-Hu} % AUTHOR NAMES                  

\usepackage{hyperref}
    \hypersetup{colorlinks=true, linkcolor=blue, citecolor=blue, filecolor=blue,
                urlcolor=blue, unicode=false}
    \urlstyle{same}
                                


\begin{document}

\title{Verification and Validation Report: \progname} 
\author{\authname}
\date{\today}
	
\maketitle

\pagenumbering{roman}

\section{Revision History}

\begin{tabularx}{\textwidth}{p{3cm}p{2cm}X}
\toprule {\bf Date} & {\bf Version} & {\bf Notes}\\
\midrule
Date 1 & 1.0 & Notes\\
Date 2 & 1.1 & Notes\\
\bottomrule
\end{tabularx}

~\newpage

\section{Symbols, Abbreviations and Acronyms}

\renewcommand{\arraystretch}{1.2}
\begin{tabular}{l l} 
  \toprule		
  \textbf{symbol} & \textbf{description}\\
  \midrule 
  T & Test\\
  \bottomrule
\end{tabular}\\

\wss{symbols, abbreviations or acronyms -- you can reference the SRS tables if needed}

\newpage

\tableofcontents

\listoftables %if appropriate

\listoffigures %if appropriate

\newpage

\pagenumbering{arabic}

\hspace{2em}This document contains the team's verification and validation report for the TeleHealth
Insights project. This document features functional requirements evaluation, nonfunctional requirements
evaluation, unit testing, changes due to testing, automated testing, trace to requirements, trace to modules,
and code coverage metrics.

\section{Functional Requirements Evaluation}
\hspace{2em}The following section covers all the functional requirements tests specified in the project's
VnV Plan document. The coverage can be traced in Table X.

\subsection{Authentication}
\hspace{2em}The test cases below focus on ensuring users can safely and securely login, create and
access their accounts without worrying about others accessing their information.

\begin{mdframed}[linewidth=0.5mm] \par
  \textbf{Test Case Identifier:} FR-ST-A1 \par
  \textbf{Input:} Selection of Parent account role for login \par
  \textbf{Expected Output:} The expected result is the Parent account role is selected and User is brought to the Parent login screen \par
  \textbf{Actual Output:} \par
  \textbf{Expected and Actual Output Match:} True \par
  \textbf{Relevant Functional Requirement(s):} FR-A1
\end{mdframed}

\subsection{Data Collection and Storage}
\hspace{2em}The test cases below foucs on ensuring data is collected and stored correctly. We test to make sure
no identifable information is stored in the database and we also check that all multimedia data is linked correctly to user assignment.

\begin{mdframed}[linewidth=0.5mm] \par
  \textbf{Test Case Identifier:} FR-ST-DSC1 \par
  \textbf{Input:} Insertion of multimedia files into the database \par
  \textbf{Expected Output:} A success message in the console for both storing and retrieving the data; the retrieved files are uncorrupted and match the original files \par
  \textbf{Actual Output:} A success message in the console and a link to multimedia file \par
  \textbf{Expected and Actual Output Match:} True \par
  \textbf{Relevant Functional Requirement(s):} FR-DSC1
\end{mdframed}

\begin{mdframed}[linewidth=0.5mm] \par
  \textbf{Test Case Identifier:} FR-ST-DSC2 \par
  \textbf{Input:} Insertion of a test assessment session with video, audio files, flagged occurrences, and timestamps for each assessment question \par
  \textbf{Expected Output:} Creation of a JSON file containing the flagged occurrences and timestamps stored alongside the session data \par
  \textbf{Actual Output:} A JSON file was created in AWS with the correct expected output \par
  \textbf{Expected and Actual Output Match:} True \par
  \textbf{Relevant Functional Requirement(s):} FR-DSC2
\end{mdframed}

\begin{mdframed}[linewidth=0.5mm] \par
  \textbf{Test Case Identifier:} FR-ST-DSC3 \par
  \textbf{Input:} Attempted insertion of a record containing personally identifiable information (e.g. address) \par
  \textbf{Expected Output:} The consol throws an error as no such field exists for persoanl information \par
  \textbf{Actual Output:} The database throws an invalid payload error\par
  \textbf{Expected and Actual Output Match:} True \par
  \textbf{Relevant Functional Requirement(s):} FR-DSC3
\end{mdframed}

\begin{mdframed}[linewidth=0.5mm] \par
  \textbf{Test Case Identifier:} FR-ST-DSC4 \par
  \textbf{Input:} Insertion of multiple sessions, each tagged with a unique user identifier \par
  \textbf{Expected Output:} All session data is stored and correctly grouped under their respective unique user identifiers \par
  \textbf{Actual Output:} The database creates folders based on the unique identifers \par
  \textbf{Expected and Actual Output Match:} True \par
  \textbf{Relevant Functional Requirement(s):} FR-DSC4
\end{mdframed}

\begin{mdframed}[linewidth=0.5mm] \par
  \textbf{Test Case Identifier:} FR-ST-DSC5 \par
  \textbf{Input:} Insertion of an assessment report linked to a patient's unique identifier \par
  \textbf{Expected Output:} The report is successfully stored, linked to the corresponding patient identifier\par
  \textbf{Actual Output:} The assement is put into the correct folder and is added to the JSON that links multimedia to assignment \par
  \textbf{Expected and Actual Output Match:} True \par
  \textbf{Relevant Functional Requirement(s):} FR-DSC5
\end{mdframed}


\subsection{Video and Audio Data Analysis}
\hspace{2em}The test cases below ensure that both video and audio data is correctly accessed, processed and stored 
in its respective user folder with no errors.

\begin{mdframed}[linewidth=0.5mm] \par
  \textbf{Test Case Identifier:} FR-ST-VDA1 \par
  \textbf{Input:} Request by the analysis model to access video and audio data from a completed session \par
  \textbf{Expected Output:} All requested videos and audio files are processed successfully with a corresponding success message logged \par
  \textbf{Actual Output:} A sucess message in the console after video and audio are finished processing \par
  \textbf{Expected and Actual Output Match:} True \par
  \textbf{Relevant Functional Requirement(s):} FR-ST-VDA1
\end{mdframed}

\begin{mdframed}[linewidth=0.5mm] \par
  \textbf{Test Case Identifier:} FR-ST-VDA2, FR-ST-VDA3 \par
  \textbf{Input:} Video and audio data containing speech disturbances, interruptions, and other irregularities for analysis \par
  \textbf{Expected Output:} A JSON file is generated that records the number of disturbances \par
  \textbf{Actual Output:} A JSON file is created in the correct user folder with a link to the video and contains bias timestamps\par
  \textbf{Expected and Actual Output Match:} True \par
  \textbf{Relevant Functional Requirement(s):} FR-ST-VDA2, FR-ST-VDA3
\end{mdframed}

\subsection{Data Processing and Display}
\hspace{2em}The test cases below 

\begin{mdframed}[linewidth=0.5mm] \par
  \textbf{Test Case Identifier:} FR-ST-A1 \par
  \textbf{Input:} Selection of Parent account role for login \par
  \textbf{Expected Output:} The expected result is the Parent account role is selected and User is brought to the Parent login screen \par
  \textbf{Actual Output:} \par
  \textbf{Expected and Actual Output Match:} True \par
  \textbf{Relevant Functional Requirement(s):} FR-A1
\end{mdframed}

\subsection{System Set Up}
\hspace{2em}The test cases below 

\begin{mdframed}[linewidth=0.5mm] \par
  \textbf{Test Case Identifier:} FR-ST-A1 \par
  \textbf{Input:} Selection of Parent account role for login \par
  \textbf{Expected Output:} The expected result is the Parent account role is selected and User is brought to the Parent login screen \par
  \textbf{Actual Output:} \par
  \textbf{Expected and Actual Output Match:} True \par
  \textbf{Relevant Functional Requirement(s):} FR-A1
\end{mdframed}

\subsection{Assessment Interface}
\hspace{2em}The test cases below 

\begin{mdframed}[linewidth=0.5mm] \par
  \textbf{Test Case Identifier:} FR-ST-A1 \par
  \textbf{Input:} Selection of Parent account role for login \par
  \textbf{Expected Output:} The expected result is the Parent account role is selected and User is brought to the Parent login screen \par
  \textbf{Actual Output:} \par
  \textbf{Expected and Actual Output Match:} True \par
  \textbf{Relevant Functional Requirement(s):} FR-A1
\end{mdframed}

\section{Nonfunctional Requirements Evaluation}
\hspace{2em}The following section covers all the nonfunctional requirements specified in the project’s
VnV Plan document. The coverage can be traced in Table X.

\subsection{Look and Feel Requirements}
\hspace{2em}The test cases below 

\begin{mdframed}[linewidth=0.5mm] \par
  \textbf{Test Case Identifier:} FR-ST-A1 \par
  \textbf{Input:} Selection of Parent account role for login \par
  \textbf{Expected Output:} The expected result is the Parent account role is selected and User is brought to the Parent login screen \par
  \textbf{Actual Output:} \par
  \textbf{Expected and Actual Output Match:} True \par
  \textbf{Relevant Nonfunctional Requirement(s):} FR-A1
\end{mdframed}
		
\subsection{Usability and Humanity}
\hspace{2em}The test cases below 

\begin{mdframed}[linewidth=0.5mm] \par
  \textbf{Test Case Identifier:} FR-ST-A1 \par
  \textbf{Input:} Selection of Parent account role for login \par
  \textbf{Expected Output:} The expected result is the Parent account role is selected and User is brought to the Parent login screen \par
  \textbf{Actual Output:} \par
  \textbf{Expected and Actual Output Match:} True \par
  \textbf{Relevant Nonfunctional Requirement(s):} FR-A1
\end{mdframed}

\subsection{Performance}
\hspace{2em}The test cases outlined below ensures proper performance and stability of our system and database.
\begin{mdframed}[linewidth=0.5mm]
  \textbf{Test Case Identifier:} PR-ST-SL1 \par
  \textbf{Input/Condition:} User navigates through various web pages. \par
  \textbf{Expected Output/Results:} All web pages load completely with all functionalities within MAX\_LOAD\_TIME. \par
  \textbf{Actual Output/Results:} All web pages load with correct data within MAX\_LOAD\_TIME. \par
  \textbf{Expected and Actual Output Match:} True \par
  \textbf{Relevant Functional Requirement(s):} PR-ST-SL1
\end{mdframed}

\begin{mdframed}[linewidth=0.5mm]
  \textbf{Test Case Identifier:} PR-ST-SL2 \par
  \textbf{Input/Condition:} A session is recorded during which two faces appear and a keyword is said. \par
  \textbf{Expected Output/Results:} The latency between video and recorded playback remains below SHORT\_PROCESSING\_TIME. \par
  \textbf{Actual Output/Results:} The latency is within the \\SHORT\_PROCESSING\_TIME when reviewing on clinician side \par
  \textbf{Expected and Actual Output Match:} True \par
  \textbf{Relevant Functional Requirement(s):} PR-ST-SL2
\end{mdframed}

\begin{mdframed}[linewidth=0.5mm]
  \textbf{Test Case Identifier:} PR-ST-SL3 \par
  \textbf{Input/Condition:} A video recorded during an assessment session is stored and later retrieved. \par
  \textbf{Expected Output/Results:} The retrieved video meets or exceeds AVERAGE\_RESOLUTION. \par
  \textbf{Actual Output/Results:} Video is AVERAGE\_RESOLUTION\par
  \textbf{Expected and Actual Output Match:} True \par
  \textbf{Relevant Functional Requirement(s):} PR-ST-SL3
\end{mdframed}

\begin{mdframed}[linewidth=0.5mm]
  \textbf{Test Case Identifier:} PR-ST-PA1, PR-ST-PA3 \par
  \textbf{Input/Condition:} Analysis model loaded with sample audio and video data containing known speech disturbances and multiple faces. \par
  \textbf{Expected Output/Results:} The model detects speech and multiple faces with an accuracy of VERY\_HIGH\_SUCCESS\_RATE. \par
  \textbf{Actual Output/Results:} The model detects multiple faces with \\VERY\_HIGH\_SUCCESS\_RATE but not speeches \par
  \textbf{Expected and Actual Output Match:} False \par
  \textbf{Relevant Functional Requirement(s):} PR-ST-PA1, PR-ST-PA3
\end{mdframed}

\begin{mdframed}[linewidth=0.5mm]
  \textbf{Test Case Identifier:} PR-ST-RFT1 \par
  \textbf{Input/Condition:} Simulate a common user errors (e.g., invalid inputs). \par
  \textbf{Expected Output/Results:} The system displays clear error messages for at least VERY\_HIGH\_SUCCESS\_RATE of the errors encountered. \par
  \textbf{Actual Output/Results:} System gives correct feedback to user with a VERY\_HIGH\_SUCCESS\_RATE \par
  \textbf{Expected and Actual Output Match:} True \par
  \textbf{Relevant Functional Requirement(s):} PR-ST-RFT1
\end{mdframed}

\begin{mdframed}[linewidth=0.5mm]
  \textbf{Test Case Identifier:} PR-ST-CR2 \par
  \textbf{Input/Condition:} Data stored in the database approaches the annual MIN\_STORAGE threshold. \par
  \textbf{Expected Output/Results:} The system accommodates the data volume without performance degradation. \par
  \textbf{Actual Output/Results:} The system accommodates the \\MIN\_STORAGE threshold with room to increase data storage\par
  \textbf{Expected and Actual Output Match:} True \par
  \textbf{Relevant Functional Requirement(s):} PR-ST-CR2
\end{mdframed}

\begin{mdframed}[linewidth=0.5mm]
  \textbf{Test Case Identifier:} PR-ST-LR1 \par
  \textbf{Input/Condition:} Monitor system stability over successive updates on the release build. \par
  \textbf{Expected Output/Results:} The system’s failure rate remains below LOW\_FAILURE\_RATE during updates. \par
  \textbf{Actual Output/Results:} system failur rate remains below \\LOW\_FAILURE\_RATE during deployment of versions \par
  \textbf{Expected and Actual Output Match:} True \par
  \textbf{Relevant Functional Requirement(s):} PR-ST-LR1
\end{mdframed}

\begin{mdframed}[linewidth=0.5mm]
  \textbf{Test Case Identifier:} PR-ST-LR2 \par
  \textbf{Input/Condition:} The system is run on multiple operating systems (Windows, macOS). \par
  \textbf{Expected Output/Results:} The system functions correctly on all tested platforms without issues. \par
  \textbf{Actual Output/Results:} The system functions correctly on multiple operating systems\par
  \textbf{Expected and Actual Output Match:} True \par
  \textbf{Relevant Functional Requirement(s):} PR-ST-LR2
\end{mdframed}

\subsection{Operational and Environmental}
\hspace{2em}The test cases below 

\begin{mdframed}[linewidth=0.5mm] \par
  \textbf{Test Case Identifier:} FR-ST-A1 \par
  \textbf{Input:} Selection of Parent account role for login \par
  \textbf{Expected Output:} The expected result is the Parent account role is selected and User is brought to the Parent login screen \par
  \textbf{Actual Output:} \par
  \textbf{Expected and Actual Output Match:} True \par
  \textbf{Relevant Nonfunctional Requirement(s):} FR-A1
\end{mdframed}

\subsection{Maintainability and Support}
\hspace{2em}The test cases below 

\begin{mdframed}[linewidth=0.5mm] \par
  \textbf{Test Case Identifier:} FR-ST-A1 \par
  \textbf{Input:} Selection of Parent account role for login \par
  \textbf{Expected Output:} The expected result is the Parent account role is selected and User is brought to the Parent login screen \par
  \textbf{Actual Output:} \par
  \textbf{Expected and Actual Output Match:} True \par
  \textbf{Relevant Nonfunctional Requirement(s):} FR-A1
\end{mdframed}

\subsection{Cultural}
\hspace{2em}The test cases below 

\begin{mdframed}[linewidth=0.5mm] \par
  \textbf{Test Case Identifier:} FR-ST-A1 \par
  \textbf{Input:} Selection of Parent account role for login \par
  \textbf{Expected Output:} The expected result is the Parent account role is selected and User is brought to the Parent login screen \par
  \textbf{Actual Output:} \par
  \textbf{Expected and Actual Output Match:} True \par
  \textbf{Relevant Nonfunctional Requirement(s):} FR-A1
\end{mdframed}

\subsection{Security}
\hspace{2em}The test cases below 

\begin{mdframed}[linewidth=0.5mm] \par
  \textbf{Test Case Identifier:} FR-ST-A1 \par
  \textbf{Input:} Selection of Parent account role for login \par
  \textbf{Expected Output:} The expected result is the Parent account role is selected and User is brought to the Parent login screen \par
  \textbf{Actual Output:} \par
  \textbf{Expected and Actual Output Match:} True \par
  \textbf{Relevant Nonfunctional Requirement(s):} FR-A1
\end{mdframed}

\subsection{Compliance}
\hspace{2em}The test cases below 

\begin{mdframed}[linewidth=0.5mm] \par
  \textbf{Test Case Identifier:} FR-ST-A1 \par
  \textbf{Input:} Selection of Parent account role for login \par
  \textbf{Expected Output:} The expected result is the Parent account role is selected and User is brought to the Parent login screen \par
  \textbf{Actual Output:} \par
  \textbf{Expected and Actual Output Match:} True \par
  \textbf{Relevant Nonfunctional Requirement(s):} FR-A1
\end{mdframed}
	
\section{Comparison to Existing Implementation}	

As this project does not have existing implementations, this section is not appropriate for the TeleHealth Insights project.

\section{Unit Testing}

\section{Changes Due to Testing}

\wss{This section should highlight how feedback from the users and from 
the supervisor (when one exists) shaped the final product.  In particular 
the feedback from the Rev 0 demo to the supervisor (or to potential users) 
should be highlighted.}

\section{Automated Testing}

\subsection{Linters}
To maintain a good coding standard, we integrated linters into
our development workflow. For JavaScript files, we rely on Prettier to
automatically format code, ensuring consistent indentation and spacing.
By running Prettier as part of our pre-commit checks, any formatting concerns
are addressed before merging into our main repository, which helps minimize merge
conflicts and maintain a clean codebase.

\subsection{Unit Testing}
We use Jest as our primary JavaScript testing framework to automatically verify critical parts 
of our code before changes are merged into the main branch. This approach helps us catch issues 
early, maintain code quality, and keep the overall system stable.

\subsection{Continuous Integration}
We used continuous integration (CI) pipeline to automate test execution
and provide immediate feedback whenever new code is committed. We configure GitHub
Actions trigger to run our Jest unit tests, linters and document tests on each pull request 
or direct push to main, ensuring that only code meeting quality standards is always met.


		
\section{Trace to Requirements}
		
\section{Trace to Modules}		

\section{Code Coverage Metrics}

\bibliographystyle{plainnat}
\bibliography{../../refs/References}

\newpage{}
\section*{Appendix --- Reflection}

The information in this section will be used to evaluate the team members on the
graduate attribute of Reflection.

The purpose of reflection questions is to give you a chance to assess your own
learning and that of your group as a whole, and to find ways to improve in the
future. Reflection is an important part of the learning process.  Reflection is
also an essential component of a successful software development process.  

Reflections are most interesting and useful when they're honest, even if the
stories they tell are imperfect. You will be marked based on your depth of
thought and analysis, and not based on the content of the reflections
themselves. Thus, for full marks we encourage you to answer openly and honestly
and to avoid simply writing ``what you think the evaluator wants to hear.''

Please answer the following questions.  Some questions can be answered on the
team level, but where appropriate, each team member should write their own
response:


\begin{enumerate}
  \item What went well while writing this deliverable? 
  \item What pain points did you experience during this deliverable, and how
    did you resolve them?
  \item Which parts of this document stemmed from speaking to your client(s) or
  a proxy (e.g. your peers)? Which ones were not, and why?
  \item In what ways was the Verification and Validation (VnV) Plan different
  from the activities that were actually conducted for VnV?  If there were
  differences, what changes required the modification in the plan?  Why did
  these changes occur?  Would you be able to anticipate these changes in future
  projects?  If there weren't any differences, how was your team able to clearly
  predict a feasible amount of effort and the right tasks needed to build the
  evidence that demonstrates the required quality?  (It is expected that most
  teams will have had to deviate from their original VnV Plan.)
\end{enumerate}

\end{document}