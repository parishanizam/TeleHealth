\documentclass[12pt, titlepage]{article}

\usepackage{booktabs}
\usepackage{tabularx}
\usepackage{hyperref}
\hypersetup{
    colorlinks,
    citecolor=black,
    filecolor=black,
    linkcolor=red,
    urlcolor=blue
}
\usepackage[round]{natbib}
\usepackage{mdframed}
\usepackage{enumitem}
\usepackage{parskip}

\input{../Comments}
\input{../Common}

\begin{document}

\title{Verification and Validation Report: \progname} 
\author{\authname}
\date{\today}
	
\maketitle

\pagenumbering{roman}

\section{Revision History}

\begin{tabularx}{\textwidth}{p{3cm}p{2cm}X}
\toprule {\bf Date} & {\bf Version} & {\bf Notes}\\
\midrule
Date 1 & 1.0 & Notes\\
Date 2 & 1.1 & Notes\\
\bottomrule
\end{tabularx}

~\newpage

\section{Symbols, Abbreviations and Acronyms}

\renewcommand{\arraystretch}{1.2}
\begin{tabular}{l l} 
  \toprule		
  \textbf{symbol} & \textbf{description}\\
  \midrule 
  T & Test\\
  \bottomrule
\end{tabular}\\

\wss{symbols, abbreviations or acronyms -- you can reference the SRS tables if needed}

\newpage

\tableofcontents

\listoftables %if appropriate

\listoffigures %if appropriate

\newpage

\pagenumbering{arabic}

\hspace{2em}This document contains the team's verification and validation report for the TeleHealth
Insights project. This document features functional requirements evaluation, nonfunctional requirements
evaluation, unit testing, changes due to testing, automated testing, trace to requirements, trace to modules,
and code coverage metrics.

\section{Functional Requirements Evaluation}
\hspace{2em}The following section covers all the functional requirements tests specified in the project's
VnV Plan document. The coverage can be traced in Table X.

\subsection{Authentication}
\hspace{2em}The test cases below focus on ensuring users can safely and securely login, create and
access their accounts without worrying about others accessing their information.

\begin{mdframed}[linewidth=0.5mm] \par
  \textbf{Test Case Identifier:} FR-ST-A1 \par
  \textbf{Input:} Selection of Parent account role for login \par
  \textbf{Expected Output:} The expected result is the Parent account role is selected and User is brought to the Parent login screen \par
  \textbf{Actual Output:} \par
  \textbf{Expected and Actual Output Match:} True \par
  \textbf{Relevant Functional Requirement(s):} FR-A1
\end{mdframed}

\subsection{Data Collection and Storage}
\hspace{2em}The test cases below 

\begin{mdframed}[linewidth=0.5mm] \par
  \textbf{Test Case Identifier:} FR-ST-A1 \par
  \textbf{Input:} Selection of Parent account role for login \par
  \textbf{Expected Output:} The expected result is the Parent account role is selected and User is brought to the Parent login screen \par
  \textbf{Actual Output:} \par
  \textbf{Expected and Actual Output Match:} True \par
  \textbf{Relevant Functional Requirement(s):} FR-A1
\end{mdframed}

\subsection{Video and Audio Data Analysis}
\hspace{2em}The test cases below 

\begin{mdframed}[linewidth=0.5mm] \par
  \textbf{Test Case Identifier:} FR-ST-A1 \par
  \textbf{Input:} Selection of Parent account role for login \par
  \textbf{Expected Output:} The expected result is the Parent account role is selected and User is brought to the Parent login screen \par
  \textbf{Actual Output:} \par
  \textbf{Expected and Actual Output Match:} True \par
  \textbf{Relevant Functional Requirement(s):} FR-A1
\end{mdframed}

\subsection{Data Processing and Display}
\hspace{2em}The test cases below 

\begin{mdframed}[linewidth=0.5mm] \par
  \textbf{Test Case Identifier:} FR-ST-A1 \par
  \textbf{Input:} Selection of Parent account role for login \par
  \textbf{Expected Output:} The expected result is the Parent account role is selected and User is brought to the Parent login screen \par
  \textbf{Actual Output:} \par
  \textbf{Expected and Actual Output Match:} True \par
  \textbf{Relevant Functional Requirement(s):} FR-A1
\end{mdframed}

\subsection{System Set Up}
\hspace{2em}The test cases below 

\begin{mdframed}[linewidth=0.5mm] \par
  \textbf{Test Case Identifier:} FR-ST-A1 \par
  \textbf{Input:} Selection of Parent account role for login \par
  \textbf{Expected Output:} The expected result is the Parent account role is selected and User is brought to the Parent login screen \par
  \textbf{Actual Output:} \par
  \textbf{Expected and Actual Output Match:} True \par
  \textbf{Relevant Functional Requirement(s):} FR-A1
\end{mdframed}

\subsection{Assessment Interface}
\hspace{2em}The test cases below 

\begin{mdframed}[linewidth=0.5mm] \par
  \textbf{Test Case Identifier:} FR-ST-A1 \par
  \textbf{Input:} Selection of Parent account role for login \par
  \textbf{Expected Output:} The expected result is the Parent account role is selected and User is brought to the Parent login screen \par
  \textbf{Actual Output:} \par
  \textbf{Expected and Actual Output Match:} True \par
  \textbf{Relevant Functional Requirement(s):} FR-A1
\end{mdframed}

\section{Nonfunctional Requirements Evaluation}
\hspace{2em}The following section covers all the nonfunctional requirements specified in the project’s
VnV Plan document. The coverage can be traced in Table X.

\subsection{Look and Feel Requirements}
\hspace{2em}The test cases below 

\begin{mdframed}[linewidth=0.5mm] \par
  \textbf{Test Case Identifier:} FR-ST-A1 \par
  \textbf{Input:} Selection of Parent account role for login \par
  \textbf{Expected Output:} The expected result is the Parent account role is selected and User is brought to the Parent login screen \par
  \textbf{Actual Output:} \par
  \textbf{Expected and Actual Output Match:} True \par
  \textbf{Relevant Nonfunctional Requirement(s):} FR-A1
\end{mdframed}
		
\subsection{Usability and Humanity}
\hspace{2em}The test cases below 

\begin{mdframed}[linewidth=0.5mm] \par
  \textbf{Test Case Identifier:} FR-ST-A1 \par
  \textbf{Input:} Selection of Parent account role for login \par
  \textbf{Expected Output:} The expected result is the Parent account role is selected and User is brought to the Parent login screen \par
  \textbf{Actual Output:} \par
  \textbf{Expected and Actual Output Match:} True \par
  \textbf{Relevant Nonfunctional Requirement(s):} FR-A1
\end{mdframed}

\subsection{Performance}
\hspace{2em}The test cases below 

\begin{mdframed}[linewidth=0.5mm] \par
  \textbf{Test Case Identifier:} FR-ST-A1 \par
  \textbf{Input:} Selection of Parent account role for login \par
  \textbf{Expected Output:} The expected result is the Parent account role is selected and User is brought to the Parent login screen \par
  \textbf{Actual Output:} \par
  \textbf{Expected and Actual Output Match:} True \par
  \textbf{Relevant Nonfunctional Requirement(s):} FR-A1
\end{mdframed}

\subsection{Operational and Environmental}
\hspace{2em}The test cases below 

\begin{mdframed}[linewidth=0.5mm] \par
  \textbf{Test Case Identifier:} FR-ST-A1 \par
  \textbf{Input:} Selection of Parent account role for login \par
  \textbf{Expected Output:} The expected result is the Parent account role is selected and User is brought to the Parent login screen \par
  \textbf{Actual Output:} \par
  \textbf{Expected and Actual Output Match:} True \par
  \textbf{Relevant Nonfunctional Requirement(s):} FR-A1
\end{mdframed}

\subsection{Maintainability and Support}
\hspace{2em}The test cases below 

\begin{mdframed}[linewidth=0.5mm] \par
  \textbf{Test Case Identifier:} FR-ST-A1 \par
  \textbf{Input:} Selection of Parent account role for login \par
  \textbf{Expected Output:} The expected result is the Parent account role is selected and User is brought to the Parent login screen \par
  \textbf{Actual Output:} \par
  \textbf{Expected and Actual Output Match:} True \par
  \textbf{Relevant Nonfunctional Requirement(s):} FR-A1
\end{mdframed}

\subsection{Cultural}
\hspace{2em}The test cases below 

\begin{mdframed}[linewidth=0.5mm] \par
  \textbf{Test Case Identifier:} FR-ST-A1 \par
  \textbf{Input:} Selection of Parent account role for login \par
  \textbf{Expected Output:} The expected result is the Parent account role is selected and User is brought to the Parent login screen \par
  \textbf{Actual Output:} \par
  \textbf{Expected and Actual Output Match:} True \par
  \textbf{Relevant Nonfunctional Requirement(s):} FR-A1
\end{mdframed}

\subsection{Security}
\hspace{2em}The test cases below 

\begin{mdframed}[linewidth=0.5mm] \par
  \textbf{Test Case Identifier:} FR-ST-A1 \par
  \textbf{Input:} Selection of Parent account role for login \par
  \textbf{Expected Output:} The expected result is the Parent account role is selected and User is brought to the Parent login screen \par
  \textbf{Actual Output:} \par
  \textbf{Expected and Actual Output Match:} True \par
  \textbf{Relevant Nonfunctional Requirement(s):} FR-A1
\end{mdframed}

\subsection{Compliance}
\hspace{2em}The test cases below 

\begin{mdframed}[linewidth=0.5mm] \par
  \textbf{Test Case Identifier:} FR-ST-A1 \par
  \textbf{Input:} Selection of Parent account role for login \par
  \textbf{Expected Output:} The expected result is the Parent account role is selected and User is brought to the Parent login screen \par
  \textbf{Actual Output:} \par
  \textbf{Expected and Actual Output Match:} True \par
  \textbf{Relevant Nonfunctional Requirement(s):} FR-A1
\end{mdframed}
	
\section{Comparison to Existing Implementation}	

As this project does not have existing implementations, this section is not appropriate for the TeleHealth Insights project.

\section{Unit Testing}

\section{Changes Due to Testing}

\wss{This section should highlight how feedback from the users and from 
the supervisor (when one exists) shaped the final product.  In particular 
the feedback from the Rev 0 demo to the supervisor (or to potential users) 
should be highlighted.}

\section{Automated Testing}
		
\section{Trace to Requirements}
		
\section{Trace to Modules}		

\section{Code Coverage Metrics}

\bibliographystyle{plainnat}
\bibliography{../../refs/References}

\newpage{}
\section*{Appendix --- Reflection}

The information in this section will be used to evaluate the team members on the
graduate attribute of Reflection.

\input{../Reflection.tex}

\begin{enumerate}
  \item What went well while writing this deliverable? 
  \item What pain points did you experience during this deliverable, and how
    did you resolve them?
  \item Which parts of this document stemmed from speaking to your client(s) or
  a proxy (e.g. your peers)? Which ones were not, and why?
  \item In what ways was the Verification and Validation (VnV) Plan different
  from the activities that were actually conducted for VnV?  If there were
  differences, what changes required the modification in the plan?  Why did
  these changes occur?  Would you be able to anticipate these changes in future
  projects?  If there weren't any differences, how was your team able to clearly
  predict a feasible amount of effort and the right tasks needed to build the
  evidence that demonstrates the required quality?  (It is expected that most
  teams will have had to deviate from their original VnV Plan.)
\end{enumerate}

\end{document}