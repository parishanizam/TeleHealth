\documentclass[12pt, titlepage]{article}

\usepackage{amsmath, mathtools}

\usepackage[round]{natbib}
\usepackage{amsfonts}
\usepackage{amssymb}
\usepackage{graphicx}
\usepackage{colortbl}
\usepackage{xr}
\usepackage{hyperref}
\usepackage{longtable}
\usepackage{xfrac}
\usepackage{tabularx}
\usepackage{float}
\usepackage{siunitx}
\usepackage{booktabs}
\usepackage{multirow}
\usepackage[section]{placeins}
\usepackage{caption}
\usepackage{fullpage}

\hypersetup{
bookmarks=true,     % show bookmarks bar?
colorlinks=true,       % false: boxed links; true: colored links
linkcolor=red,          % color of internal links (change box color with linkbordercolor)
citecolor=blue,      % color of links to bibliography
filecolor=magenta,  % color of file links
urlcolor=cyan          % color of external links
}

\usepackage{array}

\externaldocument{../../SRS/SRS}

\input{../../Comments}
\input{../../Common}

\begin{document}

\title{Module Interface Specification for \progname{}}

\author{\authname}

\date{\today}

\maketitle

\pagenumbering{roman}

\section{Revision History}

\begin{tabularx}{\textwidth}{p{3cm}p{2cm}X}
\toprule {\bf Date} & {\bf Version} & {\bf Notes}\\
\midrule
Date 1 & 1.0 & Notes\\
Date 2 & 1.1 & Notes\\
\bottomrule
\end{tabularx}

~\newpage

\section{Symbols, Abbreviations and Acronyms}

See SRS Documentation at \wss{give url}

\wss{Also add any additional symbols, abbreviations or acronyms}

\newpage

\tableofcontents

\newpage

\pagenumbering{arabic}

\section{Introduction}

The following document details the Module Interface Specifications for
\wss{Fill in your project name and description}

Complementary documents include the System Requirement Specifications
and Module Guide.  The full documentation and implementation can be
found at \url{...}.  \wss{provide the url for your repo}

\section{Notation}

\wss{You should describe your notation.  You can use what is below as
  a starting point.}

The structure of the MIS for modules comes from \citet{HoffmanAndStrooper1995},
with the addition that template modules have been adapted from
\cite{GhezziEtAl2003}.  The mathematical notation comes from Chapter 3 of
\citet{HoffmanAndStrooper1995}.  For instance, the symbol := is used for a
multiple assignment statement and conditional rules follow the form $(c_1
\Rightarrow r_1 | c_2 \Rightarrow r_2 | ... | c_n \Rightarrow r_n )$.

The following table summarizes the primitive data types used by \progname. 

\begin{center}
\renewcommand{\arraystretch}{1.2}
\noindent 
\begin{tabular}{l l p{7.5cm}} 
\toprule 
\textbf{Data Type} & \textbf{Notation} & \textbf{Description}\\ 
\midrule
character & char & a single symbol or digit\\
integer & $\mathbb{Z}$ & a number without a fractional component in (-$\infty$, $\infty$) \\
natural number & $\mathbb{N}$ & a number without a fractional component in [1, $\infty$) \\
real & $\mathbb{R}$ & any number in (-$\infty$, $\infty$)\\
\bottomrule
\end{tabular} 
\end{center}

\noindent
The specification of \progname \ uses some derived data types: sequences, strings, and
tuples. Sequences are lists filled with elements of the same data type. Strings
are sequences of characters. Tuples contain a list of values, potentially of
different types. In addition, \progname \ uses functions, which
are defined by the data types of their inputs and outputs. Local functions are
described by giving their type signature followed by their specification.

\section{Module Decomposition}

The following table is taken directly from the Module Guide document for this project.

\begin{table}[h!]
\centering
\begin{tabular}{p{0.3\textwidth} p{0.6\textwidth}}
\toprule
\textbf{Level 1} & \textbf{Level 2}\\
\midrule

{Hardware-Hiding} & ~ \\
\midrule

\multirow{7}{0.3\textwidth}{Behaviour-Hiding} & Input Parameters\\
& Output Format\\
& Output Verification\\
& Temperature ODEs\\
& Energy Equations\\ 
& Control Module\\
& Specification Parameters Module\\
\midrule

\multirow{3}{0.3\textwidth}{Software Decision} & {Sequence Data Structure}\\
& ODE Solver\\
& Plotting\\
\bottomrule

\end{tabular}
\caption{Module Hierarchy}
\label{TblMH}
\end{table}

\newpage
~\newpage

\section{MIS of Media Processing Module} \label{MediaProcessingModule}

\subsection{Module}
MediaProcessingModule

\subsection{Uses}
VideoProcessingModule, AudioProcessingModule

\subsection{Syntax}

\subsubsection{Exported Constants}
N/A

\subsubsection{Exported Access Programs}
\begin{center}
  \begin{tabular}{p{3cm} p{4cm} p{4cm} p{5cm}}
  \hline
  \textbf{Name} & \textbf{In} & \textbf{Out} & \textbf{Exceptions} \\
  \hline
  processMedia & \raggedright\arraybackslash mediaFile: str, mediaType: str, assessmentID: str & \raggedright\arraybackslash report: MediaAnalysisReport & \raggedright\arraybackslash MediaProcessingException \\
  \end{tabular}
\end{center}

\subsection{Semantics}

\subsubsection{State Variables}
\begin{itemize}
\item processedMedia: Map(assessmentID, MediaAnalysisReport) - stores combined results from video and audio analysis.
\end{itemize}

\subsubsection{Environment Variables}
N/A

\subsubsection{Assumptions}
\begin{itemize}
\item mediaType specifies whether the file is video or audio (e.g., "video", "audio").
\item Delegates processing to VideoProcessingModule or AudioProcessingModule based on mediaType.
\end{itemize}

\subsubsection{Access Routine Semantics}

\noindent processMedia():
\begin{itemize}
\item transition:
\begin{itemize}
\item If mediaType is "video", calls processVideo from VideoProcessingModule.
\item If mediaType is "audio", calls processAudio from AudioProcessingModule.
\item Combines results into a single MediaAnalysisReport per assessment completed.
\end{itemize}
\item output: Returns a MediaAnalysisReport with details from video and audio analyses.
\item exception: Throws MediaProcessingException if the file cannot be processed or delegated.
\end{itemize}

\subsubsection{Local Functions}
N/A

\newpage

\section{MIS of Video Processing Module} \label{VideoProcessingModule}

\subsection{Module}
VideoProcessingModule

\subsection{Uses}
N/A
MediaProcessingModule

\subsection{Syntax}

\subsubsection{Exported Constants}
N/A

\subsubsection{Exported Access Programs}
\begin{center}
  \begin{tabular}{p{3cm} p{4cm} p{4cm} p{5cm}}
  \hline
  \textbf{Name} & \textbf{In} & \textbf{Out} & \textbf{Exceptions} \\
  \hline
  processVideo & \raggedright\arraybackslash videoFile: str, assessmentID: str & \raggedright\arraybackslash report: VideoAnalysisReport & \raggedright\arraybackslash VideoProcessingException \\
  \end{tabular}
\end{center}

\subsection{Semantics}

\subsubsection{State Variables}
\begin{itemize}
\item processedVideos: Map(assessmentID, VideoAnalysisReport) - stores results of processed videos.
\end{itemize}

\subsubsection{Environment Variables}
N/A

\subsubsection{Assumptions}
\begin{itemize}
\item Video files are in a supported format (e.g., MP4, AVI).
\item Video processing is done within a time threshold for real-time feedback.
\end{itemize}

\subsubsection{Access Routine Semantics}

\noindent processVideo():
\begin{itemize}
\item transition: Analyzes the video to identify any disturbances, bias, or cheating patterns.
\item output: Returns a detailed VideoAnalysisReport containing flagged events and metrics.
\item exception: Throws VideoProcessingException if the file cannot be processed or analyzed.
\end{itemize}

\subsubsection{Local Functions}
N/A

\section{MIS of Audio Processing Module} \label{AudioProcessingModule}

\subsection{Module}
AudioProcessingModule

\subsection{Uses}
N/A
MediaProcessingModule

\subsection{Syntax}

\subsubsection{Exported Constants}
N/A

\subsubsection{Exported Access Programs}

\begin{center}
  \begin{tabular}{p{3cm} p{4cm} p{4cm} p{5cm}}
  \hline
  \textbf{Name} & \textbf{In} & \textbf{Out} & \textbf{Exceptions} \\
  \hline
  processAudio & \raggedright\arraybackslash audioFile: str, assessmentID: str & \raggedright\arraybackslash report: AudioAnalysisReport & \raggedright\arraybackslash AudioProcessingException \\
  \end{tabular}
\end{center}

\subsection{Semantics}

\subsubsection{State Variables}
\begin{itemize}
\item processedAudio: Map(assessmentID, AudioAnalysisReport) - stores results of processed audio.
\end{itemize}

\subsubsection{Environment Variables}
N/A

\subsubsection{Assumptions}
\begin{itemize}
\item Audio files are in a supported format (e.g., WAV, MP3).
\item Background noise levels are detectable and quantifiable.
\end{itemize}

\subsubsection{Access Routine Semantics}

\noindent processAudio():
\begin{itemize}
\item transition: Analyzes the audio for disturbances such as background noise or interruptions.
\item output: Returns a detailed AudioAnalysisReport with flagged issues.
\item exception: Throws AudioProcessingException if the file cannot be processed or analyzed.
\end{itemize}

\subsubsection{Local Functions}
N/A

\section{MIS of Logging Module} \label{LoggingModule}

\subsection{Module}
LoggingModule

\subsection{Uses}
N/A

\subsection{Syntax}

\subsubsection{Exported Constants}
N/A

\subsubsection{Exported Access Programs}

\begin{center}
\begin{tabular}{p{3cm} p{4cm} p{4cm} p{5cm}}
\hline
\textbf{Name} & \textbf{In} & \textbf{Out} & \textbf{Exceptions} \\
\hline
logEvent & \raggedright\arraybackslash eventType: str, message: str & \raggedright\arraybackslash status: bool & \raggedright\arraybackslash LoggingException \\
\hline
fetchLogs & \raggedright\arraybackslash logType: str, timeRange: TimeRange & \raggedright\arraybackslash logData: List(LogEntry) & \raggedright\arraybackslash LogFetchException \\
\hline
clearLogs & \raggedright\arraybackslash logType: str, timeRange: TimeRange & \raggedright\arraybackslash status: bool & \raggedright\arraybackslash LogClearException \\
\hline
\end{tabular}
\end{center}

\subsection{Semantics}

\subsubsection{State Variables}
\begin{itemize}
\item logs: Map(logType, List(LogEntry)) - stores all logs categorized by type.
\end{itemize}

\subsubsection{Environment Variables}
N/A

\subsubsection{Assumptions}
\begin{itemize}
\item Logs are categorized by their type and timestamp for easy retrieval.
\item Each log entry includes metadata such as the timestamp, severity, and module of origin.
\end{itemize}

\subsubsection{Types of Logs}

Event Logs

Tracks application activities, such as user logins, file uploads, and media processing events.
Example: "User 'parent1' logged in successfully at 14:32:00."
Error Logs

Captures unexpected behaviors, exceptions, or failures in the application.
Example: "VideoProcessingException: Unable to analyze video file 'session123.mp4' due to corrupted data."
Audit Logs

Records critical changes and actions for accountability, such as user role changes or log clearances.
Example: "Admin user 'clinician1' updated parent access rights at 16:45:00."
Performance Logs

Monitors application performance metrics like response times, memory usage, and processing durations.
Example: "Media processing for 'session456' completed in 3.2 seconds with 200MB memory usage."
Debug Logs

Includes detailed information for troubleshooting during development or maintenance.
Example: "Entering function 'processVideo' with input file 'session789.mp4'."
Security Logs

Tracks security-related events such as failed logins, access violations, or token expirations.
Example: "Security alert: Failed login attempt for user 'parent2' at 18:12:30."
\subsubsection{Access Routine Semantics}

\noindent logEvent():
\begin{itemize}
\item transition: Adds a new log entry to the logs map under the appropriate eventType.
\item output: Returns true if the log is successfully recorded.
\item exception: Throws LoggingException if the log entry cannot be added.
\end{itemize}

\noindent fetchLogs():
\begin{itemize}
\item transition: Retrieves all logs of the specified logType within the provided timeRange.
\item output: Returns a list of LogEntry objects matching the criteria.
\item exception: Throws LogFetchException if no logs are found or retrieval fails.
\end{itemize}

\noindent clearLogs():
\begin{itemize}
\item transition: Removes all logs of the specified logType within the provided timeRange.
\item output: Returns true if logs are successfully cleared.
\item exception: Throws LogClearException if logs cannot be cleared.
\end{itemize}

\subsubsection{Local Functions}
N/A

\newpage

\bibliographystyle {plainnat}
\bibliography {../../../refs/References}

\newpage

\section{Appendix} \label{Appendix}

\wss{Extra information if required}

\newpage{}

\section*{Appendix --- Reflection}

\wss{Not required for CAS 741 projects}

The information in this section will be used to evaluate the team members on the
graduate attribute of Problem Analysis and Design.

\input{../../Reflection.tex}

\begin{enumerate}
  \item What went well while writing this deliverable? 
  \item What pain points did you experience during this deliverable, and how
    did you resolve them?
  \item Which of your design decisions stemmed from speaking to your client(s)
  or a proxy (e.g. your peers, stakeholders, potential users)? For those that
  were not, why, and where did they come from?
  \item While creating the design doc, what parts of your other documents (e.g.
  requirements, hazard analysis, etc), it any, needed to be changed, and why?
  \item What are the limitations of your solution?  Put another way, given
  unlimited resources, what could you do to make the project better? (LO\_ProbSolutions)
  \item Give a brief overview of other design solutions you considered.  What
  are the benefits and tradeoffs of those other designs compared with the chosen
  design?  From all the potential options, why did you select the documented design?
  (LO\_Explores)
\end{enumerate}


\end{document}