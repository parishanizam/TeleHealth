\documentclass[12pt, titlepage]{article}

\usepackage{amsmath, mathtools}

\usepackage[round]{natbib}
\usepackage{amsfonts}
\usepackage{amssymb}
\usepackage{graphicx}
\usepackage{colortbl}
\usepackage{xr}
\usepackage{hyperref}
\usepackage{longtable}
\usepackage{xfrac}
\usepackage{tabularx}
\usepackage{float}
\usepackage{siunitx}
\usepackage{booktabs}
\usepackage{multirow}
\usepackage[section]{placeins}
\usepackage{caption}
\usepackage{fullpage}

\hypersetup{
bookmarks=true,     % show bookmarks bar?
colorlinks=true,       % false: boxed links; true: colored links
linkcolor=red,          % color of internal links (change box color with linkbordercolor)
citecolor=blue,      % color of links to bibliography
filecolor=magenta,  % color of file links
urlcolor=cyan          % color of external links
}

\usepackage{array}

\externaldocument{../../SRS/SRS}

%% Comments

\usepackage{color}

\newif\ifcomments\commentstrue %displays comments
%\newif\ifcomments\commentsfalse %so that comments do not display

\ifcomments
\newcommand{\authornote}[3]{\textcolor{#1}{[#3 ---#2]}}
\newcommand{\todo}[1]{\textcolor{red}{[TODO: #1]}}
\else
\newcommand{\authornote}[3]{}
\newcommand{\todo}[1]{}
\fi

\newcommand{\wss}[1]{\authornote{blue}{SS}{#1}} 
\newcommand{\plt}[1]{\authornote{magenta}{TPLT}{#1}} %For explanation of the template
\newcommand{\an}[1]{\authornote{cyan}{Author}{#1}}

%% Common Parts

\newcommand{\progname}{Software Engineering} % PUT YOUR PROGRAM NAME HERE
\newcommand{\authname}{Team \#22, TeleHealth Insights
\\ Mitchell Weingust
\\ Parisha Nizam
\\ Promish Kandel
\\ Jasmine Sun-Hu} % AUTHOR NAMES                  

\usepackage{hyperref}
    \hypersetup{colorlinks=true, linkcolor=blue, citecolor=blue, filecolor=blue,
                urlcolor=blue, unicode=false}
    \urlstyle{same}
                                


\begin{document}

\title{Module Interface Specification for \progname{}}

\author{\authname}

\date{\today}

\maketitle

\pagenumbering{roman}

\section{Revision History}

\begin{tabularx}{\textwidth}{p{3cm}p{2cm}X}
\toprule {\bf Date} & {\bf Version} & {\bf Notes}\\
\midrule
Date 1 & 1.0 & Notes\\
Date 2 & 1.1 & Notes\\
\bottomrule
\end{tabularx}

~\newpage

\section{Symbols, Abbreviations and Acronyms}

See SRS Documentation at \wss{give url}

\wss{Also add any additional symbols, abbreviations or acronyms}

\newpage

\tableofcontents

\newpage

\pagenumbering{arabic}

\section{Introduction}

The following document details the Module Interface Specifications for TeleHealth Insights. It is an at-home bilingual speech 
assessment system with video and audio analysis features. The system is designed 
to provide clear guidance to parents when administering the assessment to their 
children, in an environment where speech-language pathologists (SLPs) are 
unavailable. By streamlining the assessment process, the project aims to provide a 
convenient and comprehensive solution for SLPs to assess and support their patients'
speech and language development remotely.
% \wss{Fill in your project name and description}

Complementary documents include the System Requirement Specifications
and Module Guide.  The full documentation and implementation can be
found at \url{https://github.com/parishanizam/TeleHealth} 
% \wss{provide the url for your repo}

\section{Notation}

\wss{You should describe your notation.  You can use what is below as
  a starting point.}

The structure of the MIS for modules comes from \citet{HoffmanAndStrooper1995},
with the addition that template modules have been adapted from
\cite{GhezziEtAl2003}.  The mathematical notation comes from Chapter 3 of
\citet{HoffmanAndStrooper1995}.  For instance, the symbol := is used for a
multiple assignment statement and conditional rules follow the form $(c_1
\Rightarrow r_1 | c_2 \Rightarrow r_2 | ... | c_n \Rightarrow r_n )$.

The following table summarizes the primitive data types used by \progname. 

\begin{center}
\renewcommand{\arraystretch}{1.2}
\noindent 
\begin{tabular}{l l p{7.5cm}} 
\toprule 
\textbf{Data Type} & \textbf{Notation} & \textbf{Description}\\ 
\midrule
character & char & a single symbol or digit\\
integer & $\mathbb{Z}$ & a number without a fractional component in (-$\infty$, $\infty$) \\
natural number & $\mathbb{N}$ & a number without a fractional component in [1, $\infty$) \\
real & $\mathbb{R}$ & any number in (-$\infty$, $\infty$)\\
\bottomrule
\end{tabular} 
\end{center}

\noindent
The specification of \progname \ uses some derived data types: sequences, strings, and
tuples. Sequences are lists filled with elements of the same data type. Strings
are sequences of characters. Tuples contain a list of values, potentially of
different types. In addition, \progname \ uses functions, which
are defined by the data types of their inputs and outputs. Local functions are
described by giving their type signature followed by their specification.

\section{Module Decomposition}

The following table is taken directly from the Module Guide document for this project.

\begin{table}[h!]
\centering
\begin{tabular}{p{0.3\textwidth} p{0.6\textwidth}}
\toprule
\textbf{Level 1} & \textbf{Level 2}\\
\midrule

{Hardware-Hiding} & N/A \\
\midrule

\multirow{7}{0.3\textwidth}{Behaviour-Hiding} & Clinician GUI\\
& Parent GUI\\
& Authentication Module\\
& Result Storage Module\\
& Real-Time Feedback Module\\ 
& Report Generation Module\\
& Media Processing Module\\
& Video Processing Module\\
& Audio Processing Module\\
& Logging Module\\
& Question Bank Module\\
& Mandarian Question Bank\\
& English Question Bank\\
& Repetition Question Bank Module\\
& Matching Question Bacnk Module\\
\midrule

\multirow{3}{0.3\textwidth}{Software Decision} & {APP Controller}\\
& API Gateway\\
\bottomrule

\end{tabular}
\caption{Module Hierarchy}
\label{TblMH}
\end{table}

\newpage
~\newpage
\section{MIS of Clinician GUI \label{mClinicianGUI} }

\subsection{Module}

clinicianGUI

\subsection{Uses}

\begin{itemize}
  \item AppController
\end{itemize}

\subsection{Syntax}

\subsubsection{Exported Constants}

N/A

\subsubsection{Exported Access Programs}

\begin{center}
\begin{tabular}{p{8cm} p{2cm} p{2cm} p{2cm}}
\hline
\textbf{Name} & \textbf{In} & \textbf{Out} & \textbf{Exceptions} \\
\hline
displayLoginPage & - & - & - \\
displayClinicianDashboardPage & - & - & - \\
displayAddClientPage & - & - & - \\
displayPatientOverviewPage & - & - & - \\
displayPatientAssessmentResultsAnalysisPage & - & - & - \\
displayBiasReviewPage & - & - & - \\
displayFlagBiasPage & - & - & - \\
\hline
\end{tabular}
\end{center}

\subsection{Semantics}

\subsubsection{State Variables}
N/A

\subsubsection{Environment Variables}
N/A

\subsubsection{Assumptions}
N/A

\subsubsection{Access Routine Semantics}

\noindent displayLoginPage():
\begin{itemize}
\item transition: Navigates to and displays the clinician login page for the system.
\item output: N/A
\item exception: N/A
\end{itemize}

\noindent displayClinicianDashboardPage():
\begin{itemize}
\item transition: Navigates to and displays the clinician dashboard page for accessing a clinician's list of clients.
\item output: N/A
\item exception: N/A
\end{itemize}

\noindent displayAddClientPage():
\begin{itemize}
\item transition: Navigates to and displays the add client page for adding a new client to a clinician's list.
\item output: N/A
\item exception: N/A
\end{itemize}

\noindent displayPatientOverviewPage():
\begin{itemize}
\item transition: Navigates to and displays the patient overview page for accessing all of the patient's previous assessments.
\item output: N/A
\item exception: N/A
\end{itemize}

\noindent displayPatientAssessmentResultsAnalysisPage():
\begin{itemize}
\item transition: Navigates to and displays the patient assessment results analysis page for accessing all of the results of a particular assessment.
\item output: N/A
\item exception: N/A
\end{itemize}

\noindent displayBiasReviewPage():
\begin{itemize}
\item transition: Navigates to and displays the bias review page for reviewing and removing bias from a particular question in an assessment.
\item output: N/A
\item exception: N/A
\end{itemize}

\noindent displayFlagBiasPage():
\begin{itemize}
\item transition: Navigates to and displays the flag bias page for reviewing and flagging bias on a particular question in an assessment.
\item output: N/A
\item exception: N/A
\end{itemize}

\subsubsection{Local Functions}
N/A

~\newpage
\section{MIS of Parent GUI \label{mParentGUI} }

\subsection{Module}

parentGUI

\subsection{Uses}

\begin{itemize}
  \item AppController
\end{itemize}

\subsection{Syntax}

\subsubsection{Exported Constants}

N/A

\subsubsection{Exported Access Programs}

\begin{center}
\begin{tabular}{p{8cm} p{4cm} p{2cm} p{2cm}}
\hline
\textbf{Name} & \textbf{In} & \textbf{Out} & \textbf{Exceptions} \\
\hline
\wss{accessProg} & - & - & - \\
\hline
\end{tabular}
\end{center}

\subsection{Semantics}

\subsubsection{State Variables}
N/A

\subsubsection{Environment Variables}
\begin{itemize}
  \item microphoneInput
  \item cameraInput
\end{itemize}

\subsubsection{Assumptions}
N/A

\subsubsection{Access Routine Semantics}

\noindent displayLoginPage():
\begin{itemize}
\item transition: Navigates to and displays the parent login page for the system.
\item output: N/A
\item exception: N/A
\end{itemize}

\noindent displayCreateAccountPage():
\begin{itemize}
\item transition: Navigates to and displays the parent account creation page for creating a new account for the system.
\item output: N/A
\item exception: N/A
\end{itemize}

\noindent displayHomePage():
\begin{itemize}
\item transition: Navigates to and displays the homepage for a parent account, with the ability to start a new assessment.
\item output: N/A
\item exception: N/A
\end{itemize}

\noindent displayAssessmentSelectionPage():
\begin{itemize}
\item transition: Navigates to and displays the assessment selection page for selecting the type of assessment for the user.
\item output: N/A
\item exception: N/A
\end{itemize}

\noindent displayParentChecklistPage():
\begin{itemize}
\item transition: Navigates to and displays the parent checklist page for informing parents about the requirements of the assessment.
\item output: N/A
\item exception: N/A
\end{itemize}

\noindent displayInputCheckPage():
\begin{itemize}
\item transition: Navigates to and displays the input check page for testing input devices.
\item output: N/A
\item exception: N/A
\end{itemize}

\noindent displayAssessmentInstructionsPage():
\begin{itemize}
\item transition: Navigates to and displays the assessment instructions page for the child to read and engage with to learn how to interact with the assessment interface.
\item output: N/A
\item exception: N/A
\end{itemize}

\noindent displayAssessmentQuestionsPage():
\begin{itemize}
\item transition: Navigates to and displays the assessment questions page for displaying a question and its corresponding answers for the user to select.
\item output: N/A
\item exception: N/A
\end{itemize}

\noindent displayCompletionPage():
\begin{itemize}
\item transition: Navigates to and displays the completion page to confirm to the user that the assessment is complete and their results have been saved.
\item output: N/A
\item exception: N/A
\end{itemize}

\subsubsection{Local Functions}
N/A

~\newpage
\section{MIS of App Controller \label{mAppController} }

\subsection{Module}

AppController

\subsection{Uses}

\begin{itemize}
  \item APIGateway
\end{itemize}

\subsection{Syntax}

\subsubsection{Exported Constants}

N/A

\subsubsection{Exported Access Programs}

\begin{center}
\begin{tabular}{p{8cm} p{4cm} p{2cm} p{2cm}}
\hline
\textbf{Name} & \textbf{In} & \textbf{Out} & \textbf{Exceptions} \\
\hline
accessAPIGateway & - & - & - \\
\hline
\end{tabular}
\end{center}

\subsection{Semantics}

\subsubsection{State Variables}
N/A

\subsubsection{Environment Variables}
N/A

\subsubsection{Assumptions}
N/A

\subsubsection{Access Routine Semantics}

\noindent accessAPIGateway():
\begin{itemize}
\item transition: Controller accesses the API Gateway.
\item output: N/A
\item exception: N/A
\end{itemize}

\subsubsection{Local Functions}
N/A

~\newpage
\section{MIS of API Gateway \label{mAppController} }

\subsection{Module}

APIGateway

\subsection{Uses}

\begin{itemize}
  \item Authentication
  \item ResultStorage
  \item MediaProcessing
  \item Logging
  \item QuestionBank
  \item RealTimeFeedback
  \item ReportGeneration
\end{itemize}

\subsection{Syntax}

\subsubsection{Exported Constants}

N/A

\subsubsection{Exported Access Programs}

\begin{center}
\begin{tabular}{p{8cm} p{4cm} p{2cm} p{2cm}}
\hline
\textbf{Name} & \textbf{In} & \textbf{Out} & \textbf{Exceptions} \\
\hline
accessAuthentication & - & - & - \\
accessResultStorage & - & - & - \\
accessMediaProcessing & - & - & - \\
accessLogging & - & - & - \\
accessQuestionBank & - & - & - \\
accessRealTimeFeedback & - & - & - \\
accessReportGeneration & - & - & - \\
\hline
\end{tabular}
\end{center}

\subsection{Semantics}

\subsubsection{State Variables}
N/A

\subsubsection{Environment Variables}
N/A

\subsubsection{Assumptions}

\wss{Try to minimize assumptions and anticipate programmer errors via
  exceptions, but for practical purposes assumptions are sometimes appropriate.}

\subsubsection{Access Routine Semantics}

\noindent \wss{accessProg}():
\begin{itemize}
\item transition: \wss{if appropriate} 
\item output: \wss{if appropriate} 
\item exception: \wss{if appropriate} 
\end{itemize}

\wss{A module without environment variables or state variables is unlikely to
  have a state transition.  In this case a state transition can only occur if
  the module is changing the state of another module.}

\wss{Modules rarely have both a transition and an output.  In most cases you
  will have one or the other.}

\subsubsection{Local Functions}

\wss{As appropriate} \wss{These functions are for the purpose of specification.
  They are not necessarily something that is going to be implemented
  explicitly.  Even if they are implemented, they are not exported; they only
  have local scope.}

\newpage

\bibliographystyle {plainnat}
\bibliography {../../../refs/References}

\newpage

\section{Appendix} \label{Appendix}

\wss{Extra information if required}

\newpage{}

\section*{Appendix --- Reflection}

\wss{Not required for CAS 741 projects}

The information in this section will be used to evaluate the team members on the
graduate attribute of Problem Analysis and Design.

The purpose of reflection questions is to give you a chance to assess your own
learning and that of your group as a whole, and to find ways to improve in the
future. Reflection is an important part of the learning process.  Reflection is
also an essential component of a successful software development process.  

Reflections are most interesting and useful when they're honest, even if the
stories they tell are imperfect. You will be marked based on your depth of
thought and analysis, and not based on the content of the reflections
themselves. Thus, for full marks we encourage you to answer openly and honestly
and to avoid simply writing ``what you think the evaluator wants to hear.''

Please answer the following questions.  Some questions can be answered on the
team level, but where appropriate, each team member should write their own
response:


\begin{enumerate}
  \item What went well while writing this deliverable? \\
  \\Promish: I think as a group we were very coordinated and had the important parts, like the module
  hierarchy diagram, completed before our TA meeting. This allowed us to ask Chris if our design looked good and if 
  there was any feedback he could provide.
  \item What pain points did you experience during this deliverable, and how
    did you resolve them? \\
    \\Promish: The hardest part of this deliverable is knowing that right after it, we have to build everything within three weeks.
    I found myself second-guessing between what would make a good design and what would make a feasible design. We overcame
    this by talking to our supervisor and setting up a good timeline.
  \item Which of your design decisions stemmed from speaking to your client(s)
  or a proxy (e.g., your peers, stakeholders, potential users)? For those that
  were not, why, and where did they come from? \\
  \\Promish: Most of the design came from prior knowledge of working in co-op as a backend and frontend developer.
  We also had our supervisor help with the frontend UI design, as there was a mock-up of what she wanted overall or what she didn't like.
  \item While creating the design doc, what parts of your other documents (e.g.,
  requirements, hazard analysis, etc.), if any, needed to be changed, and why? \\
  \\Promish: Nothing needed to change as we had already discussed in-depth what we were building and had regular meetups with our supervisor.
  We referenced the SRS to ensure that our design met our goals and extended goals, but we didn't have to change any previous document because of our design.
  \item What are the limitations of your solution? Put another way, given
  unlimited resources, what could you do to make the project better? (LO\_ProbSolutions) \\
  \\Promish: I would say a big limitation is the time aspect of making the capstone. I feel like given more time, we would be able to build it like an agile team,
  so that features are constantly tested with users. We are also limited in terms of getting video metadata, so the accuracy of our video analysis might suffer because of that.
  Finally, we are also limited by our skills; there are some aspects of the design that we couldn't do because we didn't know how to and didn't have the time to figure it out.
  \item Give a brief overview of other design solutions you considered. What
  are the benefits and trade-offs of those other designs compared with the chosen
  design? From all the potential options, why did you select the documented design?
  (LO\_Explores) \\
  \\Promish: For our backend, we thought about a monolithic architecture style because it would be easier to implement, but the trade-off is that it's not as maintainable.
  Our supervisor stressed that maintainability is a big priority for her, so we ended up going with a microservice architecture style. We also considered the various interactions between how
  the clinician and the parent UI. We previously thought about the parent making an account and the clinician would just search for the parent to add them to their client list. However,
  we thought it was best if the clinician added the parent to their client list and then gave them a code. This way, we would reduce troll accounts and other security gaps.
\end{enumerate}



\end{document}