\documentclass[12pt, titlepage]{article}

\usepackage{amsmath, mathtools}

\usepackage[round]{natbib}
\usepackage{amsfonts}
\usepackage{amssymb}
\usepackage{graphicx}
\usepackage{colortbl}
\usepackage{xr}
\usepackage{hyperref}
\usepackage{longtable}
\usepackage{xfrac}
\usepackage{tabularx}
\usepackage{float}
\usepackage{siunitx}
\usepackage{booktabs}
\usepackage{multirow}
\usepackage[section]{placeins}
\usepackage{caption}
\usepackage{fullpage}

\hypersetup{
bookmarks=true,     % show bookmarks bar?
colorlinks=true,       % false: boxed links; true: colored links
linkcolor=red,          % color of internal links (change box color with linkbordercolor)
citecolor=blue,      % color of links to bibliography
filecolor=magenta,  % color of file links
urlcolor=cyan          % color of external links
}

\usepackage{array}

\externaldocument{../../SRS/SRS}

\input{../../Comments}
\input{../../Common}

\begin{document}

\title{Module Interface Specification for \progname{}}

\author{\authname}

\date{\today}

\maketitle

\pagenumbering{roman}

\section{Revision History}

\begin{tabularx}{\textwidth}{p{3cm}p{2cm}X}
\toprule {\bf Date} & {\bf Version} & {\bf Notes}\\
\midrule
Date 1 & 1.0 & Notes\\
Date 2 & 1.1 & Notes\\
\bottomrule
\end{tabularx}

~\newpage

\section{Symbols, Abbreviations and Acronyms}

See SRS Documentation at \wss{give url}

\wss{Also add any additional symbols, abbreviations or acronyms}

\newpage

\tableofcontents

\newpage

\pagenumbering{arabic}

\section{Introduction}

The following document details the Module Interface Specifications for
\wss{Fill in your project name and description}

Complementary documents include the System Requirement Specifications
and Module Guide.  The full documentation and implementation can be
found at \url{...}.  \wss{provide the url for your repo}

\section{Notation}

\wss{You should describe your notation.  You can use what is below as
  a starting point.}

The structure of the MIS for modules comes from \citet{HoffmanAndStrooper1995},
with the addition that template modules have been adapted from
\cite{GhezziEtAl2003}.  The mathematical notation comes from Chapter 3 of
\citet{HoffmanAndStrooper1995}.  For instance, the symbol := is used for a
multiple assignment statement and conditional rules follow the form $(c_1
\Rightarrow r_1 | c_2 \Rightarrow r_2 | ... | c_n \Rightarrow r_n )$.

The following table summarizes the primitive data types used by \progname. 

\begin{center}
\renewcommand{\arraystretch}{1.2}
\noindent 
\begin{tabular}{l l p{7.5cm}} 
\toprule 
\textbf{Data Type} & \textbf{Notation} & \textbf{Description}\\ 
\midrule
character & char & a single symbol or digit\\
integer & $\mathbb{Z}$ & a number without a fractional component in (-$\infty$, $\infty$) \\
natural number & $\mathbb{N}$ & a number without a fractional component in [1, $\infty$) \\
real & $\mathbb{R}$ & any number in (-$\infty$, $\infty$)\\
\bottomrule
\end{tabular} 
\end{center}

\noindent
The specification of \progname \ uses some derived data types: sequences, strings, and
tuples. Sequences are lists filled with elements of the same data type. Strings
are sequences of characters. Tuples contain a list of values, potentially of
different types. In addition, \progname \ uses functions, which
are defined by the data types of their inputs and outputs. Local functions are
described by giving their type signature followed by their specification.

\section{Module Decomposition}

The following table is taken directly from the Module Guide document for this project.

\begin{table}[h!]
\centering
\begin{tabular}{p{0.3\textwidth} p{0.6\textwidth}}
\toprule
\textbf{Level 1} & \textbf{Level 2}\\
\midrule

{Hardware-Hiding} & ~ \\
\midrule

\multirow{7}{0.3\textwidth}{Behaviour-Hiding} & Input Parameters\\
& Output Format\\
& Output Verification\\
& Temperature ODEs\\
& Energy Equations\\ 
& Control Module\\
& Specification Parameters Module\\
\midrule

\multirow{3}{0.3\textwidth}{Software Decision} & {Sequence Data Structure}\\
& ODE Solver\\
& Plotting\\
\bottomrule

\end{tabular}
\caption{Module Hierarchy}
\label{TblMH}
\end{table}

\newpage
~\newpage

\section{MIS of Question Bank Module} \label{QuestionBankModule}

\subsection{Module}

QuestionBankModule

\subsection{Uses}

EnglishQuestionBankModule, MandarinQuestionBankModule

\subsection{Syntax}

\subsubsection{Exported Constants}

N/A

\subsubsection{Exported Access Programs}

\begin{center}
\begin{tabular}{p{4cm} p{3cm} p{4cm} p{5cm}}
\hline
\textbf{Name} & \textbf{In} & \textbf{Out} & \textbf{Exceptions} \\
\hline
selectQuestionBank & \raggedright\arraybackslash language: str & \raggedright\arraybackslash questionBank: ADT & \raggedright\arraybackslash InvalidLanguageException \\
\hline
retrieveQuestion & \raggedright\arraybackslash language: str, questionID: str & \raggedright\arraybackslash question: JSON object & \raggedright\arraybackslash NotFoundException \\
\hline
\end{tabular}
\end{center}

\subsection{Semantics}

\subsubsection{State Variables}

\begin{itemize}
  \item activeQuestionBanks: Map(language, ADT) - maps language to its respective question bank module.
\end{itemize}

\subsubsection{Environment Variables}

N/A

\subsubsection{Assumptions}

\begin{itemize}
  \item Supported languages include English and Mandarin.
  \item Each question bank module is preloaded with language-specific questions.
\end{itemize}

\subsubsection{Access Routine Semantics}

\noindent selectQuestionBank():
\begin{itemize}
\item transition: Selects the question bank module corresponding to the input language.
\item output: Returns the selected question bank module.
\item exception: Throws \texttt{InvalidLanguageException} if the input language is not supported.
\end{itemize}

\noindent retrieveQuestion():
\begin{itemize}
\item transition: None
\item output: Retrieves the question from the appropriate question bank module.
\item exception: Throws \texttt{NotFoundException} if the questionID does not exist in the selected module.
\end{itemize}

\subsubsection{Local Functions}

N/A

\section{MIS of English Question Bank Module} \label{EnglishQuestionBankModule}

\subsection{Module}

EnglishQuestionBankModule

\subsection{Uses}

MatchingQuestionBankModule, RepetitionQuestionBankModule

\subsection{Syntax}

\subsubsection{Exported Constants}

N/A

\subsubsection{Exported Access Programs}

\begin{center}
\begin{tabular}{p{3cm} p{4cm} p{4cm} p{5cm}}
\hline
\textbf{Name} & \textbf{In} & \textbf{Out} & \textbf{Exceptions} \\
\hline
getQuestion & \raggedright\arraybackslash questionID: str & \raggedright\arraybackslash question: JSON object & \raggedright\arraybackslash NotFoundException \\
\hline
addQuestion & \raggedright\arraybackslash question: JSON object & \raggedright\arraybackslash status: bool & \raggedright\arraybackslash StorageException \\
\hline
\end{tabular}
\end{center}

\subsection{Semantics}

\subsubsection{State Variables}

\begin{itemize}
  \item englishQuestions: Map(questionID, JSON object) - stores English questions.
\end{itemize}

\subsubsection{Environment Variables}

N/A

\subsubsection{Assumptions}

\begin{itemize}
  \item Questions are either matching or repetition type.
  \item Matching and Repetition modules are used to handle the respective types.
\end{itemize}

\subsubsection{Access Routine Semantics}

\noindent getQuestion():
\begin{itemize}
\item transition: None
\item output: Returns the question corresponding to the questionID.
\item exception: Throws \texttt{NotFoundException} if the questionID does not exist.
\end{itemize}

\noindent addQuestion():
\begin{itemize}
\item transition: Adds the input question to \texttt{englishQuestions}.
\item output: Returns \texttt{true} if the question is successfully added.
\item exception: Throws \texttt{StorageException} if there is an issue storing the question.
\end{itemize}

\subsubsection{Local Functions}

N/A

\section{MIS of Mandarin Question Bank Module} \label{MandarinQuestionBankModule}

\subsection{Module}

MandarinQuestionBankModule

\subsection{Uses}

MatchingQuestionBankModule, RepetitionQuestionBankModule

\subsection{Syntax}

\subsubsection{Exported Constants}

N/A

\subsubsection{Exported Access Programs}

\begin{center}
\begin{tabular}{p{3cm} p{4cm} p{4cm} p{5cm}}
\hline
\textbf{Name} & \textbf{In} & \textbf{Out} & \textbf{Exceptions} \\
\hline
getQuestion & \raggedright\arraybackslash questionID: str & \raggedright\arraybackslash question: JSON object & \raggedright\arraybackslash NotFoundException \\
\hline
addQuestion & \raggedright\arraybackslash question: JSON object & \raggedright\arraybackslash status: bool & \raggedright\arraybackslash StorageException \\
\hline
\end{tabular}
\end{center}

\subsection{Semantics}

\subsubsection{State Variables}

\begin{itemize}
  \item mandarinQuestions: Map(questionID, JSON object) - stores Mandarin questions.
\end{itemize}

\subsubsection{Environment Variables}

N/A

\subsubsection{Assumptions}

\begin{itemize}
  \item Questions are either matching or repetition type.
  \item Matching and Repetition modules are used to handle the respective types.
\end{itemize}

\subsubsection{Access Routine Semantics}

\noindent getQuestion():
\begin{itemize}
\item transition: None
\item output: Returns the question corresponding to the questionID.
\item exception: Throws \texttt{NotFoundException} if the questionID does not exist.
\end{itemize}

\noindent addQuestion():
\begin{itemize}
\item transition: Adds the input question to \texttt{englishQuestions}.
\item output: Returns \texttt{true} if the question is successfully added.
\item exception: Throws \texttt{StorageException} if there is an issue storing the question.
\end{itemize}

\subsubsection{Local Functions}

N/A

\section{MIS of Matching Question Bank Module} \label{MatchingQuestionBankModule}

\subsection{Module}

MatchingQuestionBankModule

\subsection{Uses}

N/A

\subsection{Syntax}

\subsubsection{Exported Constants}

N/A

\subsubsection{Exported Access Programs}

\begin{center}
\begin{tabular}{p{5cm} p{3cm} p{3cm} p{5cm}}
\hline
\textbf{Name} & \textbf{In} & \textbf{Out} & \textbf{Exceptions} \\
\hline
storeMatchingQuestion & \raggedright\arraybackslash question: JSON object & \raggedright\arraybackslash status: bool & \raggedright\arraybackslash StorageException \\
\hline
retrieveMatchingQuestion & \raggedright\arraybackslash questionID: str & \raggedright\arraybackslash question: JSON object & \raggedright\arraybackslash NotFoundException \\
\hline
\end{tabular}
\end{center}

\subsection{Semantics}

\subsubsection{State Variables}

\begin{itemize}
  \item matchingQuestions: Map(questionID, JSON object) - stores matching questions.
\end{itemize}

\subsubsection{Environment Variables}

N/A

\subsubsection{Assumptions}

\begin{itemize}
  \item Questions have a unique ID.
  \item Data is stored in a JSON format for flexibility.
\end{itemize}

\subsubsection{Access Routine Semantics}

\noindent storeMatchingQuestion():
\begin{itemize}
  \item transition: Adds the question to \texttt{matchingQuestions}.
  \item output: Returns \texttt{true} if successfully stored.
  \item exception: Throws \texttt{StorageException} if there is a storage error.
\end{itemize}

\noindent retrieveMatchingQuestion():
\begin{itemize}
  \item transition: None
  \item output: Returns the matching question corresponding to the questionID.
  \item exception: Throws \texttt{NotFoundException}
\end{itemize}

\subsubsection{Local Functions}

N/A

\section{MIS of Repetition Question Bank Module} \label{RepetitionQuestionBankModule}

\subsection{Module}

RepetitionQuestionBankModule

\subsection{Uses}

N/A

\subsection{Syntax}

\subsubsection{Exported Constants}

N/A

\subsubsection{Exported Access Programs}

\begin{center}
\begin{tabular}{p{5cm} p{3cm} p{3cm} p{5cm}}
\hline
\textbf{Name} & \textbf{In} & \textbf{Out} & \textbf{Exceptions} \\
\hline
storeRepetitionQuestion & \raggedright\arraybackslash question: JSON object & \raggedright\arraybackslash status: bool & \raggedright\arraybackslash StorageException \\
\hline
retrieveRepetitionQuestion & \raggedright\arraybackslash questionID: str & \raggedright\arraybackslash question: JSON object & \raggedright\arraybackslash NotFoundException \\
\hline
\end{tabular}
\end{center}

\subsection{Semantics}

\subsubsection{State Variables}

\begin{itemize}
  \item repetitionQuestions: Map(questionID, JSON object) - stores matching questions.
\end{itemize}

\subsubsection{Environment Variables}

N/A

\subsubsection{Assumptions}

\begin{itemize}
  \item Questions have a unique ID.
  \item Data is stored in a JSON format.
\end{itemize}

\subsubsection{Access Routine Semantics}

\noindent storeRepetitionQuestion():
\begin{itemize}
  \item transition: Adds the question to \texttt{repetitionQuestions}.
  \item output: Returns \texttt{true} if successfully stored.
  \item exception: Throws \texttt{StorageException} if there is a storage error.
\end{itemize}

\noindent retrieveRepetitionQuestion():
\begin{itemize}
  \item transition: None
  \item output: Returns the matching question corresponding to the questionID.
  \item exception: Throws \texttt{NotFoundException}
\end{itemize}

\subsubsection{Local Functions}

N/A

\newpage

\bibliographystyle {plainnat}
\bibliography {../../../refs/References}

\newpage

\section{Appendix} \label{Appendix}

\wss{Extra information if required}

\newpage{}

\section*{Appendix --- Reflection}

\wss{Not required for CAS 741 projects}

The information in this section will be used to evaluate the team members on the
graduate attribute of Problem Analysis and Design.

\input{../../Reflection.tex}

\begin{enumerate}
  \item What went well while writing this deliverable? 
  \item What pain points did you experience during this deliverable, and how
    did you resolve them?
  \item Which of your design decisions stemmed from speaking to your client(s)
  or a proxy (e.g. your peers, stakeholders, potential users)? For those that
  were not, why, and where did they come from?
  \item While creating the design doc, what parts of your other documents (e.g.
  requirements, hazard analysis, etc), it any, needed to be changed, and why?
  \item What are the limitations of your solution?  Put another way, given
  unlimited resources, what could you do to make the project better? (LO\_ProbSolutions)
  \item Give a brief overview of other design solutions you considered.  What
  are the benefits and tradeoffs of those other designs compared with the chosen
  design?  From all the potential options, why did you select the documented design?
  (LO\_Explores)
\end{enumerate}


\end{document}