\documentclass[12pt, titlepage]{article}

\usepackage{fullpage}
\usepackage[round]{natbib}
\usepackage{multirow}
\usepackage{multicol}
\usepackage{booktabs}
\usepackage{tabularx}
\usepackage{graphicx}
\usepackage{float}
\usepackage{hyperref}
\usepackage{longtable}
\usepackage{pdflscape}
\hypersetup{
    colorlinks,
    citecolor=blue,
    filecolor=black,
    linkcolor=red,
    urlcolor=blue
}

\input{../../Comments}
\input{../../Common}

\newcounter{acnum}
\newcommand{\actheacnum}{AC\theacnum}
\newcommand{\acref}[1]{AC\ref{#1}}

\newcounter{ucnum}
\newcommand{\uctheucnum}{UC\theucnum}
\newcommand{\uref}[1]{UC\ref{#1}}

\newcounter{mnum}
\newcommand{\mthemnum}{M\themnum}
\newcommand{\mref}[1]{M\ref{#1}}

\begin{document}

\title{Module Guide for \progname{}} 
\author{\authname}
\date{\today}

\maketitle

\pagenumbering{roman}

\section{Revision History}

\begin{tabularx}{\textwidth}{p{3cm}p{2cm}p{2cm}X}
\toprule {\bf Date} & {\bf Version} & {\bf Member} & {\bf Notes}\\
\midrule
1/13/2025 & 1.0 & Mitchell Weingust & Added 10 - Clinician Dashboard Interfaces\\
1/14/2025 & 1.1 & Mitchell Weingust & Added 12 - Timeline\\
1/14/2025 & 1.2 & Mitchell Weingust & Added 10 - Clinician Dashboard FSM\\
1/14/2025 & 1.3 & Mitchell Weingust & Added 7 - Module Decomposition\\
\bottomrule
\end{tabularx}

\newpage

\section{Reference Material}

This section records information for easy reference.

\subsection{Abbreviations and Acronyms}

\renewcommand{\arraystretch}{1.2}
\begin{tabular}{l l} 
  \toprule		
  \textbf{symbol} & \textbf{description}\\
  \midrule 
  AC & Anticipated Change\\
  DAG & Directed Acyclic Graph \\
  M & Module \\
  MG & Module Guide \\
  OS & Operating System \\
  R & Requirement\\
  SC & Scientific Computing \\
  SRS & Software Requirements Specification\\
  \progname & Explanation of program name\\
  UC & Unlikely Change \\
  FSM & Finite State Machine \\
  \wss{etc.} & \wss{...}\\
  \bottomrule
\end{tabular}\\

\newpage

\tableofcontents

\listoftables

\listoffigures

\newpage

\pagenumbering{arabic}

\section{Introduction}

Decomposing a system into modules is a commonly accepted approach to developing
software.  A module is a work assignment for a programmer or programming
team~\citep{ParnasEtAl1984}.  We advocate a decomposition
based on the principle of information hiding~\citep{Parnas1972a}.  This
principle supports design for change, because the ``secrets'' that each module
hides represent likely future changes.  Design for change is valuable in SC,
where modifications are frequent, especially during initial development as the
solution space is explored.  

Our design follows the rules layed out by \citet{ParnasEtAl1984}, as follows:
\begin{itemize}
\item System details that are likely to change independently should be the
  secrets of separate modules.
\item Each data structure is implemented in only one module.
\item Any other program that requires information stored in a module's data
  structures must obtain it by calling access programs belonging to that module.
\end{itemize}

After completing the first stage of the design, the Software Requirements
Specification (SRS), the Module Guide (MG) is developed~\citep{ParnasEtAl1984}. The MG
specifies the modular structure of the system and is intended to allow both
designers and maintainers to easily identify the parts of the software.  The
potential readers of this document are as follows:

\begin{itemize}
\item New project members: This document can be a guide for a new project member
  to easily understand the overall structure and quickly find the
  relevant modules they are searching for.
\item Maintainers: The hierarchical structure of the module guide improves the
  maintainers' understanding when they need to make changes to the system. It is
  important for a maintainer to update the relevant sections of the document
  after changes have been made.
\item Designers: Once the module guide has been written, it can be used to
  check for consistency, feasibility, and flexibility. Designers can verify the
  system in various ways, such as consistency among modules, feasibility of the
  decomposition, and flexibility of the design.
\end{itemize}

The rest of the document is organized as follows. Section
\ref{SecChange} lists the anticipated and unlikely changes of the software
requirements. Section \ref{SecMH} summarizes the module decomposition that
was constructed according to the likely changes. Section \ref{SecConnection}
specifies the connections between the software requirements and the
modules. Section \ref{SecMD} gives a detailed description of the
modules. Section \ref{SecTM} includes two traceability matrices. One checks
the completeness of the design against the requirements provided in the SRS. The
other shows the relation between anticipated changes and the modules. Section
\ref{SecUse} describes the use relation between modules.

\section{Anticipated and Unlikely Changes} \label{SecChange}

This section lists possible changes to the system. According to the likeliness
of the change, the possible changes are classified into two
categories. Anticipated changes are listed in Section \ref{SecAchange}, and
unlikely changes are listed in Section \ref{SecUchange}.

\subsection{Anticipated Changes} \label{SecAchange}

Anticipated changes are the source of the information that is to be hidden
inside the modules. Ideally, changing one of the anticipated changes will only
require changing the one module that hides the associated decision. The approach
adapted here is called design for
change.

\begin{description}
\item[\refstepcounter{acnum} \actheacnum \label{acHardware}:] The specific
  hardware on which the software is running.
\item[\refstepcounter{acnum} \actheacnum \label{acInput}:] The format of the
  initial input data.
\item ...
\end{description}

\wss{Anticipated changes relate to changes that would be made in requirements,
design or implementation choices.  They are not related to changes that are made
at run-time, like the values of parameters.}

\subsection{Unlikely Changes} \label{SecUchange}

The module design should be as general as possible. However, a general system is
more complex. Sometimes this complexity is not necessary. Fixing some design
decisions at the system architecture stage can simplify the software design. If
these decision should later need to be changed, then many parts of the design
will potentially need to be modified. Hence, it is not intended that these
decisions will be changed.

\begin{description}
\item[\refstepcounter{ucnum} \uctheucnum \label{ucIO}:] Input/Output devices
  (Input: File and/or Keyboard, Output: File, Memory, and/or Screen).
\item ...
\end{description}

\section{Module Hierarchy} \label{SecMH}

This section provides an overview of the module design. Modules are summarized
in a hierarchy decomposed by secrets in Table \ref{TblMH}. The modules listed
below, which are leaves in the hierarchy tree, are the modules that will
actually be implemented.

\begin{multicols}{2}
  \begin{description}
    \item [\refstepcounter{mnum} \mthemnum \label{mClinicianGUI}:] Clinician GUI Module
    \item [\refstepcounter{mnum} \mthemnum \label{mParentGUI}:] Parent GUI Module
    \item [\refstepcounter{mnum} \mthemnum \label{mAppController}:] App Controller Module
    \item [\refstepcounter{mnum} \mthemnum \label{mAPIGateway}:] API Gateway Module
    \item [\refstepcounter{mnum} \mthemnum \label{mAuth}:] Authentication Module
    \item [\refstepcounter{mnum} \mthemnum \label{mResultStorage}:] Result Storage Module
    \item [\refstepcounter{mnum} \mthemnum \label{mMediaProcessing}:] Media Processing Module
    \item [\refstepcounter{mnum} \mthemnum \label{mLogging}:] Logging Module
    \item [\refstepcounter{mnum} \mthemnum \label{mQuestionBank}:] Question Bank Module
    \item [\refstepcounter{mnum} \mthemnum \label{mRealTimeFeedback}:] Real-Time Feedback Module
    \item [\refstepcounter{mnum} \mthemnum \label{mReportGeneration}:] Report Generation Module
    \item [\refstepcounter{mnum} \mthemnum \label{mVideoProcessing}:] Video Processing Module
    \item [\refstepcounter{mnum} \mthemnum \label{mAudioProcessing}:] Audio Processing Module
    \item [\refstepcounter{mnum} \mthemnum \label{mEnglishBank}:] English Question Bank Module
    \item [\refstepcounter{mnum} \mthemnum \label{mMandarinBank}:] Mandarin Question Bank Module
    \item [\refstepcounter{mnum} \mthemnum \label{mMatchingBank1}:] Matching Question Bank Module 
    \item [\refstepcounter{mnum} \mthemnum \label{mRepetitionBank1}:] Repetition Question Bank Module
  \end{description}
  \end{multicols}

\begin{description}
\item [\refstepcounter{mnum} \mthemnum \label{mHH}:] Hardware-Hiding Module
\item ...
\end{description}


\begin{table}[h!]
\centering
\begin{tabular}{p{0.3\textwidth} p{0.6\textwidth}}
\toprule
\textbf{Level 1} & \textbf{Level 2}\\
\midrule

{Hardware-Hiding Module} & ~ \\
\midrule

\multirow{7}{0.3\textwidth}{Behaviour-Hiding Module} & ?\\
& ?\\
& ?\\
& ?\\
& ?\\
& ?\\
& ?\\ 
& ?\\
\midrule

\multirow{3}{0.3\textwidth}{Software Decision Module} & {?}\\
& ?\\
& ?\\
\bottomrule

\end{tabular}
\caption{Module Hierarchy}
\label{TblMH}
\end{table}

\section{Connection Between Requirements and Design} \label{SecConnection}

The design of the system is intended to satisfy the requirements developed in
the SRS. In this stage, the system is decomposed into modules. The connection
between requirements and modules is listed in Table~\ref{TblRT}.

\wss{The intention of this section is to document decisions that are made
  ``between'' the requirements and the design.  To satisfy some requirements,
  design decisions need to be made.  Rather than make these decisions implicit,
  they are explicitly recorded here.  For instance, if a program has security
  requirements, a specific design decision may be made to satisfy those
  requirements with a password.}

\section{Module Decomposition} \label{SecMD}

Modules are decomposed according to the principle of ``information hiding''
proposed by \citet{ParnasEtAl1984}. The \emph{Secrets} field in a module
decomposition is a brief statement of the design decision hidden by the
module. The \emph{Services} field specifies \emph{what} the module will do
without documenting \emph{how} to do it. For each module, a suggestion for the
implementing software is given under the \emph{Implemented By} title. If the
entry is \emph{OS}, this means that the module is provided by the operating
system or by standard programming language libraries.  \emph{\progname{}} means the
module will be implemented by the \progname{} software.

Only the leaf modules in the hierarchy have to be implemented. If a dash
(\emph{--}) is shown, this means that the module is not a leaf and will not have
to be implemented.

\subsection{Hardware Hiding Modules (\mref{mHH})}

\begin{description}
\item[Secrets:]The data structure and algorithm used to implement the virtual
  hardware.
\item[Services:]Serves as a virtual hardware used by the rest of the
  system. This module provides the interface between the hardware and the
  software. So, the system can use it to display outputs or to accept inputs.
\item[Implemented By:] OS
\end{description}

\subsection{Behaviour-Hiding Module}

\begin{description}
\item[Secrets:]The contents of the required behaviours.
\item[Services:]Includes programs that provide externally visible behaviour of
  the system as specified in the software requirements specification (SRS)
  documents. This module serves as a communication layer between the
  hardware-hiding module and the software decision module. The programs in this
  module will need to change if there are changes in the SRS.
\item[Implemented By:] --
\end{description}

\subsubsection{Clinician GUI (\mref{mClinicianGUI})}

\begin{description}
\item[Secrets:]The interactive and visual components that allow Clinicians to interact with the system, through the App Controller (\mref{mAppController}),
               to access patient data and information, and make informed decisions.
\item[Services:]To show application functionality to clinicians, accepting user inputs (choosing assessments to review,
                flagging bias questions) and displaying outputs (assessment summaries).
\item[Implemented By:] [Your Program Name Here]
\item[Type of Module:] Library
\end{description}

\subsubsection{Parent GUI (\mref{mParentGUI})}

\begin{description}
\item[Secrets:]The interactive and visual components that allow Parents to interact with the system, through the App Controller (\mref{mAppController}),
               to setup and engage in the assessment with their child.
\item[Services:] To show application functionality to parents, accepting user inputs (selecting answers to questions,
                 completing setup) and displaying outputs (question visuals, button selections).
\item[Implemented By:] [Your Program Name Here]
\item[Type of Module:] Library
\end{description}

\subsubsection{Etc.}


\subsection{Software Decision Module}

\begin{description}
\item[Secrets:] The design decision based on mathematical theorems, physical
  facts, or programming considerations. The secrets of this module are
  \emph{not} described in the SRS.
\item[Services:] Includes data structure and algorithms used in the system that
  do not provide direct interaction with the user. 
  % Changes in these modules are more likely to be motivated by a desire to
  % improve performance than by externally imposed changes.
\item[Implemented By:] --
\end{description}

\subsubsection{AppController Module (\mref{mAppController})}

\begin{description}
  \item[Secrets:]The interactions between the GUIs (\mref{mClinicianGUI}, \mref{mParentGUI}) and the API Gateway (\mref{mAPIGateway}), acting as a means to interface with the software modules.
  \item[Services:]Enables the user to pass information from the GUIs to the backend services.
  \item[Implemented By:] [Your Program Name Here]
  \item[Type of Module:] Library
  \end{description}

  \subsubsection{API Gateway Module (\mref{mAPIGateway})}

  \begin{description}
    \item[Secrets:]The interactions between the App Controller (\mref{mAppController}) and the inter-dependencies of all other software modules, including inherited modules (
    \mref{mResultStorage},
    \mref{mMediaProcessing},
    \mref{mLogging},
    \mref{mQuestionBank},
    \mref{mRealTimeFeedback},
    \mref{mReportGeneration},
    \mref{mAudioProcessing},
    \mref{mEnglishBank},
    \mref{mMandarinBank},
    \mref{mMatchingBank1},
    \mref{mRepetitionBank1}).
    \item[Services:]Enables the user to access the system and interact with its components, consisting of the Patient, Client, and Admin views.
    \item[Implemented By:] [Your Program Name Here]
    \item[Type of Module:] Library
    \end{description}

\section{Traceability Matrix} \label{SecTM}

This section shows two traceability matrices: between the modules and the
requirements and between the modules and the anticipated changes.

% the table should use mref, the requirements should be named, use something
% like fref
\begin{table}[H]
\centering
\begin{tabular}{p{0.2\textwidth} p{0.6\textwidth}}
\toprule
\textbf{Req.} & \textbf{Modules}\\
\midrule
R1 & \mref{mHH}, \mref{mInput}, \mref{mParams}, \mref{mControl}\\
R2 & \mref{mInput}, \mref{mParams}\\
R3 & \mref{mVerify}\\
R4 & \mref{mOutput}, \mref{mControl}\\
R5 & \mref{mOutput}, \mref{mODEs}, \mref{mControl}, \mref{mSeqDS}, \mref{mSolver}, \mref{mPlot}\\
R6 & \mref{mOutput}, \mref{mODEs}, \mref{mControl}, \mref{mSeqDS}, \mref{mSolver}, \mref{mPlot}\\
R7 & \mref{mOutput}, \mref{mEnergy}, \mref{mControl}, \mref{mSeqDS}, \mref{mPlot}\\
R8 & \mref{mOutput}, \mref{mEnergy}, \mref{mControl}, \mref{mSeqDS}, \mref{mPlot}\\
R9 & \mref{mVerifyOut}\\
R10 & \mref{mOutput}, \mref{mODEs}, \mref{mControl}\\
R11 & \mref{mOutput}, \mref{mODEs}, \mref{mEnergy}, \mref{mControl}\\
\bottomrule
\end{tabular}
\caption{Trace Between Requirements and Modules}
\label{TblRT}
\end{table}

\begin{table}[H]
\centering
\begin{tabular}{p{0.2\textwidth} p{0.6\textwidth}}
\toprule
\textbf{AC} & \textbf{Modules}\\
\midrule
\acref{acHardware} & \mref{mHH}\\
\acref{acInput} & \mref{mInput}\\
\acref{acParams} & \mref{mParams}\\
\acref{acVerify} & \mref{mVerify}\\
\acref{acOutput} & \mref{mOutput}\\
\acref{acVerifyOut} & \mref{mVerifyOut}\\
\acref{acODEs} & \mref{mODEs}\\
\acref{acEnergy} & \mref{mEnergy}\\
\acref{acControl} & \mref{mControl}\\
\acref{acSeqDS} & \mref{mSeqDS}\\
\acref{acSolver} & \mref{mSolver}\\
\acref{acPlot} & \mref{mPlot}\\
\bottomrule
\end{tabular}
\caption{Trace Between Anticipated Changes and Modules}
\label{TblACT}
\end{table}

\section{Use Hierarchy Between Modules} \label{SecUse}

In this section, the uses hierarchy between modules is
provided. \citet{Parnas1978} said of two programs A and B that A {\em uses} B if
correct execution of B may be necessary for A to complete the task described in
its specification. That is, A {\em uses} B if there exist situations in which
the correct functioning of A depends upon the availability of a correct
implementation of B.  Figure \ref{FigUH} illustrates the use relation between
the modules. It can be seen that the graph is a directed acyclic graph
(DAG). Each level of the hierarchy offers a testable and usable subset of the
system, and modules in the higher level of the hierarchy are essentially simpler
because they use modules from the lower levels.

\wss{The uses relation is not a data flow diagram.  In the code there will often
be an import statement in module A when it directly uses module B.  Module B
provides the services that module A needs.  The code for module A needs to be
able to see these services (hence the import statement).  Since the uses
relation is transitive, there is a use relation without an import, but the
arrows in the diagram typically correspond to the presence of import statement.}

\wss{If module A uses module B, the arrow is directed from A to B.}

\begin{figure}[H]
\centering
%\includegraphics[width=0.7\textwidth]{UsesHierarchy.png}
\caption{Use hierarchy among modules}
\label{FigUH}
\end{figure}

%\section*{References}

\section{User Interfaces}

\wss{Design of user interface for software and hardware.  Attach an appendix if
needed. Drawings, Sketches, Figma}

\hspace{1.5em}The interface below depicts the initial interface a clinician would see upon logging into their account in the system.
\begin{figure}[H]
  \centering
  \includegraphics[scale=0.9]{images/Clinician-Dashboard.png}
  \caption{Clinician Dashboard}
\end{figure}

\hspace{1.5em}The interface below depicts the interface a clinician would see upon selecting the Add Client button on the previous Clinician Dashboard screen.
\begin{figure}[H]
  \centering
  \includegraphics[scale=0.9]{images/Add-Client.png}
  \caption{Add Client}
\end{figure}

\hspace{1.5em}The interface below depicts the patient overview, which can be reached from the Clinician Dashboard by selecting a name from the client list.
\begin{figure}[H]
  \centering
  \includegraphics[scale=0.9]{images/Patient-Overview.png}
  \caption{Patient Overview}
\end{figure}

\hspace{1.5em}The interface below depicts the patient assessment results analysis, which can be reached from the Patient Overview by selecting an assessment date from the list of assessments.
\begin{figure}[H]
  \centering
  \includegraphics[scale=0.9]{images/Patient-Assessment-Results-Analysis-1.png}
  \caption{Patient Assessment Results Analysis (1)}
\end{figure}

\hspace{1.5em}The interface below depicts a continuation of the patient assessment results analysis, which can be reached from the previous figure, by scrolling the scrollbar on the right edge of the screen.
\begin{figure}[H]
  \centering
  \includegraphics[scale=0.9]{images/Patient-Assessment-Results-Analysis-2.png}
  \caption{Patient Assessment Results Analysis (2)}
\end{figure}

\hspace{1.5em}The interface below depicts the bias review, which can be reached from the Patient Assessment Results Analysis by selecting Review on any of the questions on an assessment.
\begin{figure}[H]
  \centering
  \includegraphics[scale=0.9]{images/Review-Bias.png}
  \caption{Bias Review}
\end{figure}

\hspace{1.5em}The interface below depicts a question review page, where no bias has been detected. The ability to Flag Bias is present in the bottom right corner, to give the Clinician the ability to manually reflect bias in a question.
\begin{figure}[H]
  \centering
  \includegraphics[scale=0.9]{images/Flag-Bias.png}
  \caption{Flag Bias}
\end{figure}

\newpage

\hspace{1.5em}The below finite state machine depicts how the overall system can be interacted with, as well as which actions lead to changes in states in the system. Included in this Finite State Machine
              are Clinician Dashboard and Assessment, which are further broken down in the following Figures.
\begin{figure}[H]
  \centering
  \includegraphics[scale=0.6]{images/state_diagram.drawio.png}
  \caption{FSM - TeleHealth Insights System}
\end{figure}

\newpage

\hspace{1.5em}The below finite state machine depicts how the clinician can interface with the dashboard, as well as which interactions lead to changes in states in the system.
\begin{figure}[H]
  \centering
  \includegraphics[scale=0.6]{images/FSM_Clinician_Dashboard.png}
  \caption{FSM - Clinician Dashboard}
\end{figure}

\newpage

\section{Design of Communication Protocols}

N/A

\newpage
\begin{landscape}
\section{Timeline}
\scriptsize
\begin{longtable}{|c|c|c|c|c|c|c|c|}
  \hline
      \textbf{Milestone} & \textbf{Module/Pages} & \textbf{Objective} & \textbf{Mitchell} & \textbf{Parisha} & \textbf{Promish} & \textbf{Jasmine} & \textbf{Date} \\ \hline
      Assessment & Question Bank Module & ~ & X & X & ~ & ~ & 1/19/25 \\ \hline
      Assessment & English Question Bank  Module & ~ & X & X & ~ & ~ & 1/19/25 \\ \hline
      Assessment & Matching Question Bank  Module & ~ & X & X & ~ & ~ & 1/19/25 \\ \hline
      Assessment & Repetition Question Bank  Module & ~ & X & X & ~ & ~ & 1/19/25 \\ \hline
      Assessment & ~ & Verification and Validation Testing & X & X & X & X & 1/19/25 \\ \hline
      Controllers & API Gateway & ~ & ~ & ~ & X & ~ & 1/19/25 \\ \hline
      Controllers & App Controller & ~ & ~ & ~ & X & ~ & 1/19/25 \\ \hline
      Controllers & ~ & Verification and Validation Testing & X & X & X & X & 1/19/25 \\ \hline
      Assessment GUI & Assessment Selection Page & ~ & ~ & ~ & ~ & X & 1/19/25 \\ \hline
      Assessment GUI & Parent Checklist Page & ~ & ~ & X & ~ & ~ & 1/22/25 \\ \hline
      Assessment GUI & Input Check Page & ~ & ~ & ~ & ~ & X & 1/22/25 \\ \hline
      Assessment GUI & Assessment Questions Page & ~ & X & ~ & ~ & ~ & 1/22/25 \\ \hline
      Clinician Dashboard & Result Storage Module & ~ & ~ & ~ & X & ~ & 1/22/25 \\ \hline
      Assessment GUI & Assessment Instructions Page & ~ & ~ & X & ~ & ~ & 1/25/25 \\ \hline
      Assessment GUI & Tutorial Page & ~ & X & ~ & ~ & ~ & 1/25/25 \\ \hline
      Assessment GUI & Assessment Completion Page & ~ & ~ & ~ & ~ & X & 1/25/25 \\ \hline
      Assessment GUI & ~ & Verification and Validation Testing & X & X & X & X & 1/25/25 \\ \hline
      Clinician Dashboard & Report Generation Module & ~ & ~ & ~ & X & ~ & 1/25/25 \\ \hline
      Clinician Dashboard & ~ & Verification and Validation Testing & X & X & X & X & 1/25/25 \\ \hline
      Clinician Dashboard GUI & Clinician Dashboard Overview Page & ~ & X & ~ & ~ & ~ & 1/28/25 \\ \hline
      Clinician Dashboard GUI & Patient Overview Page & ~ & ~ & X & ~ & ~ & 1/28/25 \\ \hline
      Clinician Dashboard GUI & Patient Assessment Results Analysis Page & ~ & ~ & ~ & ~ & X & 1/28/25 \\ \hline
      Media Processing & Media Processing Module & ~ & ~ & ~ & X & ~ & 1/28/25 \\ \hline
      Clinician Dashboard GUI & Bias Review Page & ~ & ~ & ~ & ~ & X & 1/31/25 \\ \hline
      Clinician Dashboard GUI & Add New Client Page & ~ & X & ~ & ~ & ~ & 1/31/25 \\ \hline
      Clinician Dashboard GUI & ~ & Verification and Validation Testing & X & X & X & X & 1/31/25 \\ \hline
      Homepage & Authentication Module & ~ & ~ & ~ & X & ~ & 1/31/25 \\ \hline
      Homepage GUI & Select Account Type Page & ~ & ~ & X & ~ & ~ & 1/31/25 \\ \hline
      Homepage GUI & Login Page (Parent) Page & ~ & ~ & X & ~ & ~ & 2/3/25 \\ \hline
      Homepage GUI & Login Page (Clinician) Page & ~ & ~ & X & ~ & ~ & 2/3/25 \\ \hline
      Homepage GUI & Create Account Page & ~ & X & ~ & ~ & ~ & 2/3/25 \\ \hline
      Homepage GUI & Homepage (Parent) Page & ~ & ~ & ~ & ~ & X & 2/3/25 \\ \hline
      Homepage GUI & ~ & Verification and Validation Testing & X & X & X & X & 2/3/25 \\ \hline
      Media Processing & Video Processing Module & ~ & ~ & ~ & X & ~ & 2/3/25 \\ \hline
      Media Processing & Audio Processing Module & ~ & X & ~ & ~ & ~ & 2/6/25 \\ \hline
      Media Processing & ~ & Verification and Validation Testing & X & X & X & X & 2/6/25 \\ \hline
      Miscellaneous & Logging Module & ~ & ~ & X & ~ & ~ & 2/6/25 \\ \hline
      Miscellaneous & Real-Time Feedback Module & ~ & ~ & ~ & X & ~ & 2/6/25 \\ \hline
      Miscellaneous & ~ & Verification and Validation Testing & X & X & X & X & 2/6/25 \\ \hline
      Admin & Add Clinician Page & ~ & ~ & ~ & ~ & X & 2/6/25 \\ \hline
      Admin & ~ & Verification and Validation Testing & X & X & X & X & 2/6/25 \\ \hline
      Rev0 & ~ & Full System Testing & X & X & X & X & 2/8/25 \\ \hline
      Rev0 & ~ & Rev0 Practice & X & X & X & X & 2/9/25 \\ \hline
      Rev0 & ~ & Rev0 Presentation & X & X & X & X & 2/10/25 \\ \hline
\end{longtable}
\normalsize
\end{landscape}

\bibliographystyle {plainnat}
\bibliography{../../../refs/References}

\newpage{}

\end{document}