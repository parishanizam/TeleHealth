\documentclass{article}

\usepackage{booktabs}
\usepackage{tabularx}
\usepackage{hyperref}
\usepackage{comment}
\usepackage{enumerate}
\usepackage{adjustbox}
\usepackage{booktabs}
\usepackage{multirow}
\usepackage{makecell}
\usepackage{geometry}
\usepackage{graphicx}
\usepackage[shortlabels]{enumitem}
\usepackage{float}
\usepackage{array}
\usepackage{pdflscape}
\usepackage{tabularx,ragged2e,booktabs,caption}
\usepackage{longtable}
\usepackage{ulem}

\hypersetup{
    colorlinks=true,       % false: boxed links; true: colored links
    linkcolor=red,          % color of internal links (change box color with linkbordercolor)
    citecolor=green,        % color of links to bibliography
    filecolor=magenta,      % color of file links
    urlcolor=cyan           % color of external links
}

\title{Hazard Analysis\\\progname}

\author{\authname}

\date{}

%% Comments

\usepackage{color}

\newif\ifcomments\commentstrue %displays comments
%\newif\ifcomments\commentsfalse %so that comments do not display

\ifcomments
\newcommand{\authornote}[3]{\textcolor{#1}{[#3 ---#2]}}
\newcommand{\todo}[1]{\textcolor{red}{[TODO: #1]}}
\else
\newcommand{\authornote}[3]{}
\newcommand{\todo}[1]{}
\fi

\newcommand{\wss}[1]{\authornote{blue}{SS}{#1}} 
\newcommand{\plt}[1]{\authornote{magenta}{TPLT}{#1}} %For explanation of the template
\newcommand{\an}[1]{\authornote{cyan}{Author}{#1}}

%% Common Parts

\newcommand{\progname}{Software Engineering} % PUT YOUR PROGRAM NAME HERE
\newcommand{\authname}{Team \#22, TeleHealth Insights
\\ Mitchell Weingust
\\ Parisha Nizam
\\ Promish Kandel
\\ Jasmine Sun-Hu} % AUTHOR NAMES                  

\usepackage{hyperref}
    \hypersetup{colorlinks=true, linkcolor=blue, citecolor=blue, filecolor=blue,
                urlcolor=blue, unicode=false}
    \urlstyle{same}
                                


\begin{document}

\maketitle
\thispagestyle{empty}

~\newpage

\pagenumbering{roman}

\begin{table}[hp]
\caption{Revision History} \label{TblRevisionHistory}
\begin{tabularx}{\textwidth}{p{1.5cm}p{1cm}p{3.5cm}X}
\toprule {\textbf{Date}} & {\textbf{Vers.}} & {\textbf{Contributors}} & {\textbf{Notes}}\\
\midrule
10/24/2024 & 1.0 & Jasmine Sun-Hu & Added Sections 1,2,3\\
10/25/2024 & 1.1 & Jasmine Sun-Hu & Added Drafts of Section 4, Reflection\\
10/25/2024 & 1.2 & Promish Kandel & Added FMEA Table, Sections 6,7, Reflection\\
10/25/2024 & 1.3 & Mitchell Weingust & Document Review, Added List of Tables/Figures, Fixed Grammar\\
\bottomrule
\end{tabularx}
\end{table}

~\newpage

\tableofcontents
\listoffigures
\listoftables

~\newpage

\pagenumbering{arabic}

\section{Introduction}

\hspace{1.5em} This document outlines the hazard analysis for TeleHealth Insights, an at-home bilingual speech 
assessment system with video and audio analysis features. The system will provide clear guidance to
parents when administering the assessment to their children so speech language pathologists can assess and support
their patients' speech and language development remotely. A hazard is defined as a 
property or condition in a system, that when combined with a condition in the environment, has the potential to cause harm or 
damage. A hazard is not limited to safety, it can also be related to system security, user sensitivity and 
unexpected human or technology interactions. The purpose of this document is to identify any hazards to the project, and 
develop new safety and security requirements from a Failure Mode and Effect Analysis (FMEA).

\section{Scope and Purpose of Hazard Analysis}

\hspace{1.5em} The hazard analysis focuses on identifying, evaluating, and mitigating any hazards that could negatively 
impact the speech language assessment platform. This includes both technical and user interaction hazards,
as the project will include handling sensitive patient data. The analysis will cover a variety of aspects of the system, such as 
data handling, software stability and user sensitivity. \\
\indent The purpose of the hazard analysis is to identify any risks that could 
affect data privacy and security, system reliability, data collection accuracy, and compliance with relevant standards. A 
hazard analysis minimizes these risks, as the loss from unaddressed hazards could involve patient safety, data breaches, 
and legal or financial consequences for the organization administering the assessment.

\section{System Boundaries and Components}

The system referred to throughout the document consists of several major components: 
\begin{enumerate}
    \item \textbf{User Interface (UI)}: The front-end platform where users interact with the system. It allows users to navigate through the assessment, accept inputs, and display results.
    \item \textbf{Backend Server}: The back-end platform handles data collection and processing, business logic, and communication between all system components.
        \begin{itemize}
            \item \textbf{Authentication}: Manages the login and access control mechanisms.
        \end{itemize}
    \item Assessment Recording
        \begin{itemize}
            \item \textbf{Video Recording Module}: Responsible for capturing and transmitting video data during an assessment session.
            \item \textbf{Audio Recording Module}: Responsible for capturing and transmitting audio data during an assessment session.
        \end{itemize}
    \item Assessment Analysis
        \begin{itemize}
            \item \textbf{Video Analysis Model}: Processes and analyzes the video recording for disturbances and other behaviours against assessment instructions.
            \item \textbf{Audio Analysis Model}: Processes and analyzes the audio recording for disturbances and other behaviours against assessment instructions.
        \end{itemize}
    \item \textbf{Database}: A centralized storage for all assessment results, recordings, analysis results, user data, and any other data as necessary.
\end{enumerate}

The system boundary for this project includes the entire platform, consisting of the user interface, backend server, 
assessment recording, assessment analysis, and the database. Components such as the user's device (e.g. computer or 
tablet used for the assessment), and any third-party services used, are external to the system and outside the control of 
the capstone team. Therefore, they will not be directly considered in the hazard analysis. The connections to external hardware, such as a webcam, speakers, or microphone,
will be considered within the system boundary and may be included in the hazard analysis.

\section{Critical Assumptions}
\begin{enumerate}
    \item Users have reliable internet access while using the web application.
    \item The user's device is compatible with the web application and has the necessary capabilities and system requirements to run the assessment session.
    \item Users will keep their account details confidential.
    \item Any third-party hosting services (e.g. cloud storage or hosting) used by the backend server are reliable and secure according to industry standards.
    \item When a user is taking the assessment they will have a functional microphone and camera that meet the minimum requirements for recording assessment sessions.
\end{enumerate}

\section{Failure Mode and Effect Analysis}
\subsection{Hazards Out of Scope}
The following are hazards that could occur outside the control of the system, thus they can't be fixed or mitigated.
\begin{itemize}
    \item Wi-Fi shuts down during assessment
    \item User hardware malfunctions during assessment
\end{itemize}
\subsection{Failure Mode \& Effect Analysis Table}
The following FMEA table is a breakdown of the hazards that could occur within the system, along with recommended actions to mitigate them.

\newgeometry{margin=0.5in}
\begin{landscape}
\begin{longtable}{|>{\raggedright\arraybackslash\centering}p{2.5cm}|>{\raggedright\arraybackslash}p{3.5cm}|>{\raggedright\arraybackslash}p{3.5cm}|>{\raggedright\arraybackslash}p{3.5cm}|>{\raggedright\arraybackslash}p{4.5cm}|>{\raggedright\arraybackslash}p{2.8cm}|>{\raggedright\arraybackslash}p{2cm}|}
\caption{Failure Mode and Effect Analysis} \label{FMEA}\\
\hline
 Component & Failure Modes & Effects of Failure & Causes of Failure & Recommended Action & SR & Ref.  \\
 \endfirsthead
 \multicolumn{7}{c}
 {Table \thetable\ Continued from previous page}\\
 \hline
 Component & Failure Modes & Effects of Failure & Causes of Failure & Recommended Action & SR & Ref.  \\
 \endhead
 \multicolumn{7}{r}{{Continued on next page}} \\
\endfoot
\multicolumn{7}{r}{{Concluded}} \\
\endlastfoot
 \hline
 Database
 & 
 \begin{enumerate}
    \item SQL injection attack
    \item Unauthorized access
 \end{enumerate}
 & 
  \begin{enumerate}
    \item Loss of confidentiality, integrity and availability of user data and assessment data.
    \item Breach of sensitive patient data, violation of HIPAA.
 \end{enumerate}
& 
  \begin{enumerate}
     \item Inadequate input validation or un-parameterized SQL queries.
     \item Administrative access is not validated and user accesses database directly.
 \end{enumerate}
&
  \begin{enumerate}
     \item Implement periodic data backups, prioritize and implement thorough database access controls, and use parameterized queries.
     \item Add multi-factor authentication.
 \end{enumerate}

&  
\begin{enumerate}
     \item PR-RFT2, PR-RFT3
     \item SR-AC3, SR-AC4
 \end{enumerate}
&
\begin{enumerate}
     \item HA-D1
     \item HA-D2
 \end{enumerate}
 \\
 \hline
 Authentication
 & 
 \begin{enumerate}
    \item Parent gets clinician level access
    \item Users can't login
 \end{enumerate}
 & 
  \begin{enumerate}
    \item Unauthorized access to sensitive patient data, leading to potential HIPAA violations and data breaches.
    \item Users are unable to access their accounts or data, leading to poor user experience.
 \end{enumerate}
& 
  \begin{enumerate}
     \item Improper role-based access control implementation resulting in errors in user role assignment.
     \item Errors in authentication logic or server downtime.
 \end{enumerate}
&
  \begin{enumerate}
     \item Do regular access audits to ensure clear separation of user roles.
     \item Implement a fallback login and add error handling for feedback.
 \end{enumerate}

&  
\begin{enumerate}
     \item FR-A1,FR-A3
     \item FR-A2,FR-A4
 \end{enumerate}
&
\begin{enumerate}
     \item HA-A1
     \item HA-A2
 \end{enumerate}
 \\
 \hline
 Video Analysis Model
 & 
 \begin{enumerate}
    \item Model cannot access video recording
    \item Model cannot detect user actions during analysis
 \end{enumerate}
 & 
  \begin{enumerate}
    \item Video-based analysis is incomplete or fails, which may hinder decision-making based on video data.
    \item Reduced accuracy in behaviour detection or activity recognition.
 \end{enumerate}
& 
  \begin{enumerate}
     \item Missing file permissions, incorrect file paths, or server-side issues.
     \item Insufficient training data, low video resolution, or model overfitting to specific data types.
 \end{enumerate}
&
  \begin{enumerate}
     \item Validate file paths before processing, ensure proper access permissions, and log all access attempts for debugging.
     \item Retrain model with more diverse data, improve preprocessing techniques like video upscaling, and evaluate model performance.
 \end{enumerate}

&  
\begin{enumerate}
     \item FR-VADA1
     \item FR-VADA3
 \end{enumerate}
&
\begin{enumerate}
     \item HA-VAM1
     \item HA-VAM2
 \end{enumerate}
 \\
 \hline
 Audio Analysis Model
 & 
 \begin{enumerate}
    \item Model cannot access audio recording
    \item Model cannot detect audio cues during analysis
 \end{enumerate}
 & 
  \begin{enumerate}
    \item Audio-based analysis is incomplete or fails, impacting the overall data analysis outcome.
    \item Missed events or actions during analysis.
 \end{enumerate}
& 
  \begin{enumerate}
     \item File corruption, incorrect file format, or lack of access permissions.
     \item Inadequate training on diverse audio samples or background noise interference.
 \end{enumerate}
&
  \begin{enumerate}
     \item Validate audio files before analysis, provide user guidelines for supported formats, and log access errors.
     \item Use noise reduction preprocessing, retrain the model with varied audio data.
 \end{enumerate}

&  
\begin{enumerate}
     \item FR-VADA1
     \item FR-VADA3
 \end{enumerate}
&
\begin{enumerate}
     \item HA-AAM1
     \item HA-AAM2
 \end{enumerate}
 \\
 \hline
 Video Recording
 & 
 \begin{enumerate}
    \item Video recording is blurry
 \end{enumerate}
 & 
  \begin{enumerate}
    \item The video analysis model may miss critical details, leading to inaccurate analysis.
 \end{enumerate}
& 
  \begin{enumerate}
     \item Low-resolution recording settings, poor camera quality, or motion blur.
 \end{enumerate}
&
  \begin{enumerate}
     \item Apply post-processing filters.
 \end{enumerate}

&  
\begin{enumerate}
     \item FR-SS3
 \end{enumerate}
&
\begin{enumerate}
     \item HA-VR1
 \end{enumerate}
 \\
 \hline
 Audio Recording 
 & 
 \begin{enumerate}
    \item Audio recording has background noise interference
 \end{enumerate}
 & 
  \begin{enumerate}
    \item Poor quality audio makes it difficult for the model to detect speech or audio events accurately.
 \end{enumerate}
& 
  \begin{enumerate}
     \item Faulty recording equipment, interference, or poor recording environment.
 \end{enumerate}
&
  \begin{enumerate}
     \item Filter noise using software tools, and provide best practices for recording.
 \end{enumerate}

&  
\begin{enumerate}
     \item FR-SS2
 \end{enumerate}
&
\begin{enumerate}
     \item HA-AR1
 \end{enumerate}
 \\
 \hline
 Backend Server
 & 
 \begin{enumerate}
    \item Data loss during processing
    \item Server crashes due to user overload
 \end{enumerate}
 & 
  \begin{enumerate}
    \item Partial or complete loss of data during video/audio processing could result in incomplete analysis
    \item Users may be unable to complete the assessment, or the server crashing could destroy user data.
 \end{enumerate}
& 
  \begin{enumerate}
     \item Server overload or incorrect handling of data transfer.
     \item High traffic overload, memory leaks, or unhandled exceptions.
 \end{enumerate}
&
  \begin{enumerate}
     \item Use robust data storage solutions such as a temporary cache before saving.
     \item Monitor server health and use proper exception handling to manage unexpected errors.
 \end{enumerate}

&  
\begin{enumerate}
     \item FR-DSC1, FR-DSC2
     \item PR-CR1, PR-CR2, PR-CR3
 \end{enumerate}
&
\begin{enumerate}
     \item HA-BS1
     \item HA-BS2
 \end{enumerate}
 \\
 \hline
 User Interface
 & 
 \begin{enumerate}
    \item Error in navigation structure/flow
    \item Button components aren't clickable
 \end{enumerate}
 & 
  \begin{enumerate}
    \item Users cannot move through the application smoothly, leading to frustration and a poor user experience.
    \item Users cannot complete quizzes, or perform interactions with clickable objects and are unable to proceed through the interface.
 \end{enumerate}
& 
  \begin{enumerate}
     \item Incorrect routing logic or implementation.
     \item Errors in implementation.
 \end{enumerate}
&
  \begin{enumerate}
     \item Test navigation paths thoroughly and implement error logging for navigation failures.
     \item Test UI components with different devices and browsers.
 \end{enumerate}

&  
\begin{enumerate}
     \item LF-AR2, LF-AR5
     \item UH-AR1, LF-AR4
 \end{enumerate}
&
\begin{enumerate}
     \item HA-UI1
     \item HA-UI2
 \end{enumerate}
 \\
 \hline
\end{longtable}
\end{landscape}
\restoregeometry
\newpage


\section{Safety and Security Requirements}
\subsection{Security Requirements}
\begin{enumerate}[{HA-SER}1. ]
    \item The system shall validate all SQL queries and ensure that input data is properly sanitized to prevent SQL injection attacks.\\
    \textbf{Rationale: }Prevent unauthorized access, data corruption, and breaches of sensitive user information.\\
    \textbf{Fit criterion: }All SQL queries must be parameterized, and inputs must be validated for known SQL injection vulnerabilities before execution.\\ 
  \end{enumerate}
\begin{enumerate}[{HA-SER}2. ]
    \item The system shall implement multi-factor authentication (MFA) for all users accessing sensitive patient data.\\
    \textbf{Rationale: }Unauthorized access to patient data, which could lead to data breaches and violations of HIPAA.\\
    \textbf{Fit criterion: }Users accessing sensitive data must pass a multi-factor authentication process within 2 minutes of a code being sent\\
  \end{enumerate}
\begin{enumerate}[{HA-SER}3. ]
    \item The system shall monitor server health and implement exception handling mechanisms to manage unexpected errors.\\
    \textbf{Rationale: }Minimize downtime and data loss due to server crashes or overload\\
    \textbf{Fit criterion: }The system should trigger a warning alert for server overload within 2 minutes, with exception handling enabling automatic recovery or failover mechanisms within 5 minutes.\\
  \end{enumerate}
\section{Roadmap}
\hspace{1.5em} The hazard analysis has identified several new security requirements that were 
not initially considered. However, due to the time constraints of the capstone 
project, not all of these requirements will be implemented at this stage. The 
team has decided to prioritize the implementation of requirement HA-SER1, as it
is essential to ensure the security of assessment data and recordings stored in the database. Requirements HA-SER2 and
HA-SER3 will be addressed in future development phases after the capstone timeline.
\newpage

\section*{Appendix --- Reflection}

The purpose of reflection questions is to give you a chance to assess your own
learning and that of your group as a whole, and to find ways to improve in the
future. Reflection is an important part of the learning process.  Reflection is
also an essential component of a successful software development process.  

Reflections are most interesting and useful when they're honest, even if the
stories they tell are imperfect. You will be marked based on your depth of
thought and analysis, and not based on the content of the reflections
themselves. Thus, for full marks we encourage you to answer openly and honestly
and to avoid simply writing ``what you think the evaluator wants to hear.''

Please answer the following questions.  Some questions can be answered on the
team level, but where appropriate, each team member should write their own
response:


\begin{enumerate}
    \item What went well while writing this deliverable? 
    
    \hspace{1.5em} The structured format and clear guidelines from both the document outline and lecture slides for writing the hazard 
    analysis, as well as the student examples from previous years were helpful and provided a clear idea of what was 
    expected from us for this deliverable.\\

    \item What pain points did you experience during this deliverable, and how
    did you resolve them?

    \hspace{1.5em} One of our main challenges was clearly defining the system components and boundaries, especially in deciding which 
    external components to include or exclude in the hazard analysis. The team considered including third-party hosting services 
    and user devices as part of the boundary, but after further discussion, we decided to limit our scope to only include 
    components the team can have control over. The reasoning for this is because including external components would have 
    added complexities out of our control, such as third-party security protocols and user device management, which could 
    vary and introduce risks that are outside our scope to address. This approach lets us focus on designing reliable 
    interactions with external components, but does not require addressing the risks of the external components themselves.\\

    \item Which of your listed risks had your team thought of before this
    deliverable, and which did you think of while doing this deliverable? For
    the latter ones (ones you thought of while doing the Hazard Analysis), how
    did they come about?

    \hspace{1.5em} Before conducting the hazard analysis, our team had already considered risks 
    related to video and audio processing by asking 'what if' questions, such 
    as "what if the video is broken?" or "what if the audio is noisy?". However, 
    during the hazard analysis, we began to explore deeper concerns related to database 
    management and the hazards associated with it, which directly informed the new security 
    requirements. We started by discussing the importance of securing patient data and 
    recognized that for a software hazard analysis, database security was crucial to 
    prevent any potential data leakage. This led to further discussions about other risks, 
    such as SQL injection attacks and incorrect assignment of user roles, which are now 
    captured in the updated security requirements.\\


    \item Other than the risk of physical harm (some projects may not have any
    appreciable risks of this form), list at least 2 other types of risk in
    software products. Why are they important to consider?

    1. Data Security Risks: Risks such as unauthorized access, data breaches, and data leaks are critical because of the 
    increased reliance on digital storage of sensitive information. A data breach could result in severe legal and 
    financial repercussions, especially in healthcare applications where patient confidentiality is extremely important. \\\\
	  2. User Experience (UX) Risks: Things like confusing navigation or unresponsive interfaces can lead to user 
    frustration, decreased usage, and even refusal to use the software. This is particularly important for non-technical 
    users such as parents and children in a remote setting.

\end{enumerate}

\end{document}