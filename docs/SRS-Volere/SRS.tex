% THIS DOCUMENT IS FOLLOWS THE VOLERE TEMPLATE BY Suzanne Robertson and James Robertson
% ONLY THE SECTION HEADINGS ARE PROVIDED
%
% Initial draft from https://github.com/Dieblich/volere
%
% Risks are removed because they are covered by the Hazard Analysis
\documentclass[12pt]{article}

\usepackage{booktabs}
\usepackage{tabularx}
\usepackage{hyperref}
\hypersetup{
    bookmarks=true,         % show bookmarks bar?
      colorlinks=true,      % false: boxed links; true: colored links
    linkcolor=red,          % color of internal links (change box color with linkbordercolor)
    citecolor=green,        % color of links to bibliography
    filecolor=magenta,      % color of file links
    urlcolor=cyan           % color of external links
}

\newcommand{\lips}{\textit{Insert your content here.}}

%% Comments

\usepackage{color}

\newif\ifcomments\commentstrue %displays comments
%\newif\ifcomments\commentsfalse %so that comments do not display

\ifcomments
\newcommand{\authornote}[3]{\textcolor{#1}{[#3 ---#2]}}
\newcommand{\todo}[1]{\textcolor{red}{[TODO: #1]}}
\else
\newcommand{\authornote}[3]{}
\newcommand{\todo}[1]{}
\fi

\newcommand{\wss}[1]{\authornote{blue}{SS}{#1}} 
\newcommand{\plt}[1]{\authornote{magenta}{TPLT}{#1}} %For explanation of the template
\newcommand{\an}[1]{\authornote{cyan}{Author}{#1}}

%% Common Parts

\newcommand{\progname}{Software Engineering} % PUT YOUR PROGRAM NAME HERE
\newcommand{\authname}{Team \#22, TeleHealth Insights
\\ Mitchell Weingust
\\ Parisha Nizam
\\ Promish Kandel
\\ Jasmine Sun-Hu} % AUTHOR NAMES                  

\usepackage{hyperref}
    \hypersetup{colorlinks=true, linkcolor=blue, citecolor=blue, filecolor=blue,
                urlcolor=blue, unicode=false}
    \urlstyle{same}
                                


\begin{document}

\title{Software Requirements Specification for \progname: subtitle describing software} 
\author{\authname}
\date{\today}
	
\maketitle

~\newpage

\pagenumbering{roman}

\tableofcontents

~\newpage

\section*{Revision History}

\begin{tabularx}{\textwidth}{p{3cm}p{2cm}X}
\toprule {\textbf{Date}} & {\textbf{Version}} & {\textbf{Notes}}\\
\midrule
October 3 & 1.0 & PK \& JS added Functional Requirements 9.4,9.5,9.6,\\
October 3 & 1.0 & Pk \& JS added sections 1,3,5\\
\bottomrule
\end{tabularx}

~\\

~\newpage
\section{Purpose of the Project}
\subsection{User Business}
\hspace{2em}The project being outlined in this document is an at-home bilingual speech 
assessment system with video and audio analysis features. The system is designed 
to provide clear guidance to parents when administering the assessment to their 
children, in an environment where speech-language pathologists (SLPs) are 
unavailable. By streamlining the assessment process, the project aims to provide a 
convenient and comprehensive solution for SLPs to assess and support their patients'
speech and language development remotely. 
\subsection{Goals of the Project}
\begin{itemize}
  \item[1.2.1] \textbf{Intuitive Parent Interface:}  
  The system must provide an intuitive interface that helps parents administer 
  language assessments effectively. It should be easy to navigate with clear and 
  meaningful symbols, and it must provide real-time feedback to ensure parents are 
  aware their interactions are being processed throughout the assessment.

  \item[1.2.2] \textbf{Engaging Child Interaction:}  
  The system must feature an engaging interface for children to keep them attentive 
  during the assessment. The design should be simple yet visually appealing, using 
  colors and images to attract the child’s attention to the questions and selections,
  ensuring that children remain engaged throughout the assessment.

  \item[1.2.3] \textbf{Reliable Assessment Data for SLPs:}  
  The system must provide reliable and accurate assessment data for speech-language 
  pathologists (SLPs) by capturing additional contextual data. This includes 
  identifying background interference, signs of bias, and potential test 
  complications. The system should also filter out noise and detect multiple users 
  to prevent external guidance from affecting the assessment results.

  \item[1.2.4] \textbf{Data Security:}  
  The system must ensure that all sensitive health and personal data is securely 
  stored and accessed. It should implement a strong security protocol to securely 
  store, retrieve, and manage sensitive data, ensuring the privacy and confidentiality
  of the users.

  \item[1.2.5] \textbf{Cross-Platform Compatibility:}  
  The system must provide cross-platform compatibility, ensuring that it functions 
  seamlessly across different devices and screen sizes. It should be accessible to 
  both parents and children, rendering correctly on all screen formats, whether on 
  phones, tablets, or desktops.
\end{itemize}

\section{Stakeholders}
\subsection{Client}
\lips
\subsection{Customer}
\lips
\subsection{Other Stakeholders}
\lips
\subsection{Hands-On Users of the Project}
\lips
\subsection{Personas}
\lips
\subsection{Priorities Assigned to Users}
\lips
\subsection{User Participation}
\lips
\subsection{Maintenance Users and Service Technicians}
\lips

\section{Mandated Constraints}
\subsection{Solution Constraints}
\begin{itemize}
  \item[3.1.1] The platform must be accessible as a website to provide ease of access to users without requiring special software installations.
  \item[3.1.2] The platform must adhere to HIPAA or relevant data protection regulations to ensure patient data privacy and security.
  \item[3.1.3] Access to the platform must be restricted to authorized users, with secure authentication processes in place.
  \item[3.1.4] The system must be capable of scaling to accommodate an increasing number of users and growing data storage needs as the client expands.
  \item[3.1.5] The platform must support assessment sessions of up to at least 30 minutes to align with standard telehealth consultation times.
  \item[3.1.6] The platform must support adaptable video and audio quality based on internet bandwidth, ensuring clarity and reliability during assessments.
  \item[3.1.7] The platform must comply with WCAG 2.1 accessibility standards, making it accessible to users with varying needs.
  \item[3.1.8] Patient records must be retained for a minimum of 7 years from the last visit or at least 1 year after the patient turns 18, whichever is 
  longer, in accordance with California law.
\end{itemize}
\subsection{Implementation Environment of the Current System}
\begin{itemize}
  \item[3.2.1] The platform’s hosting environment must meet HIPAA-compliance standards to ensure data security.
  \item[3.2.2] The development framework must support scalable, secure, and efficient web application development, compatible with existing technical 
  infrastructure.
\end{itemize}
\subsection{Partner or Collaborative Applications}
\begin{itemize}
  \item[3.3.1] The platform must be capable of exporting data as an Excel file, allowing for easy sharing, analysis, and compatibility with other systems
  that clinicians may use for data processing.
\end{itemize}
\subsection{Off-the-Shelf Software}
\begin{itemize}
  \item[3.4.1] \color{red} There are no mandated off-the-shelf software constraints. \color{black}
\end{itemize}
\subsection{Anticipated Workplace Environment}
\begin{itemize}
  \item[3.5.1] The platform must be compatible across a range of devices, including desktops, tablets, and mobile phones.
\end{itemize}
\subsection{Schedule Constraints}
\begin{itemize}
  \item[3.6.1] The proof-of-concept shall be complete and demonstrated between Nov. 11-22, 2024.
  \item[3.6.2] Revision 0 of the project shall be complete and demonstrated between February 3-14, 2025.
  \item[3.6.3] The final product shall be complete and demonstrated between March 24-30, 2025.
\end{itemize}
\subsection{Budget Constraints}
\begin{itemize}
  \item[3.7.1] The project budget must not exceed \$750 CAD. 
\end{itemize}
\subsection{Enterprise Constraints}
\begin{itemize}
  \item[3.8.1] \color{red} There are no mandated enterprise constraints. \color{black}
\end{itemize}

\section{Naming Conventions and Terminology}
\subsection{Glossary of All Terms, Including Acronyms, Used by Stakeholders
involved in the Project}
\lips

\section{Relevant Facts And Assumptions}
\subsection{Relevant Facts}
\begin{itemize}
  \item[5.1.1] The project is subject to healthcare privacy laws like HIPAA, ensuring that patient data is securely stored and managed.
  \item[5.1.2] The client has requested a web-based platform, indicating a preference for accessibility without the need for specialized 
  software installations.
  \item[5.1.3] The platform will have two primary user roles. The clinicians who perform assessments and review results and the parents who 
  administer the assessment to their children who are the patients.
\end{itemize}
\subsection{Business Rules}
\begin{itemize}
  \item[5.2.1] Only authorized users (clinicians) can access patient data.
  \item[5.2.2] Patient records must be retained for at least 7 years from the last visit, or 1 year after the patient turns 18, whichever is longer, 
  to comply with California state law.
  \item[5.2.3] The platform must comply with WCAG 2.1 to ensure it is accessible to users with disabilities.
  \item[5.2.4] All patient data must be encrypted both in transit and at rest to maintain confidentiality and meet regulatory standards.
  \item[5.2.5] The platform must generate reports based on assessment data, which can be reviewed and stored within the system 
  \item[5.2.6] Video and audio recordings must automatically adjust to optimize based on internet bandwidth, ensuring quality without excessive 
  buffering or latency.
\end{itemize}
\subsection{Assumptions}
\begin{itemize}
  \item[5.3.1] All users of the system have reliable internet connections that can support telehealth video sessions.
  \item[5.3.2] All patient data will be stored on servers located in regions that comply with healthcare data residency regulations.
  \item[5.3.3] The platform is assumed to be accessible from various devices (desktops, tablets, mobile phones), though it may perform optimally on 
  desktops.
  \item[5.3.4] Assessments will not exceed 30 minutes per session to fit standard telehealth consultation times.
  \item[5.3.5] It is assumed that users (both clinicians and patients) have a basic level of comfort with using web applications and online 
  communication tools.
  \item[5.3.6] The platform may need to accommodate additional users and storage demands as the client scales its telehealth services over time.
\end{itemize}

\section{The Scope of the Work}
\subsection{The Current Situation}
\lips
\subsection{The Context of the Work}
\lips
\subsection{Work Partitioning}
\lips
\subsection{Specifying a Business Use Case (BUC)}
\lips

\section{Business Data Model and Data Dictionary}
\subsection{Business Data Model}
\lips
\subsection{Data Dictionary}
\lips

\section{The Scope of the Product}
\subsection{Product Boundary}
\lips
\subsection{Product Use Case Table}
\lips
\subsection{Individual Product Use Cases (PUC's)}
\lips

\section{Functional Requirements}

\subsection{Authentication}
\textbf{A1: } Description.\\
\textit{Insert formal Specification}\\
\textbf{Rationale: } Insert Rational\\
\textbf{Fit criterion: } Insert criterion here 

\subsection{System Setup}
\textbf{SS1: } Description.\\
\textit{Insert formal Specification}\\
\textbf{Rationale: } Insert Rational\\
\textbf{Fit criterion: } Insert criterion here 

\subsection{User Interactions and Question Handling}
\textbf{UIQH1: } Description.\\
\textit{Insert formal Specification}\\
\textbf{Rationale: } Insert Rational\\
\textbf{Fit criterion: } Insert criterion here 

\subsection{Data Collection and Storage}
\textbf{FR-DCS1: }The database shall store multimedia files including video, audio, and structured data for 
each session.\\
\textit{Insert formal Specification}\\
\textbf{Rationale: }Video and audio files will provide extra information such as parent interference and 
other forms of bias/ cheating on the assessment.\\
\textbf{Fit criterion: }The system must successfully store and retrieve at least 1GB of video, audio and 
structured data per session without any data corruption. \\\\
\textbf{FR-DSC2: }The system shall not store any personally identifiable textual information (e.g., patient name, 
address, or medical record number) in the database.\\
\textit{Insert formal Specification}\\
\textbf{Rationale: }To maintain privacy and ensure compliance with data protection regulations such as HIPAA, 
identifying textual information must be excluded from storage in the database.\\
\textbf{Fit criterion: }An automated process shall verify and confirm that 100\% of records in the database 
accessible by clinicians are anonymized and contain no identifying textual information.\\\\
\textbf{FR-DSC3: }The database shall group all stored data by a unique user identifier to ensure data can be linked 
to specific users without storing identifiable information.\\
\textit{Insert formal Specification}\\
\textbf{Rationale: }Using a unique user identifier allows for data organization and retrieval by patient without 
compromising patient privacy, supporting the requirement for anonymized data storage.\\
\textbf{Fit criterion: }The system must assign a unique identifier to every user and confirm through testing 
that 100\% of session data is properly grouped and retrievable under that identifier, with no misassociated 
data.\\\\

\subsection{Video and Audio Data Analysis}
\textbf{FR-VADA1: }The analysis model shall have access to the video and audio recordings of each session.\\
\textit{Insert formal Specification}\\
\textbf{Rationale: }The data contains essential visual and auditory information that can help clinicians 
efficiently assess any speech-related disturbances and non-verbal cues.\\
\textbf{Fit criterion: }The model must successfully retrieve and process video data from 100\% of 
completed assessment sessions without encountering data access errors.\\\\
\textbf{FR-VADA2: }The analysis model shall identify speech disturbances, including interruptions, parental 
assistance on the assessment, or other irregularities in the background.\\
\textit{Insert formal Specification}\\
\textbf{Rationale: }Detecting disturbances is critical for accurate assessment of speech disorders without 
bias so that clinicians and speech language pathologists can accurately provide diagnosis and treatment.\\
\textbf{Fit criterion: }The model must accurately identify and log at least 95\% of speech disturbances 
from a set of test videos, validated against human observations.\\\\
\textbf{FR-VADA3: }The system shall flag detected disturbances and associate them with specific timestamps in the 
video recordings.\\
\textit{Insert formal Specification}\\
\textbf{Rationale: }Flagging disturbances and marking the exact points where they occur enables clinicians and 
speech-language pathologists to quickly review the relevant portions of the assessment, reducing the time needed 
for manual analysis.\\
\textbf{Fit criterion: }For each session, the model must accurately attach time stamps to disturbances
identified in VADA2 with at least 95\% accuracy.\\\\

\subsection{Data Processing and Display}
\textbf{FR-DPD1: }The system shall retrieve processed assessment results from the database for report generation.\\
\textit{Insert formal Specification}\\
\textbf{Rationale: }In order to generate reports, the system must access and extract the necessary data from the database, ensuring that all relevant assessment information is included.\\
\textbf{Fit criterion: }The system shall successfully retrieve all assessment data without errors within 10 seconds of a query being made.\\
\\
\noindent\textbf{FR-DPD2: }The system shall automatically generate a comprehensive report based on the retrieved assessment data, including flagged occurrences, timestamps, and patient performance metrics.\\
\textit{Insert formal Specification}\\
\textbf{Rationale: }Automatically generating a report provides a streamlined process for clinicians to review the patient’s performance, saving time on manual data compilation.\\
\textbf{Fit criterion: }The report must include all of the required data for each session, and must be generated within 10 seconds of the request.\\
\\
\noindent\textbf{FR-DPD3: }The system shall display the generated report in a user-friendly format, accessible through the platform’s interface.\\
\textit{Insert formal Specification}\\
\textbf{Rationale: }Clinicians need to be able to easily view and interpret the report to assess patient progress and determine next steps for therapy.\\
\textbf{Fit criterion: }The report must be displayed within the clinician's dashboard, formatted with charts and tables where applicable, and fully load within 10 seconds.\\
\\
\noindent\textbf{FR-DPD4: }The system shall store the generated report in the database, linked to the corresponding patient’s unique user identifier.\\
\textit{Insert formal Specification}\\
\textbf{Rationale: }Storing the report ensures that clinicians can access previous assessment results, enabling them to track patient progress over time.\\
\textbf{Fit criterion: }The report must be stored in the database with a unique identifier and timestamp, and be retrievable for at least 7 years after creation.\\
\\
\noindent\textbf{FR-DPD5: }Clinicians shall be able to securely access previously generated reports from the database at any time.\\
\textit{Insert formal Specification}\\
\textbf{Rationale: }Clinicians need on-demand access to reports to monitor progress and make informed treatment decisions during follow-up sessions.\\
\textbf{Fit criterion: }Clinicians must be able to access 100\% of stored reports within 10 seconds.\\

\section{Look and Feel Requirements}
\subsection{Appearance Requirements}
\lips
\subsection{Style Requirements}
\lips

\section{Usability and Humanity Requirements}
\subsection{Ease of Use Requirements}
\lips
\subsection{Personalization and Internationalization Requirements}
\lips
\subsection{Learning Requirements}
\lips
\subsection{Understandability and Politeness Requirements}
\lips
\subsection{Accessibility Requirements}
\lips

\section{Performance Requirements}
\subsection{Speed and Latency Requirements}
\lips
\subsection{Safety-Critical Requirements}
\lips
\subsection{Precision or Accuracy Requirements}
\lips
\subsection{Robustness or Fault-Tolerance Requirements}
\lips
\subsection{Capacity Requirements}
\lips
\subsection{Scalability or Extensibility Requirements}
\lips
\subsection{Longevity Requirements}
\lips

\section{Operational and Environmental Requirements}
\subsection{Expected Physical Environment}
\lips
\subsection{Wider Environment Requirements}
\lips
\subsection{Requirements for Interfacing with Adjacent Systems}
\lips
\subsection{Productization Requirements}
\lips
\subsection{Release Requirements}
\lips

\section{Maintainability and Support Requirements}
\subsection{Maintenance Requirements}
\lips
\subsection{Supportability Requirements}
\lips
\subsection{Adaptability Requirements}
\lips

\section{Security Requirements}
\subsection{Access Requirements}
\lips
\subsection{Integrity Requirements}
\lips
\subsection{Privacy Requirements}
\lips
\subsection{Audit Requirements}
\lips
\subsection{Immunity Requirements}
\lips

\section{Cultural Requirements}
\subsection{Cultural Requirements}
\lips

\section{Compliance Requirements}
\subsection{Legal Requirements}
\lips
\subsection{Standards Compliance Requirements}
\lips

\section{Open Issues}
\lips

\section{Off-the-Shelf Solutions}
\subsection{Ready-Made Products}
\lips
\subsection{Reusable Components}
\lips
\subsection{Products That Can Be Copied}
\lips

\section{New Problems}
\subsection{Effects on the Current Environment}
\lips
\subsection{Effects on the Installed Systems}
\lips
\subsection{Potential User Problems}
\lips
\subsection{Limitations in the Anticipated Implementation Environment That May
Inhibit the New Product}
\lips
\subsection{Follow-Up Problems}
\lips

\section{Tasks}
\subsection{Project Planning}
\lips
\subsection{Planning of the Development Phases}
\lips

\section{Migration to the New Product}
\subsection{Requirements for Migration to the New Product}
\lips
\subsection{Data That Has to be Modified or Translated for the New System}
\lips

\section{Costs}
\lips
\section{User Documentation and Training}
\subsection{User Documentation Requirements}
\lips
\subsection{Training Requirements}
\lips

\section{Waiting Room}
\lips

\section{Ideas for Solution}
\lips

\newpage{}
\section*{Appendix --- Reflection}

The information in this section will be used to evaluate the team members on the
graduate attribute of Lifelong Learning.  Please answer the following questions:

\begin{enumerate}
  \item What knowledge and skills will the team collectively need to acquire to
  successfully complete this capstone project?  Examples of possible knowledge
  to acquire include domain specific knowledge from the domain of your
  application, or software engineering knowledge, mechatronics knowledge or
  computer science knowledge.  Skills may be related to technology, or writing,
  or presentation, or team management, etc.  You should look to identify at
  least one item for each team member.
  \item For each of the knowledge areas and skills identified in the previous
  question, what are at least two approaches to acquiring the knowledge or
  mastering the skill?  Of the identified approaches, which will each team
  member pursue, and why did they make this choice?
\end{enumerate}

\end{document}