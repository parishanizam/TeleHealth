% THIS DOCUMENT IS FOLLOWS THE VOLERE TEMPLATE BY Suzanne Robertson and James Robertson
% ONLY THE SECTION HEADINGS ARE PROVIDED
%
% Initial draft from https://github.com/Dieblich/volere
%
% Risks are removed because they are covered by the Hazard Analysis
\documentclass[12pt]{article}

\usepackage{booktabs}
\usepackage{tabularx}
\usepackage{hyperref}
\hypersetup{
    bookmarks=true,         % show bookmarks bar?
      colorlinks=true,      % false: boxed links; true: colored links
    linkcolor=red,          % color of internal links (change box color with linkbordercolor)
    citecolor=green,        % color of links to bibliography
    filecolor=magenta,      % color of file links
    urlcolor=cyan           % color of external links
}

\newcommand{\lips}{\textit{Insert your content here.}}

%% Comments

\usepackage{color}

\newif\ifcomments\commentstrue %displays comments
%\newif\ifcomments\commentsfalse %so that comments do not display

\ifcomments
\newcommand{\authornote}[3]{\textcolor{#1}{[#3 ---#2]}}
\newcommand{\todo}[1]{\textcolor{red}{[TODO: #1]}}
\else
\newcommand{\authornote}[3]{}
\newcommand{\todo}[1]{}
\fi

\newcommand{\wss}[1]{\authornote{blue}{SS}{#1}} 
\newcommand{\plt}[1]{\authornote{magenta}{TPLT}{#1}} %For explanation of the template
\newcommand{\an}[1]{\authornote{cyan}{Author}{#1}}

%% Common Parts

\newcommand{\progname}{Software Engineering} % PUT YOUR PROGRAM NAME HERE
\newcommand{\authname}{Team \#22, TeleHealth Insights
\\ Mitchell Weingust
\\ Parisha Nizam
\\ Promish Kandel
\\ Jasmine Sun-Hu} % AUTHOR NAMES                  

\usepackage{hyperref}
    \hypersetup{colorlinks=true, linkcolor=blue, citecolor=blue, filecolor=blue,
                urlcolor=blue, unicode=false}
    \urlstyle{same}
                                


\begin{document}

\title{Software Requirements Specification for \progname: subtitle describing software} 
\author{\authname}
\date{\today}
	
\maketitle

~\newpage

\pagenumbering{roman}

\tableofcontents

~\newpage

\section*{Revision History}

\begin{tabularx}{\textwidth}{p{3cm}p{2cm}X}
\toprule {\textbf{Date}} & {\textbf{Version}} & {\textbf{Notes}}\\
\midrule
Date 1 & 1.0 & Notes\\
Date 2 & 1.1 & Notes\\
\bottomrule
\end{tabularx}

~\\

~\newpage
\section{Purpose of the Project}
\subsection{User Business}
\lips
\subsection{Goals of the Project}
\lips
\section{Stakeholders}
\subsection{Client}
\lips
\subsection{Customer}
\lips
\subsection{Other Stakeholders}
\lips
\subsection{Hands-On Users of the Project}
\lips
\subsection{Personas}
\lips
\subsection{Priorities Assigned to Users}
\lips
\subsection{User Participation}
\lips
\subsection{Maintenance Users and Service Technicians}
\lips

\section{Mandated Constraints}
\subsection{Solution Constraints}
\lips
\subsection{Implementation Environment of the Current System}
\lips
\subsection{Partner or Collaborative Applications}
\lips
\subsection{Off-the-Shelf Software}
\lips
\subsection{Anticipated Workplace Environment}
\lips
\subsection{Schedule Constraints}
\lips
\subsection{Budget Constraints}
\lips
\subsection{Enterprise Constraints}
\lips

\section{Naming Conventions and Terminology}
\subsection{Glossary of All Terms, Including Acronyms, Used by Stakeholders
involved in the Project}
\lips

\section{Relevant Facts And Assumptions}
\subsection{Relevant Facts}
\lips
\subsection{Business Rules}
\lips
\subsection{Assumptions}
\lips

\section{The Scope of the Work}
\subsection{The Current Situation}
\lips
\subsection{The Context of the Work}
\lips
\subsection{Work Partitioning}
\lips
\subsection{Specifying a Business Use Case (BUC)}
\lips

\section{Business Data Model and Data Dictionary}
\subsection{Business Data Model}
\lips
\subsection{Data Dictionary}
\lips

\section{The Scope of the Product}
\subsection{Product Boundary}
\lips
\subsection{Product Use Case Table}
\lips
\subsection{Individual Product Use Cases (PUC's)}
\lips

\section{Functional Requirements}

\subsection{Authentication}
\textbf{A1: } Description.\\
\textit{Insert formal Specification}\\
\textbf{Rationale: } Insert Rational\\
\textbf{Fit criterion: } Insert criterion here 

\subsection{System Setup}
\textbf{SS1: } Description.\\
\textit{Insert formal Specification}\\
\textbf{Rationale: } Insert Rational\\
\textbf{Fit criterion: } Insert criterion here 

\subsection{User Interactions and Question Handling}
\textbf{UIQH1: } Description.\\
\textit{Insert formal Specification}\\
\textbf{Rationale: } Insert Rational\\
\textbf{Fit criterion: } Insert criterion here 

\subsection{Data Collection and Storage}
\textbf{DCS1: }The database shall store multimedia files including video, audio, and JSON format files for each session.\\
\textit{Insert formal Specification}\\
\textbf{Rationale: } These file types are necessary to capture the full scope of the speech-language assessment, 
including patient responses and the structured data associated with each session (e.g., flagged occurrences, 
timestamps).\\
\textbf{Fit criterion: } The system must successfully store and retrieve at least 1GB of video, audio, and JSON 
data per session without data corruption. \\\\
\textbf{DCS2: }The database shall record the video, audio, flagged occurrences (e.g., errors or critical moments
 during the assessment), and timestamps for each question asked during the assessment.\\
\textit{Insert formal Specification}\\
\textbf{Rationale: } Storing flagged occurrences and timestamps lets clinicians perform detailed analysis 
of patient responses and enables them to review specific moments of interest efficiently.\\
\textbf{Fit criterion: } The database shall include video and audio files for 100 percent of assessment sessions,
 and each recording must have flagged occurrences and timestamps associated with every question asked, 
 retrievable via query. \\\\
\textbf{DSC3: }The system shall not store any personally identifiable textual information (e.g., patient name, address, 
or medical record number) in the database.\\
\textit{Insert formal Specification}\\
\textbf{Rationale: } To maintain privacy and ensure compliance with data protection regulations such as HIPAA, 
identifying textual information must be excluded from storage in the database.\\
\textbf{Fit criterion: } ??.\\\\
\textbf{DSC4: }The database shall group all stored data by a unique user identifier to ensure data can be linked to 
specific users without storing identifiable information.\\
\textit{Insert formal Specification}\\
\textbf{Rationale: } Using a unique user identifier allows for data organization and retrieval by patient without 
compromising patient privacy, supporting the requirement for anonymized data storage.\\
\textbf{Fit criterion: } The system must assign a unique identifier to every user and confirm through testing 
that all session data is properly grouped and retrievable under that identifier, with no misassociated data.\\\\
\textbf{DSC5: }Description.\\
\textit{Insert formal Specification}\\
\textbf{Rationale: } Insert Rational\\
\textbf{Fit criterion: } Insert criterion here\\

\subsection{Video and Audio Data Analysis}
\textbf{VADA1: } Description.\\
\textit{Insert formal Specification}\\
\textbf{Rationale: } Insert Rational\\
\textbf{Fit criterion: } Insert criterion here \\\\
\textbf{VADA2: } The analysis model shall have access to the video recordings of each session for the purpose of processing
 and analyzing patient speech patterns and behavior.\\
\textit{Insert formal Specification}\\
\textbf{Rationale: } The video data contains essential visual and auditory information that the model needs to analyze in order to assess speech-related disturbances and non-verbal cues.\\
\textbf{Fit criterion: } Insert criterion here \\\\

\subsection{Data Processing and Display}
\textbf{DPD1: }The system shall retrieve processed assessment results from the database for report generation.\\
\textit{Insert formal Specification}\\
\textbf{Rationale: }Inorder to generate reports, the system must access and extract the necessary data from the database, ensuring that all relevant assessment information is included.\\
\textbf{Fit criterion: }The system shall successfully retrieve all assessment data without errors within 5 seconds of a query being made.\\
\\
\noindent\textbf{DPD2: }The system shall automatically generate a comprehensive report based on the retrieved assessment data, including flagged occurrences, timestamps, and patient performance metrics.\\
\textit{Insert formal Specification}\\
\textbf{Rationale: }Automatically generating a report provides a streamlined process for clinicians to review the patient’s performance, saving time on manual data compilation.\\
\textbf{Fit criterion: }The report must include 100\% of the required data for each session (video, audio, flagged disturbances, timestamps), and be generated within 10 seconds of the request.\\
\\
\noindent\textbf{DPD3: }The system shall display the generated report in a user-friendly format, accessible through the platform’s interface.\\
\textit{Insert formal Specification}\\
\textbf{Rationale: }Clinicians need to be able to easily view and interpret the report to assess patient progress and determine next steps for therapy.\\
\textbf{Fit criterion: }The report must be displayed within the clinician's dashboard, formatted with charts and tables where applicable, and fully load within 3 seconds.\\
\\
\noindent\textbf{DPD4: }The system shall store the generated report in the database, linked to the corresponding patient’s unique user identifier.\\
\textit{Insert formal Specification}\\
\textbf{Rationale: }Storing the report ensures that clinicians can access previous assessment results, enabling them to track patient progress over time.\\
\textbf{Fit criterion: } The report must be stored in the database with a unique identifier and timestamp, and be retrievable for at least 5 years after creation.\\
\\
\noindent\textbf{DPD5: }Clinicians shall be able to securely access previously generated reports from the database at any time.\\
\textit{Insert formal Specification}\\
\textbf{Rationale: }Clinicians need on-demand access to reports to monitor progress and make informed treatment decisions during follow-up sessions.\\
\textbf{Fit criterion: }Clinicians must be able to access 100\% of stored reports within 3 seconds via a secure, role-based access system.\\

\section{Look and Feel Requirements}
\subsection{Appearance Requirements}
\lips
\subsection{Style Requirements}
\lips

\section{Usability and Humanity Requirements}
\subsection{Ease of Use Requirements}
\lips
\subsection{Personalization and Internationalization Requirements}
\lips
\subsection{Learning Requirements}
\lips
\subsection{Understandability and Politeness Requirements}
\lips
\subsection{Accessibility Requirements}
\lips

\section{Performance Requirements}
\subsection{Speed and Latency Requirements}
\lips
\subsection{Safety-Critical Requirements}
\lips
\subsection{Precision or Accuracy Requirements}
\lips
\subsection{Robustness or Fault-Tolerance Requirements}
\lips
\subsection{Capacity Requirements}
\lips
\subsection{Scalability or Extensibility Requirements}
\lips
\subsection{Longevity Requirements}
\lips

\section{Operational and Environmental Requirements}
\subsection{Expected Physical Environment}
\lips
\subsection{Wider Environment Requirements}
\lips
\subsection{Requirements for Interfacing with Adjacent Systems}
\lips
\subsection{Productization Requirements}
\lips
\subsection{Release Requirements}
\lips

\section{Maintainability and Support Requirements}
\subsection{Maintenance Requirements}
\lips
\subsection{Supportability Requirements}
\lips
\subsection{Adaptability Requirements}
\lips

\section{Security Requirements}
\subsection{Access Requirements}
\lips
\subsection{Integrity Requirements}
\lips
\subsection{Privacy Requirements}
\lips
\subsection{Audit Requirements}
\lips
\subsection{Immunity Requirements}
\lips

\section{Cultural Requirements}
\subsection{Cultural Requirements}
\lips

\section{Compliance Requirements}
\subsection{Legal Requirements}
\lips
\subsection{Standards Compliance Requirements}
\lips

\section{Open Issues}
\lips

\section{Off-the-Shelf Solutions}
\subsection{Ready-Made Products}
\lips
\subsection{Reusable Components}
\lips
\subsection{Products That Can Be Copied}
\lips

\section{New Problems}
\subsection{Effects on the Current Environment}
\lips
\subsection{Effects on the Installed Systems}
\lips
\subsection{Potential User Problems}
\lips
\subsection{Limitations in the Anticipated Implementation Environment That May
Inhibit the New Product}
\lips
\subsection{Follow-Up Problems}
\lips

\section{Tasks}
\subsection{Project Planning}
\lips
\subsection{Planning of the Development Phases}
\lips

\section{Migration to the New Product}
\subsection{Requirements for Migration to the New Product}
\lips
\subsection{Data That Has to be Modified or Translated for the New System}
\lips

\section{Costs}
\lips
\section{User Documentation and Training}
\subsection{User Documentation Requirements}
\lips
\subsection{Training Requirements}
\lips

\section{Waiting Room}
\lips

\section{Ideas for Solution}
\lips

\newpage{}
\section*{Appendix --- Reflection}

The information in this section will be used to evaluate the team members on the
graduate attribute of Lifelong Learning.  Please answer the following questions:

\begin{enumerate}
  \item What knowledge and skills will the team collectively need to acquire to
  successfully complete this capstone project?  Examples of possible knowledge
  to acquire include domain specific knowledge from the domain of your
  application, or software engineering knowledge, mechatronics knowledge or
  computer science knowledge.  Skills may be related to technology, or writing,
  or presentation, or team management, etc.  You should look to identify at
  least one item for each team member.
  \item For each of the knowledge areas and skills identified in the previous
  question, what are at least two approaches to acquiring the knowledge or
  mastering the skill?  Of the identified approaches, which will each team
  member pursue, and why did they make this choice?
\end{enumerate}

\end{document}