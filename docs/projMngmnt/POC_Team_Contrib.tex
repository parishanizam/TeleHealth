\documentclass{article}

\usepackage{float}
\restylefloat{table}

\usepackage{booktabs}

\title{Team Contributions: POC\\\progname}

\author{\authname}

\date{}

\input{../Comments}
\input{../Common}

\begin{document}

\maketitle

This document summarizes the contributions of each team member up to the POC
Demo.  The time period of interest is the time between the beginning of the term
and the POC demo.

\section{Demo Plans}

In the proof of concept demonstration, the team will showcase an interactive interface that displays the front 
camera of the device to detect and respond to specific visual inputs. When a user places their face in front 
facing view of the camera, the application will recognize when the user uses hand gestures in view of the camera, 
or when multiple faces are detected in the view. The interface will provide feedback by displaying what action it 
detects below the displayed camera feed within 3 seconds of the action itself. 

\section{Team Meeting Attendance}

\wss{For each team member how many team meetings have they attended over the
time period of interest.  This number should be determined from the meeting
issues in the team's repo.  The first entry in the table should be the total
number of team meetings held by the team.}

\begin{table}[H]
\centering
\begin{tabular}{ll}
\toprule
\textbf{Student} & \textbf{Meetings}\\
\midrule
Total & Num\\
Name 1 & Num\\
Name 2 & Num\\
Name 3 & Num\\
Name 4 & Num\\
Name 5 & Num\\
\bottomrule
\end{tabular}
\end{table}

\wss{If needed, an explanation for the counts can be provided here.}

\section{Supervisor/Stakeholder Meeting Attendance}

\begin{table}[H]
\centering
\begin{tabular}{ll}
\toprule
\textbf{Student} & \textbf{Meetings}\\
\midrule
Total & 6\\
Mitchell Weingust & 6\\
Jasmine Sun-Hu & 6\\
Parisha Nizam & 5\\
Promish Kandel & 5\\
\bottomrule
\end{tabular}
\end{table}

Promish and Parisha each missed one meeting when they were sick. All team members attended supervisor and stakeholder meetings when the circumstances allowed it.
If a team member missed a meeting, they caught themselves up by reviewing meeting notes and talking with other teammates.

\section{Lecture Attendance}

\wss{For each team member how many lectures have they attended over the time
period of interest.  This number should be determined from the lecture issues in
the team's repo.  The first entry in the table should be the total number of
lectures since the beginning of the term.}

\begin{table}[H]
\centering
\begin{tabular}{ll}
\toprule
\textbf{Student} & \textbf{Lectures}\\
\midrule
Total & Num\\
Name 1 & Num\\
Name 2 & Num\\
Name 3 & Num\\
Name 4 & Num\\
Name 5 & Num\\
\bottomrule
\end{tabular}
\end{table}

\wss{If needed, an explanation for the lecture attendance can be provided here.}

\section{TA Document Discussion Attendance}

\begin{table}[H]
\centering
\begin{tabular}{ll}
\toprule
\textbf{Student} & \textbf{Lectures}\\
\midrule
Total & 3\\
Mitchell Weingust & 3\\
Jasmine Sun-Hu & 3\\
Parisha Nizam & 3\\
Promish Kandel & 3\\
\bottomrule
\end{tabular}
\end{table}

\section{Commits}

\wss{For each team member how many commits to the main branch have been made
over the time period of interest.  The total is the total number of commits for
the entire team since the beginning of the term.  The percentage is the
percentage of the total commits made by each team member.}

\begin{table}[H]
\centering
\begin{tabular}{lll}
\toprule
\textbf{Student} & \textbf{Commits} & \textbf{Percent}\\
\midrule
Total & Num & 100\% \\
Name 1 & Num & \% \\
Name 2 & Num & \% \\
Name 3 & Num & \% \\
Name 4 & Num & \% \\
Name 5 & Num & \% \\
\bottomrule
\end{tabular}
\end{table}

\wss{If needed, an explanation for the counts can be provided here.  For
instance, if a team member has more commits to unmerged branches, these numbers
can be provided here.  If multiple people contribute to a commit, git allows for
multi-author commits.}

\section{Issue Tracker}

\wss{For each team member how many issues have they authored (including open and
closed issues (O+C)) and how many have they been assigned (only counting closed
issues (C only)) over the time period of interest.}

\begin{table}[H]
\centering
\begin{tabular}{lll}
\toprule
\textbf{Student} & \textbf{Authored (O+C)} & \textbf{Assigned (C only)}\\
\midrule
Mitchell Weingust & 143 & 40 \\
Jasmine Sun-Hu & 11 & 43 \\
Parisha Nizam & 3 & 44 \\
Promish Kandel & 4 & 37 \\
\bottomrule
\end{tabular}
\end{table}

For the authored issues, Mitchell was elected to create the issues for the team for organizational purposes. Assigned issues are distributed very evenly, disparities are due to 
cases where when dividing the workload one team member is assigned a few larger sections, and another is assigned multiple smaller sections.

\section{CICD}

\wss{Say how CICD will be used in your project}

\wss{If your team has additional metrics of productivity, please feel free to
add them to this report.}

\end{document}