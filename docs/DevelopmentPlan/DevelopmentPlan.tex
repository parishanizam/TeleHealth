\documentclass{article}

\usepackage{booktabs}
\usepackage{tabularx}
\usepackage{longtable}
\usepackage{ulem}
\usepackage{indentfirst}

\title{Development Plan\\\progname}

\author{\authname}

\date{}

\input{../Comments}
\input{../Common}

\begin{document}

\maketitle

\begin{table}[hp]
\caption{Revision History} \label{TblRevisionHistory}
\begin{tabularx}{\textwidth}{p{1.3cm}p{1cm}p{3cm}X}
\toprule
\textbf{Date} & \textbf{Vers} & \textbf{Developer(s)} & \textbf{Change}\\
\midrule
20/09/24 & 1.0 & Jasmine Sun-Hu, Mitchell Weingust & Added: Team Identifiers, Confidential Information, Intellectual Property,
Copyright License, Team Meeting Plan\\
22/09/24 & 1.1 & Jasmine Sun-Hu & Added: Team Communication Plan, Team Member Roles, Workflow Plan, Project Decomposition and Scheduling\\
22/09/24 & 1.2 & Mitchell Weingust & Added: Appendix - Team Charter\\
23/09/24 & 1.3 & Parisha Nizam & Added: Coding Standard, Expected Technologies, POC \\
23/09/24 & 1.4 & Parisha Nizam & POC additions \\
23/09/24 & 1.5 & Jasmine, Mitchell, Parisha, Promish & Added: Reflection \\
24/09/24 & 1.6 & Promish, Mitchell, Parisha, Jasmine & Final Review\\
06/11/24 & 1.7 & Jasmine Sun-Hu & Updated POC Demonstration Plan\\
03/23/25 & 2.0 & Parisha Nizam & Implemented TA Feedback: \href{https://github.com/parishanizam/TeleHealth/issues/185}{Updated Revision Table Format and POC plan details with model usage Pg 1,9,10} \\
03/23/25 & 2.1 & Parisha Nizam & Implemented TA Feedback: \href{https://github.com/parishanizam/TeleHealth/issues/180}{Updated Tools with reasons why we chose it and advantages/disadvantages Pg 13-16}\\
03/23/25 & 2.2 & Parisha Nizam & Implemented TA Feedback: \href{https://github.com/parishanizam/TeleHealth/issues/235}{Updated Target Metrics Pg 20}\\
\bottomrule
\end{tabularx}
\end{table}

\newpage{}

\section*{Introduction}

This document contains the outline, roadmap and logistics for TeleHealth Insights, a project to build a web 
interface that will help parents administer language assessment tests at home for children with speech
difficulties. The development plan includes team roles and responsibilities, a general timeline, and 
necessary logistic details.

\section{Confidential Information?}

There is no confidential information to protect, therefore there is no agreement.

\section{IP to Protect}

There is no intellectual property to protect, therefore there is no agreement.

\section{Copyright License}

Apache License Version 2.0 (Apache-2.0) \\
\url{https://github.com/parishanizam/TeleHealth/blob/main/LICENSE}

\section{Team Meeting Plan}

The team will meet in-person at least once a week every Monday from 3:30-4:30 pm. Exceptions to this 
may include when the University is closed, statuatory holidays, or a group consensus to postpone the meeting
is agreed upon. Additional meetings can be held in person or virtually through the team's discord server on 
a per need basis. Location and timing will be decided as a group at least 3 hours in advance. Team meetings will
be structured as follows:
\begin{itemize}
  \item 5-10 minutes of progress check-in 
  \item 5 minutes of agenda debrief
  \item 40 minutes of executing the agenda
  \item 5-10 minutes of discussing next steps
\end{itemize}
\indent The meeting chair will be decided at least 24 hours prior to the meeting, and rotate on a weekly basis.

Meetings with the project's supervisor will take place in-person every Tuesday from 9:45-10:15 am. Exceptions 
to this may include when the University is closed, statuatory holidays, or a group consensus to postpone the 
meeting is agreed upon.

\section{Team Communication Plan}

Communication is essential for a successful project. The following is an outline of how the team will 
communicate, the tools/platforms we will use, and the expectations for each team member regarding communication.\\

Communication Tools
\begin{itemize}
  \item GitHub: a GitHub repository will be used for code versioning, project tracking, and technical documentation.\\
  Additional details:
  \begin{itemize}
    \item Project board: used to track milestones and visualize the workflow of the project using Kanban style columns
    \item Issues: used to keep track of tasks and meetings, and delegating tasks among team members. Labels are used
     to categorize issues.
  \end{itemize}
  \item Discord: a discord server will be used for day-to-day communication and online meetings. Below outlines the 
  discord server structure:
  \begin{itemize}
    \item General: text channel for general updates, quick questions and informal discussions
    \item Documents-and-resources: text channel for relevant files or useful links that do not belong in the github 
    project folder
    \item Meetings: text channel for co-ordinating ad-hoc meetings between some or all team members.
    \item External-meetings: text channel for co-ordinating and reviewing meeting agendas with individuals 
    outside the core capstone team (e.g. capstone supervisor).
    \item Help: text channel for questions or issues that need prioritized attention.
    \item Office: voice channel to hold any online meetings.
  \end{itemize}
  \item E-mail: school emails will be used to communicate with individuals outside the core capstone team such as the 
  capstone supervisor, capstone professor, external professionals, etc.
\end{itemize}

\section{Team Member Roles}

All team members are responsible for writing documentation, coding, testing, and creating/commenting on issues no 
matter their role.\\
\indent The roles decided for each team member were selected as a result of their past experiences and strengths in group projects.
Each team member agreed to the responsibilities detailed in the below table, and believe they can achieve their goals.
Should there be any concerns or flexibility among roles, an individual memeber of the team can discuss their responsibilities,
and the roles can be rebalanced, modified, or changed.\\

\vspace{0.5cm}
\begin{tabularx}{\textwidth}{|l|l|X|}
  \hline
  \textbf{Team Member} & \textbf{Role} & \textbf{Responsibilities} \\
  \hline
  Mitchell Weingust & Team Liaison & Chairs external meetings, handles the communication between the team and the capstone 
  supervisor, course instructors, TAs and any other external individuals relevant to the project.\\
  \cline{2-3} & Machine Learning Audio Lead & In charge of overseeing the audio analysis of the machine learning model.\\
  \hline
  Parisha Nizam & Project Manager & Chairs team meetings, oversees the project timeline, ensures milestones and deadlines 
  are met, and that team members contribute appropriately.\\
  \hline
  Promish Kandel & Lead Developer & In charge of managing and leading the technical design, coordination and testing of the 
  project, responsible for helping teammates with technical challenges. \\
  \cline{2-3} & Machine Learning Video Lead & In charge of overseeing the video analysis of the machine learning model.\\
  \hline
  Jasmine Sun-Hu & UI/UX Design Lead & In charge of overseeing the user interface and user experience components of the project, 
  responsible for user research and usability testing.\\
  \hline
\end{tabularx}

\section{Workflow Plan}

\subsection{Git Strategy}
\begin{itemize}
	\item The main branch will contain code that has been approved, tested, and is considered production-ready by 
  the team.
	\item Branches will be created based on deliverables, features, or bug fixes and be named clearly based on their 
  purpose (e.g. problem-statement, development-plan, front-end/navbar, etc.) 
  \item Once the deliverable, feature or bug fix is completed on its branch, it can be deleted after being merged into the main branch.
	\item Once a deliverable or bug fix is complete, a pull request will be made. The PR description will include:
	\begin{itemize}
    \item The purpose of the change
    \item The issue number it addresses with a link to the issue
    \item Details of testing and/or any relevant references
    \item Contributors
  \end{itemize}
	\item All PRs require at least one teammate to review it, with the exception of documentation updates that are 
  able to automatically merge.
	\item GitHub PR comments will be used to provide feedback
	\item Issues will be created on Git for project management (see 7.2)
\end{itemize}

\subsection{Issue Management}
\begin{itemize}
	\item Every new task (documentation, features, bugs, updates, meetings) will be logged as a GitHub Issue.
	\item Team members may either use one of the provided templates or create a blank issue. For blank issues, 
  they must include:
	\begin{itemize}
    \item Title that gives a clear overview of the issue
    \item Description of the task, including any necessary background or context
    \item Links to related issues necessary for the completion of the current issue if applicable.
    \item Label tags, Milestone category, and Project assignment*.
    \item Assignees
  \end{itemize}
  \item *Label tags are based on the type of issue (e.g. documentation), milestone categories 
  are the type of deliverable (e.g. Development Plan), and all issues are displayed on one project board (see 8.1)
  \item All issues are to be tracked in the project board:
  \begin{itemize}
    \item Issues that are ready to be picked up by anyone is put into the Backlog
    \item Issues that are being worked on are to be put into the In progress
    \item Issues that are being review, is put into the In review
    \item Issues that are finish and merged into the main branch can be put into the Closed
  \end{itemize}
\end{itemize}

\subsection{Use of CI/CD}
\begin{itemize}
	\item GitHub Actions will be used to create and manage continuous integration workflow scripts.
	\begin{itemize}
    \item Running a code linting tool and automatically enforcing style guides
    \item Restricting the ability to merge pull requests that do not meet our coding standards (see section 11)
    \item Ensure that the code builds successfully on each push
  \end{itemize}
	\item If all tests pass, and at least one team members review the pull request, the branch can be approved and merged into the main branch.
	\item The lead developer will ensure all necessary code is covered by automated tests.
	\item A rollback strategy will be included in the CD pipeline in case the project needs 
  to be reverted to a previously stable state.
\end{itemize}

\section{Project Decomposition and Scheduling}

\subsection{GitHub Projects}

GitHub Projects will be used to manage and keep track of the project's schedule. 
\textbf{The link to our project board can be found \href{https://github.com/users/parishanizam/projects/2/views/1}{here}.}\\
The project board is organized by rows and columns, where each row is a different 
milestone that can be minimized and expanded, and each column is as follows:
\begin{itemize}
  \item Backlog: Issues that have assigned low priority
  \item To Do: Newly created issues
  \item In Progress: Issues actively being worked on
  \item In Review: Issues awaiting approval from other teammates
  \item Done: Completed Issues
\end{itemize}
Each issue is categorized into a milestone, and team members will update their statuses as they 
progress through the project deliverables.

\subsection{Project Timeline}

\subsection*{Forming Team + Project Selection (Due Sept 16)}
	\begin{itemize}
		\item Form a team (Sept 6)
		\item Meet with potential supervisors (Sept 12)
		\item Select project
	\end{itemize}

\subsection*{Project Planning Documentation (Due Sept 24)}
	\begin{itemize}
		\item Draft Problem Statement
		\item Create POC Plan
		\item Create Development Plan
	\end{itemize}

	\subsection*{Requirements Document Revision 0 (Due October 9)}
	\begin{itemize}
		\item Research stakeholders
		\item Define scope, purpose and context of the system
		\item Define use cases and functional requirements
		\item Define non-functional requirements
		\item Brainstorm potential challenges
		\item Create traceability matrices and graphs
	\end{itemize}

	\subsection*{Hazard Analysis 0 (Due October 23)}
	\begin{itemize}
		\item Define scope and purpose of hazard Analysis
		\item Define boundaries, components and assumptions
		\item Create FMEA table
		\item Define safety and security requirements
		\item Create roadmap
	\end{itemize}

	\subsection*{V\&V Plan 0 (Due November 1)}
	\begin{itemize}
		\item Define V\&V plan and logistics
		\item Define system tests for functional and nonfunctional requirements
		\item Define unit tests
	\end{itemize}

	\subsection*{Proof of Concept Demonstration (November 11 - 22)}
	\begin{itemize}
		\item Prepare POC demonstration
	\end{itemize}

	\subsection*{Design Document (Due January 15)}
	\begin{itemize}
		\item List potential changes
		\item Define connections between requirements and design
		\item Define module decomposition
		\item Design user interface
		\item Schedule timeline
	\end{itemize}

	\subsection*{Revision 0 Demonstration (February 3 - February 14)}
	\begin{itemize}
		\item Plan demonstration 
		\item Conduct user testing and feedback 
		\item Finalize demonstration
	\end{itemize}

	\subsection*{V\&V Report Revision 0 (Due March 7)}
	\begin{itemize}
		\item Evaluate functional requirements
		\item Evaluate nonfunctional requirements
		\item Evaluate testing methods
		\item Trace to requirements and modules
		\item Evaluate code coverage metrics
	\end{itemize}

	\subsection*{Final Demonstration (Revision 1) (March 24 - March 30)}
	\begin{itemize}
		\item Test and finalize project
		\item Create script and slideshow
		\item Practice/prepare demonstration presentation
	\end{itemize}

	\subsection*{EXPO Demonstration (April TBD)}
	\begin{itemize}
		\item Create EXPO event poster
		\item Set up project for live user interaction
		\item Prepare main talking points
	\end{itemize}

	\subsection*{Final Documentation (Revision 1) (April 2)}
	\begin{itemize}
		\item Problem statement
		\item Development plan
		\item Proof of Concept (POC) Plan
		\item Requirements Document
		\item Hazard Analysis
		\item Design Document
		\item V\&V Plan
		\item V\&V Report
		\item User’s Guide
		\item Source Code
	\end{itemize}

\section{Proof of Concept Demonstration Plan}

\noindent{\textbf{Video Implementation and Analysis:}} The main risk and most challenging aspect of this project is for the system to collect video data that will be sufficiently accurate for us to store and analyze. 
These at home assessments will be done in different settings with varying lightings, and camera qualities. Ensuring that video capturing remains complete and accurate across all settings presents a significant challenge.  Video integration also poses the risk of compromising data security. As we are handling videos of children and parents, it is important to maintain security. Video data should be encrypted during the storage and processing to ensure privacy of the clients and will be compliant to privacy standards. \\  
\\
\noindent \textbf{POC Plan:} 
In the proof of concept demonstration, the team will showcase an interactive interface that displays the front 
camera of the device to detect and respond to specific visual inputs. When a user places their face in front 
facing view of the camera, the application will recognize when the user uses hand gestures in view of the camera, 
or when multiple faces are detected in the view. The interface will provide feedback by displaying what action it 
detects below the displayed camera feed within 3 seconds of the action itself. Due to time constraints, the model will come from MediaPipe's face detection model. We will build upon this model
to allow for a well tested solution that we can implement improvements on. 

\subsection{Will a part of the implementation be difficult?}
Technologies such as using video \& video recording with analytics, and using machine learning are relatively new to members on the team due to limited experience. Research and open source models recommeded by Dr.Yuan will assist in providing resources and support. \\

\subsection{Will testing be difficult?} 
Unit testing will not be difficult, as each developer responsible for creating their code is also responsible for testing it. They should be familiar with the limitations and boundaries where errors could occur. In addition, there will be one other team member responsible for further testing to provide another perspective. We will also being using automated testing using CI and testing libraries including Jest and PyTest. Analyzing mock data as well as real data for final testing may be more challenging. \\

\subsection{Is a required library difficult to install?}
No there are no current libraries that will be difficult to install or use. Most
team members have used the planned libraries before or popular libraries have documentation to
follow.\\

\pagebreak

\subsection{Will portability be a concern?}
Portability will not be a concern. We will be starting with a web-based application that can be opened on any device. We have chosen to use the framwork Expo to easily integrate the application as a mobile application on cross platforms including IOS and android.\\


\textbf{Proof of Concept Demonstration}\\

A walkthrough of a simple mock assessment process on the application involving two users. The assesement will be completed on a laptop using integrated live video. POC will demonstrate the ability of the web based application to do the following: \\
\begin{itemize}
  \item Open the POC demo application
  \item Turn on the camera
  \item A person positions themselves in front of the camera such that their face is within the frame
  \item A person can perform hand gestures within the camera view
  \item The interface will provide feedback detecting the hand gesture within 3 seconds, using visual cues to show detection
  \item Another person shows their face within the camera view at the same time as the first person
  \item The interface will provide feedback detecting multiple faces within 3 seconds.
\end{itemize}

\section{Expected Technology}

The following tools and technologies will be used for the development, testing and deployment of the application 

\begin{table}[htbp]
  \caption{Tool Evaluation Part 1}
  \label{TblToolEval}
  \renewcommand{\arraystretch}{1.2}   % Increase row spacing
  \setlength{\tabcolsep}{5pt}         % Adjust column spacing if needed
  \begin{tabularx}{\textwidth}{p{2cm} p{4cm} X}
  \toprule
  \textbf{Tool} & \textbf{Explanation} & \textbf{Advantages / Disadvantages}\\
  \midrule

  \textbf{JavaScript} 
  & JavaScript will be the main programming language used as the basis for both front-end and back-end development.
  & \textbf{Advantages:}
    \begin{itemize}
      \item Large ecosystem of libraries and frameworks
      \item Single language for front-end \& back-end
      \item Lots of documentation and resources 
    \end{itemize}
    \textbf{Disadvantages:}
    \begin{itemize}
      \item Dynamically typed (risk of runtime errors)
      \item Not optimal for CPU-intensive tasks
    \end{itemize}
  \\
  \midrule

  \textbf{Python} 
  & Python will be the main language for data analysis and machine learning.
  & \textbf{Advantages:}
    \begin{itemize}
      \item Extensive scientific and ML libraries
      \item Easy to read and prototype quickly
      \item Most fitted language for creating ML model
    \end{itemize}
    \textbf{Disadvantages:}
    \begin{itemize}
      \item Virtual environment management can be tricky
    \end{itemize}
  \\
  \midrule

  \textbf{React and Tailwind} 
  & Used for cross-platform mobile development (React Native) and utility-first styling (Tailwind).
  & \textbf{Advantages:}
    \begin{itemize}
      \item Single codebase for iOS/Android
      \item Rapid styling with existing classes, great look
    \end{itemize}
    \textbf{Disadvantages:}
    \begin{itemize}
      \item Performance overhead vs. fully native apps
    \end{itemize}
  \\
  \midrule
  \bottomrule
  \end{tabularx}
\end{table}

% -----------------------------------------------------------------

\begin{table}[htbp]
  \caption{Tool Evaluation Part 2}
  \label{TblToolEval}
  \renewcommand{\arraystretch}{1.2}   % Increase row spacing
  \setlength{\tabcolsep}{5pt}         % Adjust column spacing if needed
  \begin{tabularx}{\textwidth}{p{2cm} p{4cm} X}
  \toprule
  \textbf{Tool} & \textbf{Explanation} & \textbf{Advantages / Disadvantages}\\
  \midrule

  \textbf{Node.js} 
  & Will be used for building the back-end server.
  & \textbf{Advantages:}
    \begin{itemize}
      \item Non-blocking I/O ideal for real-time apps
      \item Massive npm ecosystem, easy to install dependencies etc
    \end{itemize}
    \textbf{Disadvantages:}
    \begin{itemize}
      \item Can struggle with CPU-heavy tasks
      \item Learning curve for some members on team
    \end{itemize}
  \\
  \midrule

  \textbf{Figma} 
  & Will be used to design a prototype of the application and test its usability.
  & \textbf{Advantages:}
    \begin{itemize}
      \item Real-time collaboration, can design entire application together
      \item Easy sharing of prototypes
    \end{itemize}
    \textbf{Disadvantages:}
    \begin{itemize}
      \item Some advanced features require paid plan
      \item Time Consuming - Manual 
    \end{itemize}
  \\
  \midrule

  \textbf{VSCode} 
  & Code editor to write application code.
  & \textbf{Advantages:}
    \begin{itemize}
      \item Large extension marketplace
      \item Built-in Git integration
    \end{itemize}
    \textbf{Disadvantages:}
    \begin{itemize}
      \item Can become resource-intensive with many plugins, learning curve to figure out what is missing to make extensions work
    \end{itemize}
  \\
  \midrule

  \textbf{GitHub with GitHub Projects} 
  & Version control tool to manage tasks and track progress.
  & \textbf{Advantages:}
    \begin{itemize}
      \item Centralised platform for code, issues, and docs
      \item Easy code review process with pull requests
    \end{itemize}
    \textbf{Disadvantages:}
    \begin{itemize}
      \item Requires Git proficiency to avoid merge conflicts
    \end{itemize}
  \\
  \midrule

  % -----------------------------------------------------------------

  \textbf{GitHub Actions} 
  & Automating CI pipeline to handle testing, linting, and building the project.
  & \textbf{Advantages:}
    \begin{itemize}
      \item Deep GitHub integration
      \item Large marketplace of reusable actions
    \end{itemize}
    \textbf{Disadvantages:}
    \begin{itemize}
      \item Complex workflows need YAML expertise
    \end{itemize}
  \\
  \midrule

  \textbf{ESLint} 
  & Linter to enforce coding standards and best practices.
  & \textbf{Advantages:}
    \begin{itemize}
      \item Ensures consistent code style
      \item Highly configurable
    \end{itemize}
    \textbf{Disadvantages:}
    \begin{itemize}
      \item Overly strict rules can be frustrating
      \item Initial setup time can get long 
    \end{itemize}
  \\
  \midrule

  \textbf{Prettier} 
  & Format library to ensure consistent code formatting; integrates with ESLint.
  & \textbf{Advantages:}
    \begin{itemize}
      \item Automatically formats code
      \item Reduces style debates among developers
    \end{itemize}
    \textbf{Disadvantages:}
    \begin{itemize}
      \item May conflict with other formatters if not synced
    \end{itemize}
  \\
  \midrule

  \textbf{Jest} 
  & Testing framework for JavaScript.
  & \textbf{Advantages:}
    \begin{itemize}
      \item Zero-config setup for many JS apps
      \item Good mocking and snapshot features
    \end{itemize}
    \textbf{Disadvantages:}
    \begin{itemize}
      \item Can slow down with large test suites
    \end{itemize}
  \\
  \midrule

  \textbf{Pytest} 
  & Testing framework for Python.
  & \textbf{Advantages:}
    \begin{itemize}
      \item Auto-discovers test files
      \item Large plugin ecosystem
    \end{itemize}
    \textbf{Disadvantages:}
    \begin{itemize}
      \item Advanced config can be complex
    \end{itemize}
  \\
  \midrule

  \textbf{AWS - S3} 
  & Will be used to store all data in S3 Buckets.
  & \textbf{Advantages:}
    \begin{itemize}
      \item Scalable, pay-as-you-go
      \item Easy integration with AWS ecosystem
    \end{itemize}
    \textbf{Disadvantages:}
    \begin{itemize}
      \item Costs can grow with large data - We don't want to spend money
      \item Requires secure config to prevent data leaks
    \end{itemize}
  \\
  \midrule

  \textbf{Netlify} 
  & Will be used to deploy the frontend of the application.
  & \textbf{Advantages:}
    \begin{itemize}
      \item Automatic builds from Git
      \item Simple drag-and-drop deployments
    \end{itemize}
    \textbf{Disadvantages:}
    \begin{itemize}
      \item Limited server-side options (static hosting)
      \item May require paid tiers for advanced features
    \end{itemize}
  \\
  \midrule

  \textbf{Render} 
  & Will be used to deploy the backend of the application.
  & \textbf{Advantages:}
    \begin{itemize}
      \item Straightforward Docker-based deployment
      \item Automated SSL and domain management
    \end{itemize}
    \textbf{Disadvantages:}
    \begin{itemize}
      \item Costs can occur with scaling needs
    \end{itemize}
  \\
  \midrule

  \textbf{Deepgram} 
  & Using Deepgram to create a bias detection algorithm via keyword spotting.
  & \textbf{Advantages:}
    \begin{itemize}
      \item High accuracy STT with advanced features
      \item No need to maintain our own ML model
    \end{itemize}
    \textbf{Disadvantages:}
    \begin{itemize}
      \item External API dependency
    \end{itemize}
  \\
  \bottomrule
  \end{tabularx}
\end{table}


% -------------------------------------------------
\begin{table}[htbp]
  \caption{Tool Evaluation Part 3}
  \label{TblToolEval}
  \renewcommand{\arraystretch}{1.2}   % Increase row spacing
  \setlength{\tabcolsep}{5pt}         % Adjust column spacing if needed
  \begin{tabularx}{\textwidth}{p{2cm} p{4cm} X}
  \toprule
  \textbf{Tool} & \textbf{Explanation} & \textbf{Advantages / Disadvantages}\\

  \midrule 

  \textbf{GitHub Actions} 
  & Automating CI pipeline to handle testing, linting, and building the project.
  & \textbf{Advantages:}
    \begin{itemize}
      \item Deep GitHub integration
      \item Large marketplace of reusable actions
    \end{itemize}
    \textbf{Disadvantages:}
    \begin{itemize}
      \item Complex workflows need YAML expertise
    \end{itemize}
  \\
  \midrule

  \textbf{ESLint} 
  & Linter to enforce coding standards and best practices.
  & \textbf{Advantages:}
    \begin{itemize}
      \item Ensures consistent code style
      \item Highly configurable
    \end{itemize}
    \textbf{Disadvantages:}
    \begin{itemize}
      \item Overly strict rules can be frustrating
      \item Initial setup time can get long 
    \end{itemize}
  \\
  \midrule

  \textbf{Prettier} 
  & Format library to ensure consistent code formatting; integrates with ESLint.
  & \textbf{Advantages:}
    \begin{itemize}
      \item Automatically formats code
      \item Reduces style debates among developers
    \end{itemize}
    \textbf{Disadvantages:}
    \begin{itemize}
      \item May conflict with other formatters if not synced
    \end{itemize}
  \\
  \midrule

  \textbf{Jest} 
  & Testing framework for JavaScript.
  & \textbf{Advantages:}
    \begin{itemize}
      \item Zero-config setup for many JS apps
      \item Good mocking and snapshot features
    \end{itemize}
    \textbf{Disadvantages:}
    \begin{itemize}
      \item Can slow down with large test suites
    \end{itemize}
  \\
  \bottomrule
  \end{tabularx}
\end{table}


% -------------------------------------------------
\begin{table}[htbp]
  \caption{Tool Evaluation Part 4}
  \label{TblToolEval}
  \renewcommand{\arraystretch}{1.2}   % Increase row spacing
  \setlength{\tabcolsep}{5pt}         % Adjust column spacing if needed
  \begin{tabularx}{\textwidth}{p{2cm} p{4cm} X}
  \toprule
  \textbf{Tool} & \textbf{Explanation} & \textbf{Advantages / Disadvantages}\\

  \midrule 

  \textbf{AWS - S3} 
  & Will be used to store all data in S3 Buckets.
  & \textbf{Advantages:}
    \begin{itemize}
      \item Scalable, pay-as-you-go
      \item Easy integration with AWS ecosystem
    \end{itemize}
    \textbf{Disadvantages:}
    \begin{itemize}
      \item Costs can grow with large data - We don't want to spend money
      \item Requires secure config to prevent data leaks
    \end{itemize}
  \\
  \midrule

  \textbf{Netlify} 
  & Will be used to deploy the frontend of the application.
  & \textbf{Advantages:}
    \begin{itemize}
      \item Automatic builds from Git
      \item Simple drag-and-drop deployments
    \end{itemize}
    \textbf{Disadvantages:}
    \begin{itemize}
      \item Limited server-side options (static hosting)
      \item May require paid tiers for advanced features
    \end{itemize}
  \\
  \midrule

  \textbf{Render} 
  & Will be used to deploy the backend of the application.
  & \textbf{Advantages:}
    \begin{itemize}
      \item Straightforward Docker-based deployment
      \item Automated SSL and domain management
    \end{itemize}
    \textbf{Disadvantages:}
    \begin{itemize}
      \item Costs can occur with scaling needs
    \end{itemize}
  \\
  \midrule

  \textbf{Deepgram} 
  & Using Deepgram to create a bias detection algorithm via keyword spotting.
  & \textbf{Advantages:}
    \begin{itemize}
      \item High accuracy STT with advanced features
      \item No need to maintain our own ML model
    \end{itemize}
    \textbf{Disadvantages:}
    \begin{itemize}
      \item External API dependency
    \end{itemize}
  \\
  \bottomrule
  \end{tabularx}
\end{table}

\pagebreak

\section{Coding Standard}

The project will follow \href{https://github.com/pillarstudio/standards/blob/master/reactjs-guidelines.md}{React JS coding standards} to ensure consistency. This includes following the same variable naming convention, function structure, spacings, etc. We will be using the Linter ESLint to ensure standards are met throughout the code.\\
The project will follow \href{https://flake8.pycqa.org/en/latest/}{Flake8 standards} to ensure consistency using python for the project's data analytics and machine learning component. 

\newpage{}

\section*{Appendix --- Reflection}

\input{../Reflection.tex}

\begin{enumerate}
    \item Why is it important to create a development plan prior to starting the
    project?
    \item In your opinion, what are the advantages and disadvantages of using
    CI/CD?
    \item What disagreements did your group have in this deliverable, if any,
    and how did you resolve them?
\end{enumerate}

\subsection*{Why is it important to create a development plan prior to starting the project?}

There are several reasons why it is important to create a development plan before starting a project.
It provides a high level overview of the entire project from start to finish, which gives the team a 
clear direction. Documenting the goals, scope, and timeline helps the team understand the project early 
on. This familiarity leads to more efficient project management.

A development plan also helps identify any necessary tools and resources that will be used. Along with the 
goals, scope and timeline, this information can help a team identify challenges and risks before they occur,
and allows them to plan ahead and implement strategies to reduce these risks and challenges. 

Additionally, the development plan details team roles, meeting schedules, communication standards and a 
team charter. This sets accountability for each team member and clarifies each person's role and responsibilities,
resulting in a more collaborative and consistent working environment.

In summary, creating a development plan to define objectives and team expectations before starting the project 
reduces misunderstandings, improves collaboration and communication, and overall leads to a smoother project execution. 

\subsection*{In your opinion, what are the advantages and disadvantages of using CI/CD?}

Using Continuous Integration and Continuous Deployment (CI/CD) in a software development project offers several 
advantages. Firstly, automating testing and deployment reduces the time needed to be spent on repeating the same tests which
leads to efficient and more frequent updates. The automated process can help catch bugs early on, improve code quality 
and minimize errors. It can also keep track of which commits caused which bugs to be introduced, improving organization
and traceability. The quick feedback and blocking pull requests until all tests pass when updating the codebase allows 
the developers to address problems quickly which improves overall productivity and project efficiency.

A disadvantage of CI/CD is that it requires significant time and resources to set up properly in the beginning of a 
project, especially if a team has a lot of things they would like to achieve using CI/CD. If a lot of tests are set up, 
Not all tests may be necessary for all code updates, and there could be a lot of processing time for only a little 
benefit in some cases. It also may give the team a false sense of security that the code has been fully checked, 
and some boundary cases or errors may go unnoticed because of overly relying on the automated tests. 

\subsection*{What disagreements did your group have in this deliverable, if any, and how did you resolve them?}

For this current deliverable (Problem Statement, POC and Development Plan), our group did not have any disagreements.

\newpage{}

\section*{Appendix --- Team Charter}

\subsection*{External Goals}

\begin{itemize}
\item Have something meaningful and interesting to talk about in interviews.
\item Get a 12 in the Capstone Course.
\item Design a project that's meaningful and impactful with the purpose of helping people.
\item Have an interesting, engaging, and interactive demonstration at the Capstone EXPO.
\end{itemize}

\subsection*{Attendance}

\subsubsection*{Expectations}

\indent The team's expectations regarding meeting attendance are:
\begin{itemize}
  \item Arrive on time to the agreed upon location.
  \item If a team member is running late, they should message in the team's Discord Server, indicating how far away they are.
  \item If a team member needs to leave early, they should communicate with the team ahead of time, or within the first 5 minutes of the meeting. It is their responsibility to catch up on missed content by asking team members.
  \item If a team member needs to miss a meeting, they should inform all team members, and make a plan on how they will catch up on the missed contributions. They should refer to the remaining team members to get caught up.
\end{itemize}

\subsubsection*{Acceptable Excuse}

Acceptable Excuses for missing a meeting or a deadline include:
\begin{itemize}
  \item Personal Emergency: Family or Personal
  \item Illness: Includes Mental Health
\end{itemize}

Unacceptable Excuses for missing a meeting or a deadline include:
\begin{itemize}
  \item Technical Issues: Problems with computers or the internet can be avoided through prior planning or team communication
  \item Conflicting Workload: All team members have a full course load, but are all committed to the capstone course as well.
  \item Miscommunication or Confusion: If team members are confused about details, they should openly discuss it, and not let it delay their work.
  \item Prior Commitments: Team members should organize their schedules accordingly to avoid prior commitments and scheduling conflicts from taking precedence over their work.
\end{itemize}

\subsubsection*{In Case of Emergency}

In the case of an emergency, team members must inform the rest of the team about their absence.
They do not need to state the reason aside from there being an emergency (as it may be personal).\\
\indent Team members must communicate how far they got into their individual task, and what still needs to be completed
(if they were not able to complete their task). In the case where their task has not been completed, they must
decide whether they are able to complete their individual task (delayed), or if they need to transfer
the responsibility to another team member.\\
\indent In the case their responsibility is transferred to another team member, they should communicate with said
team member how they will catch up for one of their individual tasks in the future (so equal distribution
of work is achieved by the end of the project).

\subsection*{Accountability and Teamwork}

\subsubsection*{Quality} 

The team's expectations regarding the quality of team members' preparation for team meetings
is for all work discussed to be completed prior to the team meeting is actually completed so
that the team can continue to progress to the next task or milestone.\\
\indent The team's expectations regarding the quality of deliverables is to strive for level
4's in every rubric, and to frequently refer to rubrics and guidelines to ensure each member
is on track. Team members should review each others' work prior to submission, along with
cross-checking against the posted rubrics to ensure all guidelines are met
(and in some cases, exceeded).


\subsubsection*{Attitude}

The team's expectations regarding team members' ideas are to go in with an open-mind, and to
hear all ideas out in their entirety before coming to conclusions. This will allow all ideas
to be expressed and acknowledged.\\
\indent The team's expectations regarding interactions with each other are to maintain respect
for one another, regardless of potential conflicts or disruptions. If disagreements lead to tension,
it should be figured out and openly discussed as soon as possible to come to resolutions so the team
does not suffer as a result.\\
\indent The team's expectations regarding cooperation is to feel comfortable approaching any team member
for support and help, with proper credit given. Team members should be willing to collaborate and cooperate
with other members, as it will benefit the whole team, and the final grade they will receive in the course.
In the instance that team members are over-reliant on one another, team members can express themselves to the
individual or the team, try to better understand the situation, and work towards a solution.\\
\indent The team's expectations regarding attitudes is to try to stay positive and look towards solutions, instead
of dwelling on problems. In times of stress, team members should acknowledge their own state, and not take their
frustrations out on the rest of the team. The team will support its members through periods of stress.\\
\indent Team members should contribute equally to milestones. In the case where team members feel the work has been
split unfairly, team members can express their concerns to the whole team, and the workload can be adjusted accordingly.\\
\indent The team will adapt a conflict resolution plan:
\begin{enumerate}
  \item Clarify the source of the problem and describe the conflict.
  \item Identify and understand differing viewpoints (including barriers of understanding).
  \item Establish a common goal.
  \item Find a course of action (solution) that both sides can agree to that addresses the issue.
  \item Agree on the course of action (solution).
\end{enumerate}

\subsubsection*{Stay on Track}

\indent The team will stay on track by employing frequent (1-2 times a week, depending on the milestone)
check-ins among team members to discuss the work they've completed in the past week, along with
what they are currently working on.\\
\indent Further, the team will also employ the use of the Professor's Google Calendar to ensure deadlines are met.\\
\indent In addition, the work will be completed at-least 1 day prior to the deadline to give the team time to check over
the work, compare against rubrics, address concerns, and reflect.\\
\indent Also, the team will break down milestones into smaller tasks using issues to keep track of individual tasks.\\

\indent The team will reward members who do well by going out for an additional team social, to celebrate their accomplishments.
The team will manage members whose performance is below expectations by communicating first with the member
about their output, and address it to them as an issue. If the behaviour does not change, the team can ask a member
to speak with the TA for advice. If after implementing the advice, the behaviour still does not change, the issue can
be addressed to the professor.\\
\indent The consequences for someone not contributing their fair share include: potential loss on peer evaluation, a discussion
with the TA/professor, or a difference/loss of grades (in comparison to the team's final grade).

Target Metrics:
\begin{itemize}
  \item Attendance: 95\% of all team meetings
  \item Commits: All members must contribute to Meaningful commits per Deliverable. (Meaningful means: 10-15 lines of code (minimum) changed per commit). 
  commits should be reflective upon the amount of work divided, 1/4 workload for each member.
  \item Confirm review of all documents prior to submission
  \item Confirm review of all rubric criteria prior to submission
\end{itemize}

If someone doesn't hit their targets they must explain which target they didn't meet and why.\\
\indent If the targets are repeatedly not met, the team must make an appointment with the TA to discuss strategies on
how to better achieve these targets, and ensure equal contribution among the team.\\
\indent The incentive for reaching targets early is an additional vote on where to host the next team social.\\

\subsubsection*{Team Building}
\begin{itemize}
  \item The team will build cohesion through (minimum) monthly team socials.
  \item The details of the events will be decided upon unanimously, during a time of low-stress,
  that works for all team members.
\end{itemize}

\subsubsection*{Decision Making} 
\begin{itemize}
  \item The team will make decisions through consensus, as voting could lead to solutions that
  not every member is comfortable with.
  \item The team will handle disagreements through openly stating their viewpoints and
  reasoning, and all team members will share their views on the benefits and drawbacks of
  the strategies to reach a conclusion that works for everyone.
\end{itemize}
\end{document}