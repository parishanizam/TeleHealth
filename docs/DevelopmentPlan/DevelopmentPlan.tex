\documentclass{article}

\usepackage{booktabs}
\usepackage{tabularx}

\usepackage{indentfirst}

\title{Development Plan\\\progname}

\author{\authname}

\date{}

%% Comments

\usepackage{color}

\newif\ifcomments\commentstrue %displays comments
%\newif\ifcomments\commentsfalse %so that comments do not display

\ifcomments
\newcommand{\authornote}[3]{\textcolor{#1}{[#3 ---#2]}}
\newcommand{\todo}[1]{\textcolor{red}{[TODO: #1]}}
\else
\newcommand{\authornote}[3]{}
\newcommand{\todo}[1]{}
\fi

\newcommand{\wss}[1]{\authornote{blue}{SS}{#1}} 
\newcommand{\plt}[1]{\authornote{magenta}{TPLT}{#1}} %For explanation of the template
\newcommand{\an}[1]{\authornote{cyan}{Author}{#1}}

%% Common Parts

\newcommand{\progname}{Software Engineering} % PUT YOUR PROGRAM NAME HERE
\newcommand{\authname}{Team \#22, TeleHealth Insights
\\ Mitchell Weingust
\\ Parisha Nizam
\\ Promish Kandel
\\ Jasmine Sun-Hu} % AUTHOR NAMES                  

\usepackage{hyperref}
    \hypersetup{colorlinks=true, linkcolor=blue, citecolor=blue, filecolor=blue,
                urlcolor=blue, unicode=false}
    \urlstyle{same}
                                


\begin{document}

\maketitle

\begin{table}[hp]
\caption{Revision History} \label{TblRevisionHistory}
\begin{tabularx}{\textwidth}{p{1.5cm}p{3cm}X}
\toprule
\textbf{Date} & \textbf{Developer(s)} & \textbf{Change}\\
\midrule
20/09/24 & Jasmine Sun-Hu, Mitchell Weingust & Added: Team Identifiers, Confidential Information, Intellectual Property,
Copyright License, Team Meeting Plan\\
22/09/24 & Jasmine Sun-Hu & Added: Team Communication Plan, Team Member Roles, Workflow Plan, Project Decomposition and Scheduling\\
22/09/24 & Mitchell Weingust & Added: Appendix - Team Charter\\
Date & Name(s) & Change(s)\\
\bottomrule
\end{tabularx}
\end{table}

\newpage{}

\wss{Put your introductory blurb here.  Often the blurb is a brief roadmap of
what is contained in the report.}

\wss{Additional information on the development plan can be found in the
\href{https://gitlab.cas.mcmaster.ca/courses/capstone/-/blob/main/Lectures/L02b_POCAndDevPlan/POCAndDevPlan.pdf?ref_type=heads}
{lecture slides}.}

\section{Confidential Information?}

There is no confidential information to protect, therefore there is no agreement.

\section{IP to Protect}

There is no intellectual property to protect, therefore there is no agreement.

\section{Copyright License}

Mozilla Public License 2.0 (MPL-2.0) \\
\url{https://github.com/parishanizam/TeleHealth/blob/main/LICENSE}

\section{Team Meeting Plan}

The team will meet in-person at least once a week every Monday from 3:30-4:30 pm. Exceptions to this 
may include when the University is closed, statuatory holidays, or a group consensus to postpone the meeting
is agreed upon. Additional meetings can be held in person or virtually through the team's discord server on 
a per need basis. Location and timing will be decided as a group at least 3 hours in advance. Team meetings will
be structured as follows:
\begin{itemize}
  \item 5-10 minutes of progress check-in 
  \item 5 minutes of agenda debrief
  \item 40 minutes of executing the agenda
  \item 5-10 minutes of discussing next steps
\end{itemize}
The meeting chair will be decided at least 24 hours prior to the meeting, and rotate on a weekly basis.

Meetings with the project's supervisor will take place in-person every Tuesday from 9:45-10:15 am. Exceptions 
to this may include when the University is closed, statuatory holidays, or a group consensus to postpone the 
meeting is agreed upon.

\section{Team Communication Plan}

Communication is essential for a successful project. The following is an outline of how the team will 
communicate, the tools/platforms we will use, and the expectations for each team member regarding communication.

Communication Tools
\begin{itemize}
  \item Github: a github repository will be used for code versioning, project tracking, and technical documentation. 
  Additional details are as follows:
  \begin{itemize}
    \item Project board: used to track milestones and visualize the workflow of the project using Kanban style columns
    \item Issues: used to keep track of tasks and meetings, and delegating tasks among team members. Labels are used
     to categorize issues.
  \end{itemize}
  \item Discord: a discord server will be used for day-to-day communication and online meetings. Below outlines the 
  discord server structure:
  \begin{itemize}
    \item general: text channel for general updates, quick questions and informal discussions
    \item documents-and-resources: text channel for relevant files or useful links that do not belong in the github 
    project folder
    \item meetings: text channel for co-ordinating ad-hoc meetings between some or all team members.
    \item external-meetings: text channel for co-ordinating and reviewing meeting agendas with individuals 
    outside the core capstone team (e.g. capstone supervisor).
    \item help: text channel for questions or issues that need prioritized attention.
    \item Office: voice channel to hold any online meetings.
  \end{itemize}
  \item E-mail: school emails will be used to communicate with individuals outside the core capstone team such as the 
  capstone supervisor, capstone professor, external professionals, etc.
\end{itemize}

\section{Team Member Roles}

All team members are responsible for writing documentation, coding, testing, and creating/commenting on issues no 
matter their role.

\vspace{0.5cm}
\begin{tabularx}{\textwidth}{|l|l|X|}
  \hline
  \textbf{Team Member} & \textbf{Role} & \textbf{Responsibilities} \\
  \hline
  TBD & Team Liaison & Chairs external meetings, handles the communication between the team and the capstone 
  supervisor, course instructors, TAs and any other external individuals relevant to the project.\\
  \hline
  TBD & Project Manager & Chairs team meetings, oversees the project timeline, ensures milestones and deadlines 
  are met, and that team members contribute appropriately.\\
  \hline
  TBD & Lead Developer & In charge of managing and leading the technical design, coordination and testing of the 
  project, responsible for helping teammates with technical challenges. \\
  \hline
  TBD & UI/UX Design Lead & In charge of overseeing the user interface and user experience components of the project, 
  responsible for user research and usability testing.\\
  \hline
\end{tabularx}

\section{Workflow Plan}

\subsection{Git Strategy}
\begin{itemize}
	\item The main branch will contain code that has been approved, tested, and is considered production-ready by 
  the team.
	\item Branches will be created based on deliverables, features, or bug fixes and be named clearly based on their 
  purpose (e.g. problem-statement, development-plan, front-end/navbar, etc.) 
  \item Once the deliverable, feature or bug fix is completed and merged into the main branch it may be deleted.
	\item Once a deliverable or bug fix is complete, a pull request will be made. The PR description will include:
	\begin{itemize}
    \item The purpose of the change
    \item The issue number it addresses with a link to the issue
    \item Details of testing and/or any relevant references
  \end{itemize}
	\item All PRs require at least one teammate to review it, with the exception of documentation updates that are 
  able to automatically merge.
	\item GitHub PR comments will be used to provide feedback
	\item Issues will be created on Git for project management (see 7.2)
\end{itemize}

\subsection{Issue Management}
\begin{itemize}
	\item Every new task (documentation, features, bugs, updates, meetings) will be logged as a GitHub Issue.
	\item Team members may either use one of the provided templates or create a blank issue. For blank issues, 
  they must include:
	\begin{itemize}
    \item Title that gives a clear overview of the issue
    \item Description of the task, including any necessary background or context
    \item Links to related issues necessary for the completion of the current issue if applicable.
    \item Label tags, Milestone category, and Project assignment*.
    \item Assignees
  \end{itemize}
  \item *Label tags are based on the type of issue (e.g. documentation), milestone categories 
  are the type of deliverable (e.g. Development Plan), and all issues are displayed on one project board (see 8.1)
\end{itemize}

\subsection{Use of CI/CD}
\begin{itemize}
	\item GitHub Actions will be used to create and manage continuous integration workflow scripts.
	\begin{itemize}
    \item Running a code linting tool and automatically enforcing style guides
    \item restricting the ability to merge pull requests that do not meet our coding standards (see section 11)
    \item Ensure that the code builds successfully on each push
  \end{itemize}
	\item If all tests pass, a pull request may be merged into a branch.
	\item The lead developer will ensure all necessary code is covered by automated tests.
	\item A rollback strategy will be included in the CD pipeline in case the project needs 
  to be reverted to a previously stable state.
\end{itemize}

\section{Project Decomposition and Scheduling}

\subsection{GitHub Projects}

GitHub Projects will be used to manage and keep track of the project's schedule. 
The link to our project board can be found \href{https://github.com/users/parishanizam/projects/2/views/1}{here}
The project board is organized by rows and columns, where each row is a different 
milestone that can be minimized and expanded, and each column is as follows:
\begin{itemize}
  \item Backlog: Issues that have assigned low priority
  \item To Do: Newly created issues
  \item In Progress: Issues actively being worked on
  \item In Review: Issues awaiting approval from other teammates
  \item Done: Completed Issues
\end{itemize}
Each issue is categorized into a milestone, and team members will update their statuses as they 
progress through the project deliverables.

\subsection{Project Timeline}

\subsection{Forming Team + Project Selection (Due Sept 16)}
	\begin{itemize}
		\item Form a team (Sept 6)
		\item Meet with potential supervisors (Sept 12)
		\item Select project
	\end{itemize}

\subsection{Project Planning Documentation (Due Sept 24)}
	\begin{itemize}
		\item Draft Problem Statement
		\item Create POC Plan
		\item Create Development Plan
	\end{itemize}

	\subsection{Requirements Document Revision 0 (Due October 9)}
	\begin{itemize}
		\item Research stakeholders
		\item Define scope, purpose and context of the system
		\item Define use cases and functional requirements
		\item Define non-functional requirements
		\item Brainstorm potential challenges
		\item Create traceability matrices and graphs
	\end{itemize}

	\subsection{Hazard Analysis 0 (Due October 23)}
	\begin{itemize}
		\item Define scope and purpose of hazard Analysis
		\item Define boundaries, components and assumptions
		\item Create FMEA table
		\item Define safety and security requirements
		\item Create roadmap
	\end{itemize}

	\subsection{V\&V Plan 0 (Due November 1)}
	\begin{itemize}
		\item Define V\&V plan and logistics
		\item Define system tests for functional and nonfunctional requirements
		\item Define unit tests
	\end{itemize}

	\subsection{Proof of Concept Demonstration (November 11 - 22)}
	\begin{itemize}
		\item Prepare POC demonstration
	\end{itemize}

	\subsection{Design Document (Due January 15)}
	\begin{itemize}
		\item List potential changes
		\item Define connections between requirements and design
		\item Define module decomposition
		\item Design user interface
		\item Schedule timeline
	\end{itemize}

	\subsection{Revision 0 Demonstration (February 3 - February 14)}
	\begin{itemize}
		\item Plan demonstration 
		\item Conduct user testing and feedback 
		\item Finalize demonstration
	\end{itemize}

	\subsection{V\&V Report Revision 0 (Due March 7)}
	\begin{itemize}
		\item Evaluate functional requirements
		\item Evaluate nonfunctional requirements
		\item Evaluate testing methods
		\item Trace to requirements and modules
		\item evaluate code coverage metrics
	\end{itemize}

	\subsection{Final Demonstration (Revision 1) (March 24 - March 30)}
	\begin{itemize}
		\item Test and finalize project
		\item Create script and slideshow
		\item Practice/prepare demonstration presentation
	\end{itemize}

	\subsection{EXPO Demonstration (April TBD)}
	\begin{itemize}
		\item Create EXPO event poster
		\item Set up project for live user interaction
		\item Prepare main talking points
	\end{itemize}

	\subsection{Final Documentation (Revision 1) (April 2)}
	\begin{itemize}
		\item Problem statement
		\item Development plan
		\item Proof of Concept (POC) Plan
		\item Requirements Document
		\item Hazard Analysis
		\item Design Document
		\item V\&V Plan
		\item V\&V Report
		\item User’s Guide
		\item Source Code
	\end{itemize}

\section{Proof of Concept Demonstration Plan}

What is the main risk, or risks, for the success of your project?  What will you
demonstrate during your proof of concept demonstration to convince yourself that
you will be able to overcome this risk?

\section{Expected Technology}

\wss{What programming language or languages do you expect to use?  What external
libraries?  What frameworks?  What technologies.  Are there major components of
the implementation that you expect you will implement, despite the existence of
libraries that provide the required functionality.  For projects with machine
learning, will you use pre-trained models, or be training your own model?  }

\wss{The implementation decisions can, and likely will, change over the course
of the project.  The initial documentation should be written in an abstract way;
it should be agnostic of the implementation choices, unless the implementation
choices are project constraints.  However, recording our initial thoughts on
implementation helps understand the challenge level and feasibility of a
project.  It may also help with early identification of areas where project
members will need to augment their training.}

Topics to discuss include the following:

\begin{itemize}
\item Specific programming language
\item Specific libraries
\item Pre-trained models
\item Specific linter tool (if appropriate)
\item Specific unit testing framework
\item Investigation of code coverage measuring tools
\item Specific plans for Continuous Integration (CI), or an explanation that CI
  is not being done
\item Specific performance measuring tools (like Valgrind), if
  appropriate
\item Tools you will likely be using?
\end{itemize}

\wss{git, GitHub and GitHub projects should be part of your technology.}

\section{Coding Standard}

\wss{What coding standard will you adopt?}

\newpage{}

\section*{Appendix --- Reflection}

\wss{Not required for CAS 741}

The purpose of reflection questions is to give you a chance to assess your own
learning and that of your group as a whole, and to find ways to improve in the
future. Reflection is an important part of the learning process.  Reflection is
also an essential component of a successful software development process.  

Reflections are most interesting and useful when they're honest, even if the
stories they tell are imperfect. You will be marked based on your depth of
thought and analysis, and not based on the content of the reflections
themselves. Thus, for full marks we encourage you to answer openly and honestly
and to avoid simply writing ``what you think the evaluator wants to hear.''

Please answer the following questions.  Some questions can be answered on the
team level, but where appropriate, each team member should write their own
response:


\begin{enumerate}
    \item Why is it important to create a development plan prior to starting the
    project?
    \item In your opinion, what are the advantages and disadvantages of using
    CI/CD?
    \item What disagreements did your group have in this deliverable, if any,
    and how did you resolve them?
\end{enumerate}

\newpage{}

\section*{Appendix --- Team Charter}

\subsection*{External Goals}

\begin{itemize}
\item Have something meaningful and interesting to talk about in interviews.
\item Get a 12 in the Capstone Course.
\item Design a project that's meaningful and impactful with the purpose of helping people.
\item Have an interesting, engaging, and interactive demonstration at the Capstone EXPO.
\end{itemize}

\subsection*{Attendance}

\subsubsection*{Expectations}

\indent The team's expectations regarding meeting attendance are:
\begin{itemize}
  \item Arrive on time to the agreed upon location.
  \item If a team member is running late, they should message in the team's Discord Server, indicating how far away they are.
  \item If a team member needs to leave early, they should communicate with the team ahead of time, or within the first 5 minutes of the meeting. It is their responsibility to catch up on missed content by asking team members.
  \item If a team member needs to miss a meeting, they should inform all team members, and make a plan on how they will catch up on the missed contributions. They should refer to the remaining team members to get caught up.
\end{itemize}

\subsubsection*{Acceptable Excuse}

Acceptable Excuses for missing a meeting or a deadline include:
\begin{itemize}
  \item Personal Emergency: Family or Personal
  \item Illness: Includes Mental Health
\end{itemize}

Unacceptable Excuses for missing a meeting or a deadline include:
\begin{itemize}
  \item Technical Issues: Problems with computers or the internet can be avoided through prior planning or team communication
  \item Conflicting Workload: All team members have a full course load, but are all committed to the capstone course as well.
  \item Miscommunication or Confusion: If team members are confused about details, they should openly discuss it, and not let it delay their work.
  \item Prior Commitments: Team members should organize their schedules accordingly to avoid prior commitments and scheduling conflicts from taking precedence over their work.
\end{itemize}

\subsubsection*{In Case of Emergency}

In the case of an emergency, team members must inform the rest of the team about their absence.
They do not need to state the reason aside from there being an emergency (as it may be personal).\\
\indent Team members must communicate how far they got into their individual task, and what still needs to be completed
(if they were not able to complete their task). In the case where their task has not been completed, they must
decide whether they are able to complete their individual task (delayed), or if they need to transfer
the responsibility to another team member.\\
\indent In the case their responsibility is transferred to another team member, they should communicate with said
team member how they will catch up for one of their individual tasks in the future (so equal distribution
of work is achieved by the end of the project).

\subsection*{Accountability and Teamwork}

\subsubsection*{Quality} 

The team's expectations regarding the quality of team members' preparation for team meetings
is for all work discussed to be completed prior to the team meeting is actually completed so
that the team can continue to progress to the next task or milestone.\\
\indent The team's expectations regarding the quality of deliverables is to strive for level
4's in every rubric, and to frequently refer to rubrics and guidelines to ensure each member
is on track. Team members should review each others' work prior to submission, along with
cross-checking against the posted rubrics to ensure all guidelines are met
(and in some cases, exceeded).


\subsubsection*{Attitude}

The team's expectations regarding team members' ideas are to go in with an open-mind, and to
hear all ideas out in their entirety before coming to conclusions. This will allow all ideas
to be expressed and acknowledged.\\
\indent The team's expectations regarding interactions with each other are to maintain respect
for one another, regardless of potential conflicts or disruptions. If disagreements lead to tension,
it should be figured out and openly discussed as soon as possible to come to resolutions so the team
does not suffer as a result.\\
\indent The team's expectations regarding cooperation is to feel comfortable approaching any team member
for support and help, with proper credit given. Team members should be willing to collaborate and cooperate
with other members, as it will benefit the whole team, and the final grade they will receive in the course.
In the instance that team members are over-reliant on one another, team members can express themselves to the
individual or the team, try to better understand the situation, and work towards a solution.\\
\indent The team's expectations regarding attitudes is to try to stay positive and look towards solutions, instead
of dwelling on problems. In times of stress, team members should acknowledge their own state, and not take their
frustrations out on the rest of the team. The team will support its members through periods of stress.\\
\indent Team members should contribute equally to milestones. In the case where team members feel the work has been
split unfairly, team members can express their concerns to the whole team, and the workload can be adjusted accordingly.\\
\indent The team will adapt a conflict resolution plan:
\begin{enumerate}
  \item Clarify the source of the problem and describe the conflict.
  \item Identify and understand differing viewpoints (including barriers of understanding).
  \item Establish a common goal.
  \item Find a course of action (solution) that both sides can agree to that addresses the issue.
  \item Agree on the course of action (solution).
\end{enumerate}

\subsubsection*{Stay on Track}

\indent The team will stay on track by employing frequent (1-2 times a week, depending on the milestone)
check-ins among team members to discuss the work they've completed in the past week, along with
what they are currently working on.\\
\indent Further, the team will also employ the use of the Professor's Google Calendar to ensure deadlines are met.\\
\indent In addition, the work will be completed at-least 1 day prior to the deadline to give the team time to check over
the work, compare against rubrics, address concerns, and reflect.\\
\indent Also, the team will break down milestones into smaller tasks using issues to keep track of individual tasks.\\

\indent The team will reward members who do well by going out for an additional team social, to celebrate their accomplishments.
The team will manage members whose performance is below expectations by communicating first with the member
about their output, and address it to them as an issue. If the behaviour does not change, the team can ask a member
to speak with the TA for advice. If after implementing the advice, the behaviour still does not change, the issue can
be addressed to the professor.\\
\indent The consequences for someone not contributing their fair share include: potential loss on peer evaluation, a discussion
with the TA/professor, or a difference/loss of grades (in comparison to the team's final grade).

Target Metrics:
\begin{itemize}
  \item Attendance: 95\% of all team meetings
  \item Commits: 
  \item Confirm review of all documents prior to submission
  \item Confirm review of all rubric criteria prior to submission
\end{itemize}

If someone doesn't hit their targets they must explain which target they didn't meet and why.\\
\indent If the targets are repeatedly not met, the team must make an appointment with the TA to discuss strategies on
how to better achieve these targets, and ensure equal contribution among the team.\\
\indent The incentive for reaching targets early is an additional vote on where to host the next team social.\\

\subsubsection*{Team Building}

The team will build cohesion through (minimum) monthly team socials.\\
\indent The details of the events will be decided upon unanimously, during a time of low-stress,
that works for all team members.

\subsubsection*{Decision Making} 

The team will make decisions through consensus, as voting could lead to solutions that
not every member is comfortable with.\\
\indent The team will handle disagreements through openly stating their viewpoints and
reasoning, and all team members will share their views on the benefits and drawbacks of
the strategies to reach a conclusion that works for everyone.

\end{document}