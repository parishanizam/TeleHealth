\documentclass{article}

\usepackage{booktabs}
\usepackage{tabularx}

\title{Development Plan\\\progname}

\author{\authname}

\date{}

\input{../Comments}
\input{../Common}

\begin{document}

\maketitle

\begin{table}[hp]
\caption{Revision History} \label{TblRevisionHistory}
\begin{tabularx}{\textwidth}{p{1.5cm}p{3cm}X}
\toprule
\textbf{Date} & \textbf{Developer(s)} & \textbf{Change}\\
\midrule
20/09/24 & Jasmine Sun-Hu, Mitchell Weingust & Added: Team Identifiers, Confidential Information, Intellectual Property,
Copyright License, Team Meeting Plan\\
22/09/24 & Jasmine Sun-Hu & Added: Team Communication Plan, Team Member Roles, Workflow Plan, Project Decomposition and Scheduling\\
Date & Name(s) & Change(s)\\
\bottomrule
\end{tabularx}
\end{table}

\newpage{}

\wss{Put your introductory blurb here.  Often the blurb is a brief roadmap of
what is contained in the report.}

\wss{Additional information on the development plan can be found in the
\href{https://gitlab.cas.mcmaster.ca/courses/capstone/-/blob/main/Lectures/L02b_POCAndDevPlan/POCAndDevPlan.pdf?ref_type=heads}
{lecture slides}.}

\section{Confidential Information?}

There is no confidential information to protect, therefore there is no agreement.

\section{IP to Protect}

There is no intellectual property to protect, therefore there is no agreement.

\section{Copyright License}

Mozilla Public License 2.0 (MPL-2.0) \\
\url{https://github.com/parishanizam/TeleHealth/blob/main/LICENSE}

\section{Team Meeting Plan}

The team will meet in-person at least once a week every Monday from 3:30-4:30 pm. Exceptions to this 
may include when the University is closed, statuatory holidays, or a group consensus to postpone the meeting
is agreed upon. Additional meetings can be held in person or virtually through the team's discord server on 
a per need basis. Location and timing will be decided as a group at least 3 hours in advance. Team meetings will
be structured as follows:
\begin{itemize}
  \item 5-10 minutes of progress check-in 
  \item 5 minutes of agenda debrief
  \item 40 minutes of executing the agenda
  \item 5-10 minutes of discussing next steps
\end{itemize}
The meeting chair will be decided at least 24 hours prior to the meeting, and rotate on a weekly basis.

Meetings with the project's supervisor will take place in-person every Tuesday from 9:45-10:15 am. Exceptions 
to this may include when the University is closed, statuatory holidays, or a group consensus to postpone the 
meeting is agreed upon.

\section{Team Communication Plan}

Communication is essential for a successful project. The following is an outline of how the team will 
communicate, the tools/platforms we will use, and the expectations for each team member regarding communication.

Communication Tools
\begin{itemize}
  \item Github: a github repository will be used for code versioning, project tracking, and technical documentation. 
  Additional details are as follows:
  \begin{itemize}
    \item Project board: used to track milestones and visualize the workflow of the project using Kanban style columns
    \item Issues: used to keep track of tasks and meetings, and delegating tasks among team members. Labels are used
     to categorize issues.
  \end{itemize}
  \item Discord: a discord server will be used for day-to-day communication and online meetings. Below outlines the 
  discord server structure:
  \begin{itemize}
    \item general: text channel for general updates, quick questions and informal discussions
    \item documents-and-resources: text channel for relevant files or useful links that do not belong in the github 
    project folder
    \item meetings: text channel for co-ordinating ad-hoc meetings between some or all team members.
    \item external-meetings: text channel for co-ordinating and reviewing meeting agendas with individuals 
    outside the core capstone team (e.g. capstone supervisor).
    \item help: text channel for questions or issues that need prioritized attention.
    \item Office: voice channel to hold any online meetings.
  \end{itemize}
  \item E-mail: school emails will be used to communicate with individuals outside the core capstone team such as the 
  capstone supervisor, capstone professor, external professionals, etc.
\end{itemize}

\section{Team Member Roles}

All team members are responsible for writing documentation, coding, testing, and creating/commenting on issues no 
matter their role.

\vspace{0.5cm}
\begin{tabularx}{\textwidth}{|l|l|X|}
  \hline
  \textbf{Team Member} & \textbf{Role} & \textbf{Responsibilities} \\
  \hline
  TBD & Team Liaison & Chairs external meetings, handles the communication between the team and the capstone 
  supervisor, course instructors, TAs and any other external individuals relevant to the project.\\
  \hline
  TBD & Project Manager & Chairs team meetings, oversees the project timeline, ensures milestones and deadlines 
  are met, and that team members contribute appropriately.\\
  \hline
  TBD & Lead Developer & In charge of managing and leading the technical design, coordination and testing of the 
  project, responsible for helping teammates with technical challenges. \\
  \hline
  TBD & UI/UX Design Lead & In charge of overseeing the user interface and user experience components of the project, 
  responsible for user research and usability testing.\\
  \hline
\end{tabularx}

\section{Workflow Plan}

\subsection{Git Strategy}
\begin{itemize}
	\item The main branch will contain code that has been approved, tested, and is considered production-ready by 
  the team.
	\item Branches will be created based on deliverables, features, or bug fixes and be named clearly based on their 
  purpose (e.g. problem-statement, development-plan, front-end/navbar, etc.) 
  \item Once the deliverable, feature or bug fix is completed and merged into the main branch it may be deleted.
	\item Once a deliverable or bug fix is complete, a pull request will be made. The PR description will include:
	\begin{itemize}
    \item The purpose of the change
    \item The issue number it addresses with a link to the issue
    \item Details of testing and/or any relevant references
  \end{itemize}
	\item All PRs require at least one teammate to review it, with the exception of documentation updates that are 
  able to automatically merge.
	\item GitHub PR comments will be used to provide feedback
	\item Issues will be created on Git for project management (see 7.2)
\end{itemize}

\subsection{Issue Management}
\begin{itemize}
	\item Every new task (documentation, features, bugs, updates, meetings) will be logged as a GitHub Issue.
	\item Team members may either use one of the provided templates or create a blank issue. For blank issues, 
  they must include:
	\begin{itemize}
    \item Title that gives a clear overview of the issue
    \item Description of the task, including any necessary background or context
    \item Links to related issues necessary for the completion of the current issue if applicable.
    \item Label tags, Milestone category, and Project assignment*.
    \item Assignees
  \end{itemize}
  \item *Label tags are based on the type of issue (e.g. documentation), milestone categories 
  are the type of deliverable (e.g. Development Plan), and all issues are displayed on one project board (see 8.1)
\end{itemize}

\subsection{Use of CI/CD}
\begin{itemize}
	\item GitHub Actions will be used to create and manage continuous integration workflow scripts.
	\begin{itemize}
    \item Running a code linting tool and automatically enforcing style guides
    \item restricting the ability to merge pull requests that do not meet our coding standards (see section 11)
    \item Ensure that the code builds successfully on each push
  \end{itemize}
	\item If all tests pass, a pull request may be merged into a branch.
	\item The lead developer will ensure all necessary code is covered by automated tests.
	\item A rollback strategy will be included in the CD pipeline in case the project needs 
  to be reverted to a previously stable state.
\end{itemize}

\section{Project Decomposition and Scheduling}

\subsection{GitHub Projects}

GitHub Projects will be used to manage and keep track of the project's schedule. 
The link to our project board can be found \href{https://github.com/users/parishanizam/projects/2/views/1}{here}
The project board is organized by rows and columns, where each row is a different 
milestone that can be minimized and expanded, and each column is as follows:
\begin{itemize}
  \item Backlog: Issues that have assigned low priority
  \item To Do: Newly created issues
  \item In Progress: Issues actively being worked on
  \item In Review: Issues awaiting approval from other teammates
  \item Done: Completed Issues
\end{itemize}
Each issue is categorized into a milestone, and team members will update their statuses as they 
progress through the project deliverables.

\subsection{Project Timeline}

\subsection{Forming Team + Project Selection (Due Sept 16)}
	\begin{itemize}
		\item Form a team (Sept 6)
		\item Meet with potential supervisors (Sept 12)
		\item Select project
	\end{itemize}

\subsection{Project Planning Documentation (Due Sept 24)}
	\begin{itemize}
		\item Draft Problem Statement
		\item Create POC Plan
		\item Create Development Plan
	\end{itemize}

	\subsection{Requirements Document Revision 0 (Due October 9)}
	\begin{itemize}
		\item Research stakeholders
		\item Define scope, purpose and context of the system
		\item Define use cases and functional requirements
		\item Define non-functional requirements
		\item Brainstorm potential challenges
		\item Create traceability matrices and graphs
	\end{itemize}

	\subsection{Hazard Analysis 0 (Due October 23)}
	\begin{itemize}
		\item Define scope and purpose of hazard Analysis
		\item Define boundaries, components and assumptions
		\item Create FMEA table
		\item Define safety and security requirements
		\item Create roadmap
	\end{itemize}

	\subsection{V\&V Plan 0 (Due November 1)}
	\begin{itemize}
		\item Define V\&V plan and logistics
		\item Define system tests for functional and nonfunctional requirements
		\item Define unit tests
	\end{itemize}

	\subsection{Proof of Concept Demonstration (November 11 - 22)}
	\begin{itemize}
		\item Prepare POC demonstration
	\end{itemize}

	\subsection{Design Document (Due January 15)}
	\begin{itemize}
		\item List potential changes
		\item Define connections between requirements and design
		\item Define module decomposition
		\item Design user interface
		\item Schedule timeline
	\end{itemize}

	\subsection{Revision 0 Demonstration (February 3 - February 14)}
	\begin{itemize}
		\item Plan demonstration 
		\item Conduct user testing and feedback 
		\item Finalize demonstration
	\end{itemize}

	\subsection{V\&V Report Revision 0 (Due March 7)}
	\begin{itemize}
		\item Evaluate functional requirements
		\item Evaluate nonfunctional requirements
		\item Evaluate testing methods
		\item Trace to requirements and modules
		\item evaluate code coverage metrics
	\end{itemize}

	\subsection{Final Demonstration (Revision 1) (March 24 - March 30)}
	\begin{itemize}
		\item Test and finalize project
		\item Create script and slideshow
		\item Practice/prepare demonstration presentation
	\end{itemize}

	\subsection{EXPO Demonstration (April TBD)}
	\begin{itemize}
		\item Create EXPO event poster
		\item Set up project for live user interaction
		\item Prepare main talking points
	\end{itemize}

	\subsection{Final Documentation (Revision 1) (April 2)}
	\begin{itemize}
		\item Problem statement
		\item Development plan
		\item Proof of Concept (POC) Plan
		\item Requirements Document
		\item Hazard Analysis
		\item Design Document
		\item V\&V Plan
		\item V\&V Report
		\item User’s Guide
		\item Source Code
	\end{itemize}

\section{Proof of Concept Demonstration Plan}

What is the main risk, or risks, for the success of your project?  What will you
demonstrate during your proof of concept demonstration to convince yourself that
you will be able to overcome this risk?

\section{Expected Technology}

\wss{What programming language or languages do you expect to use?  What external
libraries?  What frameworks?  What technologies.  Are there major components of
the implementation that you expect you will implement, despite the existence of
libraries that provide the required functionality.  For projects with machine
learning, will you use pre-trained models, or be training your own model?  }

\wss{The implementation decisions can, and likely will, change over the course
of the project.  The initial documentation should be written in an abstract way;
it should be agnostic of the implementation choices, unless the implementation
choices are project constraints.  However, recording our initial thoughts on
implementation helps understand the challenge level and feasibility of a
project.  It may also help with early identification of areas where project
members will need to augment their training.}

Topics to discuss include the following:

\begin{itemize}
\item Specific programming language
\item Specific libraries
\item Pre-trained models
\item Specific linter tool (if appropriate)
\item Specific unit testing framework
\item Investigation of code coverage measuring tools
\item Specific plans for Continuous Integration (CI), or an explanation that CI
  is not being done
\item Specific performance measuring tools (like Valgrind), if
  appropriate
\item Tools you will likely be using?
\end{itemize}

\wss{git, GitHub and GitHub projects should be part of your technology.}

\section{Coding Standard}

\wss{What coding standard will you adopt?}

\newpage{}

\section*{Appendix --- Reflection}

\wss{Not required for CAS 741}

\input{../Reflection.tex}

\begin{enumerate}
    \item Why is it important to create a development plan prior to starting the
    project?
    \item In your opinion, what are the advantages and disadvantages of using
    CI/CD?
    \item What disagreements did your group have in this deliverable, if any,
    and how did you resolve them?
\end{enumerate}

\newpage{}

\section*{Appendix --- Team Charter}

\wss{borrows from
\href{https://engineering.up.edu/industry_partnerships/files/team-charter.pdf}
{University of Portland Team Charter}}

\subsection*{External Goals}

\wss{What are your team's external goals for this project? These are not the
goals related to the functionality or quality fo the project.  These are the
goals on what the team wishes to achieve with the project.  Potential goals are
to win a prize at the Capstone EXPO, or to have something to talk about in
interviews, or to get an A+, etc.}

\subsection*{Attendance}

\subsubsection*{Expectations}

\wss{What are your team's expectations regarding meeting attendance (being on
time, leaving early, missing meetings, etc.)?}

\subsubsection*{Acceptable Excuse}

\wss{What constitutes an acceptable excuse for missing a meeting or a deadline?
What types of excuses will not be considered acceptable?}

\subsubsection*{In Case of Emergency}

\wss{What process will team members follow if they have an emergency and cannot
attend a team meeting or complete their individual work promised for a team
deliverable?}

\subsection*{Accountability and Teamwork}

\subsubsection*{Quality} 

\wss{What are your team's expectations regarding the quality
of team members' preparation for team meetings and the quality of the
deliverables that members bring to the team?}

\subsubsection*{Attitude}

\wss{What are your team's expectations regarding team members' ideas,
interactions with the team, cooperation, attitudes, and anything else regarding
team member contributions?  Do you want to introduce a code of conduct?  Do you
want a conflict resolution plan?  Can adopt existing codes of conduct.}

\subsubsection*{Stay on Track}

\wss{What methods will be used to keep the team on track? How will your team
ensure that members contribute as expected to the team and that the team
performs as expected? How will your team reward members who do well and manage
members whose performance is below expectations?  What are the consequences for
someone not contributing their fair share?}

\wss{You may wish to use the project management metrics collected for the TA and
instructor for this.}

\wss{You can set target metrics for attendance, commits, etc.  What are the
consequences if someone doesn't hit their targets?  Do they need to bring the
coffee to the next team meeting?  Does the team need to make an appointment with
their TA, or the instructor?  Are there incentives for reaching targets early?}

\subsubsection*{Team Building}

The team will build cohesion through (minimum) monthly team socials.
The details of the events will be decided upon unanimously, during a time of low-stress,
that works for all team members.

\subsubsection*{Decision Making} 

\wss{How will you make decisions in your group? Consensus?  Vote? How will you
handle disagreements? }

\end{document}